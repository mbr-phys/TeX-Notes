\documentclass[a4paper, 11pt, normalem]{report}

\usepackage{../../../LaTeX-Templates/Notes}
\usepackage{subfiles}

\title{Mathematics Workshop \vspace{-20pt}}
\author{Prof R Gregory, Dr D Cerdeno, and Prof T Theuns}
\date{\vspace{-15pt}Michaelmas Term 2018 - Epiphany Term 2019}
\rhead{\hyperlink{page.1}{Contents}}

\begin{document}

\maketitle
\tableofcontents

\part{Complex Analysis}
\chapter{The Complex Plane}
\section{Basics}
Course will use Riley, Hobson and Bence Chapters 3, 24, 25

Recall $i = \sqrt{-1}$ allows us to extend our notion of numbers as we go from a $\R$eal to the $\C$omplex plane.
\begin{equation}
    \C = \{z = x + iy\;|\;x,y \in \R\}
\end{equation}
We represent the complex plane with an argand diagram.
\begin{align}
    x &= \Rl(z) = \frac{z + \bar{z}}{2} \\
    y &= \Iy(z) = \frac{z - \bar{z}}{2} \\
    r &= \mathrm{mod}(z) = \sqrt{x^2 + y^2} \\
    \theta &= \mathrm{Arg}(z) = \arctan\left(\frac{y}{x}\right)
\end{align}
Notice that $e^{i\theta} = e^{i\theta + 2i\pi}$, so $\theta$ is not uniquely defined, so choose a range for $\theta$, e.g. $\theta \in [0, 2\pi), \theta \in (-\pi,\pi]$.

\section{Complex Functions}
\begin{itemize}
    \item $\C \to \C$
    \item $z \to f(z) = u(x,y) + iv(x,y)$
    \item Like two functions of real variables, e.g. $f(z) = z^2 = (x + iy)^2 = (x^2 - y^2) + 2ixy$
    \begin{itemize}
        \item $u(x,y) = x^2 - y^2$
        \item $v(x,y) = 2xy$
    \end{itemize}
\item Exponential: $e^z, \mathrm{exp}(z) = \sum_{n=0}^{\infty} \frac{z^n}{n!}$
\end{itemize}

\section{Branch Cuts}
For real numbers, $1^2 = (-1)^2 = 1$, you have two roots; but functions should be single valued. \\
$\sqrt{}$ is fine on the Real line - choose a root and stick to it, but for the Complex plane, say $\sqrt{1} = 1$, the disc around $z = 1$ means a loss of continuity.
\begin{align}
    z &= re^{i\theta} \\
    \sqrt{z} &= \sqrt{r}e^{i\theta/2}, \theta \in (0, 2\pi)
\end{align}
We can choose this branch, but must cut the complex plane along the reals so that $\theta$ can't run higher than $2\pi$, or could choose
\begin{equation}
    \sqrt{z} = \sqrt{r}e^{i\theta/2 +i\pi}
\end{equation}
but still need the same cut, so no ambiguity in definition of the function's square root.
Here, the function's square root is double valued.
\begin{equation}
    \log(re^{i\theta}) = \log(r) + i\theta
\end{equation}
$\theta$ is not unique, $\theta + 2n\pi$ is also a legitimate answer.
Again, cut the plane somewhere, decide on the branch of the log.
Can cut on the positive reals or negatives.
Branch choices appear around zeroes because polar coordinates are singular there: $\theta$ is not specified at $r = 0$.
We say $z=0$ is a branch point, indicated with a wavy line.
Branch cuts from branch point either to infinity or another branch point.

\begin{example}
\begin{equation}
    f(z) = \sqrt{z^4 + 1}
\end{equation}
Think of this through two steps: $z \to z^4 + 1 \to \sqrt{z^4 + 1}$.
There is a branch cut around $z^4 + 1 = 0$.
There are four branch points:
\begin{align}
    z^4 &= -1 \\
    z &= e^{i\pi/4}...
\end{align}
Around a branch point:
\begin{align}
    z &= e^{i\pi/4} + \epsilon e^{i\theta + 3i\pi/4} \\
    z^4 &= -1 + 4\epsilon e^{i\theta +3i\pi/4} \\
    \sqrt{z^4 + 1} &= 2\sqrt{\epsilon}e^{i\theta/2 + 3i\pi/8}
\end{align}
This gives rise to the same problem as circling the origin for $\sqrt{z}$.
This example has other choices for branch cuts, some may be able to limit ambiguities to limits.
\end{example}

\section{Trig and Hyperbolic Functions}
Trig functions can be generalised to include complex numbers, and can be expressed as exponentials in the usual way, similarly with hyperbolics.
$\sinh$ and $\cosh$ have periodicity of $2\pi i$.
\begin{align}
    \cos(z) &= \cosh(iz) \\
    \sin(z) &= i\sinh(iz)
\end{align}

\chapter{Complex Differentiation and Cauchy-Riemann}
\section{Continuity}
Definition:
\begin{equation}
    \lim_{z\to z_0} f(z) = w \in \C \iff \forall \e > 0 \;\exists \;\delta > 0 \; s.t.\; |z-z_0| < \delta \implies |f(z) - w| < \e
\end{equation}
\emph{If you are close to a point $z_0$, then f(z) is close to $f(z_0)$.}
f is continuous at $z_0 \iff \lim_{z\to z_0} f(z) = f(z_0)$.

Note that the limit must be path independent, and the real and imaginary parts must be continuous.

\section{Differentation}
Recall
\begin{equation}
    \df = \lim_{h\to 0} \frac{f(x+h) - f(x)}{h}
\end{equation}
For $\C$ having two $\R$ dimensions, we have $\grad, \grad\cdot, \grad\times$.
\begin{itemize}
    \item[X] Grad - acts on scalars ($\R$)
    \item[X] Div - acts on vectors ($\C$), gives scalar
    \item[X] Curl - in 3D
\end{itemize}
Try
\begin{equation}
    \frac{df}{dz} = \lim_{\delta\to 0} \frac{f(z+\delta) - f(z)}{\delta}, \delta, z \in \C
\end{equation}
Require limit independent of direction.

\begin{example}
$f(z) = z^2$:
\begin{align}
    & \lim_{\delta\to 0} \frac{(z+\delta)^2 - z^2}{\delta} \\
    \implies & \lim_{\delta\to 0} \frac{\cancel{z^2} + 2z\cancel{\delta} + \delta^{\cancel{2}} - \cancel{z^2}}{\cancel{\delta}} = 2z
\end{align}
What about $f(z) = \bar{z}$?
\begin{equation}
    \frac{\bar{z} + \bar{\delta} - \bar{z}}{\delta} = \frac{\bar{\delta}}{\delta} = \exp{(-2i\mathrm{Arg}(\delta))}
\end{equation}
Not path independent, so no limit.
\end{example}

$\frac{df}{dz}$ is the complex derivative of f at z, and f is differentiable at z if this limit exists.

\section{Analytic Functions}
An analytic function is a complex function that is differentiable, at least in some region.

Definition: A neighbourhood of $z \in \C$ is an open set U such that $z \in U$.

Definition: $f: \C \to \C$ is analytic/holomorphic at $z_0 \in \C$ if $\exists$ a neighbourhood U of $z_0$ on which f is differentiable $\forall \; z \in U$.

\begin{example}
    \begin{align}
        |z|^2 &= z\bar{z} \\
        \lim_{\delta\to 0} \frac{(z+\delta)(\bar{z} + \bar{\delta}) - z\bar{z}}{\delta} &= \bar{z} + \frac{\bar{\delta}}{\delta}z
    \end{align}
If $z \neq 0$, no limit, but $z = 0$ has limit, 0, $\implies$ differentiable at $z = 0$, \underline{but not analytic.}
\end{example}
$z^n$ is differentiable everywhere, so analytic on all $\C$.

\section{Cauchy-Riemann Equations}
We have
\begin{align}
    f(z) &= f(x + iy) \\
         &= u(x,y) + iv(x,y)
\end{align}
Derivation:
\begin{align}
    \frac{f(z_0 + \delta) - f(z_0)}{\underbrace{\delta}_{z - z_0}} &= \frac{u(x_0 + \delta x, y_0 + \delta y) - u(x_0,y_0)}{\delta x + i\delta y} + i\frac{v(x_0 + \delta x, y_0 + \delta y) - v(x_0,y_0)}{\delta x + i\delta y} \\
                                                                   &= \frac{u(x_0,y_0) + \delta x \frac{\p u}{\p x}|_0 + \delta y \frac{\p u}{\p y}|_0 - u_0}{\delta x + i\delta y} + i\frac{v_0 + \delta x \frac{\p v}{\p x}|_0 + \delta y\frac{\p v}{\p y} - v_0}{\delta x + i\delta y} \\
                                                                   &= \frac{\left(\frac{\p u}{\p x} + i\frac{\p v}{\p x}\right)\delta x + i\left(\frac{\p v}{\p y} - i\frac{\p u}{\p y}\right)\delta y}{\delta x + i\delta y}
\end{align}
For complex differentiation, limit must be independed of $\delta x + i\delta y$, so the numerator must factorise as $() \times (\delta x + i\delta y)$.
\begin{equation}
    \implies \frac{\p u}{\p x} + i\frac{\p v}{\p x} = \frac{\p v}{\p y} - i\frac{\p u}{\p y}
\end{equation}
This gives the C-R relations:
\begin{align}
    \frac{\p u}{\p x} &= \frac{\p v}{\p y} \\
    \frac{\p v}{\p x} &= -\frac{\p u}{\p y}
\end{align}
Check with $z^2 = (x^2 - y^2) + 2ixy$:
\begin{align}
    \frac{\p u}{\p x} &= 2x & \frac{\p v}{\p y} &= 2x \\
    \frac{\p u}{\p y} &= -2y & \frac{\p v}{\p x} &= 2y
\end{align}
Note:
\begin{equation}
    \frac{\p u}{\p x} = \frac{\p v}{\p y} \implies \frac{\p^2 u}{\p x^2} = \frac{\p^2 v}{\p x \p y} = -\frac{\p^2 u}{\p y^2}
\end{equation}
Or
\begin{equation}
    \frac{\p^2 u}{\p x^2} + \frac{\p^2 u}{\p y^2} = \del^2 u = 0
\end{equation}
Suppose we regard z and $\bar{z}$ as independed variables.
\begin{equation}
    x = \frac{z + \bar{z}}{2},~ y = \frac{z - \bar{z}}{2i}
\end{equation}
\begin{align}
    \frac{\p}{\p \bar{z}} &= \frac{\p x}{\p \bar{z}}\frac{\p}{\p \p x} + \frac{\p y}{\p \bar{z}}\frac{\p}{\p y} = \frac{1}{2}\left(\frac{\p}{\p x} + i\frac{\p}{\p y}\right) \\
    \frac{\p}{\p z} &= \frac{\p x}{\p z}\frac{\p}{\p x} + \frac{\p y}{\p z}\frac{\p}{\p y} = \frac{1}{2}\left(\frac{\p}{\p x} - i\frac{\p}{\p y}\right) \\
    \frac{\p f}{\p \bar{z}} &= \frac{1}{2}\left(\frac{\p u}{\p x} + i\frac{\p v}{\p x} + i\frac{\p u}{\p y} - \frac{\p v}{\p y}\right) \\
                    &= \frac{1}{2}\left(\left(\frac{\p u}{\p x} - \frac{\p v}{\p y}\right) + i\left(\frac{\p v}{\p x} + \frac{\p u}{\p y}\right) \right) \\
                    &= 0 \iff \text{CR satisfied}
\end{align}
Can express analytically/differentiable as $\frac{df}{d\bar{z}} = 0$.

\chapter{Complex Integration and Cauchy's Theorem}
\section{Subsets of the Plane - Curves and Domains}
A continuous curve, $\gamma$, in $\C$ is a map.
\begin{equation}
    \gamma:~ [a,b] \subset \R \to \C
\end{equation}
$[a,b]$ is an interval, $\gamma(a)$ is the starting point, and $\gamma(b)$ is the final point.
$-\gamma$ is the 'opposite' curve - it is the same path but opposite direction.
\begin{equation}
    \gamma:~ x \in [0,2\pi] \to \C,~ z = e^{ix}
\end{equation}
This is a unit circle anti-clockwise.
$e^{-ix}$ is $-\gamma$, a unit circle but clockwise.

An open set G is said to be split if
\begin{equation}
    G = G_1 \cup G_2
\end{equation}
such that $G_1$, $G_2$ are open, not empty, and
\begin{equation}
    G_1 \cap G_2 = \phi
\end{equation}
G is connected if it does not split.
This means that any two points in G can be connected by a curve lying in G.
\begin{equation}
    z_1,z_2 \in G: \exists\; \gamma \; s.t. \; \gamma_i = z_1, \gamma_f = z_2, \gamma \subset G.
\end{equation}
G is simply connected if any pair of curves connecting any pair of points in G can be continuously deformed into each other without leacing G.

\section{Complex Integration}
For real integration, there was no ambiguity in the path. For complex integration however, we have many paths.

We must specify the curve of integration,
\begin{align}
    \int_\gamma f(z)\,dz &= \int_\gamma (u+iv)(dx + idy) \\
                         &= \int_\gamma (u\,dx - v\,dy) + i(v\,dx + u\,dy)
\end{align}
$dx$ and $dy$ are determined y $\gamma$.
\begin{align}
    dx &= x'(t)\,dt \\
    dy &= y'(t)\,dt
\end{align}
$t \in [a,b]$, parameterising $\gamma$.

\begin{example}
    Find $\int_\gamma z^2\,dz$ for $\gamma: [0,\pi] \to \C, t \to e^{it}$.
    \begin{align}
        x &= \cos t \\
        y &= \sin t \\
        u &= \cos^2t - \sin^2t = \cos 2t \\
        v &= 2\sin t\cos t = \sin 2t \\
        \implies dx &= -\sin t\, dt \\
        \implies dy &= \cos t\,dt \\
        \int_\gamma u\,dx - v\,dy &= \int_{0}^{\pi} [-\cos 2t \sin t - \sin 2t\cos t]\,dt \\
                                  &= -\int_{0}^{\pi} \sin 3t\,dt \\
                                  &= \left[\frac{1}{3}\cos 3t\right]_{0}^{\pi} \\
                                  &= -\frac{2}{3} \\
        \int_\gamma v\,dx + u\,dy &= \int_{0}^{\pi} [-\sin 2t\sin t + \cos 2t\cos 2]\,dt \\
                                  &= \int_{0}^{\pi} \cos 3t\,dt \\
                                  &= 0
    \end{align}
    Note: $\int_{-1}^{1} x^2\,dx = [\frac{1}{3}x^3]_{-1}^{1} = -\frac{2}{3}$.
    Also, $-\frac{2}{3}$ if $\gamma$ is lower semi-circle - the result seems to be path independent.
\end{example}

\section{Cauchy's Theorem}
Let $D \subset \C$ be a simply connected open subset of $\C$, and if $f: D \to \C$ is analytic on D, then for any closed curve $C \subset D$:
\begin{equation}
    \oint_C f(z)\,dz = 0.
\end{equation}
C indicates a closed curve. $\oint$ indicates integral round closed curve.
$\oint$ is a contour integral and C the contour of integration.
Justify with real analysis.

Recall Green's theorem:
\begin{equation}
    \oint_C \left[A\,dx + B\,dy\right] = \iint_S \left[\frac{\p B}{\p x} - \frac{\p A}{\p y}\right]\,dx\,dy
\end{equation}
Then
\begin{align}
    \oint_C f(z)\,dz &= \oint_C [u\,dx - v\,dy] + i[u\,dy + v\,dx] \\
                     &= \iint_S \left[-\frac{\p v}{\p x} - \frac{\p u}{\p y}\right] - i\left[\frac{\p v}{\p y} - \frac{\p u}{\p x}\right].
\end{align}
Here, we use simple-connectedness.
But f is analytic in D, so analytic in S, hence by CR relations, this integral is 0.
Note, if $\gamma_1,\gamma_2$ are 2 curves from $z_1 \to z_2$, $\gamma_1 - \gamma_2$ is a closed curve, hence
\begin{equation}
    \int_{\gamma_1} f = \int_{\gamma_2} f, \text{ for analytic f.}
\end{equation}
$C = \gamma_2 - \gamma_1$ closed.

\emph{The integral of analytic functions are independent of the path.}

This also allows us to define an integral function as we now know it is path independent:
\begin{equation}
    F(z_2) - F(z_1) = \int_\gamma f\,dz,~ \frac{dF}{dz} = f.
\end{equation}
Finally, if $C_1$ and $C_2$ are closed contours enclosing surfaces $D_1$ and $D_2$, then if f is analytic on $D = D_2 - D_1$,
\begin{equation}
    \oint_{C_1} f(z_1)\,dz = \oint_{C_2}f(z)\,dz.
\end{equation}
Note that f doesn't need to be analytic on $D_1$ or $D_2$, e.g. $C_1 = e^{it}, C_2 = 2e^{it}, t \in [0,2\pi], f(z) = \frac{1}{z^2}$. This is not analytic on $D_1$ or $D_2$, but it is on $D_2 - D_1$.

\chapter{Cauchy's Integral Formula and Taylor Series}
Last time:
\begin{equation}
    \oint_C f(z)\,dz = 0
\end{equation}
if f is analytic inside C, and
\begin{equation}
    \oint_{C_1} = \oint_{C_2}
\end{equation}
if f is analytic between $C_1$ and $C_2$. \\
Apply this to $f(z) = \frac{1}{z}$. Let $C_R$ be a circular contour $|z| = R$.
\begin{align}
    C_R &= \left\{z | z = Re^{i\theta}\right\} \\
    \oint_{C_R} \frac{dz}{z} &= \int_0^{2\pi} \frac{Re^{i\theta}i\;d\theta}{Re^{i\theta}} = 2\pi i
\end{align}
This result is independent of R.

\section{Cauchy's Integral Formula}
Let $f(z)$ be analytic on a simply-connected domain D, $C = \p D$ is the contour formed by the boundary of D.
Then for $z_0 \in D$,
\begin{equation}
    f(z_0) = \frac{1}{2\pi i}\oint_C \frac{f(z)}{z - z_0}\,dz.
\end{equation}
Note: this says that the value of a harmonic function in a domain is specified by the values on its boundary.

\subsection{Proof}
\begin{equation}
    \frac{f(z)}{z - z_0}
\end{equation}
is not analytic on D, but on $D - z_0$.

Define $D_\e = \{z\, |\, |z-z_0| \leq \e\} \subset D$.
$(D - D_\e$ f is analytic on, hence
\begin{align}
    \oint_C \frac{f(z)}{z - z_0}\,dz &= \oint_{C_\e} \frac{f(z)}{z - z_0}\,dz \\
                                     &= \int_{0}^{2\pi} \frac{f(z_0 + \e e^{i\theta})i\e e^{i\theta}}{\e e^{i\theta}}\,d\theta
\end{align}
But f is analytic in D, so differentiable at $z_0$, hence continuous at $z_0$.
By the definition of continuity,
\begin{align}
    |f(z_0 + \e e^{i\theta}) &- f(z_0)| < \delta_{(\e)} \\
    \implies f(z_0 + \e e^{i\theta}) &= f(z_0) + O(\e) \\
    \implies \oint_{C} \frac{f(z)}{z - z_0}\,dz &= 2\pi if(z_0) + O(\e)
\end{align}
Taking $\e \to 0$ gives result.

\begin{example}[Checking]
    \begin{align}
        f(z) &= z+a,z_0 = 0 \\
        \oint_{C_1} \frac{f(z)}{z}\,dz &= \oint_{C_1} \left(1 + \frac{a}{z}\right)\,dz \\
                                       &= 2\pi ia = 2\pi if(0)
    \end{align}
\end{example}

What about $f'$?
\begin{align}
    f'(z_0) &= \lim_{\delta \to 0} \frac{f(z_0 + \delta) - f(z_0)}{\delta} \\
            &= \lim_{\delta \to 0} \frac{1}{2\pi i\delta}\oint_C \left(\frac{f(z)}{z-(z_0+\delta)} - \frac{f(z)}{z - z_0}\right)\,dz \\
            &= \lim_{\delta\to 0} \frac{1}{2\pi i\delta} \oint_C \frac{f(z)\,dz\;\delta}{(z-z_0)(z-z_0-\delta)} \\
            &= \lim_{\delta\to 0} \frac{1}{2\pi i} \oint_C \frac{f(z)\,dz}{(z-z_0)(z-z_0-\delta)} \\
    f'(z_0) &= \frac{1}{2\pi i} \oint_C \frac{f(z)}{(z-z_0)^2}\,dz
\end{align}
Can repeat and prove by induction, that
\begin{equation}
    f^{(n)}(z_0) = \frac{n!}{2\pi i} \oint_C \frac{f(z)}{(z-z_0)^{n+1}}\,dz
\end{equation}
Every analytic function is infinitely differentiable.

\begin{example}
Check formula for $f'$, $f = e^{2z}, C = 2e^{i\theta}, z_0 = 1$:
    \begin{align}
        \oint_{C_2} \frac{e^{2z}}{(z-1)^2}\,dz &= \oint_{C_\e} \frac{e^{2(w+1)}}{w^2}\,dw,~ w = z-1, C_2 \to C_{\e} = 1 + \e e^{i\theta} \\
                                               &= \int_{0}^{2\pi} \frac{i\e e^{i\theta}\,d\theta \times e^{2 + 2\e e^{i\theta}}}{(\e e^{i\theta})^2} \\
        e^{2\e e^{i\theta}} &= e^{2\e(\cos\theta + i\sin\theta)} \\
                            &= e^{2\e\cos\theta(\cos(2\e\sin\theta) + i\sin(2\e\sin\theta))} \\
                            &= (1 + 2\e\cos\theta + \cdots)(1 + 2i\e\sin\theta + O(\e^2)) \\
                            &= 1 + 2\e\cos\theta + 2i\e\sin\theta + \cdots \\
                            &= 1 + 2\e e^{i\theta} + O(\e^2) \\
        \oint_{C_2} \frac{e^{2z}}{(z-1)^2}\,dz &= \int_0^{2\pi} \frac{e^2i\,d\theta\,(1 + 2\e e^{i\theta})}{\e e^{i\theta}} \\
                                               &= 2\pi i \cdot \underbrace{2e^2}_{f'(1)}
    \end{align}
\end{example}

\section{Taylor Series}
Let f be analytic on a simply-connected domain D.
Then for every $z_0 \in D$, there is a radius r such that for $|z - z_0| < r$,
\begin{equation}
    f(z) = \sum_{n=0}^{\infty} \frac{f^{(n)}(z_0)}{n!} (z - z_0)^n.
\end{equation}

\subsection{Proof}
Uses
\begin{align}
    \frac{1}{1 - \e} &= \sum_0^\infty \e^n ~\forall \;|\e| < 1 \\
    \sum_{n=0}^\infty \frac{f^{(n)}(z_0)}{n!}(z-z_0)^n &= \sum_0^\infty \frac{1}{2\pi i} \oint_C \frac{f(\zeta)\,d\zeta}{(\zeta - z_0)^{n+1}}(z - z_0)^n \\
                                                       &= \frac{1}{2\pi i}\sum_{n=0}^\infty \oint_C \frac{f(\zeta)}{\zeta - z_0} \left(\frac{z - z_0}{\zeta - z_0}\right)^n\,d\zeta
\end{align}
Make sure to take C such that $|\zeta - z_0| > |z - z_0|$.
Swap $\oint$ and $\sum$:
\begin{align}
    \sum_{n=0}^\infty \frac{f^{(n)}(z_0)}{n!} (z-z_0)^n &= \frac{1}{2\pi i} \oint_C \frac{f(\zeta)}{\zeta - z_0} \frac{1}{\left(1 - \frac{z-z_0}{\zeta-z_0}\right)}\,d\zeta \\
    &= \frac{1}{2\pi i} \oint_C \frac{f(\zeta)}{\zeta - z}\,d\zeta = f(z)
\end{align}
Note that while a finite polynomial in z is an entire function, an infinite series may not be.

If
\begin{equation}
    \mathcal{P}(z) = \sum_{n=0}^\infty a_nz^n
\end{equation}
then $\mathcal{P}(z)$ converges if
\begin{equation}
    \lim_{n\to\infty} \bigg|\frac{a_{n+1}}{a_n}z\bigg| = |z|\lim_{n\to\infty}\bigg|\frac{a_{n+1}}{a_n}\bigg| < 1
\end{equation}
which gives us a radius of convergence
\begin{align}
    R &= \frac{1}{\lim_{n\to\infty}\big|\frac{a_{n+1}}{a_n}\big|} \\
    \mathcal{P}(z) &= \sum_0^\infty z^n, \frac{a_{n+1}}{a_n} = 1, R = 1 \\
    e^z &= \sum_0^\infty \frac{z^n}{n!}, R = \infty
\end{align}

\chapter{Zeros and Poles}
\section{Liouville Theorem}
Recall:
\begin{align}
    f(z) &= \sum_{n=0}^\infty \frac{f^{(n)}(z_0)}{n!}(z - z_0)^n \tag{Taylor Series} \\
    f^{(n)}(z_0) &= \frac{n!}{2\pi i} \oint_C \frac{f(z)}{(z-z_0)^{n+1}}\,dz \tag{Cauchy Int Form}
\end{align}
Liouville: \emph{Every bounded entire function is constant.}

\subsection{Proof}
Consider
\begin{align}
    f^{(n)}(z_0) &= \frac{n!}{2\pi i}\oint_{C_R} \frac{f(z)}{(z-z_0)^{n+1}}\,dz \\
    C_R &= \{z\; |\; z = z_0 + Re^{i\theta}\}, R > |z_0|
\end{align}
Since f is bounded, $\exists\,M$ such that $|f(z)| < M$ on $\C$.
\begin{align}
    |f^{(n)}(z_0)| &= \frac{n!}{2\pi}\bigg|\oint_{C_R} \frac{f(z)}{(z-z_0)^{n+1}}\,dz\bigg| < \frac{n!}{2\pi}\oint_{C_R} \frac{|f|\;|dz|}{|z-z_0|^{n+1}} \\
                   &< \frac{n!}{2\pi} \int_{0}^{2\pi} d\theta\; \frac{M}{R^n} = \frac{n! M}{R^n}
\end{align}
but RHS $\to 0$ as $R \to \infty$, hence
\begin{equation}
    f^{(n)}(z_0) \equiv 0
\end{equation}
for all $n \geq 1$.
The only Taylor series coefficient left is $n=0$, i.e. f is constant.

\section{Zeros and Singularities}
An analytic function has a zero at $z_0$ if we can write (around $z_0$)
\begin{equation}
    f(z) = (z-z_0)^n g(z)
\end{equation}
where $g(z) \neq 0$, g is analytic in a region around $z_0$.
This zero has multiplicity n, or "a zero of order n."

\underline{Zeros of an analytic function are isolated.}
Suppose there exists a subset around zero of f that is also zero ($f(z_0) = 0$).
\begin{equation}
    f'(z_0) = \lim_{\delta\to 0} \frac{f(z_0 + \delta) - f(z_0)}{\delta}
\end{equation}
Choose to take $\delta$ along the path, so that $f(z_0 + \delta) = f(z_0) = 0$.
This gives $f'(z_0) = 0$ since the limit is path independent.
By iteration, all derivatives, $f^{(n)}(z_0) = 0$.
Therefore, f is a constant around $z_0$, i.e. 0.
Either f is identically zero, or zeros are isolated.

\underline{Definition}: Let f be analytic in a domain $D - \{z_0\}$, then f has a pole at $z_0$ if we can write
\begin{equation}
    f(z) = \frac{g(z)}{(z-z_0)^m}, g(z_0) \neq 0
\end{equation}
g is analytic at $z_0$.
The order of the pole is m.

A function is \underline{meromorphic} if it is analytic/holomorphic except for isolated poles or singularities.

Note:
\begin{align}
    \oint_C \frac{dz}{z} &= 2\pi i \\
    \oint \frac{dz}{z^2} &= 0
\end{align}
Contour integration picks out the "$\frac{1}{z}$" piece.
For analytic g:
\begin{equation}
    g(z) = g(z_0) + g'(z_0)(z-z_0) + \frac{g''(z_0)}{2}(z-z_0)^2 + \cdots
\end{equation}
Construct meromorphic f:
\begin{align}
    f(z) &= \frac{g(z)}{(z-z_0)^2} = \frac{g(z_0)}{(z-z_0)^2} + \frac{g'(z_0)}{(z-z_0)} + \frac{g''(z_0)}{2} + \cdots \\
    \oint_C f(z)\,dz &= \oint_C \frac{g(z_0)}{(z-z_0)^2} + \frac{g'(z_0)}{(z-z_0)} + \cdots
\end{align}
This leads to a general theorem about contour integration.
Suppose f has a simple pole at $z_0$ (order 1), then
\begin{align}
    \oint_C f(z)\,dz &= 2\pi i\lim_{z\to z_0} (z-z_0)f(z) \\
    f(z) &= \frac{g(z)}{z-z_0}
\end{align}
For a simple pole, hence
\begin{equation}
    (z-z_0)f = g
\end{equation}
g is analytic, so
\begin{align}
    g(z) &= g(z_0) + O(z-z_0) \\
    \implies \oint_C f(z)\,dz &= \oint_C \frac{g(z_0) + \cdots}{(z-z_0)}\,dz = 2\pi ig(z_0)
\end{align}

\begin{example}
    \begin{align}
        \oint_{C_2} \frac{dz}{z(z-1)} \\
        \frac{1}{z(z-1)} &= \frac{1}{z-1} - \frac{1}{z} \\
        f_1(z) &= \frac{1}{z-1} \to \oint_{C_2} \frac{dz}{z-1} = 2\pi i \\
        \oint_{C_2} \frac{dz}{z} &= 2\pi i \\
        \implies \oint_{C_2} f\,dz &= 0
    \end{align}
    But for pole at $z = 1$,
    \begin{equation}
        \lim_{z\to 1} (z-1)f = \lim_{z\to 1} \frac{1}{z} = 1
    \end{equation}
    and for pole at $z=0$,
    \begin{equation}
        \lim_{z\to 0} zf = \lim_{z\to 0} \frac{1}{z-1} = -1
    \end{equation}
    Sum of these is 0.
    Call this coefficient of $\frac{1}{z-z_0}$ the residue.
\end{example}

\begin{example}
    \begin{align}
        \oint_C \frac{e^{2z}}{z^2 - 4}\,dz &= 2\pi i \times (z = 2) \\
        \lim_{z\to 2} (z-2)f(z) &= \lim_{z\to 2} \frac{e^{2z}}{z+2} = \frac{e^4}{4}
    \end{align}
\end{example}

Next time: Laurent Series
\begin{equation}
    f(z) = \sum_{n\to 0}^{\infty} a_n(z - z_0)^n
\end{equation}
where
\begin{equation}
    a_n = \frac{1}{2\pi i} \oint_C \frac{f(z)}{(z-z_0)^{n+1}}\,dz
\end{equation}

\chapter{Meromorphic Functions and Laurent Series}
Recall: A meromorphic function in a domain D is holomorphic except for isolated poles.

\section{Laurent's Theorem}
Let $f(z)$ be holomorphic in the annulus $\{|z-z_0|\,\in\, (r_1,r_2)\}, f(z)$ undefined at $z_0$.
Then
\begin{align}
    f(z) &= \sum_{n=-\infty}^\infty a_n(z-z_0)^n \\
         &= \underbrace{\sum_0^\infty a_n(z-z_0)^n}_{\text{Taylor}} + \underbrace{\sum_1^\infty a_{-n}(z-z_0)^{-n}}_{\text{Principal part}} \\
    a_n &= \frac{1}{2\pi i}\oint_C \frac{f(z)}{(z-z_0)^{n+1}}\,dz
\end{align}
C is any simply-closed, noncontractible curve in the annulus.

\subsection{Proof}
Without loss of generality, take $z_0 = 0, c_1 = r_1e^{i\theta}, c_2 = r_2e^{i\theta}$.
By Cauchy,
\begin{align}
    f(z) &= \frac{1}{2\pi i}\oint_{C'} \frac{f(\zeta)}{\zeta - z}\,d\zeta \\
         &= \frac{1}{2\pi i}\Bigg[\underbrace{\oint_{C_2} \frac{f(\zeta)}{\zeta - z}\,d\zeta}_{f_2} - \underbrace{\oint_{C_1} \frac{f(\zeta)}{\zeta-z}\,d\zeta}_{f_1}\Bigg] \\
    f_2(z) &= \frac{1}{2\pi i}\oint_{C_2} \frac{f(\zeta)}{\zeta-z}\,d\zeta,~ (\zeta - z) = \zeta\left(1 - \frac{z}{\zeta}\right) \\
           &= \frac{1}{2\pi i}\oint_{C_2} \frac{f(\zeta)}{\zeta} \sum_0^\infty \left(\frac{z}{\zeta}\right)^n d\zeta \\
           &= \frac{1}{2\pi i}\oint_{C_2} \frac{f(\zeta)}{\zeta} \sum_0^N \left(\frac{z}{\zeta}\right)^nd\zeta + \frac{1}{2\pi i}\oint_{C_2} \frac{f(\zeta)}{\zeta} \sum_{N+1}^\infty \left(\frac{z}{\zeta}\right)^n d\zeta \\
           &= \sum_0^N a_nz^n + \frac{1}{2\pi i} \oint_{C_2}\left(\frac{z}{\zeta}\right)^{N+1} \frac{f(\zeta)}{\zeta - z} d\zeta \\
    \bigg| \oint_{C_2} \left(\frac{z}{\zeta}\right)^{N+1} \frac{f(\zeta)}{\zeta - z}d\zeta\bigg| &< \oint_{C_2} \bigg|\frac{z}{\zeta}\bigg|^{N+1} \bigg|\frac{f(\zeta)}{\zeta - z}\bigg|\,|d\zeta| \\
    f_2(z) &= \sum_0^N a_nz^n  \\
    f_1(z) &= -\frac{1}{2\pi i} \oint_{C_1} \frac{f(\zeta)}{\zeta - z}d\zeta \\
           &= \frac{1}{2\pi i}\oint_{C_1} \frac{f(\zeta)}{z} \sum_{0}^\infty \left(\frac{\zeta}{z}\right)^n d\zeta \\
           &= \sum_0^\infty \frac{1}{2\pi i} \oint_{C_1} f(\zeta) \zeta^n z^{-(n+1)} d\zeta \\
           &= \sum_{m=1}^\infty \frac{1}{2\pi i} \oint_{C_1} \frac{f(\zeta)}{\zeta^{1-m}}d\zeta\,z^{-m} \\
           &= \sum_{m=1}^\infty a_mz^{-m} \\
    f(z) &= f_1(z) + f_2(z) = \sum_{-\infty}^\infty a_nz^n
\end{align}
Clearly, if $f(z)$ has a pole at $z_0$, the Laurent series terminates at some finite negative n, the other of the pole.
If the series does not terminate at negative n, then f has an \unl{essential singularity} at $z_0$.

\begin{example}
    \begin{align}
        f(z) &= \exp\left(\frac{1}{z}\right) \\
             &= \sum_0^\infty \frac{1}{n!z^n},~ (x\to 0^+, f\to \infty) \\
             &= \sum_{-\infty}^0 \frac{z^n}{n!},~ (x\to 0^-, f\to 0)
    \end{align}
    The singularity is "nasty" and f takes all values (except possibly one) in any neighbourhood of $z=0$.
\end{example}

\section{Integration of meromorphic functions}
The poles or singularities of a meromorphic function are isolated.
Can now use Laurent series to identify contour integrals.
\begin{align}
    \oint_C f(z)\,dz &= \oint_C \Bigg[-\sum_{-\infty}^{-1} a_n(z-z_0)^n + \underbrace{\sum_0^\infty a_n(z-z_0)^n}_{0\text{, by Cauchy}}\Bigg] \\
    \oint_C (z-z_0)^n\,dz &= 0, \text{ unless } n = -1 \\
    \oint_C f(z)\,dz &= 2\pi i a_{-1}
\end{align}
This is the definition of $a_{-1}$, the \unl{residue} or $f(z)$ at $z_0$.

\subsection{Calculating Residues}
Techniques depend on the situation:
\begin{itemize}
    \item Simple poles:
        \begin{equation}
            f(z) = \frac{g(z)}{z-z_0}
        \end{equation}
        g is holomorphic at $z_0$, so
        \begin{equation}
            \text{Res}(f,z_0) = g(z_0) = \lim_{z\to z_0} (z-z_0)f(z)
        \end{equation}
    \item Pole of order N
        \begin{equation}
            f(z) = \frac{g(z)}{(z-z_0)^N}
        \end{equation}
        g is holomorphic at $z_0$, and $\neq 0$.
        Need the $(N-1)^{th}$ coefficient of Taylor series of g.
        \begin{align}
            \text{Res}(f,z_0) &= \frac{1}{(N-1)!} g^{(N-1)}(z_0) \\
                              &= \frac{1}{(N-1)!} \lim_{z\to z_0} \frac{d^{N-1}}{dz^{N-1}} \left[(z-z_0)^Nf\right]
        \end{align}
    \item
        \begin{align}
            f(z) &= \frac{g(z)}{h(z)} \\
            \text{Res}(f,z_0) &= \frac{g(z_0)}{h'(z_0)}
        \end{align}
        g, h are holomorphic, h has a simple zero at $z_0$.
\end{itemize}
If not a simple pole, safest to series expand.
\begin{example}
    Pole or order 2 at $z=0$:
    \begin{align}
        f(z) &= \frac{1}{z\sin z} \\
        \sin z &= z - \frac{z^3}{3!} + \frac{z^5}{5!} + \cdots \\
        f(z) &= \frac{1}{z^2(1 - \frac{z^2}{6} + \cdots)} = \frac{1 + \frac{z^2}{6} + \cdots}{z^2} \\
        \text{Res}(f,z_0) &= 0
    \end{align}
\end{example}

\section{Residue Theorem}
Let f be meromorphic in a domain D, with poles at $\{z_i\}$.
Then,
\begin{equation}
    \oint_C f(z)\,dz = 2\pi i \sum \text{Res}(f,z_i)
\end{equation}

\chapter{Contour Integration}

\section{Examples}

\begin{example}[1]
    \begin{align}
        f(z) &= \frac{\cos(\pi z)}{z^2(1-z)}
    \end{align}
    Poles at $z=1$ (simple), and $z=0$ (double).

    \begin{itemize}
        \item Look at $z=1$:
            \begin{align}
                \text{Res} &= \lim_{z\to 1} (z-1)f(z) \\
                           &= \lim_{z\to 1} -\frac{\cos(\pi z)}{z^2} = 1
            \end{align}
        \item Look at $z=0$:
            \begin{align}
                \text{Res} &= \lim_{z\to 0} \frac{d}{dz}[z^2f(z)] \\
                           &= \lim_{z\to 0} \frac{d}{dz} \left[\frac{\cos(\pi z)}{(1-z)}\right] \\
                           &= \lim_{z\to 0} \left[-\frac{\pi\sin(\pi z)}{1-z} + \frac{\cos(\pi z)}{(1-z)^2}\right] \\
                           &= 1
            \end{align}
        \item Missing pole at $z=1$:
            \begin{align}
                I &= \oint_{C_{1/2}} f(z)\,dz \\
                  &= 2\pi i (\text{Res}(f,0)) = 2\pi i
            \end{align}
        \item Including pole at $z=1$:
            \begin{align}
                I &= \oint_{C_2} f(z)\,dz \\
                  &= 2\pi i(\text{Res}(f,0) + \text{Res}(f,1)) = 4\pi i
            \end{align}
    \end{itemize}
\end{example}

\begin{example}[2]
    \begin{align}
        f(z) &= \frac{e^z}{z^2 + z + 1} \\
    \end{align}
    The first job is to identify poles:
    \begin{align}
        z^2 + z + 1 &= 0 \\
        \implies z &= -\frac{1}{2} \pm \frac{1}{2} \sqrt{1-4} \\
                   &= -\frac{1}{2} \pm \frac{i\sqrt{3}}{2} \\
                   &= e^{\pm 2\pi i/3}
    \end{align}
    Now for the residues:
    \begin{enumerate}
        \item
            \begin{align}
                \text{Res}(f,e^{2\pi i/3}) &= \lim_{z\to e^{2\pi i/3}} \frac{e^z}{\left(z + \frac{1}{2} + \frac{i\sqrt{3}}{2}\right)} \\
                                           &= \frac{e^{-1/2}e^{i\sqrt{3}/2}}{i\sqrt{3}}
            \end{align}
        \item
            \begin{align}
                \text{Res}(f,e^{-2\pi i/3}) &= \lim_{z\to e^{-2\pi i/3}} \frac{e^z}{z+\frac{1}{2}-\frac{i\sqrt{3}}{2}} \\
                                            &= \frac{e^{-1/2}e^{-i\sqrt{3}/2}}{-i\sqrt{3}}
            \end{align}
    \end{enumerate}
    \begin{align}
        \oint_{C_2} f(z)\,dz &= 2\pi i \times \left[\text{Res}(f,e^{2\pi i/3} + \text{Res}(f,e^{-2\pi i/3}\right] \\
                             &= 2\pi i \left[\frac{e^{-1/2}e^{i\sqrt{3}/2}}{i\sqrt{3}} - \frac{e^{-1/2}e^{-i\sqrt{3}/2}}{i\sqrt{3}}\right] \\
                             &= 4\pi \frac{e^{-1/2}}{\sqrt{3}} \sin\left(\frac{\sqrt{3}}{2}\right)
    \end{align}
\end{example}

\section{Zeros and Poles Theorem}
Let f be a meromorphic function inside $\C$, then
\begin{align}
    \frac{1}{2\pi i}\oint_C \frac{f'(z)}{f(z)}\,dz = n_{zeros} - n_{poles}
\end{align}
Inside the contour, you count multiplicity for these.

\subsection{Proof}
If f has a pole at $z_0$, so does $f'$ (Laurent expansion or expression for pole).
Also, clearly if f has a zero at $z_0$, $\frac{1}{f}$ has a pole there, i.e. for a zero $z_0$,
\begin{align}
    f &= (z-z_0)^n g(z) \\
    f' &= (z-z_0)^n g' + n(z-z_0)^{n-1} g \\
    \implies \frac{f'}{f} &= \underbrace{\frac{g'}{g}}_{\text{regular}} + \underbrace{\frac{n}{z-z_0}}_{\text{pole (simple), Res = n}}
\end{align}
Similarly, at a pole,
\begin{align}
    f &= \frac{g}{(z-z_0)^m} \\
    \frac{f'}{f} &= \underbrace{\frac{g'}{g}}_{\text{regular}} - \underbrace{\frac{m}{z-z_0}}_{\text{simple pole, Res = m}}
\end{align}

\section{Trigonometric Integrals}
Consider an integral pf the form
\begin{align}
    I &= \int_0^{2\pi} R(\cos(m\theta),\sin(n\theta))\,d\theta
\end{align}
R is a rational function - a ratio of polynomials.
Can reinterpret as a complex integral on
\begin{align}
    C_1:\; z &= e^{i\theta} \implies d\theta = -i\frac{dz}{z} \\
    \cos(m\theta) &= \frac{e^{im\theta} + e^{-im\theta}}{2} = \frac{z^m + z^{-m}}{2} \\
    \sin(n\theta) &= \frac{z^n - z^{-n}}{2i}
\end{align}
Since R is a rational function, transforming to a complex integral in z, we see our integrand is a rational function of z, hence meromorphic.
\begin{align}
    I &= -i \oint_C \frac{dz}{z} R\left(\frac{z^m + z^{-m}}{2}, \frac{z^n - z^{-n}}{2i}\right)
\end{align}

\begin{example}
    \begin{align}
        I &= \int_0^{2\pi} \frac{d\theta}{2 - \cos\theta} \\
          &= -i \oint_{C_1} \frac{dz}{z} \frac{1}{2 - \frac{z + z^{-1}}{2}} \\
          &= \oint_{C_1} \frac{2i\,dz}{z^2 - 4z + 1}
    \end{align}
    Integrand has poles where $z^2 - 4z + 1 = 0$:
    \begin{align}
        z &= 2 \pm \sqrt{4 - 1} = 2\pm \sqrt{3}
    \end{align}
    Pole at $2 -\sqrt{3}$ is inside unit circle, so calculate residue:
    \begin{align}
    \text{Res} &= \lim_{z\to 2-\sqrt{3}} \frac{\cancel{(z - 2+\sqrt{3})}\cdot 2i}{\cancel{(z-2+\sqrt{3})}(z-2-\sqrt{3})} \\
               &= \frac{2i}{-2\sqrt{3}} = -\frac{i}{\sqrt{3}} \\
    I &= 2\pi i \text{Res}(f,2-\sqrt{3}) = 2\pi i \times -\frac{i}{\sqrt{3}} = \frac{2\pi}{\sqrt{3}}
    \end{align}
\end{example}

\begin{example}
    \begin{align}
        I &= \int_0^{2\pi} \frac{\sin\theta\;d\theta}{(3+\cos\theta)(2+\sin\theta)} \\
          &= \oint_{C_1} -\frac{i\,dz}{z} \frac{\frac{z-z^{-1}}{2i}}{\left(3 + \frac{z+z^{-1}}{2}\right)\left(2 + \frac{z-z^{-1}}{2i}\right)} \\
          &= \oint_{C_1} \frac{-2i (z^2 - 1)\;dz}{(z^2 + 6x + 1)(z^2 + 4iz - 1)}
    \end{align}
    \begin{itemize}
        \item $z^2 + 6z + 1 = 0 \implies z = -3 \pm \sqrt{8} = -3 \pm 2\sqrt{2}$.
            Pole inside $C_1$ at $z_1 = -3 + 2\sqrt{2}$.
        \item $z^2 + 4iz - 1 = 0 \implies z = -2i \pm \sqrt{-4 + 1} = -2i \pm i\sqrt{3}$.
            Pole inside $C_1$ at $z_2 = -2i + i\sqrt{3}$.
    \end{itemize}
    Now look at residues:
    \begin{enumerate}
        \item
            \begin{align}
                \text{Res}(f,z_1) &= \lim_{z\to z_1} (z-z_1)f(z) \\
                z_1^2 &= -6z_1 - 1 \\
                \implies z_1^2 - 1 &= -2 - 6z_1 = -2 + 18 -12\sqrt{2} \\
                                   &= 4\sqrt{2}z_1 \\
                z_1^2 + 4iz - 1 &= 4z_1(\sqrt{2} + i) \\
                \text{Res}(f,z_1) &= \lim_{z\to -3+2\sqrt{2}} \frac{-2i(z^2 - 1)}{(z_1 + 3 + 2\sqrt{2})(z_1^2 + 4iz -1)} \\
                                  &= \frac{-2i \times 4\sqrt{2}z_1}{4\sqrt{2} \times 4z_1 (\sqrt{2}+i)} \frac{\sqrt{2} - i}{\sqrt{2} - i} \\
                                  &= \frac{-i(\sqrt{2} - i}{2\times 3} = -\frac{i}{6}(\sqrt{2} -i)
            \end{align}
            Some reductions work for other residues.
        \item
            \begin{equation}
                \text{Res}(f,z_2) = \frac{i}{6}(\sqrt{3} - i)
            \end{equation}
    \end{enumerate}
    Hence,
    \begin{align}
        I &= 2\pi i \times \frac{i}{6}(\sqrt{3} - i - (\sqrt{2} - i)) \\
          &= -\frac{\pi}{3} (\sqrt{3} - \sqrt{2})
    \end{align}
\end{example}

\chapter{Integrals over the real line}
\section{Real integrals}
Sometimes a real integral can be more easily performed by analytic continuation to the $\C$-plane and using residues.
\begin{example}[1]
    \begin{equation}
        I = \ifnt \frac{dx}{1 + x^2}
    \end{equation}
    \begin{itemize}
        \item In real analysis: $x =\tan u$, $dx = \sec^2u\,du$, $1 + x^2 = \sec^2u$, $x = \pm \infty, u = \pm \pi/2$.
            \begin{equation}
                I = \int_{\pi/2}^{\pi/2} \frac{sec^2u\;du}{\sec^2u} = \pi
            \end{equation}
        \item In complex analysis:
            \begin{equation}
                f(z) = \frac{1}{1 + z^2}
            \end{equation}
            f has 2 simple poles at $z = \pm i$, so, 
            \begin{align}
                \text{Res}(f,\pm i) &= \lim_{z\to\pm i} (z \mp i)f(z) \\
                                    &= \lim_{z\to\pm i} \frac{1}{z\pm i} = \frac{1}{\pm 2i}
            \end{align}
            Let C be the contour
            \begin{equation}
                [-R,R] \cup \left\{Re^{i\theta} | \theta \in [0,\pi]\right\}
            \end{equation}
            For $|z| = R$, $R|f(z)| \to 0$ as $R \to \infty$.
            The integral around semi-circle goes to 0 as $R \to \infty$.
            \begin{align}
                I &= \lim_{R\to\infty} \oint_C f(z)\,dz \\
                  &= 2\pi i \text{Res}(f,i) = \pi
            \end{align}
    \end{itemize}
\end{example}

\begin{example}[2]
    \begin{align}
        I_2 &= \ifnt \frac{x+2}{(2x^2 + 3)^2}\,dx \\
        f(z) &= \frac{z+2}{(2z^2 + 3)^2}
    \end{align}
    Double poles at $2z^2 + 3 = 0$.

    As before, $R|f(Re^{i\theta})| \to 0$ as $R \to \infty$, so take some contour as before. 
    Integral around semicircle in upper $\frac{1}{2}\C$-plane $\to 0$.
    \begin{equation}
        I_2 = \lim_{R\to\infty} \oint_C f(z)\,dz
    \end{equation}
    This encloses pole at $z = +i\sqrt{\frac{3}{2}}$.
    Note:
    \begin{align}
        (2z^2 + 3)^2 &= 4\left(z^2 + \frac{3}{2}\right)^2 \\
                     &= \left(z + i\sqrt{\frac{3}{2}}\right)^2\left(z - i\sqrt{\frac{3}{2}}\right)^2 \\
        \text{Res}\left(f, i\sqrt{\frac{3}{2}}\right) &= \lim_{z\to i\sqrt{3/2}} \frac{d}{dz}\left[ \left(z - i\sqrt{\frac{3}{2}}\right)^2f(z)\right] \\
                                                      &= \lim_{z\to i\sqrt{3/2}} \left[\frac{z+2}{4(z+i\sqrt{3/2})^2}\right] \\
                                                      &= \left[\frac{1}{4(z + i\sqrt{3/2})^2} - \frac{2(z+2)}{4(z + i\sqrt{3/2})^3}\right]\times i\sqrt{\frac{3}{2}} \\
                                                      &= \frac{1}{4(i\sqrt{6})^2} - \frac{2(i\sqrt{3/2} + 2}{4(i\sqrt{6})^3} \\
                                                      &= -\frac{1}{(i\sqrt{6})^3} = \frac{1}{i6\sqrt{6}} \\
        \implies I_2 &= 2\pi i\text{Res}\left(f, i\sqrt{\frac{3}{2}}\right) \\
                     &= \frac{\pi}{3\sqrt{6}}
    \end{align}
    The more complicated the rational function, the easier the complex method. 
\end{example}

\section{Branch cuts and contours}
Sometimes we need to be careful about constructing a contour, as contours cannot cross branch cuts.
\begin{example}
    \begin{align}
        f(z) &= z^{-\alpha} = \exp(-\alpha\log(z))
    \end{align}
    Cannot integrate f around $C_R$ if $\alpha \notin \Z$.
    \begin{equation}
        I = \ofnt \frac{u^{-y}}{1 + u}\,du
    \end{equation}
    Analytically continue to $\C$:
    \begin{align}
        f(z) &= \frac{z^{-y}}{1+z} \\
        z^{-y} &= \exp(-y\log(z)) \\
        (u + i\e)^{-y} &= u^{-y}\exp\left(-y\left(\frac{i\e}{u}\right)\right)
    \end{align}
    Here, choosing standard branch for log, $\log(re^{i\theta}) = \log(r)+i\theta, \theta \in [0,2\pi]$. 
    Requires branch cut on $\R^+$.
    Since $\theta \in [0,2\pi]$ on this branch of log, "$u - i\e$" is really $\theta = +2\pi - O(\e)$.
    \begin{align}
        \log(u-i\e) &= \log(u) + \log\left(1 - \frac{i\e}{u}\right) \\
                    &= \log(u) + 2\pi i - \frac{i\e}{u} \\
        (u-i\e)^{-y} &= u^{-y}e^{-2\pi iy + i\e y/u}
    \end{align}
    To avoid branch cut, must construct "Pacman"-style contour - circle going round at radius, R, then breaking near the positive reals to go back to the origin then curve round. 
    \begin{align}
        C &= C_R' \cup I \cup C_\e' \cup \gamma 
    \end{align}
    \begin{itemize}
        \item $C_R'$:
            \begin{equation}
                \bigg|\frac{z^{-y}}{1+z}\bigg|R \to 0
            \end{equation}
            for $|z| = R, R \to \infty$.
        \item $C_\e'$:
            \begin{equation}
                \bigg|\frac{z^{-y}}{1+}\bigg|\e \to 0
            \end{equation}
            for $|z| = \e, \e \to 0$.
        \item $\gamma_{-}$:
            \begin{align}
                z^{-y} &= u^{-y} e^{-2\pi iy} \\
                \int_{\gamma} f\,dz &= \int_R^0 \frac{u^{-y}}{1+u}e^{-2\pi i y}\,du \\
                                    &= -e^{-2\pi i y}I \\
                \oint_C f(z)\,dz &= I - e^{-2\pi i y}I \\
                                 &= (1 - e^{-2\pi i y})I
            \end{align}
    \end{itemize}
    R has a simple pole at $z = -1$. 
    \begin{align}
        \text{Res}(f, -1) &= \lim_{z\to -1} (1+z)f(z) = (-1)^{-y} \\
                          &= e^{-i\pi y} \\
        I &= \frac{2\pi i\text{Res}(f,-1)}{1 - e^{-2\pi i y}} = \frac{2\pi ie^{-i\pi y}}{1 - e^{-2\pi i y}} \\
          &= \pi\times\frac{2i}{e^{i\pi y} - e^{-\pi y}} = \frac{\pi}{\sin(\pi y)}
    \end{align}
\end{example}

\newpage
\section{Closing contours}
\begin{example}
    \begin{align}
        I &= \ifnt \frac{x\sin(\pi x)}{x^2 + 1}\,dx \\
        f(z) &= \frac{z\sin(\pi z)}{1+z^2} \\
             &= \frac{z(e^{i\pi z} - e^{-i\pi z})}{(1+z^2)(2i)}
    \end{align}
    Looking to close contour in upper/lower half of plane.
    For semicircle in upper half plane, $R|e^{i\pi z}| \to 0$, but $R|e^{-i\pi z}| \to \infty$.
    Define:
    \begin{equation}
        f_{\pm} = \pm\frac{ze^{\pm i\pi z}}{2i(1+z^2}
    \end{equation}
    Let $C_+ = [-R,R] \cup \left\{ Re^{i\theta}\,|\,| \theta \in [0,\pi]\right\}$.
    \begin{equation}
        \oint_{C_+} f_+\,dz = 2\pi i\text{Res}(f,i)
    \end{equation}
    $f_+$ has simple pole at i. 
    \begin{align}
        \text{Res}(f_+,i) &= \lim_{z\to i} (z-i)f_+ \\
                          &= \lim_{z\to i} \frac{ze^{i\pi z}}{2i(z+i)} = \frac{ie^{-\pi}}{(2i)^2} \\
                          &= \frac{e^{-\pi}}{4i} \\
        \oint_{C_+} f_+\,dz &= \frac{\pi e^{-\pi}}{2} \\
        I &= 2\text{Res}(f_+,i) = \pi e^{-\pi}
    \end{align}
\end{example}

\chapter{Summation of Series}
Cotour integrals can be used to good effect in evaluating infinite series. 

\section{Riemann-Zeta Function}
\begin{align}
    \z(s) &= \sum_{n=1}^\infty \frac{1}{n^s}
\end{align}
Look at $s \in \Z^+, (s > 1)$.
Looks like we are summing over an infinite series of poles with residues, $\frac{1}{n^s}$.

Aim: Construct meromorphic function $f_s$ such that $f_s$ has poles at integer $z = n$, residues $\frac{1}{n^s}$, and $R|f| \to 0$ for $|z| = R \to \infty$.
This last condition with lead to $\int_{CR} f\;dz = 0$.
Note: $\sin(\pi z)$ has zeros at $z = n$.
Around $z = n$:
\begin{align}
    \sin(n\pi + \pi(z-n)) &= \cos(n\pi)\sin(\pi(z-n)) \\
                          &= (01)^n\pi(z-n) + O(z-n)^3 \\
    \pi\cot(\pi z) &= \frac{1}{z-n} + O(z-n) \\
    f_s(z) &= \frac{\pi\cot(\pi z)}{z^s}
\end{align}
Simple poles at $z = n \neq 0$, residue $\frac{1}{n^s}$.
At $z = 0$, pole of order $s+1$. 
If s is even, then 
\begin{equation}
    \frac{1}{(-n)^s}  = \frac{1}{n^s}.
\end{equation}
Poles on $\Z^+$ have same residue as poles on $Z^-$.

Consider 
\begin{align}
    z &= Re^{i\theta} = R\cos\theta + iR\sin\theta \\
    e^{2i\pi z} &= e^{2i\pi R\cos\theta}e^{-2\pi R\sin\theta} \\
    \theta \in (0,\pi), \sin\theta &> 0, e^{2i\pi z} \to 0, |z| = R\to \infty \\
    \theta \in (\pi,2\pi), \sin\theta &< 0, e^{2i\pi z}\to 0, |z|=R\to\infty \\
    |\cot(\pi z)| &= \bigg|\frac{e^{2i\pi z}+1}{e^{2i\pi z} - 1}\bigg| \to 1
\end{align}
To cover $z \in \R$, take $z = \pm R, R = N + \frac{1}{2} \implies \cot(\pi(N+\frac{1}{2})) = 0$.
So take the discrete series of contours, $C_R$, $R = N + \frac{1}{2}, N \in \Z^+ \to \infty$. 

On $C_R$: $R_N|f(z)| \to 0$ as $N \to \infty$ and
\begin{equation}
    \int_{C_R} f(z)\;dz = 0
\end{equation}
At finite N, $\oint_{C_R} f\;dz$ is small, poles run from $-N$ to N.
\begin{equation}
    \sum_{res} \text{Res}(f,z_i) = 0
\end{equation}
Poles at $z \in \Z$. 
\begin{align}
    2\z(s) + \text{Res}(f_s,0) = 0
\end{align}

\begin{itemize}
    \item $s=2$: At $z=0$, has triple pole 
        \begin{align} 
            f_2 &= \frac{\pi\cot(\pi z)}{z^2} \\
            \cos(\pi z) &= 1 - \frac{(\pi z)^2}{2} + \frac{(\pi z)^4}{4!} + \cdots \\
            \sin(\pi z) &= \pi z - \frac{(\pi z)^3}{3!} + \frac{(\pi z)^5}{5!} + \cdots \\
            \frac{\pi\cot(\pi z)}{z^2} &= \pi\frac{1 - \frac{(\pi z)^2}{2} + \frac{(\pi z)^4}{4!}}{\pi z^3\left(1 - \frac{(\pi z)^2}{3!}\right)} \\
                                       &= \frac{1}{z^3}\left(1 - \frac{\pi^2z^2}{3}\right) \\
            \z(2) &= -\frac{1}{2}\text{Res}(f_2,0) = \frac{\pi^2}{6}
        \end{align}
    \item $s=4$:
        \begin{align}
            \frac{\pi\cot(\pi z)}{z^4} &= \frac{1}{z^5}\frac{1 - \frac{(\pi z)^2}{2} + \frac{(\pi z)^4}{4!}}{1 - \e} 
        \end{align}
        Let 
        \begin{align}
            \e &= \frac{(\pi z)^2}{3!} - \frac{(\pi z)^4}{4!} \\
            \frac{1}{1-\e} &= 1 + \e + \e^2 + \cdots \\
            \frac{\pi\cot(\pi z)}{z^4} &= \frac{1}{z^5}\left(1 - \frac{(\pi z)^2}{2} + \frac{(\pi z)^4}{4!} \right)\left(1 + \frac{\pi z)^2}{3!} - \frac{(\pi z)^4}{5!} + \frac{(\pi z)^4}{36} \right) \\
            \text{Residue} &= -\frac{\pi^4}{45} \\
            \z(4) &= \frac{\pi^4}{90}
        \end{align}
\end{itemize}

\section{Other Series}
What about 
\begin{equation}
    S = \sum_1^\infty \frac{(-1)^n}{n^2}
\end{equation}
$\frac{\pi}{\sin(\pi z)}$ has alternating sign for residue. 
\begin{equation}
    f(z) = \frac{\pi}{z^2\sin(\pi z)}
\end{equation}
Two copies of S on positive and negative real axis. 
As $z \to 0$:
\begin{align}
    f(z) &= \frac{1}{z^3}\left(1 + \frac{(\pi z)^2}{3!}\right) \\
    S &= -\frac{1}{2}\text{Res}(f,0) = -\frac{\pi^2}{12}
\end{align}

Finally, suppose the terms in the series are not symmetric around $z = 0$.
\begin{align}
    S &= \sum_0^\infty \frac{1}{(n+1)(n+2)} \\
    f(z) &= \frac{\pi\cot(\pi z)}{(z+1)(z+2)}
\end{align}
This gives S from poles for $z = n \geq 0$. 
Consider $m = -n-\lambda \implies m+i = -(n+(\lambda-i))$, so if $\lambda = 3$, then $m+1 = -(n+2), m+2 = -(n+1)$. 

So we have $(m+1)(m+2) = (n+1)(n+2)$. 
This gives two copies of S, but not symmetric around the origin.
\begin{align}
    \sum_{n=-3}^{-\infty} \frac{1}{(n+1)(n+2)} &= \sum_{m=0}^\infty \frac{1}{(m+1)(m+2)}
\end{align}
Thus $2S + \text{Res}(f, -1) + \text{Res}(f,-2) = 0$. 
Poles at $z = -1,-2$ are double poles. 
\begin{align}
\pi\cot(\pi z) &= \frac{1}{z-n}(1 + O(z-n)^2)
\end{align}
Both poles give residue of $-1$, hence $S = 1$.

\subfile{vecs.tex}

\end{document}
