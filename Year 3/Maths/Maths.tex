\documentclass[a4paper, 11pt, normalem]{report}

\usepackage{../../../LaTeX-Templates/Notes}
\usepackage{subfiles}

\titlecontents{chapter}% <section-type>
    [0pt]% <left>
    {}% <above-code>
    {Lecture \thecontentslabel\quad}% <numbered-entry-format>
    {}% <numberless-entry-format>
    {\dotfill\contentspage}% <filler-page-format>
\titleformat{\chapter}{\fontsize{15}{17}\bfseries\normalfont}{\textbf{Lecture \thechapter}}{1em}{}
\titleformat{\subsubsection}{\fontsize{10}{13}\bfseries\scshape}{\textbf{\thesubsubsection}}{1em}{}
\setcounter{tocdepth}{4}
% \setcounter{secnumdepth}{1}

\renewcommand{\arraystretch}{1.2}

\newcommand\R{\mathbb{R}}
\newcommand\N{\mathbb{N}}
\newcommand\C{\mathbb{C}}
\newcommand\Z{\mathbb{Z}}
\newcommand\Rl{\mathsf{Re}}
\newcommand\Iy{\mathsf{Im}}
\newcommand\p{\partial}
\newcommand\ifnt{\int_{-\infty}^{\infty}}
\newcommand\ofnt{\int_{0}^{\infty}}
\newcommand\ifsum{\sum_{-\infty}^{\infty}}
\newcommand\F{\mathcal{F}}
\newcommand\La{\mathcal{L}}
\newcommand\K{\mathcal{K}}
\newcommand\om{\omega}
\newcommand\veca{\vec{a}(\vec{r})}
\newcommand\veru{\vec{r}(u)}
\newcommand\vr{\vec{r}}
\newcommand\vs{\vec{S}}
\newcommand\va{\vec{a}}
\newcommand\hn{\hat{n}}
\newcommand\hr{\hat{r}}
\newcommand\he{\hat{e}}
\newcommand\dy{\frac{dy}{dx}}
\newcommand\df{\frac{df}{dx}}
\newcommand\dyy{\frac{d^2 y}{dx^2}}
\newcommand\dff{\frac{d^2 f}{dx^2}}
\newcommand\dz{\frac{df}{dz}}
\newcommand\lam{\lambda}
\newcommand\e{\epsilon}
\newcommand\del{\nabla}
\newcommand\grad{\vec{\nabla}}

\title{Mathematics Workshop \vspace{-20pt}}
\author{Professor Ruth Gregory}
\date{\vspace{-15pt}Michaelmas Term 2018 - Epiphany Term 2019}
\rhead{\hyperlink{page.1}{Contents}}

\begin{document}

\maketitle
\tableofcontents

\part{Complex Analysis}
\chapter{The Complex Plane}
\section{Basics}
Course will use Riley, Hobson and Bence Chapters 3, 24, 25

Recall $i = \sqrt{-1}$ allows us to extend our notion of numbers as we go from a $\R$eal to the $\C$omplex plane.
\begin{equation}
    \C = \{z = x + iy\;|\;x,y \in \R\}
\end{equation}
We represent the complex plane with an argand diagram.
\begin{align}
    x &= \Rl(z) = \frac{z + \bar{z}}{2} \\
    y &= \Iy(z) = \frac{z - \bar{z}}{2} \\
    r &= \mathrm{mod}(z) = \sqrt{x^2 + y^2} \\
    \theta &= \mathrm{Arg}(z) = \arctan\left(\frac{y}{x}\right)
\end{align}
Notice that $e^{i\theta} = e^{i\theta + 2i\pi}$, so $\theta$ is not uniquely defined, so choose a range for $\theta$, e.g. $\theta \in [0, 2\pi), \theta \in (-\pi,\pi]$.

\section{Complex Functions}
\begin{itemize}
    \item $\C \to \C$
    \item $z \to f(z) = u(x,y) + iv(x,y)$
    \item Like two functions of real variables, e.g. $f(z) = z^2 = (x + iy)^2 = (x^2 - y^2) + 2ixy$
    \begin{itemize}
        \item $u(x,y) = x^2 - y^2$
        \item $v(x,y) = 2xy$
    \end{itemize}
\item Exponential: $e^z, \mathrm{exp}(z) = \sum_{n=0}^{\infty} \frac{z^n}{n!}$
\end{itemize}

\section{Branch Cuts}
For real numbers, $1^2 = (-1)^2 = 1$, you have two roots; but functions should be single valued. \\
$\sqrt{}$ is fine on the Real line - choose a root and stick to it, but for the Complex plane, say $\sqrt{1} = 1$, the disc around $z = 1$ means a loss of continuity. 
\begin{align}
    z &= re^{i\theta} \\
    \sqrt{z} &= \sqrt{r}e^{i\theta/2}, \theta \in (0, 2\pi)
\end{align}
We can choose this branch, but must cut the complex plane along the reals so that $\theta$ can't run higher than $2\pi$, or could choose
\begin{equation}
    \sqrt{z} = \sqrt{r}e^{i\theta/2 +i\pi} 
\end{equation}
but still need the same cut, so no ambiguity in definition of the function's square root. 
Here, the function's square root is double valued.
\begin{equation}
    \log(re^{i\theta}) = \log(r) + i\theta
\end{equation}
$\theta$ is not unique, $\theta + 2n\pi$ is also a legitimate answer. 
Again, cut the plane somewhere, decide on the branch of the log. 
Can cut on the positive reals or negatives. 
Branch choices appear around zeroes because polar coordinates are singular there: $\theta$ is not specified at $r = 0$.
We say $z=0$ is a branch point, indicated with a wavy line. 
Branch cuts from branch point either to infinity or another branch point. 

\begin{example}
\begin{equation}
    f(z) = \sqrt{z^4 + 1}
\end{equation}
Think of this through two steps: $z \to z^4 + 1 \to \sqrt{z^4 + 1}$. 
There is a branch cut around $z^4 + 1 = 0$. 
There are four branch points:
\begin{align}
    z^4 &= -1 \\
    z &= e^{i\pi/4}...
\end{align}
Around a branch point:
\begin{align}
    z &= e^{i\pi/4} + \epsilon e^{i\theta + 3i\pi/4} \\
    z^4 &= -1 + 4\epsilon e^{i\theta +3i\pi/4} \\
    \sqrt{z^4 + 1} &= 2\sqrt{\epsilon}e^{i\theta/2 + 3i\pi/8}
\end{align}
This gives rise to the same problem as circling the origin for $\sqrt{z}$.
This example has other choices for branch cuts, some may be able to limit ambiguities to limits. 
\end{example}

\section{Trig and Hyperbolic Functions}
Trig functions can be generalised to include complex numbers, and can be expressed as exponentials in the usual way, similarly with hyperbolics. 
$\sinh$ and $\cosh$ have periodicity of $2\pi i$.
\begin{align}
    \cos(z) &= \cosh(iz) \\
    \sin(z) &= i\sinh(iz)
\end{align}

\chapter{Complex Differentiation and Cauchy-Riemann}
\section{Continuity}
Definition:
\begin{equation}
    \lim_{z\to z_0} f(z) = w \in \C \iff \forall \e > 0 \;\exists \;\delta > 0 \; s.t.\; |z-z_0| < \delta \implies |f(z) - w| < \e
\end{equation}
\emph{If you are close to a point $z_0$, then f(z) is close to $f(z_0)$.}
f is continuous at $z_0 \iff \lim_{z\to z_0} f(z) = f(z_0)$.

Note that the limit must be path independent, and the real and imaginary parts must be continuous. 

\section{Differentation}
Recall
\begin{equation}
    \df = \lim_{h\to 0} \frac{f(x+h) - f(x)}{h}
\end{equation}
For $\C$ having two $\R$ dimensions, we have $\grad, \grad\cdot, \grad\times$. 
\begin{itemize}
    \item[X] Grad - acts on scalars ($\R$)
    \item[X] Div - acts on vectors ($\C$), gives scalar
    \item[X] Curl - in 3D
\end{itemize}
Try 
\begin{equation}
    \frac{df}{dz} = \lim_{\delta\to 0} \frac{f(z+\delta) - f(z)}{\delta}, \delta, z \in \C
\end{equation}
Require limit independent of direction.

\begin{example}
$f(z) = z^2$:
\begin{align}
    & \lim_{\delta\to 0} \frac{(z+\delta)^2 - z^2}{\delta} \\
    \implies & \lim_{\delta\to 0} \frac{\cancel{z^2} + 2z\cancel{\delta} + \delta^{\cancel{2}} - \cancel{z^2}}{\cancel{\delta}} = 2z
\end{align}
What about $f(z) = \bar{z}$?
\begin{equation}
    \frac{\bar{z} + \bar{\delta} - \bar{z}}{\delta} = \frac{\bar{\delta}}{\delta} = \exp{(-2i\mathrm{Arg}(\delta))}
\end{equation}
Not path independent, so no limit. 
\end{example}

$\frac{df}{dz}$ is the complex derivative of f at z, and f is differentiable at z if this limit exists. 

\section{Analytic Functions}
An analytic function is a complex function that is differentiable, at least in some region. 

Definition: A neighbourhood of $z \in \C$ is an open set U such that $z \in U$.

Definition: $f: \C \to \C$ is analytic/holomorphic at $z_0 \in \C$ if $\exists$ a neighbourhood U of $z_0$ on which f is differentiable $\forall \; z \in U$.

\begin{example}
    \begin{align}
        |z|^2 &= z\bar{z} \\
        \lim_{\delta\to 0} \frac{(z+\delta)(\bar{z} + \bar{\delta}) - z\bar{z}}{\delta} &= \bar{z} + \frac{\bar{\delta}}{\delta}z
    \end{align}
If $z \neq 0$, no limit, but $z = 0$ has limit, 0, $\implies$ differentiable at $z = 0$, \underline{but not analytic.}
\end{example}
$z^n$ is differentiable everywhere, so analytic on all $\C$.

\section{Cauchy-Riemann Equations}
We have
\begin{align}
    f(z) &= f(x + iy) \\
         &= u(x,y) + iv(x,y)
\end{align}
Derivation:
\begin{align}
    \frac{f(z_0 + \delta) - f(z_0)}{\underbrace{\delta}_{z - z_0}} &= \frac{u(x_0 + \delta x, y_0 + \delta y) - u(x_0,y_0)}{\delta x + i\delta y} + i\frac{v(x_0 + \delta x, y_0 + \delta y) - v(x_0,y_0)}{\delta x + i\delta y} \\
                                                                   &= \frac{u(x_0,y_0) + \delta x \frac{\p u}{\p x}|_0 + \delta y \frac{\p u}{\p y}|_0 - u_0}{\delta x + i\delta y} + i\frac{v_0 + \delta x \frac{\p v}{\p x}|_0 + \delta y\frac{\p v}{\p y} - v_0}{\delta x + i\delta y} \\
                                                                   &= \frac{\left(\frac{\p u}{\p x} + i\frac{\p v}{\p x}\right)\delta x + i\left(\frac{\p v}{\p y} - i\frac{\p u}{\p y}\right)\delta y}{\delta x + i\delta y}
\end{align}
For complex differentiation, limit must be independed of $\delta x + i\delta y$, so the numerator must factorise as $() \times (\delta x + i\delta y)$.
\begin{equation}
    \implies \frac{\p u}{\p x} + i\frac{\p v}{\p x} = \frac{\p v}{\p y} - i\frac{\p u}{\p y}
\end{equation}
This gives the C-R relations:
\begin{align}
    \frac{\p u}{\p x} &= \frac{\p v}{\p y} \\
    \frac{\p v}{\p x} &= -\frac{\p u}{\p y} 
\end{align}
Check with $z^2 = (x^2 - y^2) + 2ixy$:
\begin{align}
    \frac{\p u}{\p x} &= 2x & \frac{\p v}{\p y} &= 2x \\
    \frac{\p u}{\p y} &= -2y & \frac{\p v}{\p x} &= 2y 
\end{align}
Note:
\begin{equation}
    \frac{\p u}{\p x} = \frac{\p v}{\p y} \implies \frac{\p^2 u}{\p x^2} = \frac{\p^2 v}{\p x \p y} = -\frac{\p^2 u}{\p y^2}
\end{equation}
Or 
\begin{equation}
    \frac{\p^2 u}{\p x^2} + \frac{\p^2 u}{\p y^2} = \del^2 u = 0
\end{equation}
Suppose we regard z and $\bar{z}$ as independed variables. 
\begin{equation}
    x = \frac{z + \bar{z}}{2},~ y = \frac{z - \bar{z}}{2i} 
\end{equation}
\begin{align}
    \frac{\p}{\p \bar{z}} &= \frac{\p x}{\p \bar{z}}\frac{\p}{\p \p x} + \frac{\p y}{\p \bar{z}}\frac{\p}{\p y} = \frac{1}{2}\left(\frac{\p}{\p x} + i\frac{\p}{\p y}\right) \\
    \frac{\p}{\p z} &= \frac{\p x}{\p z}\frac{\p}{\p x} + \frac{\p y}{\p z}\frac{\p}{\p y} = \frac{1}{2}\left(\frac{\p}{\p x} - i\frac{\p}{\p y}\right) \\
    \frac{\p f}{\p \bar{z}} &= \frac{1}{2}\left(\frac{\p u}{\p x} + i\frac{\p v}{\p x} + i\frac{\p u}{\p y} - \frac{\p v}{\p y}\right) \\
                    &= \frac{1}{2}\left(\left(\frac{\p u}{\p x} - \frac{\p v}{\p y}\right) + i\left(\frac{\p v}{\p x} + \frac{\p u}{\p y}\right) \right) \\
                    &= 0 \iff \text{CR satisfied}
\end{align}
Can express analytically/differentiable as $\frac{df}{d\bar{z}} = 0$.










\end{document}
