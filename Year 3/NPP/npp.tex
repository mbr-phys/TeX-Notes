\documentclass[a4paper, 11pt, normalem]{report}

\usepackage{../../../LaTeX-Templates/Notes}
\usepackage{subfiles}

\title{Nuclear and Particle Physics \vspace{-20pt}}
\author{Dr Maitre}
\date{\vspace{-15pt}Michaelmas Term 2018 - Epiphany Term 2019}
\rhead{\hyperlink{page.1}{Contents}}

\begin{document}

\maketitle
\tableofcontents

\part{}
\chapter{}

Use these because these are made by God himself:
\url{https://dmaitre.phyip3.dur.ac.uk/notes/npp/}

\emph{Use these notes only with link above, these will just be additional annotations}

\section{Units}

\begin{example}[Do Broglie Wavelength]
    \begin{align}
        \lambda &= \frac{\hbar}{p} \\
        p &= 20 \text{GeV/c} \\
        l &= 0.05 \hbar c\text{GeV}^{-1} \\
        \hbar c &= 0.19733 \,\text{fm}\;\text{GeV} \equiv 1 \\
        \lambda &= 0.0987 \,\text{fm}
    \end{align}
\end{example}

\section{Kinematics}

High energies mean speed close to c, so use special relativity, and get the Lorentz transforms and Tensors.

\begin{example}[4-Momenta]
    In the rest frame of a particle of mass m, 
    \begin{align}
        p &= (m, \vec{0})
    \end{align}
    How is it in a different frame?
    \begin{align}
        p &= (E,\vp) \\
        p\cdot p &\equiv p^2 = \begin{cases} \text{Rest Frame} & m^2 - \vec{0}^2 - m^2 \\ \text{Other Frames} & E^2 \vec{p}\cdot\vec{p} = E^2 - |\vec{p}|^2 \end{cases} \\
        m^2 &= E^2 - |\vec{p}|^2 \\
        E^2 &= m^c + |\vec{p}|^2 \\
        E &= \sqrt{m^2 + |\vec{p}|^2}
    \end{align}
    Note: there will be factors of c in this, but set to 1, in natural units.
\end{example}


\end{document}
