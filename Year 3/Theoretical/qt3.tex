\documentclass[a4paper, 11pt, normalem]{report}

\usepackage{../../../LaTeX-Templates/Notes}
\usepackage{subfiles}

\title{Quantum Theory 3 \vspace{-20pt}}
\author{Prof Khoze}
\date{\vspace{-15pt}Epiphany Term 2019}
\rhead{\hyperlink{page.1}{Contents}}

\begin{document}

\maketitle
\tableofcontents

\part{Scattering Theory}
\chapter{Introduction to Scattering}
\section{Two types of scattering}
\begin{itemize}
    \item elastic - initial particles remain and no new particles emerge in the collision
    \item inelastic - in the final state, there is more than just the initial particles
\end{itemize}
Will be using \unl{non-relativistic} Quantum Mechanics for this part of the course, therefore will only be studying elastic non-relativistic scattering, e.g. Rutherford experiment, $\alpha + \text{Au} \to \alpha + \text{Au}$.
\begin{itemize}
    \item Consider elastic $e^-e^+$ scattering
        \begin{equation}
            e^+ + e^- \to e^+ + e^-
        \end{equation}
        \emph{Feynman diagrams of collision - both s-channel (particles meet) and t-channel (particles interact through virtual photon).}
    \item Consider inelastic scattering
        \begin{equation}
            e^+ + e^- \to \mu^+ + \mu^-
        \end{equation}
        \emph{Feynman diagram of collision and decay into muon and anti-muon - electron and positron collide and annihilate, their energy then carried by photon which decays into muon and anti-muon.}
\end{itemize}

\section{Scattering cross-sections}
The scattering cross-section, $\sigma$, is a probabilistic quantity that characterises the 'strength' of the scattering (interaction between the particles).
$\sigma$ has dimension of area ($m^2$).
\begin{align}
    \frac{d\sigma}{d\Omega} &= \frac{1}{F} \frac{dR}{d\Omega},
\end{align}
where:
\begin{itemize}
    \item $F$ is the flux - number of incident particles per unit area per unit time ($s^{-1}m^{-2}$)
    \item $dR$ is the rate - number of scattered particles ($N$) into $d\Omega$ per unit time ($s^{-1}$)
    \item $d\Omega$ is the solid angle
    \item $\frac{d\sigma}{d\Omega}$ is the differential cross-section into the solid angle ($m^2$)
\end{itemize}
\begin{align}
    \frac{d\sigma}{d\Omega}(\theta,\phi) &= \frac{1}{F}\frac{dR}{d\Omega} \\
    \sigma_{tot} &= \int_0^{2\pi} d\phi \int_0^\pi d\theta \sin\theta \frac{d\sigma}{d\Omega}(\theta,\phi) \\
    N &= \sigma_{tot} \cdot \int F\,dt
\end{align}
\begin{itemize}
    \item $F$ is known for each experiment, part of the design
    \item $\sigma_{tot}$ is measured in experiment
        \begin{itemize}
            \item Measured in barns (b) - $1 \text{ barn} = 10^{-24} cm^{-2}$
            \item $\sigma_{\text{Thompson}} = 0.665\,b$
        \end{itemize}
    \item LHC gluon fusion into Higgs:
        \begin{itemize}
            \item $g+g \to H$
            \item \emph{Feynman diagram of gluon collision into Higgs boson}
            \item $\sigma_{\text{Higgs}} \approx 10 pb$
        \end{itemize}
\end{itemize}

\chapter{General Features of Potential Scattering in QM}
\section{The Schrodinger Equation}
Time-dependent Schrodinger equation, and reduced mass:
\begin{align}
    i\hbar\frac{\p\psi(r,t)}{\p t} &= \left(-\frac{\hbar^2}{2m}\del^2 + V(r)\right)\psi(r,t) \\
    m &= \frac{m_Am_B}{m_A+m_B}
\end{align}
$E=$ fixed and finite
\begin{align}
    E&=\frac{p^2}{2m} \\
    \psi(r,t) &= e^{-iEt/\hbar}\psi(r)
\end{align}
Leads to time-independent Schrodinger equation
\begin{align}
    E\psi(r) &= \left(-\frac{\hbar^2}{2m}\del^2 + V(r)\right)\psi(r) 
\end{align}
$\frac{p}{\hbar} = k$, $U(r) = \frac{2m}{\hbar^2}V(r)\to$ Scattering equation:
\begin{equation}
    \left(\del^2 + k^2 - U(r)\right)\psi(r) = 0
\end{equation}
For scattering:
\begin{itemize}
    \item Looking for $\psi(r)$ s.t. as $r \to \pm \infty$, 
        \begin{align}
            \psi_{inc}(r) + \psi_{scat}(r) &\equiv e^{ik\cdot r} + \frac{e^{i|k|\cdot|r|}}{r}\cdot f(k,\theta,\phi) \\
                                           &= e^{ikr\cos\theta} + \frac{e^{ikr}}{r}\cdot f(k,\theta,\phi)
        \end{align}
    \item When scattering occurs, the incoming plane waves turn into spherical waves with $\frac{1}{r}$ amplitude from point of scattering
    \item $f(k,\theta,\phi)$ is the scattering amplitude - need to determine this in order to compute $\sigma$
\end{itemize}
\begin{equation}
    \psi(r) \approx_{r \to \infty} 1\cdot e^{ik\cdot r} + f(k,\theta,\phi)\frac{e^{ik\cdot r}}{r}
\end{equation}
Now consider the probability density for normalisation,
\begin{align}
    \rho_{inc}(r) &\equiv |\psi_{inc}|^2 = 1
\end{align}
What about flux?
\begin{align}
    F &= \frac{\text{\# of incoming particles}}{\text{Area}\cdot\text{time}} \\ 
      &= v\cdot\rho = \frac{p}{m}\cdot\rho,~ \rho = 1 \\
      &= \frac{p}{m}
\end{align}
Recall $\frac{d\sigma}{d\Omega} = \frac{1}{F}\frac{dR}{d\Omega}$. So what is $dR$?
\begin{align}
    dR &= j_r r^2 d\Omega
\end{align}
So $j_r$ is the probability current density - the number of scattered particles crossing the unit area per unit time.
\begin{align}
    j_r &\equiv \frac{\hbar}{2mi}\left(\psi^*_{scat}(r)\del\psi_{scat}(r) - \left(\del\psi_{scat}(r)\right)^*\psi(r)\right) \\
        &= \frac{\hbar}{m}\text{Im}\left(\psi^*_{scat}\del\psi_{scat}\right) = \frac{\hbar k}{m}\frac{|f|^2}{r^2} \\
        &= \frac{p}{m} |f|^2\frac{1}{r^2} = F\frac{|f(k,\theta,\phi)|^2}{r^2} \\
    \implies \frac{d\sigma}{d\Omega} &= \frac{1}{F}\frac{dR}{d\Omega} = |f(k,\theta,\phi)|^2
\end{align}



















\end{document}
