\documentclass[a4paper,11pt,normalem]{article}
\usepackage{../../../LaTeX-Templates/Notes}

\rhead{}
\newcommand{\HRule}{\rule{\linewidth}{0.5mm}}

\begin{document}
{\centering
{\includegraphics[scale=0.5]{../../logo0.png}\hfill{\Large\bfseries Michaelmas 2017}}\\[1.5cm]
{\LARGE\bfseries Foundations of Physics 2A}\\[0.5cm]
\HRule \\[0.3cm]
{\huge\bfseries Quantum Mechanics}\\[0.1cm]
\HRule \\[1cm]}
\begin{center}
\begin{minipage}{0.4\textwidth}
    \begin{flushleft} \large
        \emph{Author:} \\ Matthew Rossetter
    \end{flushleft}
\end{minipage}~
\begin{minipage}{0.4\textwidth}
    \begin{flushright} \large
        \emph{Lecturer:} \\ Prof. Shaun Cole
    \end{flushright}
\end{minipage}
\end{center}
\section{Basics of QM}\label{basics-of-qm}

\begin{itemize}
\item
  Useless so just read notes
\item
  Particles described by wave functions, psi
\item
  \(P(x) dx = \psi^2 dx = \psi^* \psi dx\)
  \begin{itemize}
  \item
    \(\int_{-\infty}^{\infty} = 1\)
  \end{itemize}
\end{itemize}

\section{Operators and Expectation Values}\label{operators-and-expectation-values}

\begin{itemize}
\item
  Expectation value is average value
  \begin{itemize}
  \item
    \(\langle x \rangle = \int_{-\infty}^{\infty} xP(x) dx\)
  \item
    Doesn't always mean most expected value, just average
  \item
    page 7
  \end{itemize}
\item
  \(\hat{p}\) now means \(-i\hbar \frac{\partial}{\partial x}\)
  \begin{itemize}
  \item
    page 8
  \end{itemize}
\item
  Hamiltonian
  \begin{itemize}
  \item
    sum of potential and kinetic energy
  \item
    use \(-\frac{\hat{p}^2}{2m} \frac{\partial^2}{\partial x^2} + V\)
    for the operator version
  \item
    \(\hat{H} \psi = E \psi\)
    \begin{itemize}
    \item
      alternate form of Schrodinger
    \end{itemize}
  \item
    page 8
  \end{itemize}
\item
  Klein-Gordon for relativistic electrons
  \begin{itemize}
  \item
    probably won't need it
  \item
    page 9
  \end{itemize}
\end{itemize}

\section{The Origin Of Uncertainty}\label{the-origin-of-uncertainty}

\begin{itemize}
\item
  Page 10
\item
  Complex conjugate operators of position, momentum, etc operate on the
  function immediately to the right
\item
  Hermitian -- the complex conjugate of the integral for the expectation
  value is equivalent
  \begin{itemize}
  \item
    Swapping positions of variables and taking cplx conjugate
  \end{itemize}
\item
  Any dynamical quantity can be an operator
\item
  Expectation value only real if it is Hermitian
\item
  Can use Hermitian to show uncertainty principle
  \begin{itemize}
  \item
    try finding \(\langle xp\rangle\)
  \item
    will not get the complex conjugate
  \end{itemize}
\item
  \(\hat{x}\hat{p} \neq \hat{p}\hat{x}\)
  \begin{itemize}
  \item
    they do not commute
  \end{itemize}
\item
  Page 11
\item
  Square bracket notation for commutator
  \begin{itemize}
  \item
    \([\hat{a},\hat{b}] = \hat{a}\hat{b} - \hat{b}\hat{a}\)
  \end{itemize}
\item
  The x and p operators don't commute so aren't real and can't be
  measured
  \begin{itemize}
  \item
    can be a test for other operators too to find relations for
    uncertainty
  \item
    the difference between the two double commutators for x and p is
    $i\hbar$ which is where the $\hbar/2$ comes from for uncertainty
  \end{itemize}
\item
  Use a Gaussian function and find EVs for
  \(\langle x \rangle , \langle p \rangle , \langle x^2 \rangle , \langle p^2 \rangle\)
  to get minimum Heisenberg values
\item
  Ehrenfest's Theorem fulfils Newton's second law with EVs
\end{itemize}

\section{Schrodinger Equation and Eigenfunctions}\label{schrodinger-equation-and-eigenfunctions}

\begin{itemize}
\item
  page 14
\item
  \(ln T = \frac{E}{i\hbar}t + K\)
\item
  \(T = e^{-it\frac{E}{\hbar}}\)
\item
  Can covert from time-dep to time-indep if \(V = V(x)\), separable fns
\item
  Kronecker delta fn from Maths
  \begin{itemize}
  \item
    for distinct eigenfns and orthogonality
  \end{itemize}
\item
  Bra-ket notation for eigen stuff
  \begin{itemize}
  \item
    \(\langle\) \textbar{} for cplx conjugate
  \item
    \textbar{} \(\rangle\) for normal
  \item
    \(\langle\) \textbar{} \(\rangle\) = \(\int\)
  \end{itemize}
\item
  Eigenfns of the Hamiltonian are stationary
\end{itemize}

\section{Eigenfunctions versus Superpositions}\label{eigenfunctions-versus-superpositions}

\begin{itemize}
\item
  page 17
\item
  \(\Psi(x,t) = \Psi_{E}(x,t)e^{-\frac{iE}{\hbar}t}\)
\item
  \(\Psi(x,t) = C_{1}\Psi_{E_{1}}(x,t)e^{-\frac{iE_{1}}{\hbar}t} + C_{2}\Psi_{E_{2}}(x,t)e^{-\frac{iE_{2}}{\hbar}t}\)
\item
  Certain harmonics for standing waves
  \begin{itemize}
  \item
    only certain wavelengths and energies for these systems
  \end{itemize}
\item
  Infinite square well example
  \begin{itemize}
  \item
    see last year's notes
  \item
    normal SHO
    \begin{itemize}
    \item
      sin and cos solns
    \item
      \(\cos = 0\)
    \item
      \(\Psi(x) = A\sin(kx)\)
    \item
      \(k = \frac{\sqrt{2mE}}{\hbar}\)
    \item
      \(A = \sqrt{\frac{2}{L}}\)
    \item
      all usual maths from last year, find \(\Psi\) and integrate for
      constants
    \end{itemize}
  \item
    can make this into a full energy eigenfn: \[
    \Psi_{n}(x,t) = \sqrt{\frac{2}{L}}\sin\Big(\frac{n\pi x}{L}\Big)e^{-\frac{iE_{n}t}{\hbar}}
    \]
  \end{itemize}
\item
  Mixture of states leads to time dep
\item
  For any single eigenfn, probability is not time dep
  \begin{itemize}
  \item
    we say these are stationary states
  \item
    expectation values are not time dep
  \end{itemize}
\item
  General solution to time dep Schrodinger eqn is superposition of
  eigenstates: \[
  \Psi(x,t) = \sum_{n} c_{n}\Psi_{n}(x,t) = \sum_{n}c_{n}\sqrt{\frac{2}{L}}\sin\Big(\frac{n\pi x}{L}\Big)e^{-\frac{iE_{n}t}{\hbar}}
  \]
\item
  \(c_{n}\) must be chosen so as to normalise total wavefn
  \begin{itemize}
  \item
    must equate to one as usual
  \end{itemize}
\end{itemize}

\subsection{Orthogonality}\label{orthogonality}

\begin{itemize}
\item
  Form an overlap integral of two states n and m:
  \begin{itemize}
  \item
    \(\langle\Psi_{n}|\Psi_{m}\rangle\)
  \item
    use \href{www.wolframalpha.com}{!wa} as usual for indefinite
    integral
  \end{itemize}
\item
  when \(n \neq m\) both sin terms are for integer \(\pi\) and so both
  are zero
\item
  when \(n = m\), the denominator on the first term goes to zero and the
  rest tends to 1 as the inside tends to zero
  \begin{itemize}
  \item
    this shows it is orthonormal
  \item
    like dot product
  \end{itemize}
\item
  Span the space known from Fourier analysis for expressed as sum of sin
  waves
\end{itemize}

\subsection{Wavefns in Superposition of Eigenstates}\label{wavefns-in-superposition-of-eigenstates}

\begin{itemize}
\item
  \textbf{this is much more thorough in the notes}
\item
  General solution for superpositions: \[
  \Psi(x) = \sum_{n} c_{n}\Psi_{n}(x) \\
  |\Psi\rangle  = \sum_{n} c_{n}|\Psi_{n}\rangle
  \]
\item
  Normalisation condition for two states:
  \begin{itemize}
  \item
    \(|c_{1}|^{2} + |c_{2}|^{2} = 1\)
  \item
    for general,
    \begin{itemize}
    \item
      use Kronecker delta fn to simplify integrals to get down to:
    \item
      \(\sum_{m}|c_{m}|^{2}\)
    \item
      \textbf{always adds up to 1}
    \end{itemize}
  \end{itemize}
\item
  For average energy,
  \begin{itemize}
  \item
    using Kronecker delta again \[
    \langle E \rangle = \sum_{n}|c_{n}|^{2}E_{n}
    \]
  \end{itemize}
\item
  Only works if \(|c_{n}|^{2}\) is probability of being in state
  \(\Psi_{n}\) and measuring energy \(E_{n}\)
\item
  QM predicts only the probability of measuring particular energies
  \begin{itemize}
  \item
    non-deterministic
  \end{itemize}
\end{itemize}

\subsection{Expectation for Energy and Time dep}\label{expectation-for-energy-and-time-dep}

\begin{itemize}
\item
  Expectation value of energy is probability weighted sum of energies
  associated with each state
\item
  A superposed states probability of being in any particular state is \[
  \Psi(x,t) = \sum_{n} c_{n}\Psi_{n}e^{-\frac{iE_{n}t}{\hbar}}
  \]
\item
  it oscillates with time
\item
  Looking at the probability for a two state system:
  \begin{itemize}
  \item
    last two terms contain \(e^{\pm\frac{i(E_{2} - E_{1})t}{\hbar}}\)
  \item
    write this as \(e^{\pm i\omega t}\)
  \item
    \(\hbar\omega = E_{2} - E_{1}\) \[
    P(x,t) = |c_{1}|^{2}|\Psi_{1}|^2 + |c_{2}|^{2}|\Psi_{2}|^{2} + c_{1}^{*}c_{2}\Psi_{1}^{*}\Psi_{2}e^{-i\omega t} + c_{1}c_{2}^{*}\Psi_{1}\Psi_{2}^{*}e^{i\omega t}
    \]
  \end{itemize}
\item
  This is real as last two terms are cplx conjugate of each other and
  so, when added together, their imaginary parts cancel.
\item
  Notes that this probability varies with time
\item
  \textbf{Superposed states are not stationary states}
\item
  It is still normalised, \(\int P(x,t) = 1\), as orthogonality of
  eigenfns makes last two terms zero when integrated over x
\end{itemize}

\section{Superposition of eigenstates and transitions}\label{superposition-of-eigenstates-and-transitions}

\begin{itemize}
\item
  page 21
\item
  All expectation values are constant in time for single eigenfn
  \begin{itemize}
  \item
    A general soln which contains multiple is not
  \end{itemize}
\item
  For two terms in the infinite square well, the probability density
  oscillates with
  \begin{itemize}
  \item
    \(\omega = \frac{E_2 - E_1}{\hbar}\)
  \end{itemize}
\item
  For an electron in ground state of an atom,
  \begin{itemize}
  \item
    it can be said it is a single eigenfn so it is not dep on time
  \item
    standing wave
  \end{itemize}
\item
  QM probability distributions solve radiation problem
  \begin{itemize}
  \item
    probability density is stationary so the charge is not `moving' and
    it doesn't radiate
  \item
    otherwise everything crashes into nucleus
  \end{itemize}
\item
  The mixture of two or more different states, say \(n = 1, 2\) then it
  has a chance of decaying down
  \begin{itemize}
  \item
    can't happen at ground state as no lower state to mix with
  \end{itemize}
\end{itemize}

\subsection{How to find the cm}\label{how-to-find-the-cm}

\begin{itemize}
\item
  Going from particular soln to general assume that energy eigenfns span
  the entire space
  \begin{itemize}
  \item
    complete set of basis fns
  \item
    any arbitrary fn can be expanded as sum of these
  \end{itemize}
\item
  For any functional form \(f(x)\), can decompose it into a weighted sum
  of energy eigenfns \(c_n \Psi_n\) by calculating each
  \(c_n = \int \Psi_{n}^{*}f(x)dx*\)
  \begin{itemize}
  \item
    find the overlap of f with the wavefn
  \item
    \(\sum_{n} c_n \delta_{nm} = c_m\)
  \end{itemize}
\end{itemize}

\section{Eigenfunctions of various potentials}\label{eigenfunctions-of-various-potentials}

\begin{itemize}
\item
  page 22
\item
  Consider infinite square well, symmetric about the origin
\end{itemize}

\subsection{Infinite Square Well}\label{infinite-square-well}
\begin{itemize}
\item
  For
  \(0 < x < L,~ \Psi_{n}(x,t) = \Psi_{n}e^{-i\frac{E_{n}t}{\hbar}}\): \[
  \Psi_{n}(x) = \sqrt{\frac{2}{L}}\sin(\frac{n\pi x}{L}) ~;~ E_n = \frac{n^{2}\pi^{2}\hbar^2}{2mL^2}
  \]
\item
  Can make this symmetric about origin by using a change of variable:
  \begin{itemize}
  \item
    \(x' = x - \frac{L}{2}\)
  \item
    Energy levels are unchanged but solns for odd n will become cos
    \begin{itemize}
    \item
      seen by drawing a diagram
    \end{itemize}
  \item
    This gives \[
    \Psi_{n}(x)' = \sqrt{\frac{2}{L}}\sin(\frac{n\pi x'}{L} + \frac{n\pi}{2}) = \sqrt{\frac{2}{L}}\sin(\frac{n\pi x'}{L})\cos(\frac{n\pi}{2}) + \cos(\frac{n\pi x'}{L}\sin(\frac{n\pi}{2}))
    \]
  \end{itemize}
\item
  Depends on n
  \begin{itemize}
  \item
    odd n wavefns are even fns
    \begin{itemize}
    \item
      \(\Psi(x') = \Psi(-x')\)
    \end{itemize}
  \item
    even n wavefns are odd fns
    \begin{itemize}
    \item
      \(\Psi(x') = \Psi(-x')\)
    \end{itemize}
  \item
    Note that in either case the probability distribution
    \(P(x) = |\Psi(x)|^2\) is symmetric about \(x' = 0\)
  \end{itemize}
\end{itemize}

\subsection{Finite Square Well}\label{finite-square-well}

\begin{itemize}
\item
  Potential is now V instead of infinity
\item
  Expect bound wavefns to be similar to infinite square well but broader
  and leak out slightly
\item
  Solve Schrodinger for three regions:
  \begin{itemize}
  \item
    \(x < -\frac{L}{2}\)
  \item
    \(-\frac{L}{2} < x < \frac{L}{2}\)
  \item
    \(x > \frac{L}{2}\)
  \end{itemize}
\item
  For central region, \(V = 0\): \[
  -\frac{\hbar^2}{2m} \frac{d^2 \Psi}{dx^2} = E\Psi \\
  \Psi = A\cos{ks} + B\sin(kx) ~;~ k^2 = \frac{2mE}{\hbar^2}
  \]
\item
  For \(x > \frac{L}{2}\): \[
  -\frac{\hbar^2}{2m} \frac{d^2 \Psi}{dx^2} + V_0 \Psi = E\Psi \\
  \frac{d^2 \Psi}{dx^2} = \frac{2m}{\hbar^2}(V_0 - E)\Psi = \alpha^2 \Psi ~;~ \alpha^2 = \frac{2m}{\hbar^2}(V_0 - E)
  \]
\item
  Here we have used that for bound states \(V_0 > E\)
  \begin{itemize}
  \item
    \(-\alpha^2 < 0\)
  \item
    General soln is \(\Psi(x) = Ce^{\alpha x} + De^{-\alpha x}\)
  \item
    For soln to be normalisable, \(C = 0\), or exponent just increases
    \[
    \Psi(x) = De^{-\alpha x} ~;~ x > \frac{L}{2}
    \]
  \end{itemize}
\item
  Similarly for \(x < -\frac{L}{2}\): \[
  \Psi(x) = Fe^{\alpha x}
  \]
\item
  Not an infinite discontinuity in V, then
  \(\frac{\partial \Psi}{\partial x}\) as well as \(\Psi\) must be
  continuous at \(x = \pm \frac{L}{2}\)
  \begin{itemize}
  \item
    In principle, this is all wee need for solns
  \item
    solns will either be odd or even - \(\Psi(x) = \pm \Psi(-x)\)
  \end{itemize}
\end{itemize}

\subsubsection{Even Solns}
\begin{itemize}
\item
  page 24
\item
  Only need to consider the join at \(x = \frac{L}{2}\) as it's the same
  at \(x = -\frac{L}{2}\) by symmetry
\item
  Applying continuity:
  \begin{itemize}
  \item
    In \(\Psi\): \(A\cos{\frac{kL}{2}} = De^{-\frac{\alpha L}{2}}\)
  \item
    In \(\frac{\partial \Psi}{\partial x}\):
    \(-kA\sin{\frac{kL}{2}} = -\alpha De^{-\frac{\alpha L}{2}}\)
  \end{itemize}
\item
  We find that: \[
  (\frac{kL}{2}\tan{\frac{kL}{2}} = \frac{\alpha L}{2})
  \]
\item
  Meanwhile for defns of k and \(\alpha\), in terms of energy we also
  have: \[
  \Big(\frac{\alpha L}{2}\Big)^2 + \Big(\frac{kL}{2} \Big)^2 = \frac{2mV_0}{\hbar^2}\Big(\frac{L}{2} \Big)^2
  \]
\item
  Solving for simultaneous eqns finds allowed values for k, energy
  levels and wavefns
  \begin{itemize}
  \item
    Isn't trivial, but can find informative graphical soln by setting \[
    X = \frac{kL}{2} ~;~ Y = \frac{\alpha L}{2}
    \]
  \end{itemize}
\item
  Two eqns become: \[
  Y = X\tan X ~;~ X^2 + Y^2 = \frac{2mV_0}{\hbar^2}\Big(\frac{L}{2} \Big)^2
  \]
\item
  Plot Y(X) for both eqns and where they intersect are allowed solns
\end{itemize}

\subsubsection{Odd Solns}
\begin{itemize}
\item
  Follow same method for odd solns, so sin instead of cos \[
  Y = -X\cot X ~;~ X^2 + Y^2 = \frac{2mV_0}{\hbar^2}\Big(\frac{L}{2} \Big)^2
  \]
\item
  Look at plot on page 24:
  \begin{enumerate}
  \item
    There is always at least one bound state
    \begin{itemize}
    \item
      if \(R < \frac{\pi}{2} \to V_0 < \frac{\hbar^{2}\pi^2}{2mL^2}\)
      then this is the only state
    \end{itemize}
  \item
    For intermediate circle, there are 4 allowed states corresponding to four points of intersection
    \begin{itemize}
    \item
      Reading X and Y values determines k and \(\alpha\)
    \item
      Plugging back into eqns above gives D and F (for symmetry) in
      terms of A
    \item
      To get value for A, use normalisation
      \begin{itemize}
      \item
        split into bounds \(-\infty \leq x \leq -\frac{L}{2}\),
        \(-\frac{L}{2} \leq x \leq \frac{L}{2}\),
        \(\frac{L}{2} \leq x \leq \infty\)
      \end{itemize}
    \end{itemize}
  \item
    In lim as \(V_0 \to \infty\), radius of circle becomes very large
    and intersections occur at
    \begin{itemize}
    \item
      \(X(= \frac{kL}{2}) = \frac{n\pi}{2}\)
    \item
      energy values are same as in infinite square well
      \begin{itemize}
      \item
        \(E_n = \frac{n^{2}\hbar^{2}k^2}{2m} = \frac{\pi^{2}\hbar^{2}}{2mL^2}\)
      \end{itemize}
    \end{itemize}
  \end{enumerate}
\end{itemize}

\subsection{Commutator Algebra}\label{commutator-algebra}

\begin{itemize}
\item
  \([A, B] = AB - BA = -[B, A]\)
  \begin{itemize}
  \item
    not necessarily zero
  \item
    operators do not have to commute
  \item
    E.g. \([x, p] = i\hbar\)
  \end{itemize}
\item
  \textbf{See page 25 for commutator identity proofs} \[
  \begin{aligned}
  [A,A] = AA - AA = 0 \\
  [A + B, C] = [A, C] + [B, C] \\
  [AB, C] = A[B, C] + [A,C]B \\
  [A,BC] = [A,B]C + B[A,C]
  \end{aligned}
  \]
\end{itemize}

\subsection{Consequences of non-commutation}\label{consequences-of-non-commutation}

\begin{itemize}
\item
  All operators which commute share a common set of eigenfns
  \begin{itemize}
  \item
    page 26 for proof
  \end{itemize}
\item
  Tells a bit more about uncertainty principle
  \begin{itemize}
  \item
    measuring an operator does not disturb wavefn for measurement of
    next
  \item
    if they don't commute, then in measuring A, we change the wavefn to
    be one of the eigenfns of A
    \begin{itemize}
    \item
      if not eigenfns of B, can still expand f in terms of eigenfns of B
      \[
      f_n = \sum_{m} c_{m}g_m ~;~ c_{m} = \int g_{m}^{*}f_{n}\,dx*
      \]
    \end{itemize}
  \end{itemize}
\item
  Now if we measure B, don't get a deterministic value of B
  \begin{itemize}
  \item
    measure value \(b_m\) with probability \(|c_{m}|^2\)
  \end{itemize}
\item
  Non-commutation means the two associated variables cannot
  simultaneously have deterministic values
  \begin{itemize}
  \item
    this implies the uncertainty principle
  \end{itemize}
\end{itemize}

\subsection{The linear harmonic oscillator}\label{the-linear-harmonic-oscillator}

\begin{itemize}
\item
  Potential, \(V(x) = \frac{1}{2}kx^2\)
  \begin{itemize}
  \item
    oscillates with \(\omega = \sqrt{\frac{k}{m}}\)
  \item
    \(V(x) = \frac{1}{2}m\omega^{2}x^2\)
  \end{itemize}
\item
  There are wide applications as about equilibrium position, \(x_0\), any arbitrary continuous potential V(x) is harmonic to leading order \[
  V(x) \approx V(x_{0}) + \Bigg[\frac{dV}{dx}\Big|_{x_{0}} (x - x_{0}) \Bigg]^{= 0} + \frac{1}{2}\frac{d^{2}V}{dx^2}\Big|_{x_{0}} (x - x_{0})^2 + \cdots
  \]
\item
  Time Indep Schrodinger eqn is: \[
  \Big(\frac{\hat{p}^2}{2m} + V\Big)\Psi = E\Psi
  \] \[
  \frac{1}{2m}(\hat{p}^2 + (m\omega x)^2)\Psi = E\Psi
  \]
\end{itemize}

\subsubsection{Ladder Operators}
\begin{itemize}
\item
  Can't do this factorisation: \[
  (a^2 + b^2) = (-ib + a)(-b + a)
  \]
\item
  These are operators and they don't commute, so this doesn't work
\item
  Consider: \[
  a_{\pm} = \frac{1}{\sqrt{2\hbar m\omega}}(\mp ip + m\omega x)
  \]
\item
  The prefactor is chosen to make things neater later on \[
  a_{+}a_{-} = \Big(\frac{1}{\hbar\omega}H - \frac{1}{2} \Big)
  \] \[
  a_{-}a_{+} = \Big(\frac{1}{\hbar\omega}H + \frac{1}{2} \Big)
  \]
\item
  \emph{full derivation on page 27} \[
  H = \hbar\omega (a_{+}a_{-} + \tfrac{1}{2})
  \] \[
  H = \hbar\omega (a_{-}a_{+} - \tfrac{1}{2})
  \]
\item
  Hamiltonian does not quite factor properly and we have \[
  \hbar\omega (a_{\pm}a_{\mp} \pm \tfrac{1}{2})\Psi = E\Psi
  \]
\item
  Suppose could find soln \(\Psi_n\), with associated energy, \(E_n\)
  \begin{itemize}
  \item
    operate on this with \(a_+\) for \(a_{+}\Psi_n\) which is also a
    soln \[
    H(a_{+}\Psi_n) = \hbar\omega (a_{+}a_{-} + \tfrac{1}{2})a_{+}\Psi_n
    \] \[
    H(a_{+}\Psi_{n}) = (E_n + \hbar\omega)a_{+}\Psi_n
    \]
  \end{itemize}
\item
  If \(\Psi_n\) satisfies Schrodinger with energy, \(E_n\), then
  \(a_{+}\Psi_n\) satisfies it with energy, \(E_n + \hbar \omega\)
  \begin{itemize}
  \item
    This works similarly with \(a_{-}\)
  \end{itemize}
\item
  Need one energy to get started and find the rest of them with this
  method
  \begin{itemize}
  \item
    must be a bottom rung \[
    a_{-}\Psi_{0} = 0 \\
    \frac{1}{\sqrt{2\hbar m\omega}}(\hbar\frac{d}{dx} + m\omega x)\Psi_0 = 0
    \] \[
    \frac{d\Psi_0}{dx} = -\frac{m\omega}{\hbar}x\Psi_0
    \] \[
    \Psi_0 = Ne^{-\tfrac{m\omega x^2}{2\hbar}}
    \] \[
    \Psi_{0}(x) = \Big(\frac{m\omega}{\pi\hbar} \Big)^{\tfrac{1}{4}} e^{-\tfrac{m\omega x^2}{2\hbar}}
    \]
  \end{itemize}
\item
  This solution is the Gaussian wavefn used in the first few sections
  \begin{itemize}
  \item
    \(a = \frac{m\omega}{\hbar}\)
  \item
    associated energy is easy to find from
    \(H\Psi = \hbar\omega(a_{+}a_{-} + \tfrac{1}{2})\Psi = E\Psi\) \[
    a_{-}\Psi_{0} = 0
    \] \[
    \frac{1}{2}\hbar\omega\Psi_0 = E_{0}\Psi_{0}
    \] \[
    \therefore E_0 = \frac{\hbar\omega}{2}
    \]
  \end{itemize}
\item
  Call this state \(n = 0\), then all other states follow: \[
  \Psi_n = A_{n}a_{+}^{n}\Psi_{0}
  \] \[
  E_n = (n + \tfrac{1}{2})\hbar\omega
  \]
\item
  \(n = 0\) is the ground state for SHO
\end{itemize}

\subsubsection{Brute Force Solution}
\begin{itemize}
\item
  lots of maths - see page 28
\item
  Can be solved using the Frobenius/Power series technique
\item
  Polynomial solns known as Hermite polynomials \(H_{n}(\zeta)\)
  \begin{itemize}
  \item
    n denotes order of polynomial
  \item
    will get lots of other terms involved \[
    H_{n + 2}(\zeta) - 2\zeta H_{n + 1}(\zeta) + 2(n + 1)H_{n}(\zeta) = 0
    \]
  \end{itemize}
\item
  Above eqn fixes this
\item
  Swap between even and odd fns as you go up in n
\end{itemize}

\subsection{Properties of Harmonic Potential}\label{properties-of-harmonic-potential}

\begin{itemize}
\item
  Wavefns for ISW, FSW, and SHO have similar properties \[
  E_n = (n + \tfrac{1}{2})\hbar\omega
  \]
\item
  Runs from n = 0
\item
  System has zero point energy which is non-zero
  \begin{itemize}
  \item
    due to Heisenberg uncertainty
  \item
    can't sit motionless at the bottom of well as then psn and momentum
    are both known
  \item
    ground state of system must satisfy Heisenberg
  \item
    implies energy is greater than the minimum of the pot well
  \end{itemize}
\end{itemize}

\section{The 3-D Schrodinger Equation}\label{the-3-d-schrodinger-equation}

\begin{itemize}
\item
  page 30
\end{itemize}
\[
    i\hbar\frac{\partial \Psi}{\partial t} = -\frac{\hbar^2}{2m}\Big(\frac{\partial^2 \Psi}{\partial x^2} + \frac{\partial^2 \Psi}{\partial y^2} + \frac{\partial^2 \Psi}{\partial z^2} \Big) + V(x, y, z, t)\Psi
\] \[
    i\hbar\frac{\partial \Psi}{\partial t} -\frac{\hbar^2}{2m}\nabla^{2}\Psi(\vec{r}, t) + V(\vec{r}, t)\Psi(\vec{r}, t)
\] \[
    -\frac{\hbar^2}{2m}\nabla^{2}\Psi_{n}(\vec{r}) + V\Psi_{n}(\vec{r}) = E_{n}\Psi_{n}(\vec{r})
\]

\begin{itemize}
\item
  This is separable over space \[
  \Psi_n(x, y,  z) = X(x)Y(y)Z(z) \text{ if } V(x,y,z) = V_{x}(x) + V_{y}(y) + V_{z}(z)
  \]
\item
  Can separate into three equations for dimensions \[
  -\frac{\hbar^2}{2m}\frac{\partial^2 \Psi}{\partial x^2} + V_{x}X(x) = E_{x}X(x)
  \] \[
  -\frac{\hbar^2}{2m}\frac{\partial^2 \Psi}{\partial y^2} + V_{y}Y(y) = E_{y}Y(y)
  \] \[
  -\frac{\hbar^2}{2m}\frac{\partial^2 \Psi}{\partial z^2} + V_{z}Z(z) = E_{z}Z(z)
  \] \[
  E_x + E_y + E_z = E
  \]
\item
  Not many useful potentials can be split up like this

  \begin{itemize}
  \item
    Harmonic Oscillator can be though
  \end{itemize}
\end{itemize}

\subsection{ISW Potential}\label{isw-potential}

\begin{itemize}
\item
  Inside the well, we have explicit eqns as above
\item
  Soln is just the same as 1D case in each direction
\item
  Full wavefn and energy level: \[
  \Psi(x,y,z) = X(x)Y(y)Z(z) = \sqrt{\frac{8}{L_{x}L_{y}L_{z}}}\sin(\frac{n_x \pi x}{L_x})\sin(\frac{n_y \pi y}{L_y})\sin(\frac{n_z \pi z}{L_z})
  \] \[
  E = E_x + E_y + E_z + \frac{\pi^2 \hbar^2}{2m} \Big(\frac{n_{x}^2}{L_{x}^2} + \frac{n_{y}^2}{L_{y}^2} + \frac{n_{z}^2}{L_{z}^2} \Big)
  \]
\item
  Things simplify a bit for a cube, see notes page 31
\item
  Only one way to get ground state, (1, 1, 1), so it is non-degenerate
  \begin{itemize}
  \item
    next energy level can be (1, 1, 2), (1, 2, 1), (2, 1, 1), so it is
    three-fold degenerate
  \item
    get degeneracies because of symmetries of the potential
  \end{itemize}
\end{itemize}

\subsection{Schrodinger in Spherical Polars}\label{schrodinger-in-spherical-polars}

\begin{itemize}
\item
  page 32 \[
  x = r\sin\theta\cos\phi
  \] \[
  y = r\sin\theta\sin\phi
  \] \[
  z = r\cos\theta
  \]
\item
  Usual stuff for integration in sphericals
\item
  Usual conversion for everything else too \[
  \nabla^2 = \frac{1}{r^2}\frac{\partial}{\partial r}\Big(r^2 \frac{\partial}{\partial r} \Big) + \frac{1}{r^2 \sin\theta}\frac{\partial}{\partial \theta} \Big(\sin\theta \frac{\partial}{\partial\theta} \Big) + \frac{1}{r^2 \sin^2\theta} \frac{\partial^2}{\partial \phi^2}
  \]
\item
  Solutions are called Spherical Harmonics, \(Y(\theta, \phi)\)
\item
  Related to angular momentum
\end{itemize}

\section{Angular Momentum}\label{angular-momentum}

\begin{itemize}
\item
  \(\underline{L} = \underline{r} \times \underline{p}\)
\item
  The angular momentum operators: \[
  \hat{L}_x = -i\hbar \Big(y\frac{\partial}{\partial z} - z\frac{\partial}{\partial y} \Big)
  \] \[
  \hat{L}_y = -i\hbar \Big(z\frac{\partial}{\partial x} - x\frac{\partial}{\partial z} \Big)
  \] \[
  \hat{L}_z = -i\hbar \Big(x\frac{\partial}{\partial y} - y\frac{\partial}{\partial x} \Big)
  \]
\item
  Angular momentum operators are Hermitian
\item
  Heisenberg doesn't apply across these operators, if measuring in
  different directions
\end{itemize}

\subsection{Combinations}
\begin{itemize}
\item
  \(L_x, L_y, L_z\) do not commute \[
  [L_x, L_y] = i\hbar L_z
  \]
\item
  Same cycle for other permutations
\item
  Non-commutation means can't measure all components at once
\item
  Can't know a pair of the angular momentums
\item
  Uncertainty for this: \[
  \sigma_{L_x}\sigma_{L_y} \geq \frac{1}{2}\Bigg| \Big\langle[L_x, L_y]\Big\rangle\Bigg| = \frac{\hbar}{2}\Bigg|\Big\langle L_z \Big \rangle \Bigg|
  \]
\end{itemize}

\subsection{Total Angular momentums}
\begin{itemize}
\item
  \(L^2 = L_{x}^2 + L_{y}^2 + L_{z}^2\)
\item
  \(L^2\) commutes with each of the components, i.e.
  \([L^2, L_{i}] = 0, ~i = x,y,z\)
\item
  So can measure a single component of angular momentum alongside total
  magnitude of angular momentum
  \begin{itemize}
  \item
    Use \(L_z\) as t is the simplest
  \item
    \(L_z\) and \(L^2\) commute, so they share a common set of enfns
  \end{itemize}
\end{itemize}

\subsection{Spatial Symmetry and L}\label{spatial-symmetry-and-l}
\begin{itemize}
\item
  page 35
\item
  Angular momentum is zero for any spherically symmetric wavefn
\item
  L about the z-axis is related to angular dependence
  \begin{itemize}
  \item
    oscillations between real and imaginary in x-y plane
  \item
    \(x + iy = re^{i\theta}\)
  \end{itemize}
\end{itemize}

\section{Angular Momentum and Spherical Harmonics}\label{angular-momentum-and-spherical-harmonics}

\subsection{Angular momentum Operators in Spherical Polars}\label{angular-momentum-operators-in-spherical-polars}

\begin{itemize}
\item
  Use angular momentum operators in polar form: \[
  L_z = -i\hbar \frac{\partial}{\partial\phi}
  \] \[
  L^2 = -\hbar^2 \Big(\frac{1}{\sin\theta}\frac{\partial}{\partial \theta} \big(\sin\theta \frac{\partial}{\partial \theta} \big) + \frac{1}{\sin^2 \theta}\frac{\partial^2}{\partial \phi^2} \Big)
  \]
\item
  All of these can be obtained using the chain rule
\item
  \(L^2\) is essentially the angular part of the Laplacian
  \begin{itemize}
  \item
    use this to rewrite Schrodinger
  \end{itemize}
\item
  Angular dependence of wavefn is related to angular momentum
\end{itemize}

\subsection{Eigenfunctions of $L_z$}\label{eigenfunctions-of-l_z}

\begin{itemize}
\item
  Want to solve
\end{itemize}

\[
    L_z \Phi_m = m\hbar\Phi_m
\]

\begin{itemize}
\item
  for eigenvalues m and eigenfunctions \(\Phi_m\)
\end{itemize}

\[
    \Phi_m = \frac{1}{\sqrt{2\pi}}e^{im\phi}
\]

\begin{itemize}
\item
  The equation is satisfied for any value of m but the soln requires
  periodicity of \(2\pi\)
  \begin{itemize}
  \item
    results that m must be a integer
  \end{itemize}
\item
  Eigenvalues of \(L_z\) are \(0, \pm n\hbar,~ n \in \mathbb{R}\)
  \begin{itemize}
  \item
    z-axis could be chosen in any arbitrary direction then the orbital
    angular momentum about any axis is quantised
  \item
    m is the magnetic quantum number in atoms
    \begin{itemize}
    \item
      response to magnetic fields
    \end{itemize}
  \end{itemize}
\item
  These eigenfunctions are orthonormal
\end{itemize}

\[
    \int_{0}^{2\pi} \Phi_{n}^{*} \Phi_{m} d\phi = \delta_{nm}*
\]

\begin{itemize}
\item
  Can expand any angular function:
\end{itemize}

\[
    f(\phi) = \sum c_m \Phi_m
\]

\subsection{\texorpdfstring{Eigenfunctions of
\(L^2\)}{Eigenfunctions of L\^{}2}}\label{eigenfunctions-of-l2}

\begin{itemize}
\item
  There is a common set of eigenfunctions for \(L_z\) and \(L^2\)
  \begin{itemize}
  \item
    Call these \(Y_{lm}(\theta, \phi)\) \[
    L_z Y_{lm}(\theta, \phi) = m\hbar Y_{lm}(\theta, \phi)
    \] \[
    L^2 Y_{lm}(\theta, \phi) = l(l + 1)\hbar^2 Y_{lm}(\theta, \phi)
    \]
  \end{itemize}
\item
  Set the constant to \(l(l + 1)\)
  \begin{itemize}
  \item
    will be explained later
  \end{itemize}
\item
  \(Y_{lm}\) is separable, otherwise it wouldn't share eigenfunctions
  with \(L_z\)
  \(Y_{lm}(\theta, \phi) = \Theta_{lm}(\theta)\Phi_{m}(\theta)\)
\item
  derivation on page 38
  \begin{itemize}
  \item
    will solve this in maths lectures
  \end{itemize}
\item
  When \(m = 0\), solutions are called \textbf{Legendre polynomials},
  \(P_{l}(\cos\theta)\)
  \begin{itemize}
  \item
    l is the order of the polynomial
  \end{itemize}
\item
  For \(m \neq 0\), the solutions are \textbf{associated Legendre
  polynomials}
  \begin{itemize}
  \item
    related to the \(|m|^{th}\) derivative of the \(P_l\)
  \item
    page 39
  \end{itemize}
\item
  Since \(P_l\) is a polynomial of degree l, then its \(l + 1\)
  derivative vanishes
  \begin{itemize}
  \item
    for a fixed value of l, we require \(|m| \leq l\)
  \item
    \(2l + 1\) values of m for every l
  \end{itemize}
\item
  Normalise all of this to get \(\Theta_{lm}(\theta)\)
\end{itemize}

\[
    \Theta_{lm} = (-1)^m \Big(\frac{2l + 1}{2} \frac{(l - m)!}{(l + m)!} \Big)^{1/2} P_{l}^{m}(\cos\theta)
\]

\subsection{\texorpdfstring{Spherical Harmonics of
\(Y_{lm}\)}{Spherical Harmonics of Y\_\{lm\}}}\label{spherical-harmonics-of-y_lm}

\begin{itemize}
\item
  Eigenfunctions common to \(L_z\) and \(L^2\) are given by:
  \begin{itemize}
  \item
    with \(l \geq 0\) and \(-l \leq m \leq l\)
  \end{itemize}
\end{itemize}

\[
    Y_{lm}(\theta, \phi) = \Theta_{lm}\Phi_m = (-1)^m \Big(\frac{2l + 1}{2} \frac{(l - m)!}{(l + m)!} \Big)^{1/2} P_{l}^{m}(\cos\theta)e^{im\phi}
\]

\begin{itemize}
\item
  Convention of:
\end{itemize}
\[
    Y_{lm}^{*}(\theta, \phi )* = (-1)^m Y_{l,-m}(\theta, \phi)
\]
\begin{itemize}
\item
  page 39 for proof of values of m
\item
  Start with this:
\end{itemize}
\[
    \langle L^2 \rangle = \langle L_{x}^2 + L_{y}^2 + L_{z}^2 \rangle = \langle L_{x}^2\rangle + \langle L_{y}^2 \rangle + \langle L_{z}^2 \rangle
\]
\begin{itemize}
\item
  Operators are all Hermitian and real and non-negative
\end{itemize}
\[
    \langle L^2 \rangle \geq \langle L_{z}^2 \rangle
\]
\begin{itemize}
\item
  Continue from there to get
\end{itemize}
\[
    l(l + 1) \geq m^2
\]
\begin{itemize}
\item
  This implies the physical limit of \(|m| \leq l\)
\item
  Specifying \(L_z\) means that \(L_x\) and \(L_y\) can't have defined
  values
  \begin{itemize}
  \item
    measuring them will get a quantised level,
    \(\pm n\hbar,~~ 0 \leq n \leq l\)
  \item
    probability varies
  \end{itemize}
\item
  Can explicitly evaluate expectation values of \(\langle L_x \rangle\)
  and \(\langle L_y \rangle\)
  \begin{itemize}
  \item
    both turn out to be zero
  \item
    the probability of \(\pm \hbar\) etc are equal
  \end{itemize}
\item
  page 40
\item
  A semi-classical vector model helps understand this behaviour
  \begin{itemize}
  \item
    set magnitude to \(\sqrt{l(l + 1)}\hbar\)
  \item
    spherical harmonics as well
  \item
    can deduce m from the length etc
  \item
    dipole patterns of `heat'
  \end{itemize}
\item
  page 41 for probability distributions
  \begin{itemize}
  \item
    more interesting than the wavefunctions
  \end{itemize}
\end{itemize}
\[
    P(\theta, \phi) \sin\theta d\theta d\phi = Y_{lm}^* Y_{lm} \sin\theta d\theta d\phi
\] \[
    P(\theta, \phi) d\Omega = Y_{lm}^* Y_{lm} d\Omega
\]
\begin{itemize}
\item
  \(d\Omega\) is the element of solid angle
  \begin{itemize}
  \item
    \(\sin\theta d\theta d\phi\)
  \end{itemize}
\item
  Easier to visualise as exponents in \(Y_{lm}\) cancel out
\item
  To get \(\theta\) dependence, integrate over \(\phi\)
\end{itemize}
\[
    P(\theta) d\theta = \int_{\phi = 0}^{2\pi} |Y_{lm}(\theta, \phi)|^2 d\phi \sin\theta d\theta
\]
\section{Finding the Hydrogen Wavefunction}\label{finding-the-hydrogen-wavefunction}

\subsection{Radial Equation for Spherical Potential}\label{radial-equation-for-spherical-potential}

\[
    \frac{1}{R}\frac{\partial}{\partial r} \Big(r^2 \frac{\partial R}{\partial r} \Big) - \frac{2mr^2}{\hbar^2}(V(r) - E) = \frac{L^2 Y}{\hbar^2 Y} \]
\[
    L^2 Y = l(l + 1)\hbar^2 Y \] \[
    \frac{1}{R}\frac{\partial}{\partial r} \Big(r^2 \frac{\partial R}{\partial r} \Big) - \frac{2mr^2}{\hbar^2}(V(r) - E) - l(l + 1) = 0
\]

\begin{itemize}
\item
  This is equation for the eigenfunctions, \(R(r)\)
  \begin{itemize}
  \item
    depends on l but not m
  \item
    for each l, different eigenfunctions
    \begin{itemize}
    \item
      can label with index n
    \end{itemize}
  \item
    denote \(R_{nl}(r)\)
  \end{itemize}
\item
  page 42
  \begin{itemize}
  \item
    lots of simplifications to equation
  \item
    \(rR(r) = U_{nl}\)
  \end{itemize}
\item
  Radial equation:
\end{itemize}
\[
    -\frac{\hbar^2}{2m}\frac{d^2 U_{nl}}{dr^2} + \Bigg[V(r) + \frac{\hbar^2}{2m}\frac{l(l + 1)}{r^2} \Bigg]U_{nl} = EU_{nl}
\]
\begin{itemize}
\item
  Identical to 1D Schrodinger
  \begin{itemize}
  \item
    normal potential replaced with effective potential in square
    brackets
  \item
    extra term behaves like potential:
    \begin{itemize}
    \item
      Centripetal force: \(F = \frac{mv^2}{r}\)
    \item
      \(F = \frac{L^2}{mr^3} ~;~ L = mvr\)
    \item
      integrate from this to get the term
    \end{itemize}
  \end{itemize}
\end{itemize}

\subsection{The Hydrogen Atom}\label{the-hydrogen-atom}
\begin{itemize}
\item
  Hydrogen atom
  \begin{itemize}
  \item
    electron and proton instead of just one particle
  \item
    replace electron mass with the reduced mass:
  \item
    \(\mu = \frac{M_p m_e}{(M_p + m_e)}\)
  \end{itemize}
\item
  Use Coulomb potential
\end{itemize}
\[
    -\frac{\hbar^2}{2\mu}\frac{d^2 U_{nl}}{dr^2} + \frac{l(l + 1)\hbar^2}{2\mu r^2}U_{nl} - \frac{Ze^2}{4\pi\epsilon_{0} r}U_{nl} = EU_{nl}
\]
\begin{itemize}
\item
  Solve this for the eigenfunctions
\item
  Multiply by spherical harmonics for total
\end{itemize}
\[
    R_{nl} \propto \rho^{l} e^{-\rho} L_{n - l - 1}^{2l + 1}(2\rho) \]
\[
    \rho = kr \] \[
    \Psi_{nlm}(r,\theta,\phi) = R_{nl}(r)Y_{lm}(\theta,\phi) \] \[
    E = -13.6\frac{Z^2}{n^2}\frac{\mu}{\mu_{H}}\,eV
\]
\begin{itemize}
\item
  This only depends on n
\item
  Properties of Laguerre polynomials require \(n \geq l + 1\) so
  \(l \leq n - 1\)
  \begin{itemize}
  \item
    only features for \(1/r\) potential
  \item
    all have energies dep on m
  \end{itemize}
\item
  Degeneracy
  \begin{itemize}
  \item
    For given n, n values of l which have same energy so the level is n
    degenerate
  \item
    for each l, \(2l + 1\) values of m, so in fact each level is
  \item
    \(\sum_{l = 0}^{n} (2l + 1)\) degenerate
  \item
    \(2n^2\) degenerate later on once spin is added
  \end{itemize}
\end{itemize}
\[
    \rho = kr = \frac{\mu Ze^2}{2\pi\epsilon_{0}\hbar^2\rho_{0}}r = \frac{\mu Ze^2}{4\pi\epsilon_{0}\hbar^2 n}r \]
\[
    a = \frac{4\pi\epsilon_{0}\hbar^2}{\mu Ze^2}
\]
\begin{itemize}
\item
  a is the Bohr radius for hydrogen of \(5.29 \times 10^{-11} \,m^{-1}\)
\end{itemize}
\[
    \Psi_{nlm} \propto \Big(\frac{r}{2a} \Big)^{l}e^{-\frac{r}{an}}L_{n - l - 1}^{2l + 1} \Big(\frac{2r}{an} \Big) Y_{lm}(\theta, \phi)
\]
\begin{itemize}
\item
  Get the normalisation constant by
\end{itemize}
\[
    \int_{r = 0}^{\infty} \int_{\phi = 0}^{2\pi} \int_{\theta = 0}^{\pi} \psi^* (r,\theta,\phi)\psi(r,\theta,\phi)r^2 \sin\theta \,d\theta\, d\phi\, dr = 1 \]
\[
    E = -\frac{\mu}{2\hbar^2}\Big(\frac{Ze^2}{4\pi\epsilon_0}\Big)^2 \frac{1}{n^2}
\]
\subsection{Transitions Between Energy Levels}\label{transitions-between-energy-levels}

\begin{itemize}
\item
  If Hydrogen is some stationary state, \(\psi_{nlm}\), it should be
  stable, but perturbations can cause the electron to transition to
  another stationary state
  \begin{itemize}
  \item
    collision with atom/photon/electron
  \item
    either by absorbing energy or emitting it
  \item
    perturbations are always present so transitions - quantum jumps -
    are constantly occurring
  \item
    transitions are between discrete energy levels of n difference
  \end{itemize}
\end{itemize}
\[
    E_{\gamma} = E_{i} - E_f = 13.6 \Big(\frac{1}{n_{f}^2} - \frac{1}{n_{i}^2} \Big)eV
\]
\begin{itemize}
\item
  Transitions to ground state give rise to Lyman series of emissions
  lines with wavelengths given by
\end{itemize}
\[
    E = \frac{hc}{\lambda} = 13.6\Big(1 - \frac{1}{n^2} \Big) eV ~;~ n \geq 2
\]
\begin{itemize}
\item
  page 45 for table of Lyman series
  \begin{itemize}
  \item
    tend to a limit for shortest wavelength possible
  \end{itemize}
\end{itemize}

\section{Generalising Angular Momentum}\label{generalising-angular-momentum}

\subsection{Preview of Spin Concept}\label{preview-of-spin-concept}

\begin{itemize}
\item
  Spin is the spinning of particles around own axis
  \begin{itemize}
  \item
    electron doesn't have internal structure though
  \item
    spin is \emph{intrinsic} angular momentum
  \end{itemize}
\end{itemize}

\subsection{Review of Angular Momentum}\label{review-of-angular-momentum}
\begin{itemize}
\item
  Two components of \(\underline{L}\) cannot be measured simultaneously
  \begin{itemize}
  \item
    one component can be measured with \(L^2\) however
  \item
    Share a common set of eigenstates
  \end{itemize}
\end{itemize}

\subsection{Angular Momentum Ladder Operators}\label{angular-momentum-ladder-operators}

\begin{itemize}
\item
  page 47
\end{itemize}
\[
    \begin{aligned}
    L_{\pm} = L_x \pm iL_y \\
    [L^2, L_+] = 0
    \end{aligned}
\]
\begin{itemize}
\item
  same principle as other ladder operators
\end{itemize}
\[
    \hbar L_+ f_{\lambda, \mu} \implies L_z (L_+ f_{\lambda, \mu}) = (\mu + 1)\hbar L_+ f_{\lambda, \mu}
\]

\subsubsection{Properties}\label{properties}

\begin{itemize}
\item
  Must be a top and bottom rung
  \begin{itemize}
  \item
    max and min values for \(\mu\)
  \item
    \(L_+ f_{\lambda, \mu_{max}} = 0\)
  \end{itemize}
\item
  \(\lambda = \mu_{max} (\mu_{max} + 1)\)
  \begin{itemize}
  \item
    Legendre polynomial approach
  \item
    \(\mu_{min} = -\mu_{max}\)
  \item
    this leads to the range of values for \(m\)
  \end{itemize}
\item
  \(\mu\) can take half-integer values
  \begin{itemize}
  \item
    comes from addition of intrinsic and extrinsic
  \end{itemize}
\end{itemize}

\subsection{Overview of general angular momentum, J}\label{overview-of-general-angular-momentum-j}

\begin{itemize}
\item
  \(\underline{J}\) is angular momentum if its operator components,
  \(J_x, J_y, J_z\) satisfy
\end{itemize}
\[
    \begin{aligned}
    [J_x, J_y] &= i\hbar J_z ~;~ [J_z, J_x] = i\hbar J_y \\
    J^2 &= J_{x}^2 + J_{y}^2 + J_{z}^2 \\
    [J^2, J_x] &= 0
    \end{aligned}
\]
\begin{itemize}
\item
  Common eigenfunctions of \(J^2\) and \(J_z\)
  \begin{itemize}
  \item
    eigenvalues of \(j(j + 1)\hbar^2\) and \(m_j \hbar\) respectively
  \end{itemize}
\item
  Ladder operators
  \begin{itemize}
  \item
    \(J_zJ_{+}f_{j,m_j} = (m_j + 1)\hbar J_+ f_{j,m_j}\)
  \end{itemize}
\item
  Can't go on forever so top and bottom values for \(m\)
  \begin{itemize}
  \item
    \(m_{max} - m_{min} \in \mathbb{N}\)
  \item
    \(j = \frac{\mathbb{N}}{2}\)
  \end{itemize}
\item
  page 49
\end{itemize}
\[
    \underline{J} = \underline{L} + \underline{S}
\]

\section{Spin}\label{spin}

\begin{itemize}
\item
  \(\underline{S}\) as angular momentum spin operator
  \begin{itemize}
  \item
    \(S^2\) eigenvalues, \(s(s + 1)\hbar^2\)
  \item
    do not have spherical harmonic eigenfunctions
  \item
    eigenstates not functions of spatial coordinates
  \end{itemize}
\item
  intrinsic property of particle
\end{itemize}

\subsection{Electron Orbiting Magnetic Field}\label{electron-orbiting-magnetic-field}

\[
    \mu_l = IA = -\frac{evr}{2} = -\frac{e}{2m_e}m_e vr = -\frac{e}{2m_e}L \]
\[
    \underline{\mu}_l = -\frac{e}{2m_e}\underline{L} = -\frac{\mu_B}{\hbar}\underline{L} \]
\[
    \mu_B = \frac{e\hbar}{2m_e}
\]

\begin{itemize}
\item
  \(\mu_B\) is the Bohr magneton, natural unit of microscopic magnetic
  moment
  \begin{itemize}
  \item
    \(= 9.27 \times 10^{-24} J\,T^{-1}\)
  \item
    \(= 5.79 \times 10^{-5} eV \, T^{-1}\)
  \end{itemize}
\item
  \(-\frac{\mu_B}{\hbar}\underline{L}\) is quantum mechanical magnetic
  moment
  \begin{itemize}
  \item
    \((\mu_l)_{z} = -\frac{\mu_{B}}{\hbar}\underline{L}_z\)
  \end{itemize}
\item
  For a hydrogen atom,
  \begin{itemize}
  \item
    \((\mu_l)_{z} = -m_{l}\mu_B,~ -l \leq m_l \leq +l, \in \mathbb{N}\)
  \end{itemize}
\end{itemize}

\subsection{Dirac Notation}\label{dirac-notation}
\[
    |\chi_+ \rangle \to \begin{pmatrix} 1 \\ 0 \end{pmatrix} ~;~ |\chi_- \rangle \to \begin{pmatrix} 0 \\ 1 \end{pmatrix} \]
\[
    \langle \chi_+ | \to \begin{pmatrix} 1 & 0 \end{pmatrix} ~;~ \langle \chi_- | \to \begin{pmatrix} 0 & 1 \end{pmatrix}
\]

\begin{itemize}
\item
  kets are equivalent ti column vectors
\item
  bras to row vectors
\item
  operators to square matrices
\item
  taking complex conjugate of inner product can be seen as taking
  Hermitian conjugate of three matrices:
  \begin{itemize}
  \item
    \((ABC)^{\dagger} = C^\dagger B^\dagger A^\dagger\)
  \item
    \(\langle \psi_n | A | \psi_m \rangle* = \langle \psi_m | A^\dagger | \psi_n \rangle\)
  \item
    \(\langle \psi_n | A | \psi_m \rangle* = \langle \psi_m | A | \psi_n \rangle\),
    if A is Hermitian
  \end{itemize}
\item
  Bra-Ket inner products were introduced earlier as shorthand for
  integrals
  \begin{itemize}
  \item
    more general than this for when there are no spatial coordinates to
    integrate over
    \begin{itemize}
    \item
      i.e. spin
    \end{itemize}
  \end{itemize}
\end{itemize}

\section{Time-dependent Perturbation Theory}\label{time-dependent-perturbation-theory}

\subsection{Small Modification to Hamiltonian}\label{small-modification-to-hamiltonian}

\begin{itemize}

\item
  Solved time-dependent Schrodinger for some Hamiltonian, \(H^0\)
  \begin{itemize}
  \item
    Get eigenfunctions, \(\psi_{n}^0\) for each energy level
  \item
    \(H^0 \psi_{n}^0 = E_{n}^0 \psi_{n}^0\)
  \item
    orthonormal
  \end{itemize}
\item
  Perturb the system slightly
  \begin{itemize}
  \item
    a small bump in the square well
  \item
    want to find eigenfunctions and eigenvalues for new Hamiltonian
  \item
    can't solve this exactly but can use \emph{perturbation theory} for
    approx solution
    \begin{itemize}
    \item
      build on known solutions of unperturbed system
    \end{itemize}
  \end{itemize}
\item
  Write new Hamiltonian as
  \begin{itemize}
  \item
    \(H = H^0 + \lambda H'\)
    \begin{itemize}
    \item
      H' is the perturbation
    \end{itemize}
  \end{itemize}
\end{itemize}

\subsection{Power expansion}\label{power-expansion}

\begin{itemize}
\item
  Want to write \(E_n\) and \(\psi_n\) as power series in \(\lambda\)
\end{itemize}

\[
    E_n = E_{n}^0 + \lambda E_{n}^1 + \lambda^2 E_{n}^2 + \cdots \] \[
    \psi_n = \psi_{n}^0 + \lambda \psi_{n}^1 + \lambda^2 \psi_{n}^2 + \cdots
\]

\begin{itemize}
\item
  \(E_{n}^1\) and \(\psi_{n}^1\) refer to the first order corrections
  for the nth eigenfunctions
  \begin{itemize}
  \item
    superscript 2 is second order correction etc
  \end{itemize}
\end{itemize}

\[
    H\psi_n = E_{n}\psi_n \] \[
    (H = H^0 + \lambda H')(\psi_{0} + \lambda \psi_{n}^1 + \lambda^2 \psi_{n}^2 + \cdots) = ( E_{n}^0 + \lambda E_{n}^1 + \lambda^2 E_{n}^2 + \cdots)(\psi_0 + \lambda \psi_{n}^1 + \lambda^2 \psi_{n}^2 + \cdots)
\]

\begin{itemize}
\item
  collect powers of lambda together
\end{itemize}

\[
    H^0\psi_0 + \lambda(H^0\psi_{n}^1 + H'\psi_{n}^0) + \lambda^2 (H^0 \psi_{n}^2 + H'\psi_{n}^1) + \cdots = E_{n}^0\psi_{n}^0 + \lambda(E_{n}^0\psi_{n}^1 + E_{n}^1\psi_{n}^0) + \lambda^2(E_{n}^0\psi_{n}^2 + E_{n}^1\psi_{n}^1 + E_{n}^2\psi_{n}^0) + \cdots
\]

\subsection{First Order correction}\label{first-order-correction}

\begin{itemize}
\item
  Zeroth order expansion is just no perturbation
\item
  Look at first order in \(\lambda\):
\end{itemize}

\[
    H^0\psi_{n}^1 + H'\psi_{n}^0 = E_{n}^0\psi_{n}^1 + E_{n}^1\psi_{n}^0
\]

\begin{itemize}
\item
  Take an inner product with \(\psi_{n}^0\)
  \begin{itemize}
  \item
    use Dirac notation for maximum generality
  \item
    some terms from this will cancel so:
  \end{itemize}
\end{itemize}

\[
    E_{n}^1 = \langle \psi_{n}^1 | H' | \psi_{n}^0 \rangle
\]

\begin{itemize}
\item
  Expressing this with integrals instead yields the same result:
\end{itemize}

\[
    E_{n}^1 = \int (\psi_{n}^0)^* H' \psi_{n}^0 \,dx
\]

\begin{itemize}
\item
  Either way, the first order correction to an energy eigenvalues is the
  \textbf{expectation value} of the perturbation using the unperturbed
  eigenfunctions
\end{itemize}

\subsection{1D Square Well with Delta Function}\label{d-square-well-with-delta-function}

\begin{itemize}
\item
  With Kronecker delta, \(\delta_{nm} = 1\) if \(n = m\), or 0 otherwise
  \begin{itemize}
  \item
    The \textbf{Dirac Delta} is the continuum version:
  \end{itemize}
\end{itemize}

\[
    \int f(x)\delta(x - x_0) \,dx = f(x_0)
\]

\begin{itemize}
\item
  collapses any integral to the point marked out by the function
  \begin{itemize}
  \item
    \(\int \delta(x - x_0)\,dx = 1\)
  \end{itemize}
\item
  So if we have \(H' = \lambda\delta(x - \frac{a}{2})\) then
\end{itemize}

\[
    E_{n}^1 = \int \int (\psi_{n}^0)^* H' \psi_{n}^0 \,dx \] \[
    E_{n}^1 = \frac{2}{a}\int_{0}^{a} \sin\Big(\frac{n\pi x}{a}\Big)\lambda\delta(x - \frac{a}{2})\sin\Big(\frac{n\pi x}{a}\Big)\,dx = \frac{2\lambda}{a}\sin^2 \Big(\frac{n\pi}{2}\Big)
\]

\begin{itemize}
\item
  This is 0 if n is even
  \begin{itemize}
  \item
    no correction to even \(E_{n}^0\)
  \item
    but \(E_{n}^1 = \frac{2\lambda}{a}\) for odd n
  \item
    perturbation has no effect for even n
  \item
    odd n gets a peak and energies are shifted
  \end{itemize}
\end{itemize}

\subsection{First Order Correction to Wavefunction}\label{first-order-correction-to-wavefunction}

\begin{itemize}
\item
  Calc by writing \(\psi_{n}^1 = \sum_{n \neq 1} c_{nl}\psi_{l}^0\)
  \begin{itemize}
  \item
    substitute this into the first order correction
  \item
    take inner product with \(\psi_{l}^0\)
  \end{itemize}
\end{itemize}

\[
    c_{nl} = -\frac{\langle \psi_{l}^0 | H' | \psi_{n}^0 \rangle}{(E_{n}^0 - E_{l}^0)}
\]

\begin{itemize}
\item
  To first order, \(\psi_n \approx \psi_{n}^0 + \lambda\psi_{n}^1\)
\end{itemize}

\section{Degenerate Perturbation Theory}\label{degenerate-perturbation-theory}

\begin{itemize}
\item
  \(E_n - E_m \neq 0\) isn't always the case
\end{itemize}

\paragraph{Twofold Degeneracy}\label{twofold-degeneracy}

\begin{itemize}
\item
  Suppose energy level with exactly two states, \(\psi_{a}^0\) and
  \(\psi_{b}^0\), giving same energy of \(E^0\)

  \begin{itemize}

  \item
    any linear combination of these gives \(E^0\)
  \item
    e.g.~in Hydrogen, \(l = 0,\, m = 0,\, m_s = \pm \frac{1}{2}\)
  \end{itemize}
\item
  The perturbation \(\lambda H'\) breaks the degeneracy, we want to find
  these unperturbed states
\item
  Same first order correction:
\end{itemize}

\[
    H^0 \psi^1 + \lambda H'\psi^0 = E^0 \psi^1 + \lambda E^1 \psi^0
\]

\begin{itemize}
\item
  multiply by conjugate of one of the states, \(\psi_{a}^{0*}\), or
  \(\langle \psi_{a}^0|\), and integrate
\end{itemize}

\[
    \langle \psi_{a}^0 | H' | \psi^0 \rangle  = E^1 \langle \psi_{a}^0 | \psi^0 \rangle
\]

\begin{itemize}
\item
  full derivation on page 55
\item
  use \(\psi^0 = \alpha \psi_{a}^0 + \beta \psi_{b}^0\):
\end{itemize}

\[
    \alpha \langle \psi_{a}^0 | H' | \psi_{a}^0 \rangle + \beta \langle \psi_{a}^0 | H' | \psi_{b}^0 = E^1 \alpha \]
\[
    \alpha W_{aa} + \beta W_{ab} = \alpha E^1 \] \[
    W_{ij} = \langle \psi_{i}^0 | H' | \psi_{j}^0 \rangle = \int \psi_{i}^{0*} H' \psi_{j}^0 dx
\]

\begin{itemize}
\item
  can assemble this into eigenvalue matrix equation:
\end{itemize}

\[
    \begin{pmatrix} W_{aa} & W_{ab} \\ W_{ba} & W_{bb} \end{pmatrix} \begin{pmatrix} \alpha \\ \beta \end{pmatrix} = E^1 \begin{pmatrix} \alpha \\ \beta \end{pmatrix} \]
\[
    \implies (W_{aa} - E^1)(W_{bb} - E^1) - W_{ab}W_{ba} = 0 \] \[
    \implies (E^1)^2 - (W_{aa} + W_{bb})E^1 + (W_{aa}W_{bb} - |W_{ab}|^2) = 0 \because W_{ab} = W_{ba}^* \]
\[
    E_{\pm}^1 = \frac{1}{2} \Bigg[W_{aa} + W_{bb} \pm \sqrt{(W_{aa} - W_{bb})^2 + 4|W_{ab}|^2}\Bigg]
\]

\begin{itemize}
\item
  If two states are degenerate, both having the same energy, any linear
  combination also has the same energy
\item
  A small perturbation \(H'\) cause a small change in energy and the
  first order approximation for this, \(E^1\) is given by the solution
  of a matrix equation
\item
  \emph{see pages 56 \& 57 for examples on this}
\end{itemize}

\section{Degenerate Perturbation Theory II}\label{degenerate-perturbation-theory-ii}

\subsection{Link to non-degenerate perturbation theory}\label{link-to-non-degenerate-perturbation-theory}

\begin{itemize}
\item
  page 58
\end{itemize}

\subsection{Higher-order degeneracy}\label{higher-order-degeneracy}

\begin{itemize}
\item
  For n-fold degeneracy, get \(n \times n\) matrix, with n roots
  \begin{itemize}
  \item
    some roots may be zero
  \end{itemize}
\end{itemize}

\subsection{Example in 3D Square Well}\label{example-in-3d-square-well}

\begin{itemize}
\item
  page 58 and 59
\item
  integration n stuff
\end{itemize}

\[
    \begin{pmatrix} 1 - \omega & \kappa & 0 \\ \kappa & 1 - \omega & 0 \\ 0 & 0 & 1 - \omega \end{pmatrix} \begin{pmatrix} \alpha \\ \beta \\ \gamma \end{pmatrix} = \begin{pmatrix} 0 \\ 0 \\ 0 \end{pmatrix}, \omega = \frac{4E^1}{V_0}
\]

\subsection{Meaning of Vector components}\label{meaning-of-vector-components}

\begin{itemize}
\item
  \(\alpha, \beta, \gamma\) are the amplitudes of the states
  \(\psi_{211}, \psi_{121}, \psi_{112}\) in a general superposition
  state
\item
  page 60 has eigenvector stuff
\end{itemize}

\section{Degenerate Perturbation Theory and Hydrogen}\label{degenerate-perturbation-theory-and-hydrogen}

\subsection{Spin-Orbit Coupling}\label{spin-orbit-coupling}

\begin{itemize}
\item
  Perturbation from magnetic dipole moment generated by electron spin,
  \(\mu_s\) is
\end{itemize}

\[
    H_{so}' = -\underline{\mu}_s \cdot \underline{B} \] \[
    \underline{\mu}_s = -g_s \Big(\frac{e}{2m_e}\Big)\underline{S},~ g_s \approx 2 \]
\[
    \underline{\mu}_s = \frac{-e}{m_e}\underline{S}
\]

\begin{itemize}
\item
  From electron's PoV, can see a current loop from the nucleus
  `orbiting' it
\end{itemize}

\[
    L = \frac{4\pi m_er^3 B}{\mu_0 e} \] \[
    \underline{L} = \frac{4\pi epsilon_0 c^2 mr^3}{e}\underline{B} \] \[
    H_{so}' = \frac{e^2}{4\pi c^2 m^2 r^3}\underline{S}\cdot\underline{L}
\]

\begin{itemize}
\item
  Factor of 0.5 due to special relativity as well:
\end{itemize}

\[
    H_{so}' = \frac{e^2}{8\pi c^2 m^2 r^3}\underline{S}\cdot\underline{L}
\]

\subsection{Commutation Properties}\label{commutation-properties}

\begin{itemize}
\item
  pages 61 - 62
\item
  use total angular momentum, \(J^2\)
\end{itemize}

\[
    [J^2, \underline{L}\cdot\underline{S}] = 0
\]

\begin{itemize}
\item
  as all of the components of these commute with each other
\item
  think about joint eigenfunctions of the operators
  \(H^0, L^2, J^2, S^2, J_z\)
  \begin{itemize}
  \item
    since \(J\) is conserved
  \end{itemize}
\end{itemize}

\subsection{Adding Angular Momenta}\label{adding-angular-momenta}

\begin{itemize}
\item
  max value of \(J_z = l + s\)
  \begin{itemize}
  \item
    \(-l - s \leq J_z \leq l + s\), in integer steps
  \item
    correspond to value \(j = l + s\)
    \begin{itemize}
    \item
      from \(j(j+ 1)\hbar^2\) in \(J^2\)
    \end{itemize}
  \end{itemize}
\item
  see page 62 for visualisation and tables
\end{itemize}

\section{Degenerate Perturbation Theory and Hydrogen II}\label{degenerate-perturbation-theory-and-hydrogen-ii}

\subsection{Spin Orbit}\label{spin-orbit}

\[
    J^2 = (\underline{L} + \underline{S})\cdot(\underline{L} + \underline{S}) + L^2 + S^2 + 2\underline{L}\cdot\underline{S} \]
\[
    \underline{J}\cdot\underline{S} = \frac{1}{2}(J^2 - L^2 - S^2)
\]

\begin{itemize}
\item
  Eigenvalues of \(\underline{L}\cdot\underline{S}\) are eigenvalues of
  above:
\end{itemize}

\[
    \frac{\hbar^2}{2}[j(j+1) - l(l+1) - s(s + 1)] = \frac{\hbar^2}{2}\bigg[j(j+1) - l(l+1) - \frac{3}{4}\bigg]
\]

\begin{itemize}
\item
  Spin is \(\frac{1}{2}\)
\item
  Can plug this into the first order correction to the energy
  eigenvalues
  \begin{itemize}
  \item
    page 64
  \item
    can then be pulled outside the expression as just numbers
  \end{itemize}
\item
  Spin-orbit coupling typically leads to
  \(\frac{E_n^1}{E_n^0} \approx \frac{E_n^0}{mc^2}\)
\end{itemize}

\subsection{Correction to electron-proton potential (Darwin Term)}\label{correction-to-electron-proton-potential-darwin-term}

\begin{itemize}

\item
  Must correct the assumption that these are point particles
\item
  Position certain only to its Compton wavelength
\end{itemize}

\[
    \lambda_C = \frac{\hbar}{mc}
\]

\begin{itemize}
\item
  Charge of the electron is therefore smeared out over this volume
  \begin{itemize}
  \item
    smearing for the proton is smaller by a factor of the mass ratios so
    is negligible
  \end{itemize}
\item
  The correction made to the potential if the electron is at the same
  position as the proton is the Darwin term:
\end{itemize}

\[
    H_D^1 = \frac{\pi\hbar^2}{2m^2c^2}\frac{e^2}{4\pi\epsilon_0}\delta(\underline{r})
\]

\begin{itemize}
\item
  \textbf{See footnotes on page 65 for explanation on 3D delta function}
\item
  Independent of spin
\item
  Commutes with all angular momentum operators
  \begin{itemize}
  \item
    can use non-degenerate perturbation theory
  \end{itemize}
\end{itemize}

\[
    E_n^1 = \frac{\pi\hbar^2}{2m^2c^2}\frac{e^2}{4\pi\epsilon_0}\int_V \psi^* _{nlm} \delta(\underline{r}) \psi_{nlm}\,dV
\]

\begin{itemize}
\item
  Only acts at the origin
  \begin{itemize}
  \item
    therefore only on states which are non-zero at the origin
  \item
    this is only for \(l, m = 0\)
  \end{itemize}
\item
  Laguerre polynomials for \(l = 0\) are \(L_{n-1}^1 (0)\) so just a
  constant term in the polynomial
\item
  \emph{simplification maths on page 65}
\end{itemize}

\[
    E_n^1 = \frac{2}{mc^2}(E_n^0)^2 n
\]

\subsection{Relativistic corrections to kinetic energy}\label{relativistic-corrections-to-kinetic-energy}

\begin{itemize}
\item
  Special relativity KE is given by
\end{itemize}

\[
    \begin{aligned}
    E^2 = (T + mc^2)^2& = p^2c^2 + m^2c^4 \\
    T + mc^2 &= mc^2\sqrt{1 + \frac{p^2}{m^2c^2}} \\
    \implies T &\approx \frac{p^2}{2m} - \frac{1}{8}\frac{p^4}{m^3c^2} \\
    &= T^0 - \frac{1}{2mc^2}(T^0)^2
    \end{aligned}
\]

\begin{itemize}
\item
  Using a Taylor expansion on the square root
\item
  \(T^0\) is the non-relativistic KE, and the other therm is the lowest
  order correction for relativity
\item
  Commutation here allows for non-degenerate perturbation theory again
\end{itemize}

\[
    \begin{aligned}
    E_n^1 &= -\frac{1}{2mc^2}\int \psi_n^{0*}[H - V(r)]^2 \psi_n^0\,dV \\
    \implies &= -\frac{1}{2mc^2}[(E_n^0)^2 - 2E_n^0 \langle V(r)\rangle + \langle V(r)^2\rangle]
    \end{aligned}
\]

\begin{itemize}
\item
  \emph{Full derivations for this on page 66}
\end{itemize}

\[
    E^1 = -\frac{(E_n^0)^2}{2mc^2}\Bigg[\frac{4n}{(l + \frac{1}{2})} - 3 \Bigg]
\]

\section{Hydrogen Fine Structure}\label{hydrogen-fine-structure}

\subsection{Total First Order Correction}\label{total-first-order-correction}

\begin{itemize}
\item
  Cann add all the first order correction terms together to get final:
\[
    E_{nj}^1 = \frac{(E_n^0)^2}{2mc^2}\Bigg(3 - \frac{4n}{j + \frac{1}{2}}\Bigg)
\]
\item Also written as:
\[
    E_n^0 = -\alpha^2\frac{mc^2}{2n^2} \] \[
    \alpha = \frac{e^2}{4\pi\epsilon_0 \hbar c} \approx \frac{1}{137}
\]
\item
  \(\alpha\) is a dimensionless fine structure constant
\item
  see page 67
\item
  Collectively, all the extra terms got from expanding \(E_{nj}^1\), and
  the additional level of structure they reveal, are known as the fine
  structure.
\end{itemize}

\subsection{Ground State of Hydrogen}\label{ground-state-of-hydrogen}

\begin{itemize}
\item
  Only one value for \(j\) as \(l = 0\)
  \begin{itemize}
  \item
    \(j = \frac{1}{2}\), from the spin
  \item
    gives a relative correction to the ground state
    \begin{itemize}
    \item
      page 67
    \end{itemize}
  \end{itemize}
\end{itemize}

\[
    \frac{\Delta E_1}{E_1^0} = \frac{\alpha^2}{4}
\]

\begin{itemize}
\item
  This means the ground state is slightly lower than \(-13.6\)eV
\end{itemize}

\subsection{Quantum Number Sets and Balmer Series}\label{quantum-number-sets-and-balmer-series}

\begin{itemize}
\item
  page 68 - 69
\end{itemize}

\subsection{Lamb Splitting}\label{lamb-splitting}

\begin{itemize}
\item
  There is a difference between \(2s_{1/2}\) and \(2p_{1/2}\) not
  noticed on pages 68 - 69
  \begin{itemize}
  \item
    called Lamb shift, of order \(\alpha^5\)
  \item
    comes from interaction between electron and vacuum fluctuations of
    quantised EM field
  \item
    difference for between states of same \(n, j\) and different \(l\)
  \end{itemize}
\end{itemize}

\subsection{Hyperfine Splitting}\label{hyperfine-splitting}

\begin{itemize}
\item
  Another energy shift
  \begin{itemize}
  \item
    \(\times 10\) smaller than Lamb
  \end{itemize}
\item
  Arises from proton's spin
  \begin{itemize}
  \item
    interaction with electron's magnetic dipole changes potential
  \item
    page 70
  \end{itemize}
\end{itemize}

\section{Formalities and the Correspondence Principle}\label{formalities-and-the-correspondence-principle}

\subsection{Marginal Probability Distributions}\label{marginal-probability-distributions}

\begin{itemize}
\item
  Recap of probability distributions for 1D and 3D wavefunctions on page
  71
\item
  For wavefunctions split into radial wavefunctions and spherical
  harmonics, \(\psi(r,\theta, \phi) = R(r)Y(\theta,\phi)\)
  \begin{itemize}
  \item
    The probability distributions for the radial wavefunction and for
    the spherical harmonic can be separated entirely
  \end{itemize}
\end{itemize}

\subsection{Time-energy uncertainty principle}\label{time-energy-uncertainty-principle}

\[
    \begin{aligned}
    \Delta t = \frac{\Delta x}{v} &= \frac{m\Delta x}{p} \\
    E = \frac{p^2}{2m} &\therefore \Delta E = \frac{2p\Delta p}{2m} \\
    \Delta t \Delta E = \frac{m\Delta x}{p}\frac{2p\Delta p}{2m} &= \Delta x \Delta p \geq \frac{\hbar}{2}
    \end{aligned}
\]

\begin{itemize}
\item
  Note that position, momentum, and energy are all dynamical variables
  \begin{itemize}
  \item
    i.e. measurable characteristics of the system
  \end{itemize}
\item
  Time is not a dynamical variable
  \begin{itemize}
  \item
    \(\Delta t\) is the time it takes for the system to change
    substantially
  \item
    \textbf{Time is not an operator belonging to the particle, it is a
    parameter describing the evolution of the system}
  \end{itemize}
\item
  A state existing for a short time cannot have a definite energy
  \begin{itemize}
  \item
    \(E = \hbar\omega\), so the frequency of the system must be
    accurately known for the energy to be definite
  \item
    accurately-known frequency requires the system to be around for a
    long period of time
  \item
    short time \(\implies\) unknown frequency \(\implies\) high error in
    energy
  \end{itemize}
\item
  This is essentially
\end{itemize}

\[
    \Delta t = \Delta \langle B \rangle \Big/\Big(\frac{dB}{dt}\Big)
\]

\begin{itemize}
\item
  where B is some dynamical operator
\end{itemize}

\subsection{The time evolution of expectation values}\label{the-time-evolution-of-expectation-values}

\begin{itemize}
\item
  Will use Dirac Notation here as:
  \begin{itemize}
  \item
    a ket, \(|\psi\rangle\), can be thought of as a column vector;
  \item
    a bra, \(\langle \psi | = (|\psi\rangle)^\dagger\), as the Hermitian
    conjugate of a ket, i.e.~a row vector;
  \item
    Hermitian operators as square Hermitian matrices
  \end{itemize}
\item
  Take the time derivative of the expectation value of a Hermitian
  operator, Q, in general:
\end{itemize}

\[
    \frac{d\langle Q \rangle}{dt} = \frac{d}{dt}\langle\psi |Q|\psi\rangle = \bigg(\frac{\partial \langle\psi |}{\partial t}\bigg)Q|\psi\rangle + \langle\psi | \frac{\partial Q}{\partial t}|\psi\rangle + \langle\psi |Q \bigg(\frac{\partial |\psi\rangle}{\partial t}\bigg)
\]

\begin{itemize}
\item
  Full derivation on page 72, using
\end{itemize}

\[
    \begin{aligned}
    \frac{\partial |\psi\rangle}{\partial t} &= \bigg(-\frac{iH}{\hbar}\bigg)|\psi\rangle ~;~ \frac{\partial \langle \psi |}{\partial t} = \langle\psi |\bigg(\frac{iH}{\hbar}\bigg) \\
    \implies \frac{d\langle Q \rangle}{dt} &= \frac{i}{\hbar}\langle[H, Q]\rangle + \bigg\langle \frac{\partial Q}{\partial t}\bigg\rangle
    \end{aligned}
\]

\subsection{Formalism}\label{formalism}

\subsubsection{Ehrenfest Theorems}\label{ehrenfest-theorems}

\begin{itemize}
\item
  If \(Q = x\), then
\end{itemize}

\[
    \frac{d\langle x \rangle}{dt} = \frac{i}{\hbar}\langle[H, x]\rangle + \bigg\langle \frac{\partial x}{\partial t}\bigg\rangle
\]

\begin{itemize}
\item
  \(H = \frac{p^2}{2m} + V(x)\) so \([H, x] = -i\hbar\frac{p}{m}\)
\item
  The operator has no explicit time dependence, similarly to a
  coordinate system, as it is not the actual location of the particle
\item
  Hence, the First Ehrenfest Theorem:
\end{itemize}

\[
    \frac{d\langle x \rangle}{dt} = \frac{i}{\hbar}\bigg\langle -i\hbar\frac{p}{m}\bigg\rangle = \frac{\langle p \rangle}{m}
\]

\begin{itemize}
\item
  The second Ehrenfest Theorem can be proven similarly to get:
\end{itemize}

\[
    \frac{d\langle p \rangle}{dt} = -\bigg\langle\frac{dV}{dx}\bigg\rangle
\]

\begin{itemize}
\item
  Can combine these two together to get a correspondence to the
  classical \(F = ma\):
\end{itemize}

\[
    m\frac{d^2 \langle x \rangle}{dt^2} = \langle F \rangle
\]

\subsubsection{Virial Theorem}\label{virial-theorem}

\begin{itemize}
\item
  Let \(Q = xp\):
\end{itemize}

\[
    \begin{aligned}
    \frac{d\langle xp\rangle}{dt} &= \frac{i}{\hbar}\langle[H, xp]\rangle + \bigg\langle\frac{\partial \langle xp \rangle}{\partial t}\bigg\rangle \\
    [H, p] = i\hbar\frac{dV}{dx} \implies& [H, xp] = [H, x]p + x[H, p] = [-\frac{i\hbar}{m}p]p + x[i\hbar\frac{dV}{dx}] \\
    &= \frac{i}{\hbar}\bigg\langle -\frac{i\hbar}{m}p^2 + i\hbar x \frac{dV}{dx}\bigg\rangle \\
    &= \bigg\langle \frac{p^2}{m} - x\frac{dV}{dx}\bigg\rangle
    \end{aligned}
\]

\begin{itemize}
\item
  In equilibrium, i.e.~for a stationary state, expectation values do not
  change with time
  \begin{itemize}
  \item
    the time derivative is zero
  \end{itemize}
\end{itemize}

\[
    \bigg\langle \frac{p^2}{m} \bigg\rangle = \bigg\langle x\frac{dV}{dx}\bigg\rangle \iff \langle T \rangle = \frac{1}{2}\bigg\langle x \frac{dV}{dx}\bigg\rangle
\]

\subsection{The Correspondence Principle}\label{the-correspondence-principle}

The correspondence principle states that the behaviour of systems
described by the theory of quantum mechanics reproduces classical
physics in the limit of large quantum numbers.

In other words, it says that for large orbits and for large energies,
quantum calculations must agree with classical calculations.

Classical quantities appear in quantum mechanics in the form of expected
values of observables, and as such the Ehrenfest theorems (which predict
the time evolution of the expected values) is an example of the
correspondence principle.
\end{document}
