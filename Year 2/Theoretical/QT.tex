\documentclass[a4paper,11pt,normalem]{article}
\usepackage{../../../LaTeX-Templates/Notes}

\titlecontents{section}
    [0pt]
    {}
    {Lecture \thecontentslabel\quad}
    {}
    {\dotfill\contentspage}
    \titleformat{\section}{\fontsize{12}{15}\normalfont}{\underline{\textbf{Lecture \thesection}}}{1em}{}

\rhead{}
\newcommand{\HRule}{\rule{\linewidth}{0.5mm}}

\begin{document}
{\centering
{\includegraphics[scale=0.5]{../../logo0.png}\hfill{\Large\bfseries Epiphany 2018}}\\[1.5cm]
{\LARGE\bfseries Theoretical Physics 3}\\[0.5cm]
\HRule \\[0.3cm]
{\huge\bfseries Quantum Theory 2}\\[0.1cm]
\HRule \\[1cm]}
\begin{center}
\begin{minipage}{0.4\textwidth}
    \begin{flushleft} \large
        \emph{Author:} \\ Matthew Rossetter
    \end{flushleft}
\end{minipage}~
\begin{minipage}{0.4\textwidth}
    \begin{flushright} \large
        \emph{Lecturer:} \\ Prof. Robert Potvliege
    \end{flushright}
\end{minipage}
\end{center}

\section{}\label{lecture-1}

\emph{Course notes and audio recordings of the lectures can be found on
DUO}

\section{}\label{lecture-2}

\subsection{Vector Spaces}\label{vector-spaces}

\subsubsection{Examples in Vector Spaces}\label{examples-in-vector-spaces}

\paragraph{A. Geometric vectors}\label{a.-geometric-vectors}

Summing vectors (\emph{only valid for addition of vectors}):

\begin{enumerate}
\item
  If \(\vec{v}_1\) and \(\vec{v}_2\) are vectors, then
  \(\vec{v}_1 + \vec{v}_2\) is also a vector
  \begin{itemize}
  \item
    The plane bounded by \(\vec{v}_1\) and \(\vec{v}_2\) is a closed
    vector space under vector addition.
  \end{itemize}
\item
  \[(\vec{v}_1 + \vec{v}_2) + \vec{v}_3 = \vec{v}_1 + (\vec{v}_2 + \vec{v}_3)\]
\item
  There is a zero vector \(\vec{0}\) (vector of zero length) such that
  \(\vec{v} + \vec{0} = \vec{v}\).
\item
  Each vector has an inverse \(-\vec{v}\) such that
  \(\vec{v} + (-\vec{v}) = \vec{0}\).
\item
  \[\vec{v}_1 + \vec{v}_2 = \vec{v}_2 + \vec{v}_1\]
\item
  \(\alpha\vec{v}\) is the vector whose length is \(\alpha\) times
  \(|\vec{v}|\) in the same direction as \(\vec{v}\) for any real
  \(\alpha\). This is scalar multiplication.
\item
  \[\begin{aligned}(\alpha_1 + \alpha_2)\vec{v} = \alpha_1 \vec{v} + \alpha_2\vec{v} \\ \alpha(\vec{v}_1 + \vec{v}_2) = \alpha\vec{v}_1 + \alpha\vec{v}_2\end{aligned}\]
\item
  \[(\alpha\beta)\vec{v} = \alpha(\beta\vec{v})\]
\item
  \[1\cdot\vec{v} = \vec{v}\]
\item
  Dot product:
  \[\vec{v}_1 \cdot \vec{v}_2 = |\vec{v}_1||\vec{v}_2|\cos\theta_{12}\]
\item
  \[\vec{v}_1\cdot\vec{v}_2 = (\vec{v}_2\cdot\vec{v}_1)^* \]
\item
  Linear combinations:
  \[(\alpha\vec{v}_1 + \beta\vec{v}_2)\cdot\vec{w} = \alpha^{*}(\vec{v}_1\cdot\vec{w}) + \beta^{*}(\vec{v}_2\cdot\vec{w})\]
\item
  \[\vec{v}\cdot\vec{v} = |\vec{v}|^2 \geq 0\]
\end{enumerate}

These are the axioms of the inner product.
A vector space with inner product \(\equiv\) an inner product space

\paragraph{B. 2-component complex column vectors}\label{b.-2-component-complex-column-vectors}

\[
    V = \begin{pmatrix} a \\ b \end{pmatrix}
\]
where \(a\) and \(b\) are complex numbers

\begin{enumerate}
\item
  Addition of two vectors: \[
  \begin{aligned}
  V = \begin{pmatrix} a \\ b \end{pmatrix} ~&;~ W = \begin{pmatrix} a' \\ b' \end{pmatrix} \\
  V + W &= \begin{pmatrix} a + a' \\ b + b' \end{pmatrix}
  \end{aligned}
  \]
\item
  \[(v_1 + v_2) + v_3 = v_1 + (v_2 + v_3)\]
\item
  \[\begin{pmatrix} a \\ b\end{pmatrix} + \begin{pmatrix} 0 \\ 0 \end{pmatrix} = \begin{pmatrix} a \\ b \end{pmatrix}\]
\item
  \[\begin{pmatrix} a \\ b\end{pmatrix} + \begin{pmatrix} -a \\ -b \end{pmatrix} = \begin{pmatrix} 0 \\ 0 \end{pmatrix}\]
\item
  \[\alpha\begin{pmatrix} a \\ b\end{pmatrix} = \begin{pmatrix} \alpha a \\ \alpha b \end{pmatrix}\]
\item
  Inner product of \(v\),\(w\) is: \[
  (v,w) = \begin{pmatrix} a^* & b^* \end{pmatrix}\begin{pmatrix} a' \\ b' \end{pmatrix} = a^* a' + b^* b'
  \]
\end{enumerate}

\paragraph{C. Functions of x}\label{c.-functions-of-x}

\[
    f(x), \psi(x)
\]
These functions form a vector space.

\begin{enumerate}
\item
  \[(f + g)(x) = f(x) + g(x)\]
\item
  \[(\alpha f)(x) = \alpha f(x)\]
\item
  Inner product: \[
  (f, g) = \int_{-\infty}^{\infty} f^* (x)g(x)\,dx
  \]
\end{enumerate}

\subsection{Norm of a vector}\label{norm-of-a-vector}

The norm of a vector is defined as:

\[
    ||v|| = \sqrt{(v,v)}
\]

\begin{itemize}
\item
  Two vectors are said to be orthogonal if \((v,w) = 0\)
  \begin{itemize}
  \item
    orthonormal if there are orthogonal and have a unit norm
    \((||v|| = ||w|| = 1)\)
  \end{itemize}
\end{itemize}

\section{}\label{lecture-3}

\subsection{Hilbert Spaces}\label{hilbert-spaces}

Wave function of a harmonic oscillator:
\[
    \int_{-\infty}^\infty |\psi(x)|^2dx = 1
\]
Wave function of atomic hydrogen:
\[
    \int_{-\infty}^\infty r^2dr \int_{0}^\pi \sin\theta\,d\theta \int_0^{2\pi} d\phi \; |\psi(r,\theta,\phi)|^2 = 1
\]

\begin{itemize}
\item
  Wave functions must be square-integrable
\item
  The set of all functions forms a vector space
  \begin{itemize}
  \item
    The set of all square-integrable functions also forms a vector
    space, a subset of the above space (a subspace)
  \item
    A subspace is a vector space which is a subset of another vector
    space
  \end{itemize}
\item
  A square-integrable function refers to using the Leberque integration
\end{itemize}
Hilbert space: a complete vector space with an inner product, e.g.~the
vector space of square-integrable functions on \((-\infty, \infty)\).
The inner product is:
\[
    (\phi, \psi) = \int_{-\infty}^\infty \phi^* (x)\psi(x)dx
\]

\subsection{Bases}\label{bases}

\begin{enumerate}
\item
  Span of a set of vectors: the set of all linear combinations of these
  vectors, e.g. the span of
\[
    \left\{ \begin{pmatrix} 1 \\ 0 \\ 0 \end{pmatrix}, \begin{pmatrix} 0 \\ 1 \\ 0 \end{pmatrix}, \begin{pmatrix} 0 \\ 0 \\ 1 \end{pmatrix} \right\}
\]
is the set of linear combinations,
\[
    a\begin{pmatrix} 1 \\ 0 \\ 0 \end{pmatrix} + b\begin{pmatrix} 0 \\ 1 \\ 0 \end{pmatrix} + c\begin{pmatrix} 0 \\ 0 \\ 1 \end{pmatrix} = \begin{pmatrix} a \\ b \\ c \end{pmatrix}
\]
The span of those three vectors is the set of all 3-component column
vectors, were \(a, b, c \in \mathbb{C}\)
\item
  A set of N vectors is said to be linearly independent if it is not
  possible to write a vector from that set as a linear combination of
  the other vectors.
\end{enumerate}
\[
    \left\{ \begin{pmatrix} 1 \\ 0 \\ 0 \end{pmatrix}, \begin{pmatrix} 0 \\ 1 \\ 0 \end{pmatrix}, \begin{pmatrix} 0 \\ 0 \\ 1 \end{pmatrix} \right\}
\]
is a linearly independent set since it is not possible to find
\(\alpha\) and \(\beta\) such that
\[
    \begin{pmatrix} 0 \\ 0 \\ 1 \end{pmatrix} = \alpha\begin{pmatrix} 1 \\ 0 \\ 0 \end{pmatrix} + \beta\begin{pmatrix} 0 \\ 1 \\ 0 \end{pmatrix}
\]
Orthogonal vectors are always linearly independent.
The dimension of a finite-dimensional vector space is the max number of
vectors forming a linearly-independent set.
An infinite-dimensional vector space is one in which there is no upper
bound on the size of the linearly-independent sets.

\begin{example}[Functions of the form \(e^{inx}, n \in \mathbb{N}\)]
These functions form a linearly-independent set since any two such
functions are orthogonal.
\[
    \int_0^{2\pi} \left(e^{inx}\right)^* e^{imx} dx = 0, n \neq m
\]
\end{example}

\begin{enumerate}
\setcounter{enumi}{2}
\item
  A basis is a set of linearly-independent vectors spanning the whole
  vector space.
  An orthonormal basis is a basis whose vectors are orthonormal.
\end{enumerate}

\section{}\label{lecture-4}

\subsection{Operators I}\label{operators-i}

Examples:
\begin{enumerate}
    \item energy operator \(\to H\) 
    \item angular momentum operator \(\to \vec{L} = L_x\hat{x} + L_y\hat{y} + L_z\hat{z}\) 
    \item linear momentum operator \(\to \vec{p} = -i\hbar\vec{\nabla},~ p_x = -i\hbar\frac{\partial}{\partial x} = -i\hbar\frac{d}{dx}\)
    \item position operator \(\to x\) (in 1D)
\end{enumerate}

\begin{itemize}
\item
  operators deal with \emph{dynamical variables}
\item
  they transform wavefunctions:
\end{itemize}

\[
    p_x e^{-\frac{x^2}{a^2}} = -i\hbar\frac{d}{dx}e^{-\frac{x^2}{a^2}} = 2i\hbar\frac{x}{a^2}e^{-\frac{x^2}{a^2}}
\]

\begin{itemize}
\item
  linear operators are ones that act linearly:
  \(A(c_1v_1 + c_2v_2) = c_1Av_1 + c_2Av_2\)
\item
  non-linear operators do exist:
\end{itemize}

\[
    \begin{aligned}
    Av &= v||v|| \\
    Acv &= cv||cv|| = c|c|v||v|| \\
    &= c|c|Av \neq cAv
    \end{aligned}
\]

\begin{itemize}
\item
  many operators are \emph{unbounded}
\item
  identity operator, \(I\) such that \(Iv = v\)
\end{itemize}

\subsection{Using Linear Operators}\label{using-linear-operators}

\begin{enumerate}
\item
  adding operators:
\[
    (A + B)v = Av + Bv
\]
\item
  multiplying an operator by a scalar:
\[
    (cA)v = A(cv)
\]
\item
  product of two operators, i.e.~act on v with B first then act on the
  result with A:
\[
    (AB)v = A(Bv), ~ [AB \neq BA]
\]
\item
  invertible operator, an operator which has an inverse: \(A^{-1}\)\\
  \(A^{-1}\) being such that \[AA^{-1} = A^{-1}A = I\]\\
  singular operators are defined as non-invertible operators \[
  \begin{aligned}
  (AB)^{-1} &= B^{-1}A^{-1} \\
  (A^{-1})^{-1} &= A
  \end{aligned}
  \]
\item
  any operator \(A\) has a unique adjoint, \(A^\dagger\)\\
  \(A^\dagger\) is the operator such that for any \(v,w\) \[
  \begin{aligned}
  (v,Aw) &= (w,A^\dagger v)^* \\
  (AB)^\dagger &= B^\dagger A^\dagger \\
  (A^\dagger)^\dagger &= A \\
  (A + B)^\dagger &= A^\dagger + B^\dagger \\
  (cA)^\dagger &= c^* A^\dagger
  \end{aligned}
  \]
\end{enumerate}

\subsection{Representation by a matrix}\label{representation-by-a-matrix}
For an orthonormal basis: \(\{u_1,u_2,\cdots,u_n\}\) \[
    \begin{aligned}
    (u_i,u_j) &= \delta_{ij} = \begin{cases} 1 & i = j \\ 0 & i \neq j \end{cases} \\
    v&= c_1u_1 + c_2u_2 + \cdots + c_nu_n \\
    w &= Av \\
    w &= d_1u_1 + \cdots + d_nu_n \\
    \vec{c} &= \begin{pmatrix} c_1 \\ c_2 \\ \vdots \\ c_n \end{pmatrix} ~;~ \vec{d} = \begin{pmatrix} d_1 \\ d_2 \\ \vdots \\ d_n \end{pmatrix} \\
    \vec{d} &= \hat{A}\vec{c} \\
    \hat{A} &= \begin{pmatrix} A_{11} & A_{12} & \cdots & A_{1n} \\ \vdots & & & \\ A_{n1} & A_{n2} & \cdots & A_{nn} \end{pmatrix} \\
    A_{ij} &= (u_i, Au_j)
    \end{aligned}
\] 
This matrix represents the operator \(A\) in the basis \(\{u_1,u_2,\cdots,u_n\}\)

Example:
\[\left\{\frac{1}{\sqrt{2}}, \sqrt{\frac{3}{2}}x\right\}\] is an
orthonormal basis in the space of all functions of the form
\(f(x) = a_0 + a_1x\)

\[
    \begin{aligned}
    (u_i,u_j) &= \delta_{ij} \\
    \int_{-1}^1 u_{i}^{*}(x)u_j(x)&\,dx = \delta_{ij}*
    \end{aligned}
\]

\section{}\label{lecture-5}

\begin{itemize}
\item
  \textbf{Note:} \emph{order of presenting the basis matters, flipping
  the order of a 2 base basis transverses the matrix}
\item
  For a function, \(f = a + bx = c_1u_1(x) + c_2u_2(x)\), calculate the
  constants using the inner product
\item
  One says that the vector space spanned by \(u_1(x)\) and \(u_2(x)\) is
  isomorphic to the vector space of 2-component column vectors
\end{itemize}

\subsection{Dirac Notation}\label{dirac-notation}

\[
    \begin{aligned}
    u_1(x) &= \begin{pmatrix} 1 \\ 0 \end{pmatrix} = |1\rangle \\
    u_2(x) &= \begin{pmatrix} 0 \\ 1 \end{pmatrix} = |2\rangle \\
    f &= a + bx = \begin{pmatrix} a\sqrt{2} \\ b\sqrt{\frac{2}{3}} \end{pmatrix} = |f\rangle
    \end{aligned}
\]

\begin{itemize}
\item
  denote inner product of g and f as \((g,f) = \langle g | f \rangle\)
\end{itemize}

\[
    \begin{aligned}
    \frac{d}{dx}f &= Df = \hat{D}|f\rangle \\
    \left(g,\frac{df}{dx}\right) &= \langle g | \hat{D} | f \rangle
    \end{aligned}
\]

\begin{itemize}
\item
  The inner product of \(c|g\rangle\) and \(|f\rangle\) is
  \(c^* \langle g|f\rangle\)
\item
  Ket vectors are vectors in their own right, forming a Hilbert space
\end{itemize}

\subsection{Dual Space}\label{dual-space}

\begin{itemize}
\item
  Each state of a quantum system can be described by a vector belonging
  to a Hilbert space
\end{itemize}

\section{}\label{lecture-6}

\subsection{Degenerate Eigenvalues of an Operator}\label{degenerate-eigenvalues-of-an-operator}

\[
    \begin{aligned}
    \hat{A}|\psi\rangle &= \lambda|\psi\rangle \\
    c|\psi\rangle &= |c\psi\rangle \\
    \hat{A}|c\psi\rangle &= \hat{A}c|\psi\rangle \\
    &= c\hat{A}|\psi\rangle \\
    &= c\lambda|\psi\rangle \\
    &= \lambda c|\psi\rangle = \lambda|c\psi\rangle
    \end{aligned}
\]

\begin{itemize}
\item
  \(\lambda\) always corresponds to infinitely many different
  eigenvectors
\item
  It happens that:
\end{itemize}

\[
    \begin{aligned}
    \hat{A}|\psi_1\rangle &= \lambda_1|\psi_1\rangle \\
    \hat{A}|\psi_2\rangle &= \lambda_1|\psi_2\rangle \\
    |\psi_2\rangle &\neq |\psi_1\rangle
    \end{aligned}
\]

\begin{itemize}
\item
  i.e., \(|\psi_1\rangle\) and \(|\psi_2\rangle\) are linearly
  independent, but correspond to the same eigenvalues
  \begin{itemize}
  \item
    If so, \(\lambda\) is said to be degenerate
  \item
    e.g. for hydrogen, the \(2s\), \(2p_{m=0}\), and \(2p_{m=\pm1}\)
    states all have the same energy, \(E_2\)
  \end{itemize}
\item
  These states are orthogonal, and hence, linearly independent:
\[
    \begin{aligned}
    \int \psi_{nlm}^* (r,\phi,\theta)\,\psi_{n'l'm'}(r,\phi,\theta)\,d'r = 0
    \end{aligned}
\]
\emph{unless \(n = n'\), \(l = l'\), and \(m = m'\)}
\item
  The \(E_2\) eigenvalues of hydrogen are degenerate
\item
  The span of all the eigenvectors belonging to a degenerate eigenvalue
  is a vector space.
\item
  The degree of degeneracy of that eigenvalue is the dimension of that
  space.
  \begin{itemize}
  \item
    e.g. the degree of degeneracy of \(E_2\) is 4 - ``\(E_2\) is 4-fold
    degenerate''
  \end{itemize}
\item
  If an operator \(\hat{A}\) is represented by a matrix,
  \(\underline{\underline{A}}\), then the eigenvalues of \(\hat{A}\) are
  the same as those of \(\underline{\underline{A}}\)
  \begin{itemize}
  \item
    The eigenvectors of \(\hat{A}\) are \(\iff\) in correspondence with
    those of the matrix
  \end{itemize}
\item
  Spectrum of an operator: The set of all its eigenvalues (physicist's
  definition)
  \begin{itemize}
  \item
    \(\hat{A} - \lambda\hat{I}\)
  \item
    \(\hat{A}|\psi\rangle = \lambda|\psi\rangle\)
  \end{itemize}
\item
  Momentum operator: \(p = -i\hbar\frac{d}{dx}\)
\[
    \begin{aligned}
    p\psi(x) &= \lambda\psi(x) \\
    -i\hbar\frac{d\psi}{dx} &= \lambda\psi(x) \\
    \psi(x) &= Ce^{i\frac{\lambda}{\hbar}x} \\
    \lambda = a + ib \implies e^{i\frac{\lambda}{\hbar}x} &= e^{\frac{1}{\hbar}(ai - b)x}
    \end{aligned}
\]
\emph{for any constant C}
\[
    \begin{aligned}
    e^{-bx} \to \begin{cases} 0 & fn \to \infty \\ \infty & fn \to -\infty \end{cases}
    \end{aligned}
\]
\emph{for positive b}
\item
  \(\psi(x)\) is not square integrable if \(b \neq 0\)
\item
  If \(b = 0\), then \(e^{i\frac{a}{\hbar}x}\) remains of modulus 1, but
\[
    \int_{-\infty}^{\infty} |\psi(x)|^2dx = \int^{-\infty}_{\infty} |C|^2dx
\]
\emph{this diverges}
\item
  None of these eigenfunctions are square-integrable
\item
  \(p\) has no eigenfunctions in the Hilbert space of square-integrable
  functions
\item
  In physics, functions like \(e^{\pm ikx}\) where k is real, are also
  ``eigenfunctions'' (i.e. pseudo-eigenfunctions or generalised
  eigenfunctions)
\end{itemize}

\subsection{Dynamical Variables and Operators}\label{dynamical-variables-and-operators}

\begin{itemize}
\item
  Each state of a quantum system can be represented by a vector
  belonging to a Hilbert space, \(\mathcal{H}\)
\item
  With every dynamical variable is associated a linear operator acting
  in \(\mathcal{H}\)
  \begin{itemize}
  \item
    e.g. position, momentum, angular momentum, spin, energy
  \item
    i.e. physical quantities that may vary in time
  \end{itemize}
\item
  quantities that are constant in time are not dynamical variables
  \begin{itemize}
  \item
    e.g. the charge of the electron, etc
  \item
    therefore, they do not correspond to an operator in quantum
    mechanics
  \end{itemize}
\item
  The only values a dynamical variable can be found to have in a
  measurement are the eigenvalues of the operator associated with that
  variable
\end{itemize}

Suppose that \(|\psi\rangle\) represents a state of a quantum system,
and \(\hat{A}\) represents a dynamical variable:
\[
    \hat{A}|\psi_n\rangle = \lambda_n|\psi_n\rangle
\]
then the probability to find the result \(\lambda_n\) in an experiment is
\[
    P(\lambda_n) = \frac{|\langle\psi_n|\psi|\rangle|^2}{\langle\psi_n|\psi_n\rangle\langle\psi|\psi\rangle}
\]
Usually one takes
\[
    \begin{aligned}
    \langle{\psi|\psi\rangle} &= 1 \\
    \& \; \langle\psi_n|\psi_n\rangle &= 1 \\
    \implies P(\lambda_n) &= |\langle\psi_n|\psi\rangle|^2
    \end{aligned}
\]

\section{}\label{lecture-7}

\begin{enumerate}
\item
  Experiment
  \begin{itemize}
  \item
    System is prepared in a certain state
  \item
    measurement
  \item
    results
  \end{itemize}
\item
  Theory
  \begin{itemize}
  \item
    state of system is represented by a state vector, \(|\psi\rangle\)
  \item
    we have a theoretical description in which what is measured is described in
    terms of operators associated to dynamical variables
  \item
    probabilistic ``prediction''
  \end{itemize}
\end{enumerate}

\subsection{Consequences of the Probability Rule}\label{consequences-of-the-probability-rule}

\begin{itemize}
\item
  All the predictions of the theory are based on the state vector,
  \(|\psi\rangle\), representing the system
\item
  All one can say about the state of a quantum system is what can be
  deduced from the state vector
\item
  the state vector constrains all the information that can be known about
  the system
\item
  \(|\phi_n\rangle\) is an eigenvector
  \(\to \langle \phi_n|\phi_n\rangle \neq 0\)
\item
  the zero vector never represents a quantum state
  \(\to \langle\psi|\psi\rangle \neq 0\)
\item
  if the probability of a result, \(\lambda\), is zero, then finding
  this result is impossible (within the theoretical model used)
  \begin{itemize}
  \item
    if the probability is one, then the result will be obtained with
    certainty
  \end{itemize}
\end{itemize}

\subsection{The Principle of Superposition}\label{the-principle-of-superposition}

\begin{itemize}
\item
  if \(|\psi_1\rangle\) and \(|\psi_2\rangle\) represents a possible
  state of a system, then any linear combination of \(|\psi_1\rangle\)
  and \(|\psi_2\rangle\) also represents a possible state of the system
  \[
  \begin{aligned}
  \Psi_{100}(\vec{r},t) &= \psi_{100}(\vec{r}\exp\left[-i\left(\frac{E_1t}{\hbar}\right)\right] \\
  \Psi_{200}(\vec{r},t) &= \psi_{200}(\vec{r}\exp\left[-i\left(\frac{E_2t}{\hbar}\right)\right] \\
  \Psi(\vec{r},t) &= c_1\Psi_{100} + c_2\Psi_{200} \text{ is also a possible state}
  \end{aligned}
  \]
If \(\langle\phi_n|\phi_n\rangle = 1\), then \[
    P(\lambda_n) = \frac{|\langle|\phi_n|\psi\rangle|^2}{\langle\psi|\psi\rangle}
\]
\item
  multiplying the state vector by a non-zero complex number gives the
  same probability
\item
  the ket vectors \(c|\psi\rangle, c \in \mathbb{C}\) all represent the
  same state, regardless of the value of \(c\)
\item
  however, a linear combination of state vectors will be different
  dependent on the value of \(c\) for each state vector
\end{itemize}

\subsection{Hermitian Operators}\label{hermitian-operators}

Definition: an operator, \(\hat{A}\), is Hermitian if \[
    \langle\phi|\hat{A}|\psi\rangle = \langle\psi|\hat{A}|\phi\rangle^*
\] for any \(|\psi\rangle\), \(|\phi\rangle\)

\begin{itemize}
\item
  the eigenvalues of Hermitian operators are always real
\item
  the eigenvectors of Hermitian operators corresponding to different
  eigenvalues are orthogonal
\item
  matrices representing Hermitian operators are always Hermitian,
  i.e. equal to their conjugate transpose
\end{itemize}

\section{}\label{lecture-8}

\begin{itemize}
\item
  An operator \(\hat{A}\) is said to be Hermitian if
  \(\langle\phi|\hat{A}|\psi\rangle = \langle\psi|\hat{A}|\phi\rangle^{*}\)
  for any \(|\psi\rangle\), \(|\phi\rangle\) on which \(\hat{A}\) may
  act.
\end{itemize}

\subsection{Proof of the Orthogonality of Eigenvectors}\label{proof-of-the-orthogonality-of-eigenvectors}

\begin{itemize}
\item
  \(\hat{A}\): Hermitian such that
  \begin{itemize}
  \item
    \(\hat{A}|\psi_1\rangle = \lambda_1|\psi_1\rangle\)
  \item
    \(\hat{A}|\psi_2\rangle = \lambda_2|\psi_2\rangle\)
  \item
    \(\lambda_1 \neq \lambda_2\)
  \item
    both \(\lambda_1\) and \(\lambda_2\) are real since \(\hat{A}\) is
    Hermitian
\[
    \begin{aligned}
    \langle\psi_1|\hat{A}|\psi_2\rangle &= \lambda_2\langle\psi_1|\psi_2\rangle \\
    \langle\psi_2|\hat{A}|\psi_1\rangle * &= \lambda_2^* \langle\psi_2|\psi_1\rangle * \\
    \langle\psi_2|\hat{A}|\psi_1\rangle &= \lambda_2\langle\psi_2|\psi_1\rangle \\
    \langle\psi_2|\hat{A}|\psi_1\rangle &= \lambda_1\langle\psi_2|\psi_1\rangle \\
    0 &= \underbrace{(\lambda_1 - \lambda_2)}_{\neq 0}\underbrace{\langle\psi_2|\psi_1\rangle}_{=0}
    \end{aligned}
\]
  \end{itemize}
\item
  If \(\hat{A}\) is a Hermitian operator acting in a finite-dimensional
  Hilbert space, then it is always possible to form an orthonormal basis
  of eigenvectors of \(\hat{A}\) and this basis is complete.
\item
  A complete set of vectors is a set of vectors spanning the whole
  space.
  \begin{itemize}
  \item
    A basis is always a complete set, by definition.
  \end{itemize}
\end{itemize}

\begin{example}[1st Workshop]
\[
    \begin{aligned}
    \begin{pmatrix} 1 & 0 \\ 0 & -1 \end{pmatrix} \\
    \begin{pmatrix} 1 & 1 \\ 0 & 1 \end{pmatrix}
    \end{aligned}
\]
\begin{itemize}
\item
  The first matrix above is Hermitian, and the eigenvectors from a
  complete set.
\item
  The second matrix above is not Hermitian, and the eigenvectors do not
  form a complete set.
\end{itemize}
For infinite-dimensional spaces, there are different possibilities: 1.
Infinite square well: \[
    H = -\frac{\hbar^2}{2m}\frac{d^2}{dx^2}
\] This acts on \([-a,a]\) such that \(\psi(x=\pm a) = 0\) * There are
infinitely many eigenvalues (eigenenergies) for this 2. Free particle:
Same operator as above on \((-\infty, +\infty)\), acting on a
square-integrable function in that bound \[
    -\frac{\hbar^2}{2m}\frac{d^2}{dx^2}\psi = E\psi
\]
\begin{itemize}
\item
  This has no solution that is square-integrable
\item
  SHM \[
  \begin{aligned}
  H &= -\frac{\hbar^2}{2m}\frac{d^2}{dx^2} + \frac{1}{2}m\omega^2x^2,~ (-\infty,+\infty) \\
  H\psi_n &= E\psi_n \\
  E_n &= \hbar\omega\left(n + \frac{1}{2}\right) \\
  \psi(x) &= \sum_n c_n\psi_n(x)
  \end{aligned}
  \]
\end{itemize}
\end{example}

\subsection{Probability of Obtaining an eigenvalue}\label{probability-of-obtaining-an-eigenvalue}

\[
    \begin{aligned}
    P_i &= |\langle\phi_i|\psi\rangle|^2 \iff \\
    \langle\phi_i|\phi_1\rangle &= 1 = \langle\psi|\psi\rangle \& \\
    \hat{A}|\phi_i\rangle &= \lambda_i|\phi_i\rangle
    \end{aligned}
\]
If \(\lambda_i\) is degenerate:
\[
    \begin{aligned}
    \hat{A}|\psi_n\rangle &= \underbrace{\lambda}_{\forall n}|\psi_n\rangle \\
    \langle\phi_i|\phi_j\rangle &= \delta_{ij}
    \end{aligned}
\]
Probability of finding \(\lambda\) is:
\[
    P(\lambda) = \sum_n |\langle\phi_n|\psi\rangle|^2
\]
\begin{itemize}
\item
  This is the sum over all the eigenvectors corresponding to \(\lambda\)
\item
  ``Observable'' - a Hermitian operator with a complete set of
  eigenvectors
\end{itemize}
\[
    \begin{aligned}
    P_i(|\psi\rangle) &= |\langle\phi_i|\psi\rangle|^2 \\
    P_i(|\phi_j\rangle) &= |\langle\phi_i|\phi_j\rangle|^2 = \begin{cases} 1 & i = j \\ 0 & i \neq j \end{cases}
    \end{aligned}
\]
\begin{itemize}
\item
  Finding \(\lambda_i\) or \(\lambda_j\) is mutually exclusive:
\end{itemize}
\[
    \begin{aligned}
    \sum_i P_i(|\psi\rangle) &= 1 \\
    \sum_i |\langle\phi_i|\psi\rangle|^2 &= 1 \\
    \sum_i \langle\phi_i|\psi\rangle * \langle\phi_i|\psi\rangle &= 1 \\
    \sum_i \langle\psi|\phi_i\rangle \langle\phi_i|\psi\rangle &= 1
    \end{aligned}
\]
\begin{itemize}
\item
  One must have this, or any \(|\psi\rangle\)
\end{itemize}
\[
    \sum_i |\phi_i\rangle\langle\phi_i| = \hat{I}
\]
\begin{itemize}
\item
  The is the completeness relation
\end{itemize}

\subsection{Variance of the distribution of probability}\label{variance-of-the-distribution-of-probability}
\[
    (\Delta A)^2 = \langle\psi|\hat{A}^2|\psi\rangle - \langle\psi|\hat{A}|\psi\rangle^2
\]

\section{}\label{lecture-9}

\[
    (\Delta A)^2(\Delta B)^2 \geq -\frac{1}{4}(\langle\psi|[\hat{A},\hat{B}]|\psi\rangle)^2
\]

\begin{itemize}
\item
  system is represented by \(|\psi\rangle, \langle\psi|\psi\rangle = 1\)
\item
  two dynamical variables, \(A\) and \(B\), represented by two
  observables, \(\hat{A}\) and \(\hat{B}\)
  \begin{itemize}
  \item
    these are Hermitian operators with a complete set of eigenvalues
  \end{itemize}
\end{itemize}
\[
    [\hat{A},\hat{B}] = \hat{A}\hat{B} - \hat{B}\hat{A}
\]
\begin{itemize}
\item
  the commutator of \(\hat{A}\) and \(\hat{B}\) if
\[
    [\hat{A},\hat{B}] = 0
\]
one would say that \(\hat{A}\) and \(\hat{B}\) commute, i.e.~for any
\(|\psi\rangle \to [\hat{A},\hat{B}]|\psi\rangle = 0\)
\[
    [\hat{Q},\hat{P}] = i\hbar\hat{I}
\]
\item
  \(\hat{I}\) is the identity vector and is usually not indicated for
  simplicity
  \begin{itemize}
  \item
    \([\hat{A},\hat{I}] = 0\)
  \end{itemize}
\item
  \([\hat{A},\hat{A}] = 0\)
\item
  \([\hat{A},\hat{B}] = -[\hat{B},\hat{A}]\)
\item
  \([\hat{A},f(\hat{A})] = 0\), where \(f(\hat{A})\) can be any function
  of \(\hat{A}\)
\item
  if \([\hat{A},\hat{B}] = 0\) and \(|\phi_n\rangle\) is an eigenvector
  of \(\hat{A}\), then \(\hat{B}|\phi_n\rangle\) is also an eigenvector
  of \(\hat{A}\) corresponding to the same eigenvalue.
\item
  Proof:
\[
    \begin{aligned}
    \hat{A}|\phi_n\rangle &= \lambda_n|\phi_n\rangle \\
    \hat{A}\hat{B}|\phi_n\rangle &= \hat{B}\hat{A}|\phi_n\rangle = \lambda_n\hat{B}|\phi_n\rangle
    \end{aligned}
\]
\item
  If \(\lambda_n\) is not a degenerate eigenvalue of \(\hat{A}\), then
  \(\hat{B}|\phi_n\rangle = \mu_n|\phi_n\rangle\)
  \begin{itemize}
  \item
    \(|\phi_n\rangle\) is also an eigenvector of \(\hat{B}\)
  \end{itemize}
\item
  Proof:
  \begin{itemize}
  \item
    If \(\lambda_n\) were degenerate, then (and only then) could one
    have several linearly independent eigenvectors of \(\hat{A}\) all
    corresponding to \(\lambda_n\)
  \item
    Since we assume that \(\lambda_n\) is not degenerate,
    \(\hat{B}|\phi_n\rangle\) and \(|\phi_n\rangle\) cannot be linearly
    independent, therefore
    \(\hat{B}|\phi_n\rangle = \mu_n|\phi_n\rangle\) for some non-zero
    value of \(\mu_n\)
  \item
    If \([\hat{A},\hat{B}] = 0\), then one can find a basis of the
    Hilbert space constructed from eigenvectors common to \(\hat{A}\)
    and \(\hat{B}\), and reciprocally
  \end{itemize}
\end{itemize}

\begin{example}
For atomic hydrogen, 
\begin{itemize}
    \item \(H\) - Hamiltonian 
    \item \(\vec{L}^2\) and \(L_z\)
    \item angular momentum operators
\[
    [H,\vec{L}^2] = [H,L_z] = [\vec{L}^2,L_z] = 0
\]
\end{itemize}
One can find functions that are eigenfunctions of all these three
operators:
\[
    \begin{aligned}
    \psi_{nlm}(r,\theta,\phi) \\
    H\psi_{nlm} &= E_n\psi_{nlm} \\
    \vec{L}^2\psi_{nlm} &= \hbar^2l(l+1)\psi_{nlm} \\
    L_z\psi_{nlm} &= \hbar m\psi_{nlm}
    \end{aligned}
\]
\begin{itemize}
\item
  \(H,\vec{L}^2,L_z\) from a ``complete set of commuting observables''
  in the sense that specifying their eigenvalues (e.g. by specifying the
  corresponding quantum numbers) define their common eigenvectors
  unambiguously
\end{itemize}
\[
    (\Delta A)^2(\Delta B)^2 \geq -\frac{1}{4}(\langle\psi|[\hat{A},\hat{B}]|\psi\rangle)^2
\]
\begin{itemize}
\item
  if \(\hat{A},\hat{B}\) are Hermitian, \([\hat{A},\hat{B}] = i\hat{C}\)
  where \(\hat{C}\) is Hermitian
\end{itemize}
\[
    \langle\psi|\hat{C}|\psi\rangle = \langle\psi|\hat{C}|\psi\rangle^{ * }
\]
\begin{itemize}
\item
  the right hand-side is greater than zero
\item
  \((\Delta A)^2\) is the variance of the probability distribution
  formed by the \(P(\lambda_n)\)
\end{itemize}
\[
    \begin{aligned}
    \hat{A}|\phi_n\rangle &= \lambda_n|\phi_n\rangle \\
    \langle\phi_n|\phi_n\rangle &= 1
    \end{aligned}
\]
Probability of finding \(\lambda_n\) in the measurement is
\[
    P(\lambda_n) = |\langle\phi_n|\psi\rangle|^2
\]
\begin{itemize}
\item
  inside is the probability amplitude for finding \(\lambda_n\)
\item
  See last lecture for generalisation to degenerate eigenvalues
\end{itemize}
\[
    \langle\psi|\hat{A}|\psi\rangle = \langle A\rangle
\]
This is the expectation value of \(\hat{A}\)
\[
    \sum_n \lambda_n P(\lambda_n)
\]
\begin{itemize}
\item
  If \(|\psi\rangle\) is such that
  \(\hat{A}|\psi\rangle = \lambda|\psi\rangle\), then
  \(\langle\psi|\hat{A}|\psi\rangle = \lambda\)
\end{itemize}
\[
    \begin{aligned}
    (\Delta A)^2 &= \langle\psi|(\hat{A} - \langle A\rangle\hat{I})^2|\psi\rangle \\
    &= \langle\psi|\hat{A}^2|\psi\rangle - \langle\psi|\hat{A}|\psi\rangle^2
    \end{aligned}
\]
\(\Delta A\) is the uncertainty on \(A\)
\begin{itemize}
\item
  If we perform a measurement and get \(\lambda^{(1)}\) then again and
  get \(\lambda^{(2)}\) etc, after preparing the system to be back in
  the unmeasured state
\end{itemize}
\[
    \begin{aligned}
    \bar{\lambda} &= \frac{1}{n} \sum_j \lambda^{(j)} \\
    (\Delta A)^2 &= \langle A^2\rangle - \langle A\rangle^2 \\
    (\Delta A)^2 &\implies \sigma^2 = \frac{1}{n-1} \sum_j (\lambda^{(j)} - \bar{\lambda})^2
    \end{aligned}
\]
\end{example}

\section{}\label{lecture-10}

\begin{itemize}
\item
  If \(\Delta A = 0\), there is no dispersion
\item
  \(\Delta A = 0\) if \(|\psi\rangle\) is an eigenvector of \(\hat{A}\)
\item
  \(\hat{A}|\psi\rangle = \lambda|\psi\rangle\)
\[
    \begin{aligned}
    \hat{A}^2|\psi\rangle &= \lambda^2|\psi\rangle = \hat{A}(\hat{A}|\psi\rangle) \\
    &= \hat{A}(\lambda|\psi\rangle) - \lambda\hat{A}|\psi\rangle = \lambda^2|\psi\rangle \\
    \langle\psi|\hat{A}^2|\psi\rangle - \langle\psi|\hat{A}|\psi\rangle^2 &= \lambda^2\langle\psi|\psi\rangle - (\lambda\langle\psi|\psi\rangle)^2 \\
    &= \lambda^2 = \bar{\lambda}^2 = 0
    \end{aligned}
\]
\item
  For finite dimensional spaces, if \(|\psi\rangle\) is an eigenvector
  of \(\hat{A}\), then \(\rangle\psi|[\hat{A},\hat{B}]|\psi\rangle = 0\)
  too
\[
    \begin{aligned}
    \langle\psi|\hat{A}\hat{B} - \hat{B}\hat{A}|\psi\rangle &= \lambda^* \langle\psi|\hat{B}|\psi\rangle - \lambda\langle\psi|\hat{B}|\psi\rangle \\
    &= (\lambda - \lambda)\langle\psi|\hat{B}|\psi\rangle = 0
    \end{aligned}
\]
\emph{complex conjugate goes away since \(\hat{A}\) is Hermitian}
\item
  If \([\hat{A},\hat{B}] = 0\), then it is possible for
  \((\Delta A)^2(\Delta B)^2 = 0\)
\item
  For \(\hat{P}\) as the momentum operator,
\[
    \begin{aligned}
    \hat{P}|\phi\rangle &= p|\phi\rangle \\
    -i\hbar \frac{d}{dx} \phi(x) &= p\phi(x) \\
    \phi_p(x) &= Ce^{i\frac{px}{\hbar}}
    \end{aligned}
\]
\emph{not square summable, therefore not an element of the Hilbert
space}
\item
  For \(\hat{Q}\) as the position operator,
\[
    Q\phi(x) = x\phi(x) = a\phi(x)
\]
\emph{impossible unless \(\phi(x) = 0\), which does not qualify as an
eigenfunction}
\item
  Take \(\phi_p(x)\) as generalised eigenfunction of the momentum
  operator
\end{itemize}

\subsection{Measurement of P}\label{measurement-of-p}

\begin{itemize}
\item
  What is the probability of finding a certain value, \(p\)?
\item
  \(p\) is distributed continuously, not quantised
\item
  Better to ask for the probability of finding \(p\) between \(p_1\) and
  \(p_2\)?
\[
    P[(p_1,p_2)] = \int_{p_1}^{p_2} P(p)\,dp
\]
\item
  \(P(p)\) is the density of probability, \(P(p)\,dp\) is the probability
  to find a momentum between \(p\) and \(p+dp\)
\item
  \(P(p)\) has no physical dimensions
  \begin{itemize}
  \item
    those of the inverse of a momentum, so that \(P[(p_1,p_2)]\) is a
    pure number
\[
    P(p) = \Bigg|\int_{-\infty}^{\infty} \phi_p^{* }(x) \phi(x)\,dx\Bigg|^2 = \Bigg|C\int_{-\infty}^{\infty} e^{-i\frac{px}{\hbar}}\psi(x)\,dx\Bigg|^2
\]
  \end{itemize}
\item
  This is the Fourier transform of \(\psi(x)\)
\end{itemize}
\[
    \begin{aligned}
    \phi(k) &= \frac{1}{\sqrt{2\pi}} \int_{-\infty}^{\infty} e^{i\frac{px}{\hbar}}\psi(x)\,dx \\
    \psi(x) &= \frac{1}{\sqrt{2\pi}} \int_{-\infty}^{\infty} e^{ikx}\phi(k)\,dk \\
    &= \frac{1}{2\pi}\int_{-\infty}^{\infty} e^{ikx}dk \int_{-\infty}^{\infty} e^{-ikx'}\psi(x')\,dx' \\
    &= \frac{1}{2\pi}\int_{-\infty}^{\infty} \psi(x)\,dx  \int_{-\infty}^{\infty} e^{ik(x-x')} dx
    \end{aligned}
\]

\section{}\label{lecture-11}

\begin{itemize}
\item
  Momentum operator: \(p = -i\hbar\frac{d}{dx}\)
\item
  Position operator: \(Q = x\)
\[
    \begin{aligned}
    P\phi_k(x) &= P\left[Ce^{ikx}\right] = \hbar k\phi_k(x) \\
    \phi(k) &= \frac{1}{\sqrt{2\pi}}\int_{-\infty}^{\infty} \psi(x)e^{-ikx}dx \\
    \psi(x) &= \frac{1}{\sqrt{2\pi}}\int_{-\infty}^{\infty} \phi(k)e^{ikx}dk \\
    \delta(x-x') &= \frac{1}{2\pi}\int_{-\infty}^{\infty}e^{ik(x-x')}dk
    f(x') &= \int_{-\infty}^{\infty} \delta(x-x')f(x)\,dx
    \end{aligned}
\]
\item
  This is true for any function \(f(x)\) that is continuous at \(x=x'\)
\[
    \begin{aligned}
    \delta(x-x') &= \delta(x'-x) \\
    \int_{-\infty}^{\infty} P(k)\,dk &= 1 \implies \int_{-\infty}^{\infty} |\psi(x)|^2dx = 1 \\
    P(k) &= |\phi(k)|^2|C|^2 \\
    \phi_k(x) &= Ce^{ikx} \\
    |C|^2\int_{-\infty}^{\infty} dk&\left[\int_{-\infty}^{\infty} \psi(x)e^{-ikx}dx\right]^{ * } \cdot \left[\int_{-\infty}^{\infty}\psi(x')e^{-ikx'}dx'\right] = 1 \\
    |C|^2\int_{-\infty}^{\infty} dx&\int_{-\infty}^{\infty} \psi^* (x)\psi(x') dx' \cdot \int_{-\infty}^{\infty} e^{ik(x-x')}dk = 1 \\
    2\pi|C|^2 \int_{-\infty}^{\infty} dx& \int_{-\infty}^{\infty} \psi^* (x)\psi(x')\delta(x-x')\,dx' = 1 \\
    2\pi|C|^2 \int_{-\infty}^{\infty} dx& \psi^* (x)\psi(x) = 1 \\
    \implies 2\pi|C|^2 &= 1 \to C = \frac{1}{\sqrt{2\pi}}
    \end{aligned}
\]
\item The normalised eigenfunctions of \(P\) are:
\[
    \begin{aligned}
    \phi_k(x) &= \frac{1}{\sqrt{2\pi}}e^{ikx} \\
    \phi_p(x) &= \frac{1}{\sqrt{2\pi\hbar}}e^{ip\frac{x}{\hbar}}
    \end{aligned}
\]
\item
  Orthonormality condition here is
\end{itemize}
\[
    \begin{aligned}
    \int_{-\infty}^{\infty} \phi_k^* (x)\phi_{k'}(x)\,dx &= \frac{1}{2\pi}\int_{-\infty}^{\infty} e^{i(k-k')x}dx = \delta(k'-k) \\
    \int_{-\infty}^{\infty} \phi_n(x)\phi_{n'}(x)\,dx &= \delta_{nn'}
    \end{aligned}
\]

\subsection{Eigenfunctions of the position operator}\label{eigenfunctions-of-the-position-operator}

\[
    Q\psi(x) = x\psi(x)
\]
An eigenfunction of \(Q\) would be such that
\[
    Q\phi_q(x) \equiv q\phi_k(x) \equiv x\phi_k(x)
\]
Finally, one can take:
\[
    \begin{aligned}
    \phi_k(x) &= \delta(x-q) \\
    P[(q_1,q_2)] &= \int_{q_1}^{q_2} P(q)\,dq \\
    P(q) &= \Bigg|\int_{-\infty}^{\infty} \phi_q^* (x)\psi(x)\,dx\Bigg|^2 \\
    &= \Bigg|\int_{-\infty}^{\infty} \delta(q-x)\psi(x)\,dx\Bigg|^2 \\
    &= |\psi(q)|^2
    \end{aligned}
\]
This is the Born Rule

\begin{itemize}
\item
  Normalisation:
\[
    \int_{-\infty}^{\infty} \delta^* (x-q)\delta(x-q')\,dx = \delta(q-q')
\]
\item Discrete case: \(|\psi\rangle = \sum_n c_n|\phi_n\rangle\) if
\(\{|\phi_n\rangle\}\) is an orthonormal basis
\[
    \begin{aligned}
    c_n &= \langle\phi_n|\psi\rangle \\
    \psi(x) &= \int_{-\infty}^{\infty} \phi(p)\phi_p(x)\,dp,~ \phi(p) = \langle p|\psi\rangle \\
    \hat{Q}|x\rangle &= x|x\rangle \\
    \hat{p}|p\rangle &= p|p\rangle \\
    \psi(x) &= \langle x|\psi\rangle
    \end{aligned}
\]
\item
  \(\psi(x) = \langle x|\psi\rangle\) - wave function in position
  representation in position space
\item
  \(\phi(p) = \langle p|\psi\rangle\) - wave function in the momentum
  representation in momentum space
\end{itemize}
The last two statements are equivalent
\[
    \begin{aligned}
    |\psi\rangle &\leftrightarrow \psi(x) \\
    |x\rangle &\leftrightarrow \begin{pmatrix} a \\ b \end{pmatrix} \\
    |\psi\rangle &\leftrightarrow \phi(p) \\
    \langle x|p\rangle &= \frac{1}{\sqrt{2\pi\hbar}}e^{ip\frac{x}{\hbar}} \\
    \langle p|x\rangle &= \frac{1}{\sqrt{2\pi\hbar}}e^{-ix\frac{p}{\hbar}} \\
    \hat{Q} &\leftrightarrow x \\
    \hat{p} &\leftrightarrow -i\hbar\frac{d}{dx} \\
    \hat{p} &\leftrightarrow p \\
    \hat{Q} &\leftrightarrow -i\hbar\frac{d}{dp}
    \end{aligned}
\]
In 3D position representation:
\[
    \begin{aligned}
    P_x &= -i\hbar\frac{\partial}{\partial x} \\
    P_y &= -i\hbar\frac{\partial}{\partial y} \\
    P_z &= -i\hbar\frac{\partial}{\partial z} \\
    [x,P_x] &= [y,P_y] = [z,P_z] = i\hbar \\
    [x,y] &= [x,z] = [y,z] = 0 \\
    [x,P_y] &= [x,P_z] = \cdots = 0 \\
    [P_x,P_y] &= [P_x,P_y] = 0 \\
    [x,P_y]\psi(x,y,z) &= -i\hbar\left[x\frac{\partial}{\partial y}\psi - \frac{\partial}{\partial y}x\psi\right] = 0 \\
    \vec{P} &= P_x\hat{x} + P_y\hat{y} + P_z\hat{z} \\
    \vec{P}\phi_{\vec{p}}(\vec{r}) &= \vec{P}\phi_{\vec{p}}(\vec{r}) \\
    \vec{p} &= \hbar\vec{k} \\
    \phi_{\vec{p}}(\vec{r}) &= \frac{1}{\sqrt{2\pi\hbar}}e^{i\vec{p}\cdot\frac{\vec{r}}{\hbar}} \\
    \phi_{\vec{k}}(\vec{r}) &= \frac{1}{\sqrt{2\pi}}e^{i\hbar\vec{r}} \\
    \int \phi_k^* (\vec{r})\phi_{k'}(\vec{r})\,d^3r &= \delta^3(\vec{k} - \vec{k}') = \delta(k_x - k_{x}')\delta(k_y - k_y')\delta(k_z - k_z')
    \end{aligned}
\]

\section{}\label{lecture-12}

\begin{itemize}
\item
  Infinite square well:
  \begin{itemize}
  \item
    The Hamiltonian has infinite many discrete energy levels
  \end{itemize}
\item
  Linear harmonic oscillator:
  \begin{itemize}
  \item
    Also has infinite many discrete energy levels
  \end{itemize}
\item
  Free particle in 1D:
  \begin{itemize}
  \item
    continuum of energy levels, \(0 < E < \infty\)
  \[
      H = -\frac{\hbar^2}{2m}\frac{d^2}{dx^2}
  \]
  \end{itemize}
\item
  atom of hydrogen
  \begin{itemize}
  \item
    infinitely many discrete energy levels, corresponding to bound
    states
  \item
    and a continuum of energy levels corresponding to unbound states
  \item
    \(-13.6\,eV = T + V\)
  \item
    \(r\) must be such that \(-13.6\,eV > V(r)\)
  \item
    an electron with positive energy is in an unbound state
  \end{itemize}
\item
  in general, we have two classes - discrete and bound
\end{itemize}

\begin{enumerate}
\item
  discrete energy levels:
\[
    \begin{aligned}
    H\phi_j &= E_j\phi_j \\
    \int \phi_i^* \phi_j d^3r &= \delta_{ij}
    \end{aligned}
\]
\item
  continuum of energy levels \[
  \begin{aligned}
  H\phi_{\vec{k}} &= E_{\vec{k}}\phi_{\vec{k}} \\
  \int \phi_{\vec{k}}^*(\vec{r})\phi_{\vec{k}'}\,d^3r &= \delta(\vec{k} - \vec{k}') \\
  \int \phi_i(\vec{r})\phi_{\vec{k}}(\vec{r})\,d^3r &= 0
  \end{aligned}
  \]
\end{enumerate}

\begin{itemize}
\item
  A complete set of eigenfunctions of \(H\) necessarily include a
  continuum eigenfunctions if \(H\) has a continuous spectrum:
\[
    \psi(\vec{r}) = \sum_j c_j\phi_j(\vec{r}) + \int c_{\vec{k}}\phi_{\vec{k}}(\vec{r})\,d^3k
\]
\item
  Since the \(\phi_j\) and \(\phi_{\vec{k_{}}}\) are orthonormal
\[
    \begin{aligned}
    c_j &= \int \phi_j^*(\vec{r}')\psi(\vec{r}')\,d^3r' \\
    c_{\vec{k}} &= \int \phi_{\vec{k}}^*(\vec{r})\psi(\vec{r}')\,d^3r' \\
    \psi(\vec{r}) &= \int d^3r' \underbrace{\left[\sum_j \phi_j(\vec{r})\phi_j^*(\vec{r}') + \int d^3k \phi_{\vec{k}}(\vec{r})\phi_{\vec{k}}(\vec{r}')\right]}_{=\delta(\vec{r}-\vec{r}')}\psi(\vec{r})
    \end{aligned}
\]
\item
  Must be true for any \(\vec{r}\), and any \(\psi\)
\item
  completeness relation from lecture 8
\item
  In Dirac notation:
\[
    \langle\vec{r}|\sum_j|\phi_j\rangle\langle\phi_j| + \int d^3k |\phi_{\vec{k}}\rangle\langle\phi_{\vec{k}}| = \hat{I}|\vec{r}'\rangle|\phi_j\rangle \\
    \langle\phi_j|\psi\rangle
\]
\item
  In position representation:
\[
    \begin{aligned}
    \langle\vec{r}|\phi_j\rangle &= \phi_j(\vec{r}) = \rangle\phi_j|\vec{r}\rangle^* \\
    \langle\phi_j|\vec{r}'\rangle &= \phi_j^*(\vec{r}') \\
    \langle\vec{r}|\hat{I}|\vec{r}'\rangle &= \delta(\vec{r}-\vec{r}')
    \end{aligned}
\]
\item
  About bra vectors
\[
    \begin{aligned}
    |A\psi\rangle &= \hat{A}|\psi\rangle \\
    \langle A\psi| &= \langle\psi|\hat{A}^\dagger,~ \langle A\psi|\phi\rangle = \langle\psi|\hat{A}^\dagger|\phi\rangle \\
    \langle A\psi|\phi\rangle &= \langle\phi|A\psi\rangle^* = \langle\phi|\hat{A}|\psi\rangle^* = \langle\psi|\hat{A}^\dagger|\phi\rangle
    \end{aligned}
\]
\end{itemize}

\subsection{Unitary Transformations}\label{unitary-transformations}

\begin{itemize}
\item
  2 orientations for \(2p_m = 0\)
\item
  Relate the two by:
\[
    \begin{aligned}
    |\psi'\rangle &= \hat{R}_x(\theta)|\psi\rangle \\
    |\phi'\rangle &= \hat{R}_x(\theta)|\phi\rangle \\
    \langle\phi'|\psi'\rangle &= \langle\phi|\psi\rangle
    \end{aligned}
\]
\item
  The transformation is an isometry
\item
  In fact, it is also a unitary transformation
\end{itemize}

\section{}\label{lecture-13}

\subsection{Unitary Operators}\label{unitary-operators}

\begin{itemize}
\item
  If \(\hat{A}^\dagger = \hat{U}^{-1}\), then \(\hat{U}\) is a unitary
  operator
  \begin{itemize}
  \item
    \(\hat{U}^\dagger\hat{U} = \hat{I} = \hat{U}\hat{U}^\dagger\)
  \item
    \(\hat{U}^{-1}\hat{U} = \hat{I} = \hat{U}\hat{U}^{-1}\)
  \end{itemize}
\item
  \(\hat{U}\) is the same for all vectors of the Hilbert space
\[
    \begin{aligned}
    |\psi'\rangle &= \hat{U}|\psi\rangle \\
    |\psi\rangle &= \hat{U}^{-1}|\psi'\rangle = \hat{U}^\dagger|\psi'\rangle \\
    |\phi'\rangle &= \hat{U}|\phi\rangle \\
    |\eta\rangle &= \hat{A}|\psi\rangle \\
    |\eta'\rangle &= \hat{U}|eta\rangle = \hat{U}\hat{A}|\psi\rangle = \hat{U}\hat{A}\hat{U}^\dagger|\psi'\rangle \\
    |\eta'\rangle &= \hat{A}'|\psi'\rangle,~ \hat{A}' = \hat{U}\hat{A}\hat{U}^\dagger
    \end{aligned}
\]
\item
  Line four and seven are of the same form - but latter is written in
  terms of the transformed vectors and operators.
\item
  \(\hat{U}\) transforms:
  \begin{itemize}
  \item
    vectors \(|\psi\rangle\) into \(\hat{U}|\psi\rangle\)
  \item
    operators \(\hat{A}\) into \(\hat{U}\hat{A}\hat{U}^\dagger\)
  \end{itemize}
\item
  \(\hat{A}' = \hat{U}\hat{A}\hat{U}^\dagger\) has all the same
  properties of untransformed operator \(\hat{A}\)
\item
  If \(\hat{A}\) is Hermitian, then \(\hat{A}'\) is also Hermitian
\item
  If \(\hat{A} = \alpha\hat{B} + \beta\hat{C}\hat{D}\), then
  \(\hat{A}' = \alpha\hat{B}' + \beta\hat{C}'\hat{D}'\)
\item
  Proof:
\[
    \begin{aligned}
    \hat{A} &= \alpha\hat{B} + \beta\hat{C}\hat{D} \\
    \hat{U}\hat{A}\hat{U}^\dagger &= \alpha\hat{U}\hat{B}\hat{U}^\dagger + \beta\hat{U}\hat{C}\hat{I}\hat{D}\hat{U}^\dagger \\
    \hat{A}' &= \alpha\hat{B}' + \beta\hat{C}'\hat{D}'
    \end{aligned}
\]
\item
  \([\hat{A},\hat{B}] = [\hat{A}', \hat{B}']\)
\item
  \(\hat{A}\) and \(\hat{A}'\) have the same eigenvalues
\item
  \(\langle\phi|\hat{A}|\psi\rangle = \langle\phi'|\hat{A}'|\psi'\rangle\)
  for any \(|\psi\rangle,|\phi\rangle\)
\item
  In particular, \(\langle\phi|\psi\rangle = \langle\phi'|\psi'\rangle\)
  \begin{itemize}
  \item
    inner products are not changed by unitary transformations
  \end{itemize}
\item
  Proof:
\[
    \begin{aligned}
    |\psi'\rangle &= \hat{U}|\psi\rangle \\
    |\phi'\rangle &= \hat{U}|\phi\rangle \\
    \implies \langle\phi'| &= \langle\phi|\hat{U}^\dagger \\
    \implies \langle\phi'|\psi'\rangle &= \langle\phi|\hat{U}^\dagger\hat{U}|\psi\rangle \\
    &= \langle\phi|\psi\rangle
    \end{aligned}
\]
\item
  In particular, unitary transformations do not change the norm of the
  vector: \(\langle\psi|\psi\rangle = \langle\psi'|\psi'\rangle\)
\end{itemize}

\subsection{Time evolution of quantum systems}\label{time-evolution-of-quantum-systems}

\begin{itemize}
\item
  Time-dependent Schrodinger equation:
\[
    \begin{aligned}
    i\hbar\frac{d}{dt}|\Psi(t)\rangle &= \hat{H}|\Psi(t)\rangle \\
    |\Psi(t)\rangle &= \hat{U}(t,t_0)|\Psi(t_0)\rangle \\
    \hat{U}(t,t_0) &= \hat{U}(t,t_1)\hat{U}(t_1,t_0) \\
    \hat{U}^\dagger(t,t_0) &= \hat{U}^{-1}(t,t_0) = \hat{U}(t_0,t) \\
    \hat{U}(t_0,t_0) &= \hat{I} = \hat{U}(t_0,t)\hat{U}(t,t_0) \\
    \implies i\hbar\frac{d}{dt}\hat{U}(t,t_0) &= \hat{H}\hat{U}(t,t_0)
    \end{aligned}
\]
\item
  \(\hat{U}(t,t_0)\) is the time-evolution operator
  \begin{itemize}
  \item
    it is unitary
  \end{itemize}
\item
  If \(\hat{H}\) is time-independent, then
\[
    \begin{aligned}
    \hat{U}(t,t_0) &= \exp\left[\frac{-i\hat{H}(t-t_0)}{\hbar}\right] \\
    e^{\hat{A}} &= \hat{I} + \hat{A} + \frac{1}{2!}\hat{A}^2 + \frac{1}{3!}\hat{A}^3 + \cdots
    \end{aligned}
\]
\item
  The exponential of an operator is the Taylor expansion of that
  operator
\end{itemize}

\subsection{Expectation values of observables}\label{expectation-values-of-observables}

\[
    \begin{aligned}
    \langle\hat{A}(t)\rangle &= \langle\Psi(t)|\hat{A}|\Psi(t)\rangle \\
    &= \langle\Psi(t_0)|\underbrace{\hat{U}^\dagger(t,t_0)\hat{A}\hat{U}(t,t_0)}_{\hat{A}_H(t)}|\Psi(t_0)\rangle \\
    \hat{A}_H(t) &= \hat{U}^\dagger(t,t_0)\hat{A}\hat{U}(t,t_0) \\
    &= \hat{U}(t_0,t)\hat{A}\hat{U}^\dagger(t_0,t) \\
    \langle\hat{A}(t)\rangle &= \langle\Psi(t_0)|\hat{A}_H(t)|\Psi(t_0)\rangle
    \end{aligned}
\]

\begin{enumerate}
\item
  State vector changes in time, \(\hat{A}\) doesn't - Schrodinger
  picture
\item
  State vectors do not change in time, \(\hat{A}_H(t_{})\) does -
  Heisenberg picture
\end{enumerate}

\begin{itemize}
\item
  These two formulations are completely equivalent
\item
  Heisenberg equation of motion:
\[
    i\hbar\frac{d\hat{A}_H(t)}{dt} = [\hat{A}_H,\hat{H}] = [\hat{A},\hat{H}]
\]
if \(\hat{A}\) is time-independent.
\end{itemize}

\section{}\label{lecture-14}

\[
    \begin{aligned}
    \hat{U}^\dagger &= \hat{U}^{-1} \\
    |\psi'\rangle &= \hat{U}|\psi\rangle \\
    \hat{A}' &= \hat{U}\hat{A}\hat{U}^\dagger
    \end{aligned}
\]

\begin{itemize}
\item
  The eigenvalues of a unitary operator are real or complex numbers of
  modulus 1
\item
  The eigenvectors of a unitary operator corresponding to different
  eigenvalues are orthogonal to each other
\end{itemize}

\[
    \begin{aligned}
    \langle A\rangle(t) &= \langle\Psi(t)|\hat{A}|\Psi(t)\rangle \\
    &= \langle\Psi(t_0)|\hat{A}_H(t)|\Psi(t_0)\rangle \\
    \hat{A}_H(t) &= \hat{U}(t_0,t)\hat{A}\hat{U}^\dagger(t_0,t) \\
    i\hbar\frac{d\hat{A}_H}dt &= [\hat{A}_H,\hat{H}_H] = \hat{U}(t_0,t)[\hat{A},\hat{H}]\hat{U}^\dagger(t_0,t)
    \end{aligned}
\]

\begin{itemize}
\item
  If \([\hat{A},\hat{H}] = 0\), then \(\hat{A}_{H_{}}\) is constant in
  time
  \begin{itemize}
  \item
    \(\langle A\rangle(t)\) is also constant for any \(|\Psi\rangle\)
  \item
    \(A\) is a ``constant of motion''
\[
    \begin{aligned}
    |\psi'\rangle &= \hat{R}_x(\theta)|\psi\rangle \\
    \langle\psi'|H|\psi'\rangle &= \langle\psi|\hat{H}|\psi\rangle \\
    \langle\psi|\hat{R}_x(-\theta)\hat{H}\hat{R}_x(\theta)|\psi\rangle &= \langle\psi|\hat{H}|\psi\rangle \\
    \hat{R}_x^\dagger(theta) &= \hat{R}_x^{-1}(\theta = \hat{R}_x(-\theta) \\
    \langle\psi'| &= \langle\psi|\hat{R}_x^\dagger (\theta) \\
    &= \langle\psi|\hat{R}_x(-\theta)
    \end{aligned}
\]
  \end{itemize}
\item
  Now look at the limit when \(\theta \to \epsilon\), where \(\epsilon\)
  is near zero
\[
    \begin{aligned}
    \hat{R}_x(\pm\epsilon) &= \hat{I} \mp i\epsilon\frac{\hat{J}_x}{\hbar} \\
    \langle\psi|\left(\hat{I} + i\frac{\epsilon}{\hbar}\hat{J}_x\right)\hat{H}\left(\hat{I} - \frac{i\epsilon}{\hbar}\hat{J}_x\right)|\psi\rangle &= \langle\psi|\hat{H}|\psi\rangle \\
    \langle\psi|\hat{H}|\psi\rangle + \langle\psi|\frac{i\epsilon}{\hbar}\hat{J}_x\hat{H}|\psi\rangle + \langle\psi|\frac{-i\epsilon}{\hbar}\hat{H}\hat{J}_x|\psi\rangle &= \langle\psi|\hat{H}|\psi\rangle + \frac{i\epsilon}{\hbar}\langle\psi|[\hat{J}_x,\hat{H}]|\psi\rangle \\
    &= \langle\psi|\hat{H}|\psi\rangle \text{ for any } \psi \\
    \implies [\hat{J}_x,\hat{H}] &= 0
    \end{aligned}
\]
\item
  The requirement that the state of the atom is invariant under a
  rotation means that \(\vec{J}\) is a constant
\end{itemize}

\subsection{unitary transformations and change of bases}\label{unitary-transformations-and-change-of-bases}

\begin{itemize}
\item
  dimension of the Hilbert space, \(N\)
\item
  Consider two different orthonormal bases for that space:
\[
    \begin{aligned}
    \{|\phi_1\rangle,|\phi_2\rangle,&\cdots,|\phi_N\rangle\} \\
    \{|\psi_1\rangle,|\psi_2\rangle,&\cdots,|\psi_N\rangle\} \\
    \langle\phi_i|\phi_j\rangle &= \delta_{ij},~ \langle\psi_i,\psi_j\rangle = \delta_{ij}
    \sum_{i=1}^N |\phi_i\rangle\langle\phi_i| &= \hat{I},~ \sum_{i=1}^N |\psi_i\rangle\langle\psi_i| = \hat{I}
    \end{aligned}
\]
\item
  The last line is the Completeness relation
\item
  An operator \(\hat{A}\) is represented by a matrix
  \(\underline{\underline{A}}\) in the \(\{|\phi\rangle\}\) basis,
  \(\underline{\underline{A}}'\) in the \(\{|\psi\rangle\}\) basis
\[
    \begin{aligned}
    A_{ij} &= \langle\phi_i|\hat{A}|\phi_j\rangle \\
    A'_{ij} &= \langle\psi_i|\hat{A}\\psi_j\rangle
    \end{aligned}
\]
\item
  Because the \(\{|\phi\rangle\}\) vectors are a basis, one can always
  write each of the \(|\psi_j\rangle\) vectors as a linear combination
  of the \(|\phi_i\rangle\) vectors:
\end{itemize}

\[
    \begin{aligned}
    |\psi_j\rangle &= \sum_i U_{ji}^* |\phi_i\rangle \\
    U_{ji}^\dagger &= \langle\phi_i|\psi_j\rangle = \langle\psi_j|\phi_i\rangle^* \\
    U_{ji} &= \langle\psi_j|\phi_i\rangle \\
    \underline{\underline{U}} &= \begin{pmatrix} U_{11} & U_{12} & \cdots & U_{1N} \\ \vdots & \ddots & & \vdots \\ U_{N1} & \cdots & \cdots & U_{NN} \end{pmatrix} \\
    \underline{\underline{U}}\underline{\underline{U}}^\dagger &= \underline{\underline{I}} \\
    (\underline{\underline{U}}\underline{\underline{U}}^\dagger)_{ij} &= \sum_k U_{ik}U_{kj}^\dagger \\
    &= \sum_k \langle\psi_i|\phi_k\rangle\langle\phi_k|\psi_j\rangle \\
    &= \langle\psi|\underbrace{\sum_k |\phi_k\rangle\langle\phi_k|}_{\hat{I}}|\psi_j\rangle \\
    &= \langle\psi_i|\psi_j\rangle = \delta_{ij}
    \end{aligned}
\]

\[
    \begin{aligned}
    \hat{A}' &= \hat{U}\hat{A}\hat{U}^\dagger & |\chi\rangle &= \sum_i c_i|\phi_i\rangle \\
    \hat{c}' &= \hat{U}\hat{c} & &= \sum_i c_i'|\psi_i\rangle
    \end{aligned}
\]

\section{}\label{lecture-15}

\subsection{Spectral Decomposition}\label{spectral-decomposition}
Recall that
\(\sum_n |\phi_n\rangle\langle\phi_n| + \int d^3k |\phi_{\vec{k}}\rangle\langle\phi_{\vec{k}}| = \hat{I}\)
if and only if \(\{|\phi_n,|\phi_{\vec{k}_{}}\rangle\}\) is complete.
\[
    \begin{aligned}
    \hat{A}|\phi_n\rangle &= a_n|\phi_n\rangle & \langle\phi_i|\phi_j\rangle &= \delta_{ij} \\
    \hat{A}|\phi_{\vec{k}}\rangle &= a_{\vec{k}}|\phi_{\vec{k}}\rangle & \langle \phi_{\vec{k}}|\phi_{\vec{k}'}\rangle &= \delta(\vec{k}-\vec{k}')
    \end{aligned}
\]
\begin{itemize}
\item
  \(\hat{A}\) is a Hermitian operator
\[
    \begin{aligned}
    \hat{A} &= \hat{A}\hat{I} \\
            &= \sum_n \hat{A}|\phi_n\rangle\langle\phi_n| + \int d^3k \hat{A}|\phi_{\vec{k}}\rangle\langle\phi_{\vec{k}}| \\
            &= \sum_n a_n|\phi_n\rangle\langle\phi_n| + \int d^3k a_{\vec{k}}|\phi_{\vec{k}}\rangle\langle\phi_{\vec{k}}|
    \end{aligned}
\]
\item
  This is the spectral decomposition of \(\hat{A}\)
\end{itemize}

\subsection{Projectors}\label{projectors}

For example,
\[
    \begin{aligned}
    \hat{\mathcal{P}}_{\phi} &= |\phi\rangle\langle\phi| \text{ with } \langle\phi|\phi\rangle = 1 \\
    \hat{\mathcal{P}}_{\phi}|\psi\rangle &= |\phi\rangle\langle\phi|\psi\rangle = \langle\phi|\psi\rangle|\phi\rangle
    \end{aligned}
\]
In position representation:
\[
    \begin{aligned}
    \mathcal{P}_{\phi}\psi(\vec{r}) &= \left[\int \phi^*(\vec{r}')\psi(\vec{r}') d^3r\right]\phi(\vec{r}) \\
    \mathcal{P}_{\phi} &\equiv \phi^*(\vec{r}')\phi(\vec{r}')
    \end{aligned}
\]
in the sense that when \(\mathcal{P}_{\phi_{}}\) acts on a wave
function, \(\psi(\vec{r})\), the result is as above
\[
    \begin{aligned}
    \hat{\mathcal{P}}_\phi &= |\phi\rangle\langle\phi| \\
    \hat{\mathcal{P}}^2_\phi &= \hat{\mathcal{P}}_\phi\hat{\mathcal{P}}_\phi = |\phi\rangle\langle\phi|\phi\rangle\langle\phi| \\
    &= \phi\rangle\langle\phi| = \hat{\mathcal{P}}_\phi
    \end{aligned}
\]
\begin{itemize}
\item
  \(\hat{\mathcal{P}}_{\phi_{}}\) is idempotent
  \begin{itemize}
  \item
    operators \(\hat{A}\) such that \(\hat{A}^2 = \hat{A}\) are said to
    be idempotent
  \end{itemize}
\item
  \(\hat{\mathcal{P}}_{\phi_{}}\) is also Hermitian:
\[
    \begin{aligned}
    \langle\psi'|\hat{\mathcal{P}}_{\phi}|\psi\rangle &= \langle\psi|\hat{\mathcal{P}}_{\phi}|\psi'\rangle^{ * } \\
    &= \langle\psi'|\phi\rangle\langle\phi|\psi\rangle \\
    &= \langle\phi|\psi\rangle\langle\psi'|\phi\rangle \\
    &= \langle\psi|\phi\rangle^{ * }\langle\phi|\phi'\rangle^{ * } \\
    &= [\langle\psi|\phi\rangle\langle\phi|\psi'\rangle]^{ * }
    \end{aligned}
\]
\end{itemize}

More generally, any operator which is both idempotent and Hermitian is a
projector.
Consider a vector, \(\vec{v}\) in 3D space:
\begin{itemize}
    \item \(\vec{v} = v_x\hat{x} + v_y\hat{y} + v_z\hat{z}\) 
    \item \(\vec{w} = v_x\hat{x} + v_y\hat{y}\) - this is the projection of \(\vec{v}\) in the x-y plane 
    \item \(\vec{w} = (\hat{x}\hat{x} + \hat{y}\hat{y})\cdot\vec{v} = \hat{x}\cdot\vec{v} \hat{x} + \hat{y}\cdot\vec{v} \hat{y}\)
    \item \((\hat{x}\cdot\vec{v}\) is the same as \(|\hat{x}\rangle\langle\hat{x}|\vec{v}\rangle\)
    \item The projection in the plane is affected by \(|\hat{x}\rangle\langle\hat{x} + |\hat{y}\rangle\langle\hat{y}|\) 
    \item If \(|\phi\rangle\) and \(|\psi\rangle\) are linearly independent, \(\langle\phi|\psi\rangle = 0, \langle\phi|\phi\rangle = \langle\psi|\psi\rangle = 1\)
    \item \(|\psi\rangle\langle\phi| + |\psi\rangle\langle\psi|\) projectors in the subspace spanned by \(|\phi\rangle\) and \(|\psi\rangle\)
\[
    \sum_n |\phi_n\rangle\langle\phi_n| + \int d^3k |\phi_{\vec{k}}\rangle\langle\phi_{\vec{k}}| = \hat{I}
\]
\item
  \(\hat{\mathcal{P}}_{\phi} = |\phi\rangle\langle\phi_{}|\) is
  Hermitian
\item
  \(|\langle\phi|\psi\rangle|^2\) is the probability of finding the
  system in a state \(|\phi\rangle\) if it was in the state
  \(|\psi\rangle\) before measurement
\item
  If \(|\eta\rangle\) is an eigenvector of \(\hat{\mathcal{P}}_{\phi}\)
  with eigenvalue \(\eta_{}\):
\[
    \begin{aligned}
    \hat{\mathcal{P}}_{\phi}|\eta\rangle &= \eta|\eta\rangle \\
    |\phi\rangle\langle\phi|\eta\rangle &= \eta|\eta\rangle \\
    \langle\phi|\eta\rangle|\phi\rangle &= \eta|\eta\rangle \\
    \implies |\phi\rangle &= |\eta\rangle,~ \eta = 0,\langle\phi|\eta\rangle = 0,1
    \end{aligned}
\]
\end{itemize}
The eigenvalues of \(\hat{\mathcal{P}}_{\phi_{}}\) are 0 and 1
\begin{itemize}
    \item For \(\eta =1\) - \(|\eta\rangle = |\phi\rangle\) 
    \item For \(\eta = 0\) - \(|\eta\rangle\) can be any vector orthogonal to \(|\phi\rangle\)
\item
  Observable here - \(\hat{\mathcal{P}}_{\phi_{}}\)
\item
  Possible outcomes - \(\eta = 0,1\)
\item
  Probability of finding \(\eta = 1\) - \(|\langle\phi|\psi\rangle|^2\)
\end{itemize}

\subsection{Revision of ladder operator}\label{revision-of-ladder-operator}

\begin{itemize}
\item
  \(\hat{a}_{-} = \hat{a}\), and \(\hat{a}_{+} = \hat{a}^\dagger\)
\item
  subscript with dimension being used in - x,y,z
\end{itemize}

\[
    \begin{aligned}
    [\hat{a}_i,\hat{a}_i^\dagger] &= 1 \\
    [\hat{a}_i,\hat{a}_j^\dagger] &= 0
    \end{aligned}
\]

\section{}\label{lecture-16}

\subsection{Comments on Homework}\label{comments-on-homework}

\begin{itemize}
\item
  \([\hat{H}.\hat{U}(t,t_0)] = 0\) because if \(\hat{H}\) is
  time-independent, \(\hat{U}(t,t_0) = \exp[-i\hat{H}(t-t_0)/\hbar]\)
\end{itemize}

\subsection{Operators and Spin States}\label{operators-and-spin-states}

\begin{itemize}
\item
  Consider operators belonging to orthogonal directions, i.e. ladder
  operators
\item
  We then define the Hamiltonian,
  \(\hat{H} = \hbar\omega(\hat{a}_{x}^\dagger\hat{a}_{} + \frac{1}{2})\)
\item
  This then leads to
  \(E_n = \hbar\omega(n + \frac{1}{2}), n = 0,1,2 \implies \hat{a}_{x_{}}|\phi_n\rangle = \sqrt{n}|\phi_{n-1_{}}\rangle, \hat{a}_{x_{}}|\phi_0\rangle = 0\)
\item
  \(\hat{a}_{x_{}}^\dagger|\phi_n\rangle = \sqrt{n+1}|\phi_{n+1_{}}\)
\end{itemize}

\subsection{Angular Momentum}\label{angular-momentum}

\begin{itemize}
\item
  The orbital angular momentum operator is
  \(\vec{L} = \hat{L}_{x}\hat{i} + \hat{L}_{y}\hat{j} + \hat{L}_z\hat{k}_{}\)
\[
    \begin{aligned}
    \vec{L} &= \vec{r} \times \vec{p},~ \vec{r} = \hat{x}\hat{i} + \hat{y}\hat{j} + \hat{z}\hat{k},~ \hat{p} = \hat{p}_x\hat{i} + \hat{p}_y\hat{j} + \hat{p}_z\hat{k} \\
    \vec{L} &= \begin{vmatrix} \hat{i} & \hat{j} & \hat{k} \\ \hat{x} & \hat{y} & \hat{z} \\ \hat{p}_x & \hat{p}_y & \hat{p}_z \end{vmatrix} \\
    \implies \hat{L}_x &= \hat{y}\hat{p}_z - \hat{z}\hat{p}_y \\
    \implies \hat{L}_y &= \hat{z}\hat{p}_x - \hat{x}\hat{p}_z \\
    \implies \hat{L}_z &= \hat{x}\hat{p}_y - \hat{y}\hat{p}_x \\
    \implies \hat{L}_z &= -i\hbar\left(x\frac{\partial}{\partial y} - y\frac{\partial}{\partial x}\right) = -i\hbar\frac{\partial}{\partial \phi}
    \end{aligned}
\]
\item
  Another example is the spin operator, i.e.
  \(\vec{s} = \hat{s}_x\hat{i} + \hat{s}_y\hat{j} + \hat{s}_z\hat{k}_{}\)
\item
  An operator \(\vec{J}\) is an angular momentum operator if
  \(\hat{J}_x,\hat{J}_y,\hat{J}_{z_{}}\) are Hermitian and
  \([J_x,J_y] = i\hbar J_z\), etc
\item
  The \(J_i\)s all commute with
  \(\vec{J}^2 = \vec{J}\cdot\vec{J} = J_x^2 + J_y^2 + J_z^2\)
\item
  \(J_n = \hat{n}\cdot\vec{J}\) where \(\hat{n}\) is a unit vector in a
  given direction
\item
  \([J_n,J_n] \neq 0\) is \(\hat{n} \neq \hat{n}\),
  \([J_n,\vec{J}^2] = 0 \forall \hat{n}\)
\end{itemize}

Consider the Hilbert space \(\mathcal{H}\) spanned by the eigenvector of
\(\vec{J}^2\). Since \(\vec{J}^2\) and \(J_n\) commute, one can always
construct a basis of \(\mathcal{H}\) with simultaneous eigenvectors of
these two operators. However, since \([J_n,J_m] \neq 0\) if
\(\hat{n} \neq \hat{m}\), there is no basis of simultaneous eigenvectors
of \(\vec{J}^2, J_n, J_m\). The simultaneous eigenvectors of
\(\vec{J}^2\) and \(J_z\) are \(|jm\rangle\)

Consider the ladder operators
\(J_{+} = J_x + iJ_y,~ J_{-} = J_x - iJ_y,~ J_{+} = I_{-}^\dagger\),
\([J_{\pm},\vec{J_{}}^2] = 0\) but \([J_{+},J_{-}] \neq 0\). We find
through algebraic methods,
\begin{enumerate}
\item
  \[\begin{aligned}
   J_{+}|j,m\rangle &\propto \hbar|j,m+1\rangle,~ J_{+}|jj\rangle = 0 \\
   J_{-}|j,m\rangle &\propto \hbar|j,m-1\rangle,~ J_{-}|j-j\rangle = 0
   \end{aligned}\]
\item
  The eigenvalues for \(\vec{J}^2\) are \(j(j+1)\hbar^2\) with
  \(j = 0,\frac{1}{2},1,\frac{3}{2},\cdots\)
\item
  The eigenvectors of \(J_z\) are \(m\hbar\) with
  \(m = 0,\pm\frac{1}{2},\pm 1,\pm\frac{3}{2},\cdots\)
\item
  For simultaneous eigenvector \(|jm\rangle\) of \(\vec{J}^2\) and
  \(J_z\), the values of \(m\) and \(j\) are restricted by the
  requirement that \(m\) in the range \(-j \leq m \leq j\)
\end{enumerate}

\begin{itemize}
\item
  The eigenvectors \(|jm\rangle\) are orthonormal,
  \(\langle j'm'|jm\rangle = \delta_{jj'}\delta_{mm'}\)
\item
  \(\langle jm|jm\rangle\) has been chosen to equal 1 by choice of
  normalisation
\item
  For orbital angular momentum, \(\vec{L}\):
  \begin{itemize}
  \item
    The joint eigenfunctions of \(\vec{L}^2\) and \(L_z\) are
    \(Y_{lm}(\theta_{},\phi)\)
  \item
    \(L_zf(\phi) = -i\hbar\partial_{\phi}(f_{}(\phi)) = m\hbar f(\phi) \to f(\phi) \propto e^{im\phi}\)
  \item
    Because \(\phi\) is a position angle,
    \(e^{im(\phi + 2\pi)} = e^{im\phi}\) therefore m must be an integer
  \item
    \(\vec{L}^2Y_{lm} = \hbar^2_{}l(l+1)Y_{lm_{}}\) and
    \(L_zY_{lm_{}} = \hbar mY_{lm_{}}\) for \(-l \leq m \leq l\)
  \end{itemize}
\end{itemize}

\section{}\label{lecture-17}

Consider \([J_n,\vec{J}^2] = 0\). \(J_n\) transforms any eigenvector of
\(\vec{J}^2\) into an eigenvector of \(\vec{J}^2\) belonging to the same
value of \(j\), i.e. \(J^2\) is invariant under \(J_n\). 
Similarly consider a rotation about an axis \(\hat{n}\) by an angle \(\theta i\):
\begin{itemize}
    \item \(|jm\rangle \to \hat{R}_n(\theta_{})|jm\rangle\) 
    \item For an infinitesimal transformation - \(\hat{R}_n(\epsilon_{}) = \hat{I} - i\epsilon\frac{\hat{J}_n}{\hbar_{}}\)
    \item For a finite rotation - \(\hat{R}_n(\theta_{}) = \exp[-i\theta\hat{J}_{n_{}}/\hbar]\) * Under a rotation, an eigenstate \(|jm\rangle\) transforms into a superposition of \(|j'm'\rangle\) with \(j=j'\) * \(\langle j'm'|J_n|jm\rangle = 0\) when \(j \neq j'\)
\end{itemize}

What is the matrix representation of an angular momentum operator?

The \(\{|jm\rangle\}\) vectors form an orthonormal basis. For a given
value of j, \(J_n\) is represented by a \((2j+1)\times(2j+1)\) matrix,
since for a given j, m can take \(2j + 1\) different values and \(J_n\)
does not couple states of different values of j.

E.g. for \(j = \frac{1}{2}\), all the angular momentum operators are
represented by a \(2\times 2\) matrix. Usually, the basis is chosen to
be
\(\left\{|-\frac{1}{2},\frac{1}{2}\rangle,|\frac{1}{2},-\frac{1}{2}\rangle\right\}\)
which can be represented by \(|+\rangle,|-\rangle\)
\begin{itemize}
    \item \(J_z|+\rangle = \frac{\hbar}{2}|+\rangle\) where \(m = +\frac{1}{2}\)
and its state is spin up * \(J_z|-\rangle = -\frac{\hbar}{2}|-\rangle\)
where \(m = -\frac{1}{2}\) and its state is spin down
\end{itemize}

In this basis, \(J_z\) is represented by the matrix:
\[
    \frac{\hbar}{2}\begin{pmatrix} 1 & 0 \\ 0 & -1 \end{pmatrix}
\]
Similarly,
\[
    J_x \to \frac{\hbar}{2}\begin{pmatrix} 0 & 1 \\ 1 & 0 \end{pmatrix};~ J_y \to \frac{\hbar}{2}\begin{pmatrix} 0 & -i \\ i & 0 \end{pmatrix}
\]
These can be represented using the Pauli Matrices, i.e.
\(\sigma_x,\sigma_y,\sigma_z\), so \(J_i = \frac{\hbar}{2}\sigma_i\)
\begin{itemize}
    \item \(|+\rangle\) is represented by
\[
    \begin{pmatrix} 1 \\ 0 \end{pmatrix}
\]
\item and \(|-\rangle\) is represented by
\[
    \begin{pmatrix} 0 \\ 1 \end{pmatrix}
\]
\item so an arbitrary spin state can be expressed as
\[
    \alpha|+\rangle + \beta|-\rangle = \begin{pmatrix} \alpha \\ \beta \end{pmatrix}
\]
\end{itemize}

\subsection{2 Electron System}\label{electron-system}
Consider 2 electrons \(|+\rangle_1\) and \(|-\rangle_2\). The system is
expressed as
\(\alpha|+\rangle_1|+\rangle + \beta|+\rangle_1|-\rangle_2 + \gamma|-\rangle_1|+\rangle_2 + \delta|-\rangle_1|-\rangle_2 = |\psi\rangle_{12_{}}\),
where \(\alpha,\beta,\gamma,\delta\) are complex numbers. More
generally, the joint angular momentum state of two particles 1 and is:

\[
    \begin{aligned}
    |\psi\rangle_{12} &= \sum_{j_1,m_1,j_2,m_2} c_{j_1m_1j_2m_2}|j_1m_1\rangle_1|j_2m_2\rangle_2 \\
    \vec{J}_1 &= (J_{1x},J_{1y},J_{1z}) \text{ acts only on } |j_1m_1\rangle_1 \\
    \vec{J}_2 &= (J_{2x},J_{2y},J_{2z}) \text{ acts only on } |j_2m_2\rangle_2
    \end{aligned}
\]
\end{document}
