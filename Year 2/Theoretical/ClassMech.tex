\documentclass[a4paper,11pt,normalem]{article}
\usepackage{../../../LaTeX-Templates/Notes}

\rhead{}
\titlecontents{section}
    [0pt]
    {}
    {Lecture \thecontentslabel\quad}
    {}
    {\dotfill\contentspage}
    \titleformat{\section}{\fontsize{12}{15}\normalfont}{\underline{\textbf{Lecture \thesection}}}{1em}{}


\newcommand{\HRule}{\rule{\linewidth}{0.5mm}}

\begin{document}
{\centering
{\includegraphics[scale=0.5]{../../logo0.png}\hfill{\Large\bfseries Michaelmas 2017}}\\[1.5cm]
{\LARGE\bfseries Theoretical Physics 2}\\[0.5cm]
\HRule \\[0.3cm]
{\huge\bfseries Classical Mechanics}\\[0.1cm]
\HRule \\[1cm]}
\begin{center}
\begin{minipage}{0.4\textwidth}
    \begin{flushleft} \large
        \emph{Author:} \\ Matthew Rossetter
    \end{flushleft}
\end{minipage}~
\begin{minipage}{0.4\textwidth}
    \begin{flushright} \large
        \emph{Lecturer:} \\ Prof. Vincent Eke
    \end{flushright}
\end{minipage}
\end{center}

\section{}\label{lecture-1}
\begin{itemize}
\item DoF
\item Constraints
\end{itemize}

\section{}\label{lecture-2}

\begin{itemize}
\item Differential Formulation of Lagrangian Mechanics
\end{itemize}

\subsection{Generalised Coordinates}\label{generalised-coordinates}

\begin{itemize}
\item Page 6
\item Only need as many coordinates \(q_k\) as there are DoFs (\(N = 3M - j\))
\item Generalised velocities are \(\frac{d q_k}{dt}\)
\item Regard \(q_k\) and \(\dot{q}_k\) as independent variables
  \begin{itemize}
  \item
    \(\frac{\partial q_k}{\partial \dot{q}_k} = \frac{\partial \dot{q}_k}{\partial q_k} = 0\)
  \end{itemize}
\end{itemize}

\subsubsection{Lagrangian}\label{lagrangian}

\begin{itemize}
\item
  Page 7
\item
  Assume:
  \begin{enumerate}
  \item
    Holonomic constraints
  \item
    Constraining forces do no work
  \item
    Applied forces are conservative, such that a scalar potential energy
    function exists
    \begin{itemize}
    \item
      Potential function may change with time
    \end{itemize}
  \end{enumerate}
\item
  Generalised Coordinates don't necessarily have to be in metres - Force
  is not in Newtons, it is \(\frac{\partial W}{\partial q_k}\)
\end{itemize}

\section{}

\begin{itemize}
\item
  Ignorable coords
  \begin{itemize}
  \item
    if it doesn't appear in the Lagrangian, it can be ignored
  \item
    Canonically conjugate momentum to the coord
    \begin{itemize}
    \item
      \(p_k = \frac{\partial L}{\partial \dot{q}_k}\)
    \end{itemize}
  \item
    simplifies eqn of motion
  \item
    example:
    \begin{itemize}
    \item
      \(L = \frac{1}{2}m\dot{x}^2 \implies p = \frac{\partial L}{\partial \dot{q}_k} = m\dot{x}~\)
      is constant
    \end{itemize}
  \end{itemize}
\item
  E.g. Ladder sliding down a wall
  \begin{itemize}
  \item
    A ladder of length 2a and uniform mass/length = \(\lambda\) leans
    against a wall. It is initially stationary before sliding down the
    wall without friction. Find \(\ddot{\theta}\) where \(\theta\) is
    the ladder's angle to the horizontal, while it remains in contact
    with the wall.
  \item
    We need to define the Lagrangian in terms of the one generalised
    coord, \(\theta\)
  \item
    Define s as the distance along the ladder from the end in contact
    with the floor. Then \(dm(s) = \lambda\,ds\). Also, we can write
    \(\underline{r}(s) = -2a\cos{\theta} + s\cos{\theta} = [(s - 2a)\cos{\theta}, \sin{\theta}]\)
  \item
    Differentiating wrt time:
    \(\underline{\dot{r}}(s) = [-(s-2a)\dot{\theta}\sin{\theta},s\dot{\theta}\cos{\theta}]\)
  \item
    \(\therefore \underline{\dot{r}}^{2}(s) = \dot{\theta}^2 [s^2 + (4a^2 - 4as)\sin^{2}{\theta}]\)
  \item
    Combining this into T gives:
  \end{itemize}
\end{itemize}

\[
    T = \int_{0}^{2a} \frac{1}{2}\underline{\dot{r}}^{2}(s)\,dm(s)
\] \[
    T = \int_{0}^{2a} \frac{\lambda \dot{\theta}^2}{2}[s^2 + (4a^2 - 4as)\sin^{2}{\theta}] \, ds
\] \[
    T = \frac{\lambda \dot{\theta}^2}{2} [\frac{s^3}{3} + \sin^{2}{\theta} (4a^{2}s - 2as^2)]_{0}^{2a_{}}
\] \[
    T = \frac{4\lambda\dot{\theta}^{2}a_{1}^{3}}{3}
\] \[
    V = (\lambda 2a)g(a\sin{\theta}) \implies
\] \[
    L =  \frac{4\lambda\dot{\theta}^{2}a^{3}}{3} - \lambda 2a^{2}g\sin{\theta} = 2\lambda a^{2}(\frac{2a\dot{\theta}^2}{3} - g\sin{\theta})
\] E-L: \[
    \frac{d}{dt}(\frac{\partial L}{\partial \dot{\theta}}) - \frac{\partial L}{\partial \theta} = 0
\] \[
    \frac{d}{dt} (\frac{4a\dot{\theta}}{3}) + g\cos\theta = 0
\] \[
    \therefore \ddot{\theta} = -\frac{3g}{4a}\cos\theta
\]

\subsection{An example of an integral principle: Fermat's Principle}\label{an-example-of-an-integral-principle-fermats-principle}

\begin{itemize}
\item
  \emph{A light ray takes the quickest available path between two
  points}
\item
  Consider a boundary between two materials
  \begin{itemize}
  \item
    Refractive indices of n1 and n2
  \item
    Going from A in n1 to B in n2
  \item
    Travelling x1 in A and x2 in B
  \item
    Distance l between A and B in y dirn
  \item
    travels distance y in A, so \(l - y\) in B
  \item
    paths of s1 and s2 in A and B
  \item
    \(\theta_1\) and \(\theta_2\) are angles from sn to xn
  \end{itemize}
\end{itemize}

\[
    T = \int_{A}^{B} dt = \int_{A}^{B} \underline{ds} \] \[
    T = \frac{1}{c} \int_{A}^{B} n\,ds \] \[
    cT = n_1 s_1 + n_2 s_2 \] \[
    s_1 = \sqrt{x_{1}^{2} + y^2} \] \[
    s_2 = \sqrt{x_{2}^{2} + (l - y)^2} \] \[
    cT = n_{1}\sqrt{x_{1}^{2} + y^2} + n_{2}\sqrt{x_{2}^{2} + (l - y)^2} \]
\[
    \frac{dT}{dy} = 0 \implies \] \[
    \frac{n_1 y}{\sqrt{x_{1}^{2} + y^2}} - \frac{n_2 (l - y)}{\sqrt{x_{2}^{2} + (l - y)^2}} = 0 \]
\[
    n_{1}\sin\theta_1 - n_{2}\sin\theta_2 = 0 ~~~\text{i.e. Snell's Law}
\]

\subsection{Variational Calculus}\label{variational-calculus}

\begin{itemize}
\item
  \(I[y] = \int_{x_1}^{x_2} f(y,y',x)\,dx\)
  \begin{itemize}
  \item
    I is a functional
  \item
    for a whole path
  \item
    solution will be in form of E-L
  \end{itemize}
\item
  For any perturbation we make to the path, the value of the functional
  will not change
  \begin{itemize}
  \item
    like how looking for dy/dx = 0
  \item
    essentially same but extra dimension
  \end{itemize}
\item
  \(\epsilon\) is small number, constant, and independent of x
  \begin{itemize}
  \item
    how far away from x
  \end{itemize}
\item
  \(\eta\) tells us above or below x
  \begin{itemize}
  \item
    \(\eta(x_1) = \eta(x_2) = 0\)
  \end{itemize}
\item
  \((\frac{dI}{d\epsilon})_{\epsilon = 0_{}}\)
\item
  For \(\frac{\partial f}{\partial x} = 0\):
\end{itemize}
\[
    \frac{df}{dx} = \frac{\partial f}{\partial x} + \frac{\partial f}{\partial y}\frac{dy}{dx} + \frac{\partial f}{\partial y'}\frac{dy'}{dx} \]
\[
    \frac{d}{dx}(f - y'\frac{\partial f}{\partial y'}) = 0
\]

\begin{itemize}
\item
  \textbf{Alternate form of E-L which can be useful}
\end{itemize}

\section{}\label{lecture-4}
\subsection{Hamilton's Principle}\label{hamiltons-principle}

\[
    S[\underline{q}(t)] = \int_{t_1}^{t_2} L(\underline{q}(t), \dot{q}(t), t)\,dt \]
\[
    \frac{\delta L}{\delta q} \equiv \frac{\partial L}{\partial q} - \frac{d}{dt}(\frac{\partial L}{\partial \dot{q}})
\]

\subsection{Brachistochrome}\label{brachistochrome}

\begin{itemize}
\item
  What is the quickest route for mass m to fall under gravity from A to
  B?
  \begin{itemize}
  \item
    assuming no friction
  \end{itemize}
\item
  Conservation of energy to start with for speed:
\end{itemize}

\[
    \frac{1}{2}mv^2 + mgy = 0 \] \[
    \implies v = \sqrt{-2gy} \] \[
\]

\begin{itemize}
\item
  Total time taken:
\end{itemize}

\[
    t = \int_{A}^{B} \frac{ds}{v} ~;~ (ds)^2 = (dx)^2 + (dy)^2 \] \[
    ds = \sqrt{1 + (\frac{dy}{dx})^2} \cdot dx \] \[
    \implies t = \int_{x_{A}}^{x_{B}} \frac{\sqrt{1 + y'^{2}} \cdot dx}{\sqrt{-2gy}}
\]

\begin{itemize}
\item
  Calculus of Variations problem
  \begin{itemize}
  \item
    \(f = \sqrt{\frac{1 + y'^2}{-2gy}}\)
  \item
    no explicit dependence on x
    \begin{itemize}
    \item
      from eq. 9 on notes, we get
      \(f - y'\frac{\partial f}{\partial y'} = c\)
    \end{itemize}
  \end{itemize}
\end{itemize}

\[
    \sqrt{\frac{1 + y'^2}{-2gy}} - \frac{y' \cdot y'}{\sqrt{-2gy(1 + y'^2)}} = c \]
\[
    \frac{1}{\sqrt{-2gy(1 + y'^2)}} = c \] \[
    y(1 + y'^2) = A \] \[
    \frac{dy}{dx} = \sqrt{\frac{A - y}{y}} \] \[
    \int \sqrt{\frac{y}{A - y}} \cdot dy = \int dx = x
\]

\begin{itemize}
\item
  Try substitution:
  \begin{itemize}
  \item
    \(y = \frac{A}{2}(1 - \cos\theta) = A\sin^{2}(\frac{\theta}{2})\)
  \item
    from \(1 - \cos\theta = 2\sin^{2}(\frac{\theta}{2})\)
  \end{itemize}
\end{itemize}

\[
    dy = A\sin\frac{\theta}{2}\cos\frac{\theta}{2} \] \[
    x = \int \sqrt{\frac{A\sin^{2}\frac{\theta}{2}}{A\cos^{2}\frac{\theta}{2}}} \cdot A\sin\frac{\theta}{2}\cos\frac{\theta}{2}\,d\theta \]
\[
    x = A\int \frac{1}{2}(1 - \cos\theta)\,d\theta = \frac{A}{2}(\theta - \sin\theta) + B \]
\[
    x_A = y_A = 0 ~;~ \theta = 0 ~;~ \theta = 0 \therefore \] \[
    x = \frac{A}{2}(\theta - \sin\theta) \] \[
    y = \frac{A}{2}(1 - \cos\theta)
\]

\begin{itemize}
\item
  x and y final solns give a cycloid
\item
  see on DUO for python code for showing particle motion
\end{itemize}

\subsection{Lagrange Multiplier}\label{lagrange-multiplier}

\begin{itemize}
\item
  If you have a constraint, and want to eliminate number of variables to
  solve, use Lagrange Multipliers
\item
  Consider a pendulum of length, l
  \begin{itemize}
  \item
    The natural coordinate to choose is \(\theta\)
  \item
    If you want to use x and y then,
  \end{itemize}
\end{itemize}

\[
    \delta S = \int \Bigg[\frac{\delta L}{\delta x}\cdot \delta x + \frac{\delta L}{\delta y}\cdot \delta y \Bigg]dt
\]

\begin{itemize}
\item
  Hamilton's principle says \(\delta S = 0\) but non-independence of x
  and y means can't say each variational deriv is zero
\item
  Use \(\lambda\) as an arbitrary Lagrange multiplier
\end{itemize}

\[
    \delta S = \int \Bigg[\Big(\frac{\delta L}{\delta x} - \lambda\frac{\partial G}{\partial x} \Big)\delta x + \Big(\frac{\delta L}{\delta y} - \lambda\frac{\partial G}{\partial y} \Big)\delta y \Bigg]dt \]
\[
    \delta(F - \lambda G) = 0
\]

\subsection{Linear Oscillators}\label{linear-oscillators}

\begin{itemize}
\item
  Mech system is at equilibrium at rest
\item
  This occurs at points in space where all forces, \(F_k\) vanish
\item
  For conservative systems, this is when the potential energy is
  stationary
  \begin{itemize}
  \item
    its first derivs wrt to \(q_k\) vanish
  \end{itemize}
\item
  Near Static Equilibrium
  \begin{itemize}
  \item
    define equilibrium psn and velocity as zero
  \item
    \(\frac{\delta V}{\delta q} = 0\) at equilibrium point
  \item
    expand out and only go up to quadratic terms as near equilibrium
  \item
    Sub into E-L eqn:
  \end{itemize}
\end{itemize}

\[
    L = \frac{m}{2}\Big(\dot{q}^2 + \frac{D}{F}q^2 \Big)
\]

\begin{itemize}
\item
  Normal EoM for particle
\item
  SHO:
  \begin{itemize}
  \item
    equilibrium psn is minimum of pot energy
  \item
    close to equilibrium, we have SHM
  \item
    \(\omega = \sqrt{-\frac{D}{F}}\)
  \item
    if \(\frac{D}{F} > 0\), it is unstable
  \item
    \(\ddot{q} + \omega^2 q = 0\) is linear, homogeneous diff eqn
  \end{itemize}
\end{itemize}

\[
    q(t) = q(0)\cos(\omega t) + \frac{\dot{q}(0)}{\omega}\sin(\omega t)
\]

\begin{itemize}
\item
  Damping force
  \begin{itemize}
  \item
    frictional force to suppress motion
  \item
    opposes motion and vanishes when motion ceases
  \item
    \(F_d = -\gamma\dot{q} = -(\frac{m\omega}{Q})\dot{q}\)
  \item
    Q is dimensionless quality factor
  \item
    higher quality system leads to small Q and motion lasts longer
  \end{itemize}
\end{itemize}

\subsection{Damped Oscillators}\label{damped-oscillators}

\begin{itemize}
\item
  For Damped SHO, E-L:
\end{itemize}

\[
    \frac{d}{dt}\Big(\frac{\partial L}{\partial \dot{q}}\Big) - \frac{\partial L}{\partial q} = F_d \]
\[
    \ddot{q} + \frac{\omega}{Q}\dot{q} + \omega^2 q = 0 \] \[
    q = e^{\lambda t} \implies \lambda^2 + \frac{\omega}{Q}\lambda + \omega^2 = 0 \]
\[
    \lambda = -\frac{\omega}{2Q} \pm \sqrt{(\frac{\omega}{2Q})^2 - \omega^2} \]
\[
    \lambda = -\frac{\omega}{2Q} + \underbrace{\frac{\omega}{2Q}\sqrt{1 - 4Q^2}}_{\omega '}
\]

\begin{itemize}
\item
  The sqrt becomes positive, zero, or negative depending on Q
\item
  Overdamped: \(Q < \frac{1}{2} \implies \sqrt{} \in \mathbb{R}\)

  \begin{itemize}

  \item
    Increasing Q but keeping it under \(\frac{1}{2}\) makes it tend to
    equilibrium faster
  \end{itemize}
\end{itemize}

\[
    q(t) = e^{-\frac{\omega t}{2Q}}(A\cosh(\omega 't) + B\sinh(\omega 't)) \]
\[
    \dot{q}(t) = e^{-\frac{\omega t}{2Q}}(A\omega '\cosh(\omega 't) + B\omega '\sinh(\omega 't))\cdot -\frac{\omega}{2Q}q \]
\[
    \dot{q}(0) = B\omega ' - \frac{\omega}{2Q}q(0) \] \[
    B = \frac{1}{\omega '}(\dot{q}(0) + \frac{\omega}{2Q}q(0)) \] \[
    q(0) = A
\]

\section{}\label{lecture-5}

\subsection{Damped Oscillators Ctd}\label{damped-oscillators-ctd}
\begin{itemize}
\item
  Critically Damped: \(Q = \frac{1}{2} \implies\) 2 equal roots for
  \(\lambda = -\frac{omega}{2Q}\)
  \begin{itemize}
  \item
    Fastest return to equilibrium without overshooting
  \end{itemize}
\end{itemize}

\[
    q(t) = e^{-\frac{\omega t}{2Q}}(\underbrace{A}_{q(0)} + \underbrace{B}_{\dot{q}(0) + \frac{\omega}{2Q}q(0)}t) \]
\[
\]

\begin{itemize}
\item
  Underdamped:
  \(Q > \frac{1}{2} \implies \lambda = -\frac{\omega}{2Q} \pm i\omega ' ~;~ \omega = \frac{\omega}{2Q}\sqrt{4Q^2 - 1}\)
  \begin{itemize}
  \item
    Solution is like overdamped case but with cosh replaced by cos and
    sinh replaced by sin
  \item
    Sin/cos oscillation with exponential decay envelope
  \item
    Larger quality factor, Q, implies slower decay, and more
    oscillations
  \end{itemize}
\end{itemize}

\subsection{Driven Oscillators}\label{driven-oscillators}

\subsubsection{Oscillators Driven by an External Force}\label{oscillators-driven-by-an-external-force}

\begin{itemize}
\item
  Consider an undamped oscillator, driven by time dep external force,
  F(t)
\item
  Driving force is not part of the system
  \begin{itemize}
  \item
    do not consider force back on driver by system
  \end{itemize}
\end{itemize}

\[
    L = \underbrace{\frac{m\dot{q}^2}{2}}_{T} - \underbrace{\Bigg[\frac{m\omega^{2}q^2}{2} - F(t)q \Bigg]}_{V}
\] \[
    \ddot{q} + \omega^2 q = \frac{F(t)}{m}
\]

\begin{itemize}
\item
  Yield a second order linear, inhomogeneous differential eqn
\item
  Split into infinite number of impulsive forces and sum up
  displacements caused by each
\item
  Method is analogous to Poisson's eqn in electrostatics
  \begin{itemize}
  \item
    \(\nabla^{2}\Phi(x) = -\frac{\rho(r)}{\epsilon}\)
  \end{itemize}
\end{itemize}

\subsection{Dirac delta-fn}\label{dirac-delta-fn}

\begin{itemize}
\item
  Zero-width limit of a function/distribution of t, sharply peaked at
  \(t = t'\)
\item
  If t has dimension of time, \(\delta\) has dimension of inverse time,
  i.e. frequency
\end{itemize}

\[
    f(t') = \int_{-\infty}^{\infty} \delta(t - t')f(t)\,dt
\]

\begin{itemize}
\item
  Use this to find impulsive force at time t' as
  \(F(t) = K\delta(t - t')\)
\end{itemize}

\subsection{Response of oscillator to an indep impulsive force}\label{response-of-oscillator-to-an-indep-impulsive-force}

\[
    \int_{t' - \epsilon}^{t + \epsilon} (\ddot{q} + \omega^{2}q)dt \] \[
    \implies \dot{q}(t' + \epsilon) - \dot{q}(t' - \epsilon) + \underbrace{\omega^{2} \int_{t' - \epsilon}^{t + \epsilon} q(t)dt}_{\to 0 \text{ as } \epsilon \to 0, \text{ unless } q(t') \to \infty} = \frac{K}{m}
\]

\begin{itemize}
\item
  Instantaneous change in velocity of \(\frac{K}{m}\) from this force
  but no change in psn as no change in time
\end{itemize}

\paragraph{Evolution of SHO}\label{evolution-of-sho}

\begin{enumerate}
\item
  \(t < t'\) - normal SHO motion
\item
  \(t = t'\) - Instantaneous jump in velocity as above
\item
  \(t > t'\) - no more driving force, back to SHO
\end{enumerate}

\[
    q(t) = q(t')\cos(\omega[t - t']) + \frac{\dot{q}(t') + \tfrac{K}{m}}{\omega}\sin(\omega[t - t'])
\]

\subsection{Green's Function}\label{greens-function}

\begin{itemize}
\item
  Solution to differential equation:
\end{itemize}

\[
    \ddot{G}(t - t') + \omega^{2}G(t - t') = \delta(t - t') \] \[
    G(t - t') = \frac{1}{\omega}\sin(\omega[t - t']),~ t \geq t'
\]

\begin{itemize}
\item
  Using defn of \(\delta\)-fn, the eqn of motion for driven oscillator
  can be written as
\end{itemize}

\[
    \ddot{q} + \omega^{2}q = \frac{1}{m}\int_{-\infty}^{\infty} dt'\,F(t')\delta(t - t')
\]

\begin{itemize}
\item
  Just splitting up driving force into infinite number of impulsive
  forces
\item
  eliminate \(\delta\)-fn by using Green's fn
\end{itemize}

\[
    \ddot{q} + \omega^{2}q = \frac{1}{m}\int_{-\infty}^{\infty} dt'\,F(t')[\ddot{G}(t - t') + \omega^{2}G(t - t')] \]
\[
    q(t) = \frac{1}{m\omega} \int_{-\infty}^{t} F(t')G(t - t')dt' = \frac{1}{m\omega} \int_{-\infty}^{t} F(t')\sin(\omega[t - t'])dt'
\]

\begin{example}
\begin{itemize}
\item
  Consider driving force of the form
  \begin{itemize}
  \item
    \(F(t) = F_{0}\sin(\omega_{0}t)\) for \(t > 0\)
  \item
    \(F(t) = 0\) for \(t \leq 0\)
  \end{itemize}
\end{itemize}

\[
    q(t) = \frac{F_0}{m\omega}\int_{0}^{t} dt' \sin(\omega_{0}t')\sin(\omega(t - t'))
\]

\begin{itemize}
\item
  Use \(\sin A \sin B = \frac{1}{2} [\cos(A - B) - \cos(A + B)]\)
\end{itemize}

\[
    q(t) = \frac{F_0}{m\omega}\int_{0}^{t} \frac{1}{2}\Big[\cos[t'(\omega_{0} + \omega) - \omega t] - \cos[t'(\omega_{0} - \omega) + \omega t] \Big]dt
\]

\begin{itemize}
\item
  For \(\omega_{0} \neq \omega\), one observes oscillating behaviour
  with
\end{itemize}

\[
    q(t) = \frac{F_0}{m\omega}\Big[\frac{\sin(\omega_{0}t) + \sin(\omega t)}{2(\omega_{0} + \omega)} - \frac{\sin(\omega_{0}t) - \sin(\omega t)}{2(\omega_{0} - \omega)} \Big]
\]

\begin{itemize}
\item
  In resonance, (when \(\omega_{0} = \omega\)), q(t) grows without limit
\end{itemize}

\[
    q(t) = \frac{F_0}{m\omega}\Big[\underbrace{\frac{\sin(\omega t)}{2\omega}}_{\text{oscillates}} - \underbrace{\frac{t\cos(\omega t)}{2}}_{\text{grows}} \Big]
\]

\begin{itemize}
\item
  If we consider an underdamped SHO subject to the same driving force as
  above, then EoM can be solved to give
\end{itemize}

\[
    q(t) = q_{s}(t) + q_{t}(t)
\]

\begin{itemize}
\item
  Solution is split into the steady-state solution, \(q_{s}\), and the
  transient solution, \(q_{t}\).
  \begin{itemize}
  \item
    In this case, the Green's function for the damped oscillator is
  \end{itemize}
\end{itemize}

\[
    G(t - t') = \frac{e^{-\hat{\omega}(t - t')}}{\omega'}\sin[\omega'(t - t')], ~~ t \geq t' \]
\[
    w' = \hat{\omega}\sqrt{4Q^2 - 1} ~;~ \hat{\omega} = \frac{\omega}{2Q}
\]
\end{example}

\section{}\label{lecture-6}

\subsection{Recap}\label{recap}

\begin{itemize}
\item
  Green's function
  \begin{itemize}
  \item
    response of an oscillator to an impulsive force
  \end{itemize}
\item
  Split force up into lots of tiny impulsive forces
\end{itemize}

\[
    q(t) = \frac{1}{m\omega} \int_{-\infty}^{t} F(t')G(t - t')dt'
\]

\subsection{Underdamped Driven Oscillator Ctd}\label{underdamped-driven-oscillator-ctd}

\begin{itemize}
\item
  Displacement at time, t:
\end{itemize}

\[
    q(t) = \frac{1}{m} \int_{-\infty}^{t} F_0 \sin(\omega_{0}t')\frac{e^{-\hat{\omega}(t - t')}}{\omega'}\sin[\omega'(t - t')]dt'
\]

\begin{itemize}
\item
  To solve this, just write sins as
  \(\sin x = \frac{1}{2i}(e^{ix} - e^{-ix})\) and integrate
\item
  The steady-state soln is:
\end{itemize}

\[
    q_{s}(t) = \frac{F_0}{2m\omega'} \Bigg\{\Big[\frac{\hat{\omega}}{\hat{\omega}^2 + (\omega_{0} + \omega')^2} - \frac{\hat{\omega}}{\hat{\omega}^2 + (\omega_{0} - \omega')^2} \Big]\cos(\omega_{0}t)
\]

\[
    + \Big[\frac{\omega_{0} + \omega'}{\hat{\omega}^2 + (\omega_{0} +   \omega')^2} - \frac{\omega_{0} - \omega'}{\hat{\omega}^2 + (\omega_{0} - \omega')^2} \Big]\sin(\omega_{0}t) \Bigg\}
\]

\begin{itemize}
\item
  The transient state is:
\end{itemize}

\[
    q_{t}(t) = \frac{F_0 e^{-\hat{\omega}t}}{2m\omega'} \Bigg\{ \Big[\frac{\omega_{0} + \omega'}{\hat{\omega}^2 + (\omega_{0} + \omega')^2} + \frac{\omega_{0} + \omega'}{\hat{\omega}^2 - (\omega_{0} - \omega')^2} \Big]\sin(\omega't)
\] \[
    - \Big[\frac{\hat{\omega}}{\hat{\omega}^2 + (\omega_{0} + \omega')^2} - \frac{\hat{\omega}}{\hat{\omega}^2 + (\omega_{0} - \omega')^2} \Big]\cos(\omega't) \Bigg\} \]
\[
    q(t) = q_{s}(t) + q_{t}(t)
\]

\begin{itemize}
\item
  The transient soln oscillates with the intrinsic oscillator frequency,
  \(\omega'\), but dies away with time
\item
  The steady-state soln oscillates with driving frequency, and remains
  at large t
\item
  After the transient has died away, the dynamical behaviour is like
  that of SHO, but at the driving frequency
\item
  Note that there is no singularity at resonance

  \begin{itemize}
  \item
    The effect of resonance has been softened by the presence of damping
  \end{itemize}
\end{itemize}

\subsection{Coupled Small Oscillations}\label{coupled-small-oscillations}

\begin{itemize}
\item
  look at page 20 of notes
\item
  Consider system with two pendulums of length, l, connected half way
  down with a spring
\item
  Use \(\theta\)s as generalised coords
\item
  Use Taylor expansion of cos up to deg 2 to get potential expression
\end{itemize}

\[
    T = \frac{ml^2}{2}(\dot{\theta}_{1}^{2} + \dot{\theta}_{2}^{2}) \]
\[
    V = \frac{mgl}{2}(\theta_{1}^2 + \theta_{2}^2) \] \[
    V_{coupling} = \frac{k}{2}\Big(\frac{l}{2}\Big)^2 (\theta_2 - \theta_1)^2 \]
\[
    L = \frac{ml^2}{2}(\dot{\theta}_{1}^{2} + \dot{\theta}_{2}^{2}) - \frac{1}{2}\Bigg(mgl(\theta_{1}^2 + \theta_{2}^2) + k\Big(\frac{l}{2}\Big)^2 (\theta_2 - \theta_1)^2 \Bigg)
\]

\begin{itemize}
\item
  \(\theta_1 \theta_2\) makes things awkward, so transform to centre of
  mass, \(\theta_c = \frac{\theta_1 + \theta_2}{2}\) and relative,
  \(\theta_r = \theta_2 - \theta_1\)
\item
  Also define \(\omega_0 = \sqrt{\frac{g}{l}}\) and constant,
  \(\eta = \frac{kl}{4mg}\)
\end{itemize}

\[
    L = ml^2 \Bigg[(\dot{\theta}_{c}^2 - \omega_{0}^2\theta_{c}^2) + \frac{1}{4}(\dot{\theta}_{r}^2 - \omega_{0}^2 \Big(1 + 2\eta)\theta_{r}^2\Big) \Bigg] \]
\[
    \ddot{\theta}_{c} + \omega_{c}^2 \theta_{c} = 0 \] \[
    \ddot{\theta}_{r} + \omega_{0}^2 (1 + 2\eta)\theta_r = 0
\]

\begin{itemize}
\item
  \(\eta\) is essentially the relative strengths of the spring force to
  gravity
\item
  Total motion is linear combination of two EoMs above
\end{itemize}

\subsection{General Method}\label{general-method}

\begin{itemize}
\item
  N coupled oscillators gives N generalised coords, yielding N
  homogeneous eqns
\item
  If undergoing SHM, \emph{normal coordinates}
\end{itemize}

\begin{enumerate}
\item
  Find equilibrium config, values of generalised coords where
  generalised forces are 0
\item
  Taylor expand Lagrangian to second order in generalised coords, around
  values for the generalised coords given by equilibrium config
\item
  For N DoF, the Taylor expansion must in principal be in 2N variables
  \begin{itemize}
  \item
    In practice, fewer variables may be adequate
  \end{itemize}
\end{enumerate}

\begin{itemize}
\item
  Use Matrices
  \begin{itemize}
  \item
    page 21
  \item
    generalised coordinates are column vectors
  \end{itemize}
\end{itemize}

\[
    L = \underbrace{\underline{\dot{q}}^T \hat{\tau} \underline{\dot{q}}}_{T} - \underbrace{\underline{q}^T \hat{\upsilon}\underline{q}}_V + c
\]

\begin{itemize}
\item
  Elements of these matrices:
\end{itemize}

\[
    \tau_{jk} = \frac{1}{2} \frac{\partial^2 T}{\partial \dot{q}_j \partial \dot{q}_k}\Bigg|_{\dot{q}_{j},\dot{q}_k = 0} ~;~ \upsilon_{jk} = \frac{1}{2} \frac{\partial^2 V}{\partial \dot{q}_j \partial \dot{q}_k}\Bigg|_{q_{j},q_k = 0}
\]

\section{}\label{lecture-7}

\subsection{Recap from Last time}\label{recap-from-last-time}

\[
    T = \underline{\dot{x}} \hat{\tau} \underline{\dot{x}}
\] \[
    V = \underline{x}^T \hat{\upsilon} \underline{x}
\] \[
    L = T - V = \frac{1}{2}m(\dot{x}_{1}^2 + \dot{x}_{2}^2) - \frac{1}{2}k(x_{1}^2 + x_{2}^2 - 2x_{1}x_{2})
\]

\begin{itemize}
\item
  E-L: (\(x_1\))
\end{itemize}

\[
    m\ddot{x}_{1} - k(x_{1} - x_2) = 0
\]

\begin{itemize}
\item
  E-L: (\(x_2\))
\end{itemize}

\[
    m\ddot{x}_2 + k(x_2 - x_1) = 0
\]

\begin{itemize}
\item
  Other coordinates in each E-L is the coupling
\item
  Equivalent to
\end{itemize}

\[
    \begin{pmatrix} m & 0 \\ 0 & m \end{pmatrix}
    \begin{pmatrix}\ddot{x}_1 \\ \ddot{x}_2 \end{pmatrix} +
    \begin{pmatrix} k & -k \\ -k & k \end{pmatrix}
    \begin{pmatrix} x_1 \\ x_2 \end{pmatrix} = \underline{0}
\] \[
    2(\hat{\tau}\underline{\ddot{x}} + \hat{\upsilon}\underline{x})
\]

\begin{itemize}
\item
  Equation for coupled oscillators:
\end{itemize}

\[
    \hat{\tau}\underline{\ddot{q}} + \hat{\upsilon}\underline{q} = 0
\]

\subsection{Normal Modes}\label{normal-modes}

\begin{itemize}
\item
  Usual thing to do is guess a solution
  \begin{itemize}
  \item
    try \(\underline{q} = \underline{b}e^{i\omega t}\)
  \end{itemize}
\item
  This yields
\end{itemize}

\[
    (\hat{\upsilon} - \omega^{2}\hat{\tau})\underline{b}
\]

\begin{itemize}
\item
  Eigenvalue problem
  \begin{itemize}
  \item
    det must be zero
  \item
    \(\hat{\tau}\) must be a multiple of the identity matrix
  \end{itemize}
\item
  In the pendulum example,
  \(T = \frac{mgl}{2\omega_{0}^2}(\dot{\theta}_{1}^2 + \dot{\theta}_{2}^2)\)
\end{itemize}

\[
    \tau_{ij} = \frac{1}{2} \frac{\partial^2 T}{\partial \dot{\theta}_{i} \partial \dot{\theta}_{j}}
\] \[
    \tau_{11} = \frac{1}{2} \frac{\partial^2 T}{\partial \dot{\theta}_{1}^{2}} = \frac{mgl}{2\omega_{0}^2} ~;~ \tau_{12} = \frac{1}{2}\frac{\partial^2 T}{\partial \dot{\theta}_{1} \partial \dot{\theta}_{2}} = 0 = \tau_{21}
\] \[
    \tau_{22} = \tau_{11}
\] \[
    \hat{\tau} = \frac{mgl}{2\omega_{0}^2}\begin{pmatrix} 1 & 0 \\ 0 & 1 \end{pmatrix}
\] \[
    \omega^2 \hat{\tau} = \frac{mgl}{2}\begin{pmatrix} \lambda & 0 \\ 0 & \lambda \end{pmatrix} ~;~ \lambda = \Big(\frac{\omega}{\omega_{0}} \Big)^2
\]

\begin{itemize}
\item
  page 21
\item
  find eigenvalues from det of matrix above
\item
  Gives
\end{itemize}

\[
    \frac{b_{(j)2}}{b_{(j)1}} = \frac{1 + \eta - \lambda_{j}}{\eta}
\]

\subsection{Normal Coordinates}\label{normal-coordinates}

\[
    r_j = \sum_{k = 1}^{N} b_{(j)k}q_{k}
\]

\begin{itemize}
\item
  Note that for coupled pendulum example
  \(r_1 \propto (\theta_1 + \theta_2)\) and
  \(r_2 \propto (\theta_2 - \theta_1)\)
\item
  Any general motion is a superposition of the normal modes

  \begin{itemize}

  \item
    similar to eigenfunctions in Quantum
  \end{itemize}
\end{itemize}

\subsection{Mode Orthogonality}\label{mode-orthogonality}

\[
    \underline{b}_{i}^{T} \hat{\tau} \underline{b}_j = 0 ~;~ i \neq j
\]

\begin{enumerate}
\item
  If \(\omega_{i}\)s are all different, then \(b_i\)s are unique
\item
  If some \(\omega_i\)s are the same then there is a choice of \(b_i\)s
  but can always find orthogonal ones
\end{enumerate}

\begin{example}[Free Vibrations of a linear triatomic molecule]

\begin{itemize}
\item
  Central atom of mass, M and outer atoms have masses, m
  \begin{itemize}
  \item
    constrained to move on an axis
  \item
    connected together via springs of constant, k and natural length, l
  \item
    point masses
  \item
    Consider \(x_1, x_2, x_3\)
  \end{itemize}
\end{itemize}

\[
    T = \frac{m}{2}(\dot{x}_{1}^2 + \dot{x}_{3}^2) + \frac{M}{2}\dot{x}_{2}^2
\] \[
    \hat{\tau} = \frac{1}{2}\begin{pmatrix}m & 0 & 0 \\ 0 & M & 0 \\ 0 & 0 & m \end{pmatrix} \]
\[
    V = \frac{k}{2}\Big[(x_2 - x_1 - l)^2 + (x_3 - x_2 - l)^2 \Big] \]
\[
    \hat{\upsilon} = \frac{1}{2} \frac{\partial^2 V}{\partial x_i \partial x_j} = \frac{k}{2} \begin{pmatrix} 1 & -1 & 0 \\ -1 & 2 & -1 \\ 0 & -1 & 1 \end{pmatrix} \]
\[
    \hat{\tau}\underline{\ddot{q}} + \hat{\upsilon}\underline{q} = \underline{0}
\]

\begin{itemize}
\item
  Trial solution: \(\underline{q} = \underline{b}e^{i\omega t}\)
\end{itemize}

\[
    (\hat{\upsilon} - \omega^2 \hat{\tau})\underline{b} = 0 \] \[
    \begin{pmatrix} k - \omega^2 m & -k & 0 \\ -k & 2k - \omega^2 M & -k \\ 0 & -k & k - \omega^2 m \end{pmatrix} \begin{pmatrix} b_1 \\ b_2 \\ b_3 \end{pmatrix} = 0
\]

\begin{itemize}
\item
  The normal modes will oscillate with frequencies satisfying
\end{itemize}

\[
    \begin{vmatrix} k - \omega^2 m & -k & 0 \\ -k & 2k - \omega^2 M & -k \\ 0 & -k & k - \omega^2 m \end{vmatrix} = 0
\]

\begin{itemize}
\item
  Do some algebra and you get
\end{itemize}

\[
    \omega^2 (k - \omega^2 m)[k(M + 2m) - \omega^2 Mm] = 0
\]

\begin{itemize}
\item
  Therefore, the normal modes oscillate with frequencies
  \begin{itemize}
  \item
    \(\omega_{(1)} = 0\)
  \item
    \(\omega_{(2)} = \sqrt{\frac{k}{m}}\)
  \item
    \(\omega_{(3)} = \sqrt{\frac{k}{m}\big(1 + \frac{2m}{M}\big)}\)
  \end{itemize}
\item
  The corresponding, unnormalised mode vectors are found by substituting
  these eigenvalues into the matrix above
\item
  Doing this, you find that they are:
\end{itemize}

\[
    b_{(1)} = \begin{pmatrix} 1 \\ 1 \\ 1 \end{pmatrix} ~;~ b_{(2)} = \begin{pmatrix} -1 \\ 0 \\ 1 \end{pmatrix} ~;~ b_{(3)} = \begin{pmatrix} 1 \\ -\frac{2m}{M} \\ 1 \end{pmatrix}
\]

\begin{itemize}
\item
  Three different states:
  \begin{enumerate}
  \item
    Just a translation
    \begin{itemize}
    \item
      could have constrained the centre of mass to remain at origin and
      solve in 2 DoFs
    \end{itemize}
  \item
    The two outer atoms are oscillating in anti-phase with big M static
  \item
    The m atoms are in phase with one another, and anti-phase with M
    atom, which has a relative amplitude of \(\frac{2m}{M}\) so as to
    keep the centre of mass fixed
  \end{enumerate}
\end{itemize}
\end{example}

\section{}\label{lecture-8}

\subsection{Central Forces}\label{central-forces}

\subsubsection{Two Interacting Bodies}\label{two-interacting-bodies}

\begin{itemize}
\item
  Consider two point masses with positions \(\underline{r}_1\) and
  \(\underline{r}_2\)
\item
  Connected via central force between them
  \begin{itemize}
  \item
    Interaction potential
  \item
    Potential depends on distance between bodies in straight line
  \item
    \(V_12 = V_{12}(\underline{r}_1,\underline{r}_2) = V_{12}(|\underline{r}_1 - \underline{r}_2|) = V(r)\)
  \end{itemize}
\end{itemize}

\[
    L = \frac{1}{2}(m_1 \dot{\underline{r}}_{1}^2 + m_2 \dot{\underline{r}}_{2}^2) - \underbrace{V_1(\underline{r}_1) - V_2(\underline{r}_2)}_{\text{external potentials}} - \underbrace{V_{12}(|\underline{r}_1 - \underline{r}_2|)}_{\text{interaction potential}}
\]

\begin{itemize}
\item
  Consider an infinitesimal translation \(\epsilon\underline{a}\) and
  Taylor expand Lagrangian to first order in \(\epsilon\)
  \begin{itemize}
  \item
    page 25 for the maths
  \item
    canonically conjugate momentum is first expression of E-L equation
    \begin{itemize}
    \item
      in this case, the linear momentum
    \end{itemize}
  \item
    this comes from the translational invariance of the Lagrangian
  \end{itemize}
\end{itemize}

\subsection{Ignorable CoM}\label{ignorable-com}

\begin{itemize}
\item
  Conservation of total momentum implies that CoM
  (\(\underline{R} = \frac{m_1 \underline{r}_1 + m_2 \underline{r}_2}{(m_1 + m_2)}\))
  moves like a free particle
  \begin{itemize}
  \item
    rewrite Lagrangian in terms of \(\underline{R}\) and the relative
    coordinate, \(\underline{r} = \underline{r}_1 - \underline{r}_2\)
  \end{itemize}
\end{itemize}

\[
    L = \frac{1}{2}M\dot{\underline{R}}^2 + \frac{1}{2}\mu \underline{\dot{r}}^2 - V_{12}(r) \]
\[
    \mu = \frac{m_1 m_2}{m_1 + m_2} \] \[
    M = m_1 + m_2
\]

\begin{itemize}
\item
  Check that
\end{itemize}

\[
    \frac{1}{2}M\dot{\underline{R}}^2 + \frac{1}{2}\mu \underline{\dot{r}}^2 = \frac{1}{2}(m_1 \dot{\underline{r}}_{1}^2 + m_2 \dot{\underline{r}}_{2}^2) \]
\[
    LHS = \frac{1}{2} \Bigg[(m_1 + m_2)\Big(\frac{m_1 \underline{\dot{r}}_1 + m_2 \underline{\dot{r}}_2}{m_1 + m_2}\Big)^2 + \frac{m_1 m_2}{m_1 + m_2}(\underline{\dot{r}}_1 = \underline{\dot{r}}_2)^2 \Bigg] \]
\[
    LHS = \frac{1}{2(m_1 + m_2)} \Bigg[m_{1}^2 \underline{\dot{r}}_{1}^2 + m_{2}^2 \underline{\dot{r}}_{2}^2 + 2m_1 m_2 \underline{\dot{r}}_1 \underline{\dot{r}}_2 + m_1 m_2 (\underline{\dot{r}}_{1}^2 + \underline{\dot{r}}_{2}^2 - 2\underline{\dot{r}}_1 \underline{\dot{r}}_2) \Bigg] \]
\[
    LHS = \frac{1}{2(m_1 + m_2)} \Bigg[m_1 \underline{\dot{r}}_{1}^2 (m_1 + m_2) + m_2 \underline{\dot{r}}_{2}^2 (m_1 + m_2) \Bigg] \]
\[
    LHS = \frac{1}{2}(m_1 \dot{\underline{r}}_{1}^2 + m_2 \dot{\underline{r}}_{2}^2) = RHS
\]

\subsection{Rotational Invariance}\label{rotational-invariance}

\begin{itemize}
\item
  Transform to the centre of mass frame and remove first expression from
  Lagrangian
  \begin{itemize}
  \item
    potential on depends on distance so we get rotational invariance
  \end{itemize}
\end{itemize}

\[
    \underline{\dot{r}}^2 = \dot{x}^2 + \dot{y}^2 + \dot{z}^2 \] \[
    \underline{\dot{r}}^2 = \dot{r}^2 + r^2\dot{\theta}^2 + (r\sin\theta)^2\dot{\phi}^2
\] \[
    L = \frac{1}{2}\mu \underline{\dot{r}}^2 - V_{12}(r) \] \[
    L = \frac{1}{2}\mu [\dot{r}^2 + r^2 \dot{\theta}^2 + r^2 \dot{\phi}^2\sin^2\theta] - V_{12}(r)
\]

\begin{itemize}
\item
  \(\phi\) is ignorable, as it does not appear in L, therefore:
\end{itemize}

\[
    p_{\phi} \equiv \frac{\partial L}{\partial \dot{\phi}} = \mu r^2\dot{\phi}\sin^2\theta = \text{constant} \equiv \underbrace{J_z}_{\text{z component of } \underline{J} = \underline{r} \times \mu\dot{\underline{r}}}
\]

\begin{itemize}
\item
  Can choose coordinate system such that total angular momentum,
  \(\underline{J}\), lies along the z-axis
  \begin{itemize}
  \item
    Therefore, \(\underline{J}\) is always constant
  \end{itemize}
\end{itemize}

\[
    \underline{J} = \underline{r} \times \mu\underline{\dot{r}} ~;~ \underline{r} = \begin{pmatrix} r\cos\phi\sin\theta \\ r\sin\phi\sin\theta \\ r\cos\theta \end{pmatrix} \]
\[
    \underline{\dot{r}} = \begin{pmatrix} \dot{r}\cos\phi\sin\theta - r\dot{\phi}\sin\phi\sin\theta + r\dot{\theta}\cos\phi\cos\theta \\ \dot{r}\sin\phi\sin\theta + r\dot{\phi}\cos\phi\sin\theta + r\dot{\theta}\sin\phi\cos{\theta} \\ \dot{r}\cos\theta - r\dot{\theta}\sin\theta \end{pmatrix} \]
\[
    \underline{J} = \mu \begin{vmatrix} \underline{i} & \underline{j} & \underline{k} \\ r_i & r_j & r_k \\ \dot{r}_i & \dot{r}_j & \dot{r}_k \end{vmatrix} = \mu \begin{pmatrix} r_j \dot{r}_k - r_k\dot{r}_j \\ r_k\dot{r}_i - r_i\dot{r}_k \\ r_i\dot{r}_j - r_j\dot{r}_i \end{pmatrix}
\]

\begin{itemize}
\item
  Plug in \(\underline{\dot{r}}\):
\end{itemize}

\[
    J_z = \mu r\sin\theta [r\dot{\phi}\sin\theta] = \mu r^2 \dot{\phi} \sin^2\theta
\]

\begin{itemize}
\item
  \(\underline{J} = \underline{r} \times \mu\underline{\dot{r}}\), so
  \(\underline{p}\) and \(\underline{r}\) must be perpendicular to
  \(\underline{J}\)
\item
  Hence, these must be in a plane perpendicular to \(\underline{J}\),
  and remain in that plain as \(\underline{J}\) is constant
\item
  Choose \(\underline{J}\) to lie along z direction, so that angle of
  elevation,
\end{itemize}

\[
    \theta = \frac{\pi}{2} \] \[
    \dot{\theta} = 0 \] \[
    J = |\underline{J}| = J_z = \mu r^2 \dot{\phi}
\]

\section{}\label{lecture-9}

\subsection{Kepler's Second Law}\label{keplers-second-law}

\[
    \frac{dA}{dt} = \frac{1}{2}r^2 \dot{\phi} = \frac{J}{2\mu}
\]

\begin{itemize}
\item
  J is conserved, so the rate at which area is swept out is constant
\item
  Kepler's second law comes from this
  \begin{itemize}
  \item
    nothing to do much with gravity specifically
  \end{itemize}
\end{itemize}

\subsection{Equivalent 1D Problem}\label{equivalent-1d-problem}

\begin{itemize}
\item
  In this case, remove \(\theta\) and \(\phi\) from Lagrangian
\end{itemize}

\[
    E = T + V = \frac{1}{2}\mu[\dot{r}^2 + r^2\dot{\theta}^2 + r^2\dot{\phi}^2\sin^2\theta] + V_{12}(r) \]
\[
    \dot{\theta} = 0 ~;~ \theta = \frac{\pi}{2} ~;~ \dot{\phi} = \frac{J}{\mu r^2} \]
\[
    E = \frac{1}{2}\mu \dot{r}^2 + \underbrace{\frac{J^2}{2\mu r^2} + V_{12}(r)}_{V_{eff}(r)}
\]

\begin{itemize}
\item
  Effective potential combines interaction potential with angular
  momentum barrier
\item
  total energy E is the same as that corresponding to an effective
  Lagrangian
  \begin{itemize}
  \item \(L_{eff} = \frac{1}{2}\mu \dot{r}^2 - V_{eff}\) yields:
  \end{itemize}
\end{itemize}

\[
    \mu \ddot{r} = \frac{J^2}{\mu r^3} - \frac{\partial V_{12}}{\partial r}
\]

\begin{example}[Central Forces]
\begin{itemize}
\item
  Find the orbit shape for the following potential:
\end{itemize}

\[
    V(r) = \begin{cases} -V_0 & r \leq R \\ 0 & r > R \end{cases} \implies V_{eff}(r) = \begin{cases} \frac{J^2}{2\mu r^2} - V_0 & r \leq R \\ \frac{J}{2\mu r^2} & r > R \end{cases}
\]

\begin{itemize}
\item
  If \(\frac{J^2}{2\mu R^2} - V_0 < E < \frac{J^2}{2\mu R^2}\), then we
  have a bound orbit
  \begin{itemize}
  \item
    \(r_t\) there for E
  \item
    \(r_t \leq r \leq R\)
  \end{itemize}
\end{itemize}

\[
    r_{t}^2 = \frac{J^2}{2\mu (E + V_0)}
\]

\begin{itemize}
\item
  Set the azimuthal angle, \(\phi = 0\), and \(r = r_t\) at \(t = 0\),
  and find the shape of the orbit:
\end{itemize}

\[
    E = \frac{1}{2}\mu\dot{r}^2 + \frac{J^2}{2\mu r^2} - V_0,~ r \leq R ~~ (1)
\]

\begin{itemize}
\item
  Define \(\mu = \frac{1}{r}\), then
  \(\frac{dr}{dt} = -\frac{1}{u^2}\frac{du}{dt}\)
\end{itemize}

\[
    \dot{\phi} = \frac{d\phi}{dt} = \frac{J}{\mu r^2} = \frac{Ju^2}{\mu} \]
\[
    \frac{dr}{dt} = -\frac{1}{u^2}\frac{Ju^2}{\mu}\frac{du}{d\phi} \] \[
    \to (1): \implies \] \[
    E + V_0 - \frac{J^2 u^2}{2\mu} = \frac{\mu}{2}\frac{J^2}{\mu^2}\Big(\frac{du}{d\phi}\Big)^2 \]
\[
    \frac{du}{d\phi} = \pm \sqrt{\underbrace{\frac{2\mu}{J^2}(E + V_0)}_{\frac{1}{r_{t}^2} = u_{t}^2} - u^2} \]
\[
    \int_{U_0}^{U} -\frac{du}{\sqrt{u_{t}^2 - u^2}} = \int_{\phi_0}^{\phi} d\phi = \phi
\]

\begin{itemize}
\item
  Use substitution:
  \(u = u_t \cos\alpha, du = -u_t\sin\alpha \,d\alpha\)
\end{itemize}

\[
    \phi = \alpha + \alpha_0, ~~ t = 0 \implies \alpha_0 = \cos^{-1}\Big(\frac{u_t}{u_t}\Big) = 0 \]
\[
    \phi = +\cos^{-1}\Big(\frac{r_t}{r}\Big)
\]

\begin{itemize}
\item
  Sign is set by sign of angular momentum, J
\end{itemize}

\[
    \cos\phi = \frac{r_t}{r} \implies r = r_t\sec\phi,~ \phi < \Phi,~ R = r_t \sec\Phi
\]
\end{example}

\subsection{Gravitational Attraction}\label{gravitational-attraction}

\begin{itemize}
\item
  For gravity, we have the differential to be solved as:
\end{itemize}

\[
    \mu \ddot{r} = \frac{J^2}{\mu r^3} - \frac{k}{r^2},~ k = Gm_{1}m_{2}
\]

\begin{itemize}
\item
  solving this to determine orbits of bodies
\end{itemize}

\subsection{\texorpdfstring{The \(u = \frac{1}{r}\) transformation}{The u = \textbackslash{}frac\{1\}\{r\} transformation}}\label{the-u-frac1r-transformation}

\begin{itemize}
\item
  See last example for this transformation or page 28
\end{itemize}

\[
    \frac{d^2 u}{d\phi^2} + u = \frac{\mu k}{J^2} \] \[
    \implies \frac{1}{r} \equiv u = \frac{\mu k}{J^2} + B\cos\phi
\]

\subsection{Elliptical Orbits}\label{elliptical-orbits}

\begin{itemize}
\item
  Define \(p \equiv \frac{J^2}{\mu k}\) and
  \(\epsilon \equiv pB \implies\)
\end{itemize}

\[
    E = \frac{\mu k^2}{2J^2}(\epsilon^2 - 1) = \frac{k}{2p}(\epsilon^2 - 1)
\]

\begin{itemize}
\item
  For \(0 \leq \epsilon < 1\), bound elliptical orbit
\item
  page 29
\item
  \(\epsilon\) is the eccentricity of the elliptical orbit:
  \begin{itemize}
  \item
    \(\epsilon = 0 \implies\) circular orbit
  \item
    \(\epsilon = 1 \implies\) parabolic trajectory
  \item
    \(\epsilon > 1 \implies\) hyberbola
  \end{itemize}
\item
  elliptical orbits come from defining the \(\frac{1}{r^2}\) potential
\end{itemize}

\subsection{Third Law}\label{third-law}

\[
    T = 2\pi \sqrt{\frac{1}{G(m_1 + m_2)}a^{3/2}} \implies T^2 \propto a^2
\]

\begin{itemize}
\item
  Same proportionality constant for all planets
  \begin{itemize}
  \item
    This is approximately true as the mass of the sun largely cancels
    out the mass of the planet
  \end{itemize}
\end{itemize}

\textbf{See page 29 and 30 for Bertrand's Theorem and Quadrature}

\begin{example}[Kepler's Equation]

\begin{itemize}
\item
  Given a set of observations of positions of a solar system body, how
  would one infer the orbit of that body?
\item
  From the equation for an ellipse in the notes (Eq. 22), we can use the
  identity \(\cos^2\psi +\sin^2\psi = 1\) to introduce the
  parameterisation:
\end{itemize}

\[
    x + \frac{\epsilon p}{1 - \epsilon^2} = a\cos\psi \implies x = a(\cos\psi - \epsilon) \]
\[
    y = b\sin\psi \implies y = a\sqrt{1 - \epsilon^2}\sin\psi
\]

\begin{itemize}
\item
  It follows that:
\end{itemize}

\[
    r = \sqrt{x^2 + y^2} = a(1 - \epsilon\cos\psi)
\]
\end{example}

\section{}\label{lecture-10}

\subsection{Kepler's Equation Continued}\label{keplers-equation-continued}

\begin{itemize}
\item
  Relationship of the eccentric anomaly, \(\psi\), to the true anomaly,
  \(\phi\), for an effective particle placed a distance r and azimuthal
  angle \(\phi\) from the centre of mass
  \begin{itemize}
  \item
    The elliptical orbit, with semi-major axis a, has been placed inside
    an imaginary circle of radius a
  \end{itemize}
\item
  We can solve the effective 1D problem by quadrature
  \begin{itemize}
  \item
    Eq 17 in notes, page 27, implies:
  \end{itemize}
\end{itemize}

\[
    \int dt' = \sqrt{\frac{\mu}{2}} \int \frac{dr'}{\sqrt{E + \tfrac{k}{r'} - \tfrac{J^2}{2\mu r'^2}}}
\]

\begin{itemize}
\item
  For elliptical orbits, \(E = -\frac{k}{2a}\)
  \begin{itemize}
  \item
    Also use \(\frac{J^2}{\mu k} \equiv p\)
  \item
    \(p = a(1 - \epsilon^2)\)
  \end{itemize}
\end{itemize}

\[
    \int dt' = \sqrt{\frac{\mu}{2k}} \int \frac{r'\,dr'}{\sqrt{-\tfrac{r'^2}{2a} + r' - \tfrac{a}{2}(1 - \epsilon^2)}}
\]

\begin{itemize}
\item
  Finally, use \(r = a(1 - \epsilon\cos\psi)\)
  \begin{itemize}
  \item
    \(dr = a\epsilon\sin\psi\,d\psi\)
  \end{itemize}
\end{itemize}

\[
    \int dt' = \sqrt{\frac{\mu}{2k}} \int \frac{a^2\epsilon\sin\psi'(1 - \epsilon\cos\psi')\,d\psi'}{-\tfrac{a}{2}(1 - \epsilon\cos\psi')^2 + a(1 - \epsilon\cos\psi') - \tfrac{a}{2}(1 - \epsilon^2)} \]
\[
    \int dt' = \sqrt{\frac{\mu}{2k}} \int \frac{a^2\epsilon\sin\psi' (1 - \epsilon\cos\psi')\,d\psi'}{\sqrt{\tfrac{a}{2}}(\epsilon\sin\psi')} \]
\[
    \int dt' = \sqrt{\frac{\mu a^3}{k}} \int (1 - \epsilon\cos\psi')\,d\psi'
\]

\begin{itemize}
\item
  This integral measures the time elapsed between initial and final
  value of this eccentric anomaly, \(\psi'\)
  \begin{itemize}
  \item
    Customary to choose \(t' = 0\) for when \(\psi' = \phi' = 0\)
    \begin{itemize}
    \item
      when the orbiting body is at its perihelion
    \item
      i.e. closest to the sun
    \end{itemize}
  \end{itemize}
\item
  Hence, if \(t(\psi)\) is the time elapsed from perihelion to when the
  eccentric anomaly takes the values \(\psi\), then we have Kepler's
  equation
\end{itemize}

\[
    t(\psi) = \sqrt{\frac{\mu a^3}{k}} (\psi - \epsilon\sin\psi)
\]

\begin{itemize}
\item
  If we measure elapsed times and radii since perihelion, then Kepler's
  equation and the definition of an ellipse
  (\(r = a(1 - \epsilon\cos\psi)\)) can be solved to give the semi-major
  axis, a, and the eccentricity, \(\epsilon\)
\end{itemize}

\subsection{Hamiltonian Mechanics}\label{hamiltonian-mechanics}

\begin{example}[Swinging Atwood Machine]

\begin{itemize}
\item
  Atwood machine with slight difference
  \begin{itemize}
  \item
    One mass moves like a normal pulley, other like a pendulum
  \item
    \(DoFs = 2\)
  \end{itemize}
\end{itemize}
\end{example}

\subsection{Noether's Theorem and Invariance of L}

\emph{``If the Lagrangian is invariant under a continuous symmetry
transformation, then there are conserved quantities associated with that
symmetry, one for each parameter of the transformation. These can be
found by differentiating each coordinate wrt to the parameters of the
transformation in the immediate neighbourhood of the identity
transformation, multiplying by the conjugate momentum, and summing over
the degrees of theorem.''}

\begin{itemize}
\item
  page 31 for maths
\item
  point mass moving in free space
\end{itemize}

\[
    L = \frac{1}{2}m\dot{q}^2 \] \[
    Q = q + s \] \[
    \frac{dQ}{ds}\Bigg|_{s_{} = 0} = 1 \implies I(q) = m\dot{q}
\]

\begin{itemize}
\item
  Consider phase space approach of 6D instead of just normal space
  \begin{itemize}
  \item
    normal dimensions plus the canonically conjugate momenta
  \end{itemize}
\end{itemize}

\subsection{Legendre Transformations}\label{legendre-transformations}

\begin{itemize}
\item
  \(A(x,y)\), intro third variable z
  \begin{itemize}
  \item
    \(B(x,y,z) \equiv yz - A(x,y)\)
  \item
    use chain rule
  \item
    page 32
  \item
    y can be written in terms of x and z
  \item
    \(B = B(x,y(x,z),z) = B(x,z) = yz - A(x,y)\)
  \end{itemize}
\end{itemize}

\[
    \frac{\partial B}{\partial z}\Bigg|_x = y_{} ~;~ \frac{\partial B}{\partial x}\Bigg|_z = -\frac{\partial A}{\partial x}\Bigg|_{y}
\]

\begin{itemize}
\item
  Similar to thermodynamics:
  \begin{itemize}
  \item
    Enthalpy, X, such that
  \item
    Use for isentropic and isobaric processes
  \end{itemize}
\end{itemize}

\[
    dX = T\,dS + V\,dP
\]

\begin{itemize}
\item
  However for isothermal and isobaric processes, better to use the Gibbs
  energy:
\end{itemize}

\[
    \underbrace{G = X - TS}_{\text{Legendre transformation}} \to dG = S\,dT + V\,dp
\]

\subsection{The Hamiltonian}\label{the-hamiltonian}

\[
    H(q, p) \equiv p\dot{q} - L(q,\dot{q}) ~;~ p \equiv \frac{\partial L}{\partial \dot{q}}\Bigg|_{constant\, q_{}} \]
\[
    \frac{\partial H}{\partial p}\Bigg|_{q} = \dot{q} ~;~ \frac{\partial H}{\partial q}\Bigg|_{p} = -\frac{\partial L}{\partial q}\Bigg|_{\dot{q}} = -\frac{d}{dt}\frac{\partial L}{\partial \dot{q}}\Bigg|_{q} = -\dot{p}
\]

\section{}\label{lecture-11}

\subsection{Hamilton's Principle Continued}\label{hamiltons-principle-continued}

\begin{itemize}
\item
  \(S = \int L\,dt \implies \delta S = S[q + \delta q]\)
  \begin{itemize}
  \item
    \(S[q] = \int \delta L\,dt\)
  \end{itemize}
\item
  page 33 for maths from the following:
\end{itemize}

\[
    \delta L = \dot{q}\delta p + p\delta\dot{q} - \underbrace{\Big(\frac{\partial H}{\partial q}\delta q + \frac{\partial H}{\partial q}\delta p\Big)}_{\delta H}
\]

\begin{itemize}
\item
  \(\int \delta L\,dt = 0\)
\item
  independent infinitesimal variations in \(\delta q, \delta p\) from
  physical space in coordinate-momentum space (phase space) do not
  change the action, S
\end{itemize}

\subsection{Hamilton's Equations of Motion}\label{hamiltons-equations-of-motion}

\begin{itemize}
\item
  For N DoF, Hamiltonian is defined using the Legendre transformation:
\end{itemize}

\[
    H \equiv \Bigg(\sum_{k = 1}^{N} p_{k}\dot{q}_k \Bigg) - L
\]

\begin{itemize}
\item
  L may now be explicitly time-dependent
\item
  Include a time differential to give
\end{itemize}

\[
    dH = \sum_{k = 1}^{N} \Big(\dot{q}_k dp_k - \dot{p}_kdq_k\Big) - \frac{\partial L}{\partial t}dt
\]

\begin{itemize}
\item
  page 33 for more
\end{itemize}

\[
    \frac{\partial H}{\partial t} = -\frac{\partial L}{\partial t} \] \[
    \dot{q}_k = \frac{\partial H}{\partial p_k} ~;~ \dot{q}_k = -\frac{\partial H}{\partial q_k} ~;~ \frac{\partial H}{\partial t} = -\frac{\partial L}{\partial t}
\]

\subsection{Examples in Hamiltonians}\label{examples-in-hamiltonians}

\begin{example}[1]

\begin{itemize}
\item
  particle of mass, m, moving vertically in a uniform gravitational
  field
\end{itemize}

\[
    T = \frac{1}{2}m\dot{z}^2 ~;~ V = mgz \] \[
    L = T - V = \frac{1}{2}m\dot{z}^2 - mgz
\]

\begin{itemize}
\item
  The canonically conjugate momentum to coordinate z,
\end{itemize}

\[
    p_z = \frac{\partial L}{\partial \dot{z}} = m\dot{z}
\]

\begin{itemize}
\item
  Legendre transformation of \(L(z,\dot{z})\) to give \(H(z, p_z)\):
\end{itemize}

\[
    H(z, p_z) = p_z\dot{z} - L(z,\dot{z}) \] \[
    H(z, p_z) = p_z \cdot \frac{p_z}{m} - \frac{m}{2} \Big(\frac{p_z}{m}\Big)^2 + mgz \]
\[
    H(z, p_z) =  \frac{p_{z}^2}{2m} + mgz = T + V = E_T s
\]

\begin{itemize}
\item
  EoM:
\end{itemize}

\[
    \dot{z} = \frac{\partial H}{\partial p_z} = \frac{p_z}{m} ~;~ \dot{p}_z = -\frac{\partial H}{\partial z} = -mg
\]

\begin{itemize}
\item
  Integrate \(\dot{p}_z\) first to find \(p_z\) as a function of time
  and then integrate \(\dot{z}\) using this to find z as a function of
  time
\end{itemize}
\end{example}

\begin{example}[2]

\begin{itemize}
\item
  particle of mass m moves in gravitational field, along the spiral of
  the form, \(z = k\theta\) and \(r = cst\), where k is a constant and z
  is the vertical direction
\item
  Find the Hamiltonian, \(H(z, p)\), and the equation of motion
\item
  Show in the limit \(r \to 0\), that \(\ddot{z} \to -g\)
\end{itemize}

\[
    T = \frac{1}{2}m(\dot{x}^2 + \dot{y}^2 + \dot{z}^2) ~;~ x = r\cos\theta,~ y = r\sin\theta,~ z = k\theta \]
\[
    \dot{x} = -r\dot{\theta}\sin\theta ~;~ \dot{y} = r\dot{\theta}\cos\theta ~;~ \dot{z} = k\dot{\theta} \]
\[
    T = \frac{1}{2}m (r^2\dot{\theta}^2 + \dot{z}^2) = \frac{1}{2}m\dot{z}^2 \Big(1 + \frac{r^2}{k^2}\Big) ~;~ V = mgz \]
\[
    L = \frac{1}{2}m\dot{z}^2 \Big(1 + \frac{r^2}{k^2}\Big) - mgz
\]

\begin{itemize}
\item
  Canonically conjugate momentum to z:
\end{itemize}

\[
    p = \frac{\partial L}{\partial \dot{z}} = m\dot{z}\Big(1 + \frac{r^2}{k^2}\Big) \implies \dot{z} = \frac{p}{m\big(1 + \tfrac{r^2}{k^2}\big)} \]
\[
    H = p\dot{z} - L = \frac{p^2}{m\big(1 + \tfrac{r^2}{k^2}\big)} - \frac{p^2}{2m\big(1 + \tfrac{r^2}{k^2}\big)} + mgz \]
\[
    H(z, p) = \frac{p^2}{2m\big(1 + \tfrac{r^2}{k^2}\big)} + mgz = T + V = E_T
\]

\begin{itemize}
\item
  EoM:
\end{itemize}

\[
    \dot{z} = \frac{\partial H}{\partial p} = \frac{p}{m\big(1 + \tfrac{r^2}{k^2}\big)} ~;~ \dot{p} = -\frac{\partial H}{\partial z} = -mg
\]

\begin{itemize}
\item
  Integrating:
\end{itemize}

\[
    p(t) = p(0) - mgt
\]

\begin{itemize}
\item
  Put this into the other equation:
\end{itemize}

\[
    z = \int_{0}^{t} \frac{p(0) - mgt'}{m\big(1 + \tfrac{r^2}{k^2}\big)} dt' = z(0) + \frac{p(0)t}{m\big(1 + \tfrac{r^2}{k^2}\big)} - \frac{gt^2}{2\big(1 + \tfrac{r^2}{k^2}\big)}
\]

\begin{itemize}
\item
  Find expression for \(\ddot{z}\):
\end{itemize}

\[
    \ddot{z} = \frac{\dot{p}}{m\big(1 + \tfrac{r^2}{k^2}\big)} = \frac{-g}{\big(1 + \tfrac{r^2}{k^2}\big)}
\]

\begin{itemize}
\item
  As \(r \to 0\), \(\ddot{z} \to -g\) as denominator tends to 1
\end{itemize}
\end{example}

\begin{example}[3]
\begin{itemize}
\item
  Stretched Spring attached to a uniformly moving cart
\item
  passes \(x = 0\) at \(t = 0\)
\item
  cart moving at speed, \(v_0\)
\item
  unstretched length of the spring is negligible
\end{itemize}

\[
    L(x, \dot{x}, t) = T - V = \frac{m\dot{x}^2}{2} - \frac{k}{2}(x - v_0t)^2 \]
\[
    E-L \implies m\ddot{x} = -k(x - v_0t),~[x' = x - v_0t] \] \[
    \ddot{x}' = \ddot{x} ~;~ m\ddot{x}' = -kx'
\]

\begin{itemize}
\item
  SHM in the cart's frame of reference
\end{itemize}

\[
    p_x = \frac{\partial L}{\partial \dot{x}} = m\dot{x} \] \[
    H(x, p_x) = p_x\dot{x} - \frac{m\dot{x}^2}{2} + \frac{k}{2}(x - v_0t)^2 = \frac{p_{x}^2}{2m} + \frac{k}{2}(x - v_0t)^2 = T + V = E_T
\]

\begin{itemize}
\item
  H explicitly depends on t, therefore it is not conserved
\item
  To keep the cart moving uniformly despite the oscillating mass, energy
  flows in and out of the system
\item
  However, in terms of \(x' = x - v_0t\), \(\dot{x}' = \dot{x} = v_0\):
\end{itemize}

\[
    L(x', \dot{x}') = \frac{m\dot{x}'^2}{2} + m\dot{x}'v_0 + \frac{mv_{0}^2}{2} - \frac{kx'^2}{2} \]
\[
    p_{x'} = \frac{\partial L}{\partial \dot{x}'} = m\dot{x}' + mv_0 \]
\[
    H'(x', p_{x'}) = p_{x'}\dot{x}' - L(x', \dot{x}') \] \[
    H'(x', p_{x'}) = m\dot{x}'^2 + m\dot{x}'v_0 - \frac{m\dot{x}'^2}{2} - m\dot{x}'v_0 - \frac{mv_{0}^2}{2} + \frac{kx'^2}{2} \]
\[
    H'(x', p_{x'}) = \frac{m\dot{x}'^2}{2} + \frac{kx'^2}{2} - \frac{mv_{0}^2}{2} \]
\[
    H'(x', p_{x'}) = \frac{(p_{x'} - mv_0)^2}{2m} + \frac{kx'^2}{2} - \frac{mv_{0}^2}{2} \neq E_T
\]

\begin{itemize}
\item
  \(H'(x', p_{x'})\) has no explicit time dependence so it is conserved
  but it is no longer the total energy
\end{itemize}
\end{example}

\section{}\label{lecture-12}

\subsection{Hamilton's Principle}\label{hamiltons-principle-1}

\begin{itemize}
\item
  \textbf{A mechanical system follows the path which minimises action,
  where action is \(\int L\,dt\)}
\end{itemize}

\subsection{Canonical Transformations}\label{canonical-transformations}

\begin{itemize}
\item
  A transformation is by definition \emph{canonical}, if it preserves
  the structure of Hamilton's equations for all dynamical systems
\item
  In phase space, can consider new coords Q and momenta P as functions
  of the original coords and momenta:
\end{itemize}

\[
    Q = Q(q, p, t) ~;~ P = P(q, p, t)
\]

\begin{itemize}
\item
  Will need to define a new Hamiltonian, \(H'\) for new Q and P
\item
  page 34 for Lagrangian equivalency
\end{itemize}

\subsection{The Generating Function}\label{the-generating-function}

\begin{itemize}
\item
  \(F(q, Q, t)\)
\item
  \(\frac{dF}{dt} = \frac{\partial F}{\partial q}\dot{q} + \frac{\partial F}{\partial Q}\dot{Q} + \frac{\partial F}{\partial t}\)
\item
  see page 35
\end{itemize}

\[
    \frac{\partial L}{\partial \dot{q}} - \frac{\partial F}{\partial q} \implies p = \frac{\partial F}{\partial q} \]
\[
    \frac{\partial L'}{\partial \dot{Q}} = -\frac{\partial F}{\partial Q} \implies P = -\frac{\partial F}{\partial Q}
\]

\subsection{Transformed Hamiltonian}\label{tranformed-hamiltonian}

\[
    H'(Q, P, t) = H(q, p, t) + \frac{\partial F(q, Q, t)}{\partial t}
\]

\begin{example}
\begin{itemize}
\item
  SHO
\item
  For a mass m moving horizontally on a frictionless surface, attached
  to a spring of spring constant k, the Lagrangian can be written as:
\end{itemize}

\[
    L = \frac{1}{2}m\dot{q}^2 - \frac{1}{2}kq^2 = \frac{m}{2}(\dot{q}^2 - \omega^2 q^2),~ \omega^2 = \frac{k}{m}
\]

\begin{itemize}
\item
  The canonically conjugate momentum to coordinate q is
\end{itemize}

\[
    p = \frac{\partial L}{\partial \dot{q}} = m\dot{q}
\]

\begin{itemize}
\item
  Therefore, the Hamiltonian is:
\end{itemize}

\[
    \begin{aligned}
    H &= p\dot{q} - L = \frac{p^2}{m} - \frac{p^2}{2m} + \frac{m\omega^2 q^2}{2} \\
    &= \frac{1}{2}\Big(\frac{p^2}{m} + m\omega^2 q^2\Big)
    \end{aligned}
\]

\begin{itemize}
\item
  Now consider the Generating Function:
\end{itemize}

\[
    F(q, Q) = \frac{1}{2}m\omega q^2 \cot(2\pi Q)
\]

\begin{itemize}
\item
  This yields:
\end{itemize}

\[
    P = -\frac{\partial F}{\partial Q} = \frac{m\pi\omega q^2}{\sin^2(2\pi Q)} \]
\[
    p = \frac{\partial F}{\partial q} = m\omega q\cot(2\pi Q)
\]

\begin{itemize}
\item
  Eliminating q gives
\end{itemize}

\[
    p(P, Q) = \sqrt{\frac{m\omega P}{\pi}}\cos(2\pi Q)
\]

\begin{itemize}
\item
  This can then be used to give
\end{itemize}

\[
    q(P, Q) = \sqrt{\frac{P}{m\omega\pi}} \sin(2\pi Q)
\]

\begin{itemize}
\item
  Substituting these back into Hamiltonian:
\end{itemize}

\[
    H(q, p) = \frac{1}{2}\Big[\frac{\omega P}{\pi}\cos^2(2\pi Q) + \frac{\omega P}{\pi} \sin^2(2\pi Q)\Big] = \frac{\omega P}{2\pi} \]
\[
    \frac{\partial F}{\partial t} = 0 \implies H'(Q, P) = H(q, p) = \frac{\omega P}{2\pi}
\]

\begin{itemize}
\item
  Q is an ignorable coordinate (\(\frac{\partial H}{\partial Q} = 0\))
  and hence P is a constant of the motion
\item
  Hamilton's equations give:
\end{itemize}

\[
    \dot{Q} = \frac{\partial H'}{\partial P} = \frac{\omega}{2\pi} \implies Q(t) = \frac{\omega t}{2\pi}
\]

\begin{itemize}
\item
  Transforming back to (q, p) yields the familiar solution:
\end{itemize}

\[
    q = \sqrt{\frac{P}{\pi m \omega}}\sin(\omega t) \] \[
    p = \sqrt{\frac{m\omega P}{\pi}}\cos(\omega t) \] \[
    \sqrt{P} \propto \text{amplitude of oscillation} \] \[
    P \propto \text{energy of oscillator} \] \[
    Q \propto \text{phase of oscillator}
\]
\end{example}

\subsection{Forms of Generating Functions}\label{forms-of-generating-functions}
\begin{itemize}
\item
  forms are shown on page 36
\end{itemize}

\subsection{Poisson Brackets}\label{poisson-brackets}

\begin{itemize}
\item
  page 36 again
\item
  For N DoFs, Poisson bracket is defined as:
\end{itemize}

\[
    \{ F, G \} \equiv \sum_{k = 1}^{N} \Bigg(\frac{\partial F}{\partial q_k}\frac{\partial G}{\partial p_k} - \frac{\partial F}{\partial p_k}\frac{\partial G}{\partial q_k}\Bigg)
\]

\begin{itemize}
\item
  Can rewrite Hamilton's equations of motion:
\end{itemize}

\[
    \dot{q} = \{q, H\} ~;~ \dot{p} = \{p, H\} \] \[
    F(q, p, t) \to \dot{F} = \{F, H\} + \frac{\partial F}{\partial t}
\]

\section{}\label{lecture-13}

\subsection{Rotating Reference Frames}\label{rotating-reference-frames}

\subsubsection{Accelerating Reference Frames}\label{accelerating-reference-frames}

\begin{itemize}
\item
  page 38
\item
  changing from an inertial frame of reference, \(S\), to a linearly
  accelerating one, \(B\):
\end{itemize}

\[
    m\underline{\ddot{r}}_B = m(\underline{\ddot{r}}_S - \underline{\ddot{R}}) = \underbrace{\underline{F}}_{\text{true force}} - \underbrace{m\underline{\ddot{R}}}_{\text{fictitious force}}
\]

\subsubsection{Rotated Reference Frames in 2D}\label{rotated-reference-frames-in-2d}

\begin{enumerate}
\item
  In frame \(S\), position vector has positive x and y components A new
  fixed reference frame, \(B\), is defined wrt an original reference
  frame by an anticlockwise rotation through angle \(\theta\)
\item
  In frame \(B\), \(\underline{r}\) has a positive x component and a
  negative y component
\item
  An observer who changes their observation frame from \(S\) to \(B\)
  finds that \(\underline{r}\) appears to have rotated clockwise through
  an angle \(\theta\), from \(\underline{r}_S\) to \(\underline{r}_B\)
\end{enumerate}

\begin{itemize}
\item
  2D rotation matrix:
\end{itemize}

\[
    R = \begin{pmatrix} \cos\theta & -\sin\theta \\ \sin\theta & \cos\theta \end{pmatrix} \]
\[
    R_=\begin{pmatrix} 1 \\ 0 \end{pmatrix} = \begin{pmatrix} \cos\theta \\ \sin\theta \end{pmatrix} ~;~ R_=\begin{pmatrix} 0 \\ 1 \end{pmatrix} = \begin{pmatrix} -\sin\theta \\ \cos\theta \end{pmatrix} \]
\[
    R^{-1} = \begin{pmatrix} \cos\theta & \sin\theta \\ -\sin\theta & \cos\theta \end{pmatrix}
\]

\subsection{Rotations in 3D}\label{rotations-in-3d}

\begin{itemize}
\item
  rotational operations in 3D \textbf{do not commute}
\item
  page 39 has derivation leading to the rotation formula:
\end{itemize}

\[
    \underline{r}_B = \underline{r}_S \cos\theta + (\underline{n} \cdot \underline{r}_S)\underline{n}[1 - \cos\theta] + (\underline{r}_S \times \underline{n})\sin\theta
\]

\subsubsection{Infinitesimal Rotations}\label{infinitesimal-rotations}

\begin{itemize}
\item
  page 40
\item
  fixed in frame \(S\)
\item
  take the small angle limit for \(d\theta\), and rotation formula
  becomes:
\end{itemize}

\[
    \underline{r} + d\underline{r} = \underline{r} + (\underline{r} \times \underline{n})d\theta
\]

\begin{itemize}
\item
  velocity in \(B\) relative to \(S\)
\end{itemize}

\[
    \Big[\frac{dr}{dt}\Big]_{in\,B_{}} = -\Big(\underline{n}\frac{d\theta}{dt}\Big) \times \underline{r} = -\underbrace{\underline{\omega}}_{\text{angular velocity}} \times \underline{r}
\]

\begin{example}[Pseudo-vectors]

\begin{itemize}
\item
  Consider vectors in a particular coordinate system:
  \begin{itemize}
  \item
    \(\underline{r}_1 = (x_1, y_1, z_1)\)
  \item
    \(\underline{r}_2 = (x_2, y_2, z_2)\)
  \item
    \(\underline{u} = \underline{r}_1 \times \underline{r}_2\)
  \end{itemize}
\item
  If the coordinate system is reflected such that \(x'\) is in the
  direction of \(-x\) and similarly with the other coordinates, then in
  this reflected coordinate system, vector \(\underline{r}_1\) will have
  components:
  \begin{itemize}
  \item
    \(\underline{r}_{1}' = (-x_1, -y_1, -z_1)\)
  \end{itemize}
\item
  However,
  \(\underline{u}' = \underline{r}_{1}' \times \underline{r}_{2}'\), has
  components that are the same as \(\underline{u}\)
  \begin{itemize}
  \item
    it has not changed under reflection
  \end{itemize}
\item
  \(\underline{u}\) is a pseudo-vector
  \begin{itemize}
  \item
    it behaves like a vector under rotation but is invariant under
    reflection
  \end{itemize}
\end{itemize}
\end{example}

\subsection{Velocity and Acceleration}\label{velocity-and-acceleration}

\begin{itemize}
\item
  must account for time-dependence in frame \(S\):
\end{itemize}

\[
    \underbrace{\Big[\frac{d\underline{r}}{dt}\Big]_{in\,B_{}}}_{\text{velocity in B}} = \underbrace{\Big[\frac{d\underline{r}}{dt}\Big]_{in\,S_{}}}_{\text{velocity in S}} - \underbrace{\underline{\omega} \times \underline{r}}_{\text{motion due to motion of B relative to S}}
\]

\begin{itemize}
\item
  Use \(\Big[\frac{d}{dt}\Big]_{in\,B_{}}\) as operator
\end{itemize}

\[
    \Big[\frac{d}{dt}\Big]_{in\,B_{}} = \Bigg(\Big[\frac{d}{dt}\Big]_{in\,S_{}} - \underline{\omega} \times \Bigg)
\]

\begin{itemize}
\item
  Use this to find the acceleration:
\end{itemize}

\[
    \Big[\frac{d\underline{v}_B}{dt}\Big]_{in\,B_{}} = \underline{a}_S - 2\underline{\omega} \times \underline{v}_B - \underline{\omega} \times (\underline{\omega} \times \underline{r}) - \underline{\dot{\omega}} \times \underline{r}
\]

\subsection{Inertial Forces in a Rotating Frame}\label{inertial-forces-in-a-rotating-frame}

\[
    m\underline{\ddot{r}} = \underline{F} - \underbrace{2m\underline{\omega} \times \underline{\dot{r}}}_{\text{coriolis force}} - \underbrace{m\underline{\omega} \times (\underline{\omega} \times \underline{r})}_{\text{centrifugal force}} - \underbrace{m\underline{\dot{\omega}} \times \underline{r}}_{\text{Euler force}}
\]

\paragraph{Tying back into Central Forces}

\begin{itemize}
\item
  Comparing to Central Forces:
\end{itemize}

\[
    E = \underbrace{\overbrace{\frac{1}{2}m\dot{r}^2}^{T_{eff}} + \frac{1}{2}mr^2 \dot{\phi}^2}_{T} + \underbrace{V_{int}(r)}_{V}
\]

\begin{itemize}
\item
  Angular momentum barrier, \(\frac{1}{2}mr^2\dot{\phi}^2\) is
\end{itemize}

\[
    -\nabla (\frac{1}{2}mr^2 \dot{\phi}^2) = -m\dot{\phi}^2r
\]

\begin{itemize}
\item
  This is just the centrifugal force
\item
  Can consider the effective 1D problem as having been set in a
  rotating, non-inertial frame
\end{itemize}

\section{}\label{lecture-14}

\begin{example}[Homework]

\begin{itemize}
\item
  Sketch
\end{itemize}

\[
    V_{eff}(r) = \frac{J^2}{2mr^2} - \frac{k}{r}e^{-\frac{r}{a}}
\]

\begin{itemize}
\item
  Sketch parts independently then add together
  \begin{itemize}
  \item
    exponential not as significant so more similar to first term
  \end{itemize}
\end{itemize}
\end{example}

\subsection{Inertial Forces on Earth}\label{intertial-forces-on-earth}

\subsubsection{Local Coordinate System}\label{local-coordinate-system}

\begin{itemize}
\item
  page 41
\end{itemize}

\begin{enumerate}
\item
  Model Earth as perfect solid sphere and require an appropriate
  coordinate system to describe motion near a point on its surface
  \begin{itemize}
  \item
    Start with Cartesian coordinates with z running through poles
  \end{itemize}
\item
  Rotate coordinate system around x axis so that z emerges at a chose
  point on Earth's surface
  \begin{itemize}
  \item
    In new coordinates, looks like a rotation of \(-\theta\) about x
  \item
    in local frame, \(\omega\) has y and z components with trig
    resolutions
  \end{itemize}

\[
    \underline{\omega} = \begin{pmatrix} 0 \\ \omega\sin\theta \\ \omega\cos\theta \end{pmatrix}
\]

\item
  Displace origin of coordinate system by \(\underline{R}\) to the point
  where the local z emerges from the Earth's surface
\end{enumerate}

\subsection{Inertial Forces}\label{intertial-forces}

\begin{itemize}
\item
  need to substitute in
  \(\underline{r} = \underline{r}' + \underline{R}\)
\end{itemize}

\subsection{Forces}\label{forces}

\subsubsection{Centrifugal Forces}\label{centrifugal-forces}

\begin{itemize}
\item
  \(-m\underline{\omega} \times (\underline{\omega} \times \underline{r})\)
\item
  must be perpendicular to all terms above
\item
  points directly away from the axis of rotation
\item
  page 42 for diagrams
\item
  largest at equator, decreases smoothly as a trig function towards
  poles
\item
  adding gravity and centrifugal together gives almost standard gravity
  map but decreased
  \begin{itemize}
  \item
    centrifugal is not acting radially
  \item
    effect of gravity is shifted slightly from exactly in the middle
  \end{itemize}
\end{itemize}

\subsubsection{The Deviation of a Plumb Bob}\label{the-deviation-of-a-plumb-bob}

\begin{itemize}
\item
  Set \(\underline{r}' = (0, 0, 0)\) and \(\underline{R} = (0,0,R)\),
  where R is radius of Earth
\item
  use
\end{itemize}

\[
    \underline{\omega} = \begin{pmatrix} 0 \\ \omega\sin\theta \\ \omega\cos\theta \end{pmatrix}
\]

\begin{itemize}
\item
  obtain centrifugal force on plumb bob of mass m:
\end{itemize}

\[
    \begin{aligned}
    \underline{F}_{c} &= -m\underline{\omega} \times [\underline{\omega} \times (\underline{r}' + \underline{R})] \\
    &= -m\omega\underline{\omega} \times \begin{vmatrix} \underline{i} & \underline{j} & \underline{k} \\
    0 & \sin\theta & \cos\theta \\
    0 & 0 & R \end{vmatrix} \\
    &= -m\omega^2 \begin{vmatrix} \underline{i} & \underline{j} & \underline{k} \\ 0 & \sin\theta & \cos\theta \\ R\sin\theta & 0 & 0\end{vmatrix} \\
    &= -m\omega^2R(0, \sin\theta\cos\theta, -\sin^2\theta)
    \end{aligned}
\]

\begin{itemize}
\item
  Incorporating the effect of gravity, one obtains
\end{itemize}

\[
    m\underline{\ddot{r}}' = -m(0, \omega^2R\sin\theta\cos\theta, g - \omega^2R\sin^2\theta)
\]

\begin{itemize}
\item
  Now consider the deflection angle, \(\phi_d\), defined through
\end{itemize}

\[
    \tan\phi_d = \frac{\omega^2R\sin\theta\cos\theta}{g - \omega^2R\sin^2\theta} \approx \underbrace{\frac{\omega^2R\sin\theta\cos\theta}{g}}_{g >> \omega^2R\sin^2\theta}
\]

\begin{itemize}
\item
  For \(\theta = \frac{\pi}{4}, \frac{3\pi}{4},~ \phi_d\) is maximised
  at 1.7 milliradians
\item
  Buildings at colatitude \(\frac{\pi}{4}\) are therefore tilted by this
  amount from the true vertical if aligned with a plumb bob, or a spirit
  level
\item
  Note that \(\omega = \frac{2\pi}{86400} \text{rad }s^{-1}\),
  \(R \approx 6400\,km \implies \frac{\omega^2 r}{g} = 0.003\)
\end{itemize}

\subsubsection{Coriolis Force}\label{coriolis-force}

\begin{itemize}
\item
  page 43
\item
  Coriolis force is given by
\end{itemize}

\[
    -2m\underline{\omega} \times \underline{\dot{r}}' = -2m\omega(\dot{z}\sin\theta - \dot{y}\cos\theta, \dot{x}\cos\theta, -\dot{x}\sin\theta)
\]

\begin{itemize}
\item
  only affects moving systems, depends on velocities
\item
  often considered to be constrained in a plane, so ignore z dimension
  and simplify equation
\end{itemize}

\begin{example}[Coriolis Force]
\begin{itemize}
\item
  particle with mass, m, moves with constant relative speed, v, around
  the rim of a wheel of radius, b
\item
  wheel rolls along a fixed straight line with uniform velocity, u
\item
  taking wheel as frame of reference, find centrifugal and Coriolis
  forces
\item
  relative to the inertial frame, moving right with speed, u, and has
  its origin at O, the rotating wheel frame has angular velocity,
  \(\underline{\omega}\):
\end{itemize}

\[
    \underline{\omega} = \frac{u}{b} \text{ - into the page} \] \[
    \begin{aligned}
    \underline{F}_c &= -m\underline{\omega} \times (\underline{\omega} \times \underline{r}) \\
    &= m\omega^2b \underline{\hat{r}} \\
    &= \frac{mu^2}{b}\underline{\hat{r}}
    \end{aligned}
\]

\begin{itemize}
\item
  Coriolis force:
\end{itemize}

\[
    \begin{aligned}
    \underline{F}_{cor} &= -2m\underline{\omega} \times \overbrace{\underline{\dot{r}}}^{= \underline{v}} \\
    &= 2m\frac{uv}{b}\underline{\hat{r}} \\
    &= \frac{2v}{u}\underline{F}_c
    \end{aligned}
\]
\end{example}

\subsubsection{Loose Ends}\label{loose-ends}

\begin{itemize}
\item
  Euler force is not zero on Earth, but tends to be negligible
\item
  In rotating reference frame that is also accelerating translationally,
  \(\underline{\ddot{R}}\), general EoM given by:
\end{itemize}

\[
    m\underline{\ddot{r}} = \underline{F} - 2m\underline{\omega} \times \underline{\dot{r}} - m\underline{\omega} \times (\underline{\omega} \times \underline{r}) - m\underline{\dot{\omega}} \times \underline{r} - m\underline{\ddot{R}}
\]

\section{}\label{lecture-15}

\subsection{Examples in the Coriolis Force}\label{examples-in-the-coriolis-force}

\[
    \underline{F}_{cor} = -2m\underline{\omega} \times \underline{\dot{r}}
\]

\subsubsection{Weather Systems}\label{weather-systems}

\begin{itemize}
\item
  Storm rotating clockwise is a cyclone, in the southern hemisphere
  \begin{itemize}
  \item
    Coriolis force pushes to the left
  \end{itemize}
\item
  \(\underline{\omega} \times \underline{\dot{r}}\) is West, so
  \(-\underline{\omega} \times \underline{\dot{r}}\) is East
\item
  As pressure gradient pushes air north towards the centre of low
  pressure, the coriolis force deflects it East.
\item
  In the northern hemisphere, coriolis force acts to the right of the motion; in the southern, it acts to the left of the motion.
\end{itemize}

\subsubsection{Foucault's Pendulum}\label{foucaults-pendulum}

\begin{itemize}
\item
  Local coordinate frame:
  \begin{itemize}
  \item
    \(\underline{i}\) corresponding to east,
  \item
    \(\underline{j}\) is north,
  \item
    \(\underline{k}\) is radially outwards
  \end{itemize}
\item
  In local frame,
\end{itemize}

\[
    \underline{\omega} = \omega\sin\theta\underline{j} + \omega\cos\theta\underline{k}
\]

\begin{itemize}
\item
  \(\theta\) is the colatitude
\item
  \(\underline{\dot{\omega}} \approx 0\) for Earth and
\end{itemize}

\[
    \omega^2 \ll \omega_0^2 = \frac{g}{l}
\]

\begin{itemize}
\item
  Will neglect the Euler and centrifugal forces respectively
\item
  For the coriolis force:
\end{itemize}

\[
    \begin{aligned}
    \underline{F}_{cor} &= -2m\underline{\omega} \times \underline{\dot{r}} \\
    \underline{\omega} \times \underline{\dot{r}} &= \begin{vmatrix} \underline{i} & \underline{j} & \underline{k} \\ 0 & \omega\sin\theta & \omega\cos\theta \\ \dot{x} & \dot{y} & \dot{z} \end{vmatrix} \\
    &= \omega\begin{pmatrix} \dot{z}\sin\theta - \dot{y}\cos\theta \\ \dot{x}\cos\theta \\ -\dot{x}\sin\theta \end{pmatrix}
    \end{aligned}
\]

\begin{itemize}
\item
  Motion is largely in \(\underline{i}-\underline{j}\) plane for small
  angle oscillations, \(\therefore \dot{z} \approx 0\)
\item
  Include force due to gravity:
\end{itemize}

\[
    \underline{F} = -m\omega_0^2 (x\underline{i} + y\underline{j})
\]

\begin{itemize}
\item
  Then, ignoring vertical motion,
\end{itemize}

\[
    \begin{aligned}
    m\underline{\ddot{r}} = -m\omega_0^2 (x\underline{i} &+ y\underline{j}) - 2m\omega(-\dot{y}\cos\theta\underline{i} - \dot{x}\cos\theta\underline{j}) \\
    \begin{pmatrix} \ddot{x} \\ \ddot{y} \end{pmatrix} &= \begin{pmatrix} 2\omega\dot{y}\cos\theta - \omega_0^2 x \\ -2\omega\dot{x}\cos\theta - \omega_0^2 y \end{pmatrix}
    \end{aligned}
\]

\begin{itemize}
\item
  Set \(\zeta(t) = x(t) + iy(t)\), then
\end{itemize}

\[
    \begin{aligned}
    \ddot{\zeta} &= \ddot{x} + i\ddot{y} \\
    &= 2\omega\cos\theta\underbrace{(-i\dot{x} + \dot{y})}_{-i\dot{\zeta}} - \omega_0^2\underbrace{(x + iy)}_{\zeta} \\
    &= -2i\omega\cos\theta \dot{\zeta} - \omega_0^2 \zeta \\
    \ddot{\zeta} &+ 2i\omega\cos\theta\dot{\zeta} + \omega_0^2 \zeta = 0
    \end{aligned}
\]

\begin{itemize}
\item
  This looks like a damped oscillator with complex damping.
\item
  Try a solution, \(\zeta = e^{\lambda t}\)
\end{itemize}

\[
    \begin{aligned}
    \lambda^2 &+ 2i\omega\cos\theta\lambda + \omega_0^2 = 0 \\
    \implies \lambda &= -i\omega\cos\theta \pm \sqrt{-\omega^2\cos^2\theta - \omega_0^2} \\
    &= i\bigg(-\omega\cos\theta \pm \sqrt{\omega^2\cos^2\theta + \omega_0^2}\bigg) \\
    &= i\bigg(-\hat{\omega} \pm \sqrt{\hat{\omega} + \omega_0^2}\bigg),~ \hat{\omega} = \omega\cos\theta \ll \omega_0 \\
    &\approx i(-\hat{\omega} \pm \omega_0) \\
    \therefore \zeta = x + iy &= \underbrace{e^{-i\hat{\omega}t}}_{\text{\shortstack{precession of the \\ plane of oscillation}}}\underbrace{(\alpha e^{i\omega_0 t} + \beta e^{-i\omega_0 t})}_{\text{normal pendulum motion}} \\
    \text{Precession rate } &= \hat{\omega} = \omega\cos\theta = \omega\underbrace{\sin\lambda}_{\text{latitude}}
    \end{aligned}
\]

\subsection{The Euler Force}\label{the-euler-force}

\begin{itemize}
\item
  The moon moves away by about 4 cm a year by taking orbital angular
  momentum from the spin angular momentum of the earth
  \begin{itemize}
  \item
    Earth's days are slowing down due to this
  \item
    In 100 years, days on Earth will be about 2 ms longer
  \item
    \(\underline{\dot{\omega}} < 0\) - difference between two \(\omega\)
    vectors, points out the south pole
  \end{itemize}
\end{itemize}

\section{}\label{lecture-16}

\subsection{Rotational Inertia, Angular Momentum, and Kinetic Energy}\label{rotational-inertia-angular-momentum-and-kinetic-energy}

\begin{itemize}
\item
  See pages 44-46
\end{itemize}

\[
    I = \sum_{k = 1}^N m_k|(\underline{r}_k \cdot \underline{n})\underline{n} - \underline{r}_k|^2 = \sum_{k = 1}^\infty m_k [r_k^2 - (\underline{r}_k \cdot \underline{n})^2]
\]

\subsubsection{\texorpdfstring{Defining the Inertia Tensor,
\(\hat{I}\)}{Defining the Inertia Tensor, \textbackslash{}hat\{I\}}}\label{defining-the-inertia-tensor-hati}

\[
    \begin{aligned}
    \underline{J} &= \sum_{k = 1}^N m_k [r_k^2\underline{\omega} - (\underline{r}_k\cdot \underline{\omega})\underline{r}_k] \\
    J_x &= \sum_{k = 1}^N m_k (r_k^2 - x_k^2)\omega_x - m_kx_ky_k\omega_y - m_kx_kz_k\omega_z \\
    J_x &= \sum_{k = 1}^N m_k (r_k^2 - y_k^2)\omega_y - m_kx_ky_k\omega_x - m_ky_kz_k\omega_z \\
    J_x &= \sum_{k = 1}^N m_k (r_k^2 - z_k^2)\omega_z - m_kx_kz_k\omega_x - m_ky_kz_k\omega_y \\
    \underbrace{\begin{pmatrix} J_x \\ J_y \\ J_z \end{pmatrix}}_{\underline{J}} &= \underbrace{\begin{pmatrix} I_{xx} & I_{xy} & I_{xz} \\ I_{yx} & I_{yy} & I_{yz} \\ I_{zx} & I_{zy} & I_{zz} \end{pmatrix}}_{\hat{I}}\underbrace{\begin{pmatrix} \omega_x \\ \omega_y \\ \omega_z \end{pmatrix}}_{\underline{\omega}} \\
    I_{\alpha\beta} &= \sum_k^N m_k(r_k^2 \delta_{\alpha\beta} - r_{k\alpha}r_{k\beta})
    \end{aligned}
\]

\begin{itemize}
\item
  A ratio of vectors is called a Tensor
\item
  The Inertial tensor is the ratio of \(\underline{J}\) to
  \(\underline{\omega}\)
\item
  For continuous rigid bodies:
\end{itemize}

\[
    I_{\alpha\beta} = \int_V dx\;dy\;dz\;\rho(x,y,z)(r^2\delta_{\alpha\beta} - r_\alpha r_\beta)
\]

\begin{example}[Example of Inertia Tensor]

\begin{itemize}
\item
  Consider the Inertia tensor for a uniform cube of length a, mass m,
  with origin at the centre of mass
\item
  This is a uniform cube so it has a density
\end{itemize}

\[
    \rho = \frac{M}{a^3}
\]

\[
    \begin{aligned}
    I_{xx} &= \int_{-\frac{a}{2}}^{\frac{a}{2}} dx \int_{-\frac{a}{2}}^{\frac{a}{2}} dy \int_{-\frac{a}{2}}^{\frac{a}{2}} dz\; \rho\underbrace{(r^2 \cdot 1 - x^2)}_{y^2 + z^2} \\
    &= \frac{M}{a^3}a = \int_{-\frac{a}{2}}^{\frac{a}{2}} dy \Bigg[y^2z + \frac{z^3}{3}\Bigg]_{-\frac{a}{2}}^{\frac{a}{2}} \\
    &= \frac{M_{}}{a^2} \int_{-\frac{a}{2}}^{\frac{a}{2}} dy \cdot 2 \Bigg(\frac{y^2 a}{2} + \frac{a^3}{24}\Bigg) \\
    &= \frac{M}{a^2} \Bigg[\frac{y^3 a}{3} + \frac{a^3 y}{12} \Bigg]_{-\frac{a}{2}}^{\frac{a}{2}} \\
    &= \frac{2M_{}}{a^3} \Bigg[\frac{a^4}{24} + \frac{a^4}{24}\Bigg] = \frac{Ma^4}{6}
    \end{aligned}
\]

\begin{itemize}
\item
  From symmetry, \(I_{xx} = I_{yy} = I_{zz}\)
\end{itemize}

\[
    \begin{aligned}
    I_{xy} &= \int_{-\frac{a}{2}}^{\frac{a}{2}} dx \int_{-\frac{a}{2}}^{\frac{a}{2}} dy \int_{-\frac{a}{2}}^{\frac{a}{2}} dz\;\rho(r^2 \cdot 0 - xy) \\
    &= -\rho a \int_{-\frac{a}{2}}^{\frac{a}{2}} dx \; \underbrace{\Bigg[\frac{xy^2}{2}\Bigg]_{-\frac{a}{2}}^{\frac{a}{2}}}_{\text{even fn of y}} = 0
    \end{aligned}
\]

\begin{itemize}
\item
  From symmetry, all off-diagonal elements are zero.
\item
  Hence,
\end{itemize}

\[
    \hat{I} = \frac{Ma^2}{6}\begin{pmatrix} 1 & 0 & 0 \\ 0 & 1 & 0 \\ 0 & 0 & 1 \end{pmatrix}
\]

\begin{itemize}
\item
  If origin of coordinates is at one corner of the cube, then
\end{itemize}

\[
    \begin{aligned}
    I_{xx} = I_{yy} = I_{zz} &= \rho \int_0^a dx \int_0^a dy \int_0^a dz \;(y^2 + z^2) \\
    &= \frac{M}{a^2} \int_0^a dy \; \Bigg[y^2 a + \frac{a^3}{3}\Bigg] \\
    &= \frac{M}{a^2} \Bigg[\frac{y^3 a}{3} + \frac{a^3 y}{3}\Bigg]_{0}^{a_{}} = \frac{2Ma^2}{3} \\
    I_{xy} &= -\rho \int_0^a dx \int_0^a dy \int_0^a dz \cdot xy \\
    &= -\rho a \int_0^a dx \; \Bigg[\frac{xa^2}{2}\Bigg] = -\frac{M}{a^2}\frac{a^2}{2}\frac{a^2}{2} \\
    &= -\frac{Ma^2}{2}
    \end{aligned}
\]

\begin{itemize}
\item
  Hence,
\end{itemize}

\[
    \hat{I} = \frac{Ma^2}{12 }\begin{pmatrix} 8 & -3 & -3 \\ -3 & 8 & -3 \\ -3 & -3 & 8 \end{pmatrix}
\]

\begin{itemize}
\item
  The inertia tensor of an object is with respect to rotations about a
  particular point
\item
  Inertia tensors will always be symmetric
\end{itemize}
\end{example}

\subsection{Rotational Kinetic Energy}\label{rotational-kinetic-energy}

\[
    T = \frac{1}{2}\underline{\omega} \cdot \underbrace{\underline{J}}_{\hat{I}\underline{\omega}}
\]

\[
    T = \frac{\omega^2}{2}\underline{n}^T \hat{I} \underline{n}
\]

\begin{itemize}
\item
  \(\underline{n}^T \hat{I} \underline{n}\) is the moment of the
  inertia, \(I\) about the axis defined by \(\underline{n}\)
\end{itemize}

\section{}\label{lecture-17}

\subsection{Angular Momentum and KE}\label{angular-momentum-and-ke}

\begin{itemize}
\item
  page 46 - 47
\item
  Define CoM position and time derivative
\item
  Total Momentum \(P = \sum_{k = 1}^N m_k\underline{\dot{r}}_k\)
\item
  Use
\end{itemize}

\[
    \underline{J} = \sum_{k = 1}^N m_k(\underline{r}_k \times \underline{\dot{r}}_k)
\]

\begin{itemize}
\item
  follow derivations in notes
\end{itemize}

\subsection{Parallel and Principal Axis Theorems}\label{parallel-and-principal-axis-theorems}

\subsubsection{Displaced Axis Theorem}\label{displaced-axis-theorem}

\begin{itemize}
\item
  Can convert Inertia Tensor into other form using CoM position:
\end{itemize}

\[
    I_{\alpha\beta} = \sum_{k = 1}^N m_k(r_{k}'\delta_{\alpha\beta} - r_{k,\alpha}'r_{k,\beta}') + M(R_C^2\delta_{\alpha\beta} - R_{C,\alpha}R_{C,\beta})
\]

\begin{itemize}
\item
  full derivation on page 48
\end{itemize}

\[
    \hat{I} = \hat{I}_{CoM} + M\hat{A}
\]

\begin{itemize}
\item
  \(\hat{A}\) can be represented as a matrix, the elements of which are
  determined by the elements of the CoM position vector
\end{itemize}

\[
    A_{\alpha\beta} = R_C^2\delta_{\alpha\beta} - R_{C,\alpha}R_{C,\beta}
\]

\begin{example}[1]

\[
    \begin{aligned}
    A_{xx} &= R_C^2 - x_C^2 = y_C^2 + z_C^2 \\
    A_{xy} &= -x_Cy_C \\
    \hat{A} &= \begin{pmatrix} y_C^2 + z_C^2 & -y_Cx_C & -z_Cx_C \\ -x_Cy_C & z_C^2 + x_C^2 & -z_Cy_C \\ -x_Cz_C & -y_Cz_C & x_C^2 + y_C^2 \end{pmatrix}
    \end{aligned}
\]

\begin{itemize}
\item
  The moment of inertia for rotations about the z axis through the
  centre of mass:
\end{itemize}

\[
    I_z = \underline{\hat{z}}^T \hat{I}_{CoM} \underline{\hat{z}}
\]

\begin{itemize}
\item
  \(\underline{\hat{z}}\) is a unit vector in the z direction
\item
  For a parallel axis is a distance \(d = \sqrt{x_C^2 + y_C^2}\) away,
  the moment of inertia is \(I_d\), where
\end{itemize}

\[
    \begin{aligned}
    \underline{\hat{z}}^T \hat{I}_{d} \underline{\hat{z}} &= \underline{\hat{z}}^T \hat{I}_{CoM} \underline{\hat{z}} + M\underline{\hat{z}}^T \hat{A} \underline{\hat{z}} \\
    I_d &= I_z + M\underline{\hat{z}}^T \hat{A} \underline{\hat{z}} \\
    &= I_z + M \begin{pmatrix} 0 & 0 & 1 \end{pmatrix}\begin{pmatrix} & \hat{A} & \\ & & \\ \end{pmatrix}\begin{pmatrix} 0 \\ 0 \\ 1 \end{pmatrix} \\
    &= I_z + M \begin{pmatrix} 0 & 0 & 1 \end{pmatrix}\begin{pmatrix} -z_cx_C \\ -z_Cy_C \\ x_C^2 + y_C^2 \end{pmatrix} \\
    &= I_z + M(x_C^2 + y_C^2) \\
    &= I_z + Md^2
    \end{aligned}
\]
\end{example}

\begin{example}[2]
\begin{itemize}
\item
  Calculate \(\hat{I}\) about the centre of mass for the 3 mass point
  system shown.
\item
  Triangular system, two masses of mass \(m_1\) and one of \(m_2\)
\item
  First fine \(\hat{I}\) for the two masses at \(m_1\) about their CoM,
  a point X
\item
  Choice of axes takes advantage of symmetry, hence
\end{itemize}

\[
    \begin{aligned}
    I_{xx} &= \sum_{k = 1}^2 m_k (x_k^2 - x_k^2) = 0 \\
    I_{yy} &= \sum_{k = 1}^2 m_k (\overbrace{x_k^2}^{\pm a/2} - \overbrace{y_k^2}^{0}) = \frac{m_1 a^2}{2} \\
    \implies & I_{zz} = I_{yy}
    \end{aligned}
\]

\begin{itemize}
\item
  All of-diagonal elements will be 0
  \begin{itemize}
  \item
    they involve the product of two different components and only
    \(x_k\) is non-zero
  \end{itemize}
\end{itemize}

\[
    \hat{I}_{2m_1,X} = \frac{m_1 a^2}{2} \begin{pmatrix} 0 & 0 & 0 \\ 0 & 1 & 0 \\ 0 & 0 & 1 \end{pmatrix}
\]

\begin{itemize}
\item
  CoM of triangle is at \(x = 0, z= 0, y = \frac{m_2 h}{m_2 + 2m_1}\)
\end{itemize}

\[
    \underline{R}_C = \begin{pmatrix} 0 \\ \frac{m_2 h}{m_2 + 2m_1} \\ 0 \end{pmatrix}
\]

\begin{itemize}
\item
  Therefore, the inertia tensor defined wrt CoM of the triangle for
  these 2 masses is
\end{itemize}

\[
    \begin{aligned}
    \hat{I}_{2m_1} &= \frac{m_1 a^2}{2} \begin{pmatrix} 0 & 0 & 0 \\ 0 & 1 & 0 \\ 0 & 0 & 1 \end{pmatrix} + 2m_1 \begin{pmatrix} R_C^2 & 0 & 0 \\ 0 & 0 & 0 \\ 0 & 0 & R_C^2 \end{pmatrix} \\
    &= m_2 \begin{pmatrix} 2R_C^2 & 0 & 0 \\ 0 & \frac{a^2}{2} & 0 \\ 0 & 0 & \frac{a^2}{2} + 2R_C^2 \end{pmatrix}
    \end{aligned}
\]

\begin{itemize}
\item
  Inertia Tensor wrt CoM of triangle:
\end{itemize}

\[
    \begin{aligned}
    \hat{I}_{m_2} &= m_2 \begin{pmatrix} (h - R_C)^2 & 0 & 0 \\ 0 & 0 & 0 \\ 0 & 0 & (h - R_C)^2 \end{pmatrix} \\
    h - R_C &= \frac{h(m_2 + 2m_1) - m_2h}{m_2 + m_1} = \frac{2m_1 h}{m_2 + 2m_1} \\
    R_C^2 &= \frac{(m_1 h)^2}{(m_2 + 2m_1)^2} \\
    \therefore \hat{I} = \hat{I}_{2m_1} + \hat{I}_{m_2} &= \begin{pmatrix} 2m_1R_C^2 + m_2(h - R_C)^2 & 0 & 0 \\ 0 & \frac{m_1 a^2}{2} & 0 \\ 0 & 0 & \frac{m_1 a^2}{2} + 2m_1R_C^2 + m_2(h - R_C)^2 \end{pmatrix} \\
    &= \begin{pmatrix} \frac{2m_1m_2 h^2}{m_2 + 2m_1} & 0 & 0 \\ 0 & \frac{m_1 a^2}{2} & 0 \\ 0 & 0 & \frac{2m_1m_2 h^2}{m_2 + 2m_1} + \frac{m_1 a^2}{2} \end{pmatrix}
    \end{aligned}
\]
\end{example}

\section{}\label{lecture-18}

\subsection{Symmetry of Inertia Tensor}\label{symmetry-of-inertia-tensor}

\begin{itemize}
\item
  \(I_{\alpha\beta} = I_{\beta\alpha}\)
\end{itemize}

\subsection{Orthogonal Matrices}\label{orthogal-matrices}
\begin{itemize}
\item
  page 49
\item
  Orthogonal matrices have the transpose as their inverse:
  \begin{itemize}
  \item
    \(A^{-1} = A^T\)
  \end{itemize}
\item
  2D rotation matrix is orthogonal
\end{itemize}

\subsection{Principal Axis Theorem}\label{principal-axis-theorem}

\begin{itemize}
\item
  \(\hat{I}\) is represented by a 3x3 matrix so has three eigenvalues.
  \begin{itemize}
  \item
    These are the principal moments of inertia
  \end{itemize}
\item
  The eigenvectors from this form the principal axes
\end{itemize}

\subsubsection{\texorpdfstring{Transforming \(\hat{I}\) under rotations}{Transforming \textbackslash{}hat\{I\} under rotations}}\label{transforming-hati-under-rotations}

\begin{itemize}
\item
  The rotational KE is invariant under rotations
  \begin{itemize}
  \item
    prime indicates rotated coordinates
  \end{itemize}
\end{itemize}

\[
    \begin{aligned}
    T_{rot} &= \frac{1}{2}\underline{\omega}^T \hat{I}\underline{\omega} = \frac{1}{2}\underline{\omega}'^T \hat{I}' \underline{\omega}' \\
    \underline{\omega}' = \hat{U} \underline{\omega}
    \end{aligned}
\]

\begin{itemize}
\item
  \(\hat{U}\) is an orthogonal matrix representing a general rotation
\end{itemize}

\[
    \begin{aligned}
    \underline{\omega}'^T &= \underline{\omega}^T \hat{U}^T \\
    T_{rot} &= \frac{1}{2} (\underline{\omega}^T \hat{U}^T) \hat{I}' (\hat{U} \underline{\omega}) \\
    &= \frac{1}{2}\underline{\omega}^T (\hat{U}^T \hat{I}' \hat{U}) \underline{\omega} \\
    \implies \hat{I} &= \hat{U}^T \hat{I}' \hat{U} \\
    \implies \hat{I}' &= \hat{U}\hat{I}\hat{U}^T
    \end{aligned}
\]

\subsection{Inertia Tensor of a Dumbbell}\label{inertia-tensor-of-a-dumbbell}

\begin{itemize}
\item
  Find the inertia tensor for a dumbbell
  \begin{itemize}
  \item
    two uniform spheres of mass \(m_1\) and \(m_2\) with radii a and b
    respectively, separated by a massless rigid rod of length R
  \end{itemize}
\item
  From the symmetry of the dumbbell, one of the principal axes lies
  along the line connecting the masses
  \begin{itemize}
  \item
    choose one axis lying along this direction
  \end{itemize}
\item
  Place origin at CoM
\item
  Then the centres of \(m_1\) and \(m_2\) respectively are at
\end{itemize}

\[
    \begin{aligned}
    \frac{m_2 R}{m_1 + m_2}\hat{k} ~;~ \frac{-m_1 R}{m_1 + m_2}\hat{k}
    \end{aligned}
\]

\begin{itemize}
\item
  First calculate the inertia tensor for a sphere of mass
  \(m_1 = \frac{4}{3}\pi a^3 \rho\), which by symmetry has
  \(I_1 = I_2 = I_3\)
\item
  take u to be distance to axis of rotation of mass dm
\end{itemize}

\[
    \begin{aligned}
    dI_{3,a} &= u^2\,dm \\
    dm &= \rho\,dV = \rho (r^2 \sin\theta \;dr\,d\theta\,d\phi) \\
    u &= r\sin\theta \\
    I_{3,a} &= \int_0^{2\pi} d\phi \int_0^\pi \int_0^a r^4\sin\theta (1 - \cos^2\theta) \rho\;d\theta\,dr \\
    &= \frac{8}{15}\rho \pi a^5 = \frac{2}{5}m_1 a^2
    \end{aligned}
\]

\begin{itemize}
\item
  Similarly for the other sphere,
\end{itemize}

\[
    I_{3, b} = \frac{2}{5}m_2 b^2
\]

\begin{itemize}
\item
  To find the inertia tensor fore the dumbbell, use the displaced axis
  theorem:
\end{itemize}

\[
    \begin{aligned}
    I_1 = I_2 &= \frac{2}{5}m_1 a^2 + m_1 \frac{m_2 R}{m_1 + m_2} + \frac{2}{5}m_2 b^2 + m_2 \frac{m_1 R}{m_1 + m_2} \\
    I_3 &= \frac{2}{5}m_1 a^2 + \frac{2}{5}m_2 b^2
    \end{aligned}
\]

\begin{itemize}
\item
  Now that we have the inertia tensor for the principal axis frame, we
  can find it for any other rotated frame by applying the transformation
  \(\hat{I}' = \hat{P}\hat{I}\hat{P}^T\)
\item
  For instance, the inertia tensor for a dumbbell in a set of axes
  rotated through an angle \(\theta\) about the 1 axis can be found
  using the rotation matrix
\end{itemize}

\[
    \begin{aligned}
    \hat{P}_x &= \begin{pmatrix} 1 & 0 & 0 \\ 0 & \cos\theta & -\sin\theta \\ 0 & \sin\theta & \cos\theta \end{pmatrix} \\
    \hat{I}' = \hat{P}_x \hat{I} \hat{P}^T &= \hat{P}_x \begin{pmatrix} I_1 & 0 & 0 \\ 0 & I_2 & 0 \\ 0 & 0 & I_3 \end{pmatrix} \begin{pmatrix} 1 & 0 & 0 \\ 0 & \cos\theta & -\sin\theta \\ 0 & \sin\theta & \cos\theta \end{pmatrix} \\
    &= \cdots = \begin{pmatrix} I_1 & 0 & 0 \\ 0 & I_1\cos^2\theta + I_3\sin^2\theta & (I_3 - I_1)\sin\theta\cos\theta \\ 0 & (I_3 - I_1)\sin\theta\cos\theta & I_1\sin^2\theta + I_3\cos^2\theta \end{pmatrix}
    \end{aligned}
\]

\section{}\label{lecture-19}

\subsection{Rigid Body Dynamics}\label{rigid-body-dynamics}

\subsubsection{Euler's Equations of Motion}\label{eulers-equations-of-motion}

\begin{itemize}
\item
  page 51
\item
  take time derivative of angular momentum, \(\underline{J}\)
  \begin{itemize}
  \item
    gives torque, \(\underline{N}\)
  \end{itemize}
\end{itemize}

\[
    \begin{aligned}
    \Big(\frac{d\underline{J}}{dt}\Big)_{B} + \underline{\omega_{}} \times \underline{J} &= \underline{N} \\
    I_1\dot{\omega}_1 - \omega_2\omega_3(I_2 - I_3) &= N_1 \\
    I_2\dot{\omega}_2 - \omega_3\omega_1(I_3 - I_1) &= N_2 \\
    I_3\dot{\omega}_3 - \omega_1\omega_2(I_1 - I_2) &= N_3
    \end{aligned}
\]

\subsubsection{Torque-Free Motion}\label{torque-free-motion}

\begin{itemize}
\item
  If there are no external torques, then set Ns to 0
  \begin{itemize}
  \item
    no torques when force is all through centre of rotation
  \end{itemize}
\item
  Diagram on page 52
  \begin{itemize}
  \item
    contour lines map kinetic energy
  \item
    away from 1-2 equator, higher KE; closer to, smaller KE
  \end{itemize}
\end{itemize}

\subsubsection{Motion of a Torque-free symmetric top}\label{motion-of-a-torque-free-symmetric-top}

\begin{itemize}
\item
  Consider a symmetric top with \(I_1 = I_2 \neq I_3\)
  \begin{itemize}
  \item
    Isolated from external torques
  \item
    Spinning freely
  \end{itemize}
\item
  As \(\underline{N} = 0\), the total angular momentum and rotational KE
  are conserved as viewed from an inertial frame.
\item
  However, in the frame fixed with the principal axes of the top,
\end{itemize}

\[
    \begin{aligned}
    I_1 \dot{\omega}_1 &= (I_1 - I_3)\omega_2\omega_3 \\
    I_1\dot{\omega}_2 &= (I_3 - I_1)\omega_1\omega_3 \\
    I_3\dot{\omega}_3 &= 0
    \end{aligned}
\]

\begin{itemize}
\item
  Evidently, \(\omega_3\) is a constant
\item
  Differentiating the other equations wrt time leads to
\end{itemize}

\[
    \begin{aligned}
    I_1\ddot{\omega}_1 &= (I_1 - I_3)\omega_3\dot{\omega}_2 \\
    \implies I_1\ddot{\omega}_1 &= -\frac{(I_1 - I_3)^2}{I_1} \omega_3^2 \omega_1 \\
    \therefore \ddot{\omega}_1 &= -\Bigg[\frac{I_1 - I_3}{I_1}\omega_3 \Bigg]^2 \omega_1
    \end{aligned}
\]

\begin{itemize}
\item
  Therefore, \(\omega_{1/2}\) undergoes SHM with an angular frequency
\end{itemize}

\[
    \begin{aligned}
    \Omega_b &= \frac{|I_1 - I_3|}{I_1}\omega_3
    \end{aligned}
\]

\begin{itemize}
\item
  Thus, in the non-inertial frame attached to the principal axes of the
  top, the angular velocity, \(\underline{\omega}\), traces out a
  circular path around the 3 axis, at fixed \(\omega_3\)
\item
  The rate of this precession of \(\underline{\omega}\) is given by
  \(\Omega_b\)
\item
  In an inertial (`space'-coordinates) frame, without loss of
  generality, can assume that the angular momentum, \(\underline{J}\),
  lies in the plane of body axes 2 and 3.
\end{itemize}

\begin{example}
\begin{itemize}
\item
  Now assume a prolate symmetric top such that \(I_1, I_2 > I_3\)
\item
  Choice of axes implies that \(\phi_1(t = 0) = 0\)
\item
  Hence, Euler's equations imply
\end{itemize}

\[
    \begin{aligned}
    I_1\dot{\omega}_1 &= (I_1 - I_3)\omega_2\omega_3 \\
    I_1\dot{\omega}_2 = 0
    \end{aligned}
\]

\begin{itemize}
\item
  The instantaneous change in \(\underline{\omega}\) is entirely in one
  direction
\item
  If we consider \(d\underline{\omega}(t = 0) = d\omega_1\) in the plane
  perpendicular to \(\underline{J}\)

  \begin{itemize}

  \item
    i.e. looking down the \(\underline{J}\) axis
  \item
    call the angle between the \(\omega\) and \(\underline{J}\)
    directions \(\theta_S\)
  \end{itemize}
\end{itemize}

\[
    \begin{aligned}
    \Omega_Sdt &= \frac{d\omega_1}{\omega\sin\theta_S}
    \end{aligned}
\]

\begin{itemize}
\item
  Therefore, as viewed in the inertial frame, the 3 axis precesses
  around the angular momentum vector, \(\underline{J}\), with angular
  frequency
\end{itemize}

\[
    \begin{aligned}
    \Omega_S &= \frac{\dot{\omega}_1}{\Big(\frac{|\underline{\omega} \times \underline{J}}{|\underline{J}|} \Big)} \\
    \underline{\omega} \times \underline{J} &= \begin{vmatrix} \hat{i} & \hat{j} & \hat{k} \\ 0 & \omega_2 & \omega_3 \\ 0 & I_1\omega_2 & I_3\omega_3 \end{vmatrix} = \begin{pmatrix} I_3\omega_2\omega_3 - I_1\omega_2\omega_3 \\ 0 \\ 0 \end{pmatrix} \\
    |\underline{\omega} \times \underline{J}| &= (I_1 - I_3)\omega_2\omega_3,~ (I_1 > I_3) \\
    \implies \Omega_S &= \frac{I_1 - I_3}{I_1}\omega_2\omega_3 \Bigg/ \Bigg[\frac{(I_1 - I_3)\omega_2\omega_3}{J}\Bigg] \\
    &= \frac{J}{I_1}
    \end{aligned}
\]
\end{example}

\end{document}
