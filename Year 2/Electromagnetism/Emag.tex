\documentclass[a4paper, 11pt, normalem]{report}

\usepackage{../../../LaTeX-Templates/Notes}

\titlecontents{chapter}% <section-type>
    [0pt]% <left>
    {}% <above-code>
    {Lecture \thecontentslabel\quad}% <numbered-entry-format>
    {}% <numberless-entry-format>
    {\dotfill\contentspage}% <filler-page-format>
\titleformat{\chapter}{\fontsize{15}{17}\bfseries\normalfont}{\textbf{Lecture \thechapter}}{1em}{}
\titleformat{\subsubsection}{\fontsize{10}{13}\bfseries\scshape}{\textbf{\thesubsubsection}}{1em}{}
\setcounter{tocdepth}{4}
% \setcounter{secnumdepth}{1}

\renewcommand{\arraystretch}{1.2}

\newcommand\p{\partial}
\newcommand\E{\mathcal{E}}
\newcommand\uE{\unl{\E}}
\newcommand\B{\mathcal{B}}
\newcommand\uB{\unl{\B}}
\newcommand\del{\unl{\nabla}}
\newcommand\eno{\epsilon_0}
\newcommand\hi{\hat{i}}
\newcommand\hj{\hat{j}}
\newcommand\hk{\hat{k}}
\newcommand\hn{\hat{n}}
\newcommand\hr{\hat{r}}

\title{Foundations of Physics 2A \\ Electromagnetism \vspace{-20pt}}
\author{Professor D P Hampshire}
\date{\vspace{-15pt}Epiphany Term 2018}
\rhead{\hyperlink{page.1}{Go to TOC}}

% \begin{example}[Title]
%
% \end{example}

\begin{document}

\maketitle
\tableofcontents

\chapter{}

\textit{Summary sheets are provided on DUO to go with these}

\section{Maxwell's Equations and Classical Physics}
Feynman claims that there are seven equations that describe all of classical physics; the first four of these are Maxwell's Equations:
\begin{enumerate}
    \item From Coulomb's Law:
        \begin{equation}
            \del\cdot\uE = \frac{\rho}{\epsilon_0} \tag{M\RN{1}}
        \end{equation}
    \item Given no magnetic monopoles have been observed:
        \begin{equation}
            \del\cdot \uB = 0 \tag{M\RN{2}}
        \end{equation}
    \item From Faraday's Law of Induction:
        \begin{equation}
            \del \times \uE = -\frac{\p \B}{\p t} \tag{M\RN{3}}
        \end{equation}
    \item From Ampere's Law:
        \begin{equation}
            \del \times \uB = \mu_0 \unl{J} + \mu_0 \eno \frac{\p \E}{\p t} \tag{M\RN{4}}
        \end{equation}
        where the symbol $\del$ denotes the vector operator 'del':
        \begin{equation*}
            \del = \hi \frac{\p}{\p x} + \hj \frac{\p}{\p y} + \hk \frac{\p}{\p z}
        \end{equation*}
        \begin{itemize}
            \item $\uE$ - electric field ($V\,m^{-1}$)
            \item $\uB$ - magnetic field ($T$)
            \item $\rho$ - total charge density ($C\,m^{-3}$)
            \item $\unl{J}$ - total current density ($A\,m^{-2}$)
        \end{itemize}
    \item Force on a moving charge in a magnetic and electric field:
        \begin{equation*}
            \unl{F} = q(\uE + \unl{v} \times \uB)
        \end{equation*}
    \item Newton's Law of Motion:
        \begin{equation*}
            \unl{F} = \frac{d\unl{p}}{dt}, ~ \unl{p} = \frac{m\unl{v}}{\sqrt{1 - \frac{v^2}{c^2}}}
        \end{equation*}
    \item Newton's Law of Gravity:
        \begin{equation*}
            \unl{F} = -\frac{Gm_1 m_2}{r^2}\hat{r}_{1 \to 2}
        \end{equation*}
\end{enumerate}

\subsection{General Comments}
\begin{itemize}
    \item There is no agreed order for Maxwell's equations.
    \item Maxwell's equations are completely general and valid at every point in space and time.
    \item There is a divergence equation (i.e. $\del \cdot$) for both $\uE$ and $\uB$ and a curl equation (i.e. $\del \times$) for each as well.
    \item $\uE$ and $\uB$ always refer to the net field.
\end{itemize}

\section{Maxwell's Equations and Vector Fields}
\subsection{The Flux of a Vector Field}
A vector field is fully characterised by knowing its magnitude and direction at every point in space and time.

An arbitrarily shaped three-dimensional 'closed surface' is the surface of the volume. \\
A smooth 'open surface' is also shown which does not enclose volume and has edges.
\begin{equation*}
    d\unl{S} = \hn \cdot dS
\end{equation*}
The vector $\hn$ is the outward normal unit vector to the surface and $\unl{h}$ is an arbitrary vector field.

By definition, the surface area element, $d\unl{S}$, is $d\unl{S} = \hn \cdot dS$, where dS is the scalar area of the element.
In general, we can say "the flux of a vector $\unl{h}$ through the surface is
\begin{equation*}
    \phi = \int \unl{h} \cdot \hn dS = \int \unl{h} \cdot d\unl{S}."
\end{equation*}

\subsubsection{Comments}
\begin{itemize}
    \item The flux of charge density is current.
    \item Flux is a useful concept for describing conservation laws - energy, momentum, charge, etc.
\end{itemize}
\emph{Be careful the joys of the English word 'density'}:
\begin{itemize}
    \item Mass density - $kg\,m^{-3}$
    \item Current density - $A\,m^{-2}$
\end{itemize}

\subsection{The Divergence Theorem}
The divergence theorem is a vector calculus identity:
\begin{equation*}
    \int \unl{h}\cdot d\unl{S} = \int \del \cdot \unl{h}dV
\end{equation*}
where $\unl{h}$ is any arbitrary vector field.

\subsection{Stoke's Theorem}
\begin{equation*}
    \int \unl{h}\cdot d\unl{l} = \int (\del \times \unl{h}) \cdot d\unl{S}
\end{equation*}
Note that $\unl{l}$ and $\unl{S}$ are vectors, V is scalar.
\begin{itemize}
    \item Divergence theorem: volume integral to surface integral
    \item Stoke's theorem: surface integral to line integral
\end{itemize}

\subsection{Vector Identities}
\begin{align*}
    \del \times (\del \times \unl{A}) &= \del(\del \cdot \unl{A}) - \nabla^2 \unl{A} \\
    \del \cdot (\unl{A} \times \unl{B}) &= \unl{B}\cdot(\del \times \unl{A}) - \unl{A} \cdot (\del \times \unl{B})
\end{align*}

\chapter{}
\section{Maxwell \RN{1}}
\subsection{Deriving Gauss' law and Maxwell \RN{1} from Coulomb's Law}
\begin{wrapfigure}{r}{0.4\textwidth}
    \begin{center}
        \includegraphics[scale=0.4]{fluxvec.png}
    \end{center}
\end{wrapfigure}
The cone first passes through a sphere centred about the charge and then through the surface of the arbitrary closed shape.

Consider first the sphere: \\
The electric field, $E(r')$, has constant magnitude over the surface of the sphere and is everywhere parallel to $d\unl{S}'$, i.e. $\hn\,dS'$ so we have:
\begin{equation*}
    \int \uE(r') \cdot d\unl{S}' = \frac{q}{4\pi\eno(r')^2}\cdot 4\pi(r')^2 = \frac{q}{\eno}
\end{equation*}
Consider now the flux through the elemental area $dS$, which is part of the surface of the arbitrary shape surrounding the charge.
We have:
\begin{equation*}
    \uE(\unl{r})\cdot \hn\;dS = |E(r)||\hn||\cos\theta|dS
\end{equation*}
Since $dS$ is a projection of $dS'$ (because they are both bound by the same cone), we can relate the two of them using:
\begin{equation*}
    \frac{dS'}{\pi(r')^2} = \frac{dS}{\pi r^2}\cos\theta
\end{equation*}
Substituting, we have:
\begin{align*}
    \uE(r)\cdot\hn\;dS &= E(r)\frac{r^2}{(r')^2}dS' \\
    &= \frac{q}{4\pi\eno r^2}\frac{r^2}{(r')^2} dS' \\
    &= \frac{q}{4\pi\eno(r')^2}dS'
\end{align*}
Hence the flux through the surface $dS$ is the same as the flux through $dS'$. \\
Integrating gives:
\begin{equation*}
    \int \uE\cdot d\unl{S} = \frac{q}{\eno}
\end{equation*}
Applying superposition to a collection of charges inside the arbitrary shape gives:
\begin{equation}
    \int \uE\cdot d\unl{S} = \frac{\sum q}{\eno} = \frac{1}{\eno}\int \rho\;dV \tag{Gauss' Law}
\end{equation}
Where $\rho$ is the charge density and then we can use the divergence theorem:
\begin{equation*}
    \int \del \cdot \uE dV = \frac{1}{\eno} \int \rho\;dV
\end{equation*}
Since the volume integrals are equal for any arbitrary volume, the integrands must also be equal so:
\begin{equation*}
    \del \cdot \uE = \frac{\rho}{\eno} \tag{M\RN{1}}
\end{equation*}
We conclude that Maxwell's 1st equation is a differential form of Gauss' Law, which in turn is a form of Coulomb's Law. \\
Coulomb's Law (or equivalently Gauss' Law) is only true under static conditions.
Just as with Hooke's Law, experimentation shows us that M\RN{1} is true under all conditions.

The $\uE$-field resulting from any charge distribution gives $\del \cdot \uE = 0$ in the local regions where there is no charge.

\chapter{}
\section{Applying M\RN{1} to a Capacitor Plate}
\begin{wrapfigure}{r}{0.4\textwidth}
    \begin{center}
        \vspace{-20pt}
        \includegraphics[scale=0.4]{cap.png}
        \vspace{-90pt}
    \end{center}
\end{wrapfigure}
There is no charge density so:
\begin{align*}
    \del\cdot\uE &= \frac{\p E_x}{\p x} + \frac{\p E_y}{\p y} + \frac{\p E_z}{\p z} \\
    \del\cdot\uE &= \frac{1}{r^2}\frac{\p}{\p r}(\underbrace{r^2 E_r}_{= 0\text{ (Coulomb)}}) + \frac{1}{r\sin\theta}\frac{\p}{\p \theta}\left[E_\theta\sin\theta\right] + \frac{1}{r\sin\theta}\frac{\p}{\p \phi}(E_\phi) = 0
\end{align*}

\subsection{Comments}
\begin{itemize}
    \item $\del\cdot\uE$ describes the spatial dependence of the $\uE$-field
    \item $\del\cdot\uE = 0$ is true at every point where there is no charge
    \item $\del\cdot\uE = 0$ does not necessarily imply $\uE = 0$
    \item $\del\cdot\uE = 0$ can result from any superposition of diverging $\uE$-fields from isolated charges where $\uE$ is exactly proportional to $\frac{1}{r^2}$
    \item Understanding the physics helps to understand the mathematics
\end{itemize}

\section{Maxwell \RN{2} (No magnetic monopole)}
\subsection{Ampere's Law}
Ampere found that two parallel straight wires carrying currents $I_1$ and $I_2$ in the same direction lead to a force $F_2$ on a length L of the wire carrying $I_2$ where
\begin{equation}
    \unl{F}_2 = -\frac{\mu_0 I_1 I_2 L}{2\pi d}\hr_{1\to2} \tag{A}
\end{equation}
and d is the separation between the wires.

We can introduce a magnetic field, where
\begin{equation}
    \unl{F} = q(\unl{v}\times\uB). \tag{B}
\end{equation}
Given that the definition of current is the charge that passes per second,
\begin{equation}
    I = Q_L\cdot v \tag{C}
\end{equation}
where $Q_L$ is the charge per unit length, and v is the velocity of charges. \\
Comparing A and B with C:
\begin{equation}
    \int \uB \cdot d\unl{l} = \mu_0 I \tag{Ampere's Law}
\end{equation}
where I is the current that passes through the surface that is bounded by the path integral of the magnetic field.
\begin{align*}
    \int \uB\cdot d\unl{l} &= \uB\cdot2\pi r = \mu_0 I \\
    \implies \uB &= \frac{\mu_0 I}{2\pi r}\hat{\phi} \\
\end{align*}

\subsection{Maxwell \RN{2} from the Biot-Savart Law}
Ampere's Law can be written in the form of the Biot-Savart Law that describes the small field ($\delta\uB$) produced by a small length of a current element ($I\,d\unl{l}$) where:
\begin{align*}
    \delta \uB = \frac{\mu_0 I}{4\pi r^2}\delta\unl{l} \times \hr &= \frac{\mu_0 I}{4\pi r^2}\sin\theta\,dl\,\hat{\phi} \\
    \del \cdot (\delta\uB) = \frac{1}{\rho}\frac{\p(\rho\,\delta B_\rho)}{\p \rho} &+ \frac{1}{\rho}\underbrace{\frac{\p\, \delta B_\phi}{\p \phi}}_{= 0 \text{ (B-S)}} + \frac{1}{r\sin\theta} \frac{\p}{\p \theta}(\delta B_z) = 0 \\
    \del\cdot(\delta\uB) &= 0 \\
    \sum \delta\uB &= \uB
\end{align*}
This leads to Maxwell's second equation:
\begin{equation}
    \del\cdot\uB = 0 \tag{M\RN{2}}
\end{equation}
The net flux through both surfaces is zero. \\
Integrating Maxwell \RN{2}:
\begin{equation*}
    \int \del\cdot\uB\,dV = \int 0\cdot dV \implies \int \uB\cdot d\unl{S} = 0
\end{equation*}
Hence in all experiments to date, we have found that the net magnetic flux through any closed surface is zero.

\chapter{}
\section{Maxwell \RN{3} from Faraday's Law}
\subsection{Faraday's Fabulous Experiments}
\begin{itemize}
    \item \unl{Inductive Electromotive Force}: \\
    The voltage difference across a single loop wrapped around a coil (or solenoid) is:
    \begin{equation}
        V = \frac{d\uB}{dt}\unl{S} = \frac{\p \phi_{\B}}{\p t}, \tag{Faraday's Law of Induction}
    \end{equation}
    where $\phi_\B$ is the flux of the magnetic field.
    Outside the loop, the $\uB$-field is zero, but the changing magnetic field inside the solenoid generates an $\uE$-field outside and inside the solenoid.
    \item \unl{Motional Electromotive Force}: \\
    A loop of wire moving with velocity, $\unl{u}$, down through the homogeneous field produced between the two magnetic pole pieces.
    Using Faraday's Law of Induction,
    \begin{equation*}
        V = \frac{\p \phi_{\B}}{\p t} = \B\frac{\p S(t)}{\p t} = \B\unl{u}\omega,
    \end{equation*}
    where $S(t)$ is the instantaneous partial area of the loop with magnetic field, $\uB$, passing through it.
    Alternatively, we can think about the Lorentz force on the free positive charges in the wire in the horizontal (bottom) part of the loop of wire.
    The charges build up at one end of wire and the net force on free positive charges is
    \begin{equation*}
        \unl{F} = q(\uE + \unl{v}\times\uB).
    \end{equation*}
    In equilibrium,
    \begin{equation*}
        \uE = -\unl{v}\times\uB.
    \end{equation*}
    Therefore the net voltage produced is
    \begin{equation*}
        V = -\omega \E = \B u \omega
    \end{equation*}
    Maxwell was the first to point out (leading Einstein to relativity) that the voltage generated in the experiment can be explained using two completely different physical explanations.
\end{itemize}
Faraday's experiments demonstrate that the law of induction is correct no matter why the flux changes; whether the loop moves or the pole pieces move.

\subsection{Using Faraday's Law to find Maxwell \RN{3}}
Faraday's Law:
\begin{align*}
    \zeta_t &= -\frac{\p \phi_\B}{\p t} \\
    \zeta_t &= \int \uE \cdot d\unl{l} = -V,
\end{align*}
where $\zeta_t$ is the electromotive force (the tangential force per unit charge, integrated over length, once around the complete circuit). \\
Using Faraday's Law,
\begin{equation*}
    \zeta_t = \int \uE\cdot d\unl{l} = -\frac{\p \phi_\B}{\p t} = -\frac{\p}{\p t}\int \uB \cdot d\unl{S}.
\end{equation*}
Then using Stoke's Theorem:
\begin{equation*}
    \int (\del \times \uE) \cdot d\unl{S} = -\frac{\p}{\p t}\int \uB \cdot d\unl{S}
\end{equation*}
Therefore, the integrands can be equated to get:
\begin{equation}
    \del \times \uE = -\frac{\p \uB}{\p t} \tag{M\RN{3}}
\end{equation}

\subsection{Comments}
\begin{enumerate}
    \item The curl of $\uE$ is zero outside the coil, but $\uE$ is not zero.
    \item Maxwell \RN{3} is true at all space and time.
\end{enumerate}





















































\end{document}
