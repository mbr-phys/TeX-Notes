\documentclass[a4paper,11pt,normalem]{article}
\usepackage{../../../LaTeX-Templates/Notes}

\titlecontents{section}
    [0pt]
    {}
    {Lecture \thecontentslabel\quad}
    {}
    {\dotfill\contentspage}
    \titleformat{\section}{\fontsize{12}{15}\normalfont}{\underline{\textbf{Lecture \thesection}}}{1em}{}

\rhead{}
\newcommand{\HRule}{\rule{\linewidth}{0.5mm}}

\begin{document}
{\centering
{\includegraphics[scale=0.5]{../../logo0.png}\hfill{\Large\bfseries Epiphany 2018}}\\[1.5cm]
{\LARGE\bfseries Foundations of Physics 2B}\\[0.5cm]
\HRule \\[0.3cm]
{\huge\bfseries Condensed Matter Physics}\\[0.1cm]
\HRule \\[1cm]}
\begin{center}
\begin{minipage}{0.4\textwidth}
    \begin{flushleft} \large
        \emph{Author:} \\ Matthew Rossetter
    \end{flushleft}
\end{minipage}~
\begin{minipage}{0.4\textwidth}
    \begin{flushright} \large
        \emph{Lecturer:} \\ Prof. Douglas Halliday
    \end{flushright}
\end{minipage}
\end{center}
\section{}
\subsection{Describing Crystals}

\begin{itemize}
    \item regular periodic array of atoms - highly defined
    \item x-rays discovered in 1912
        \begin{itemize}
            \item diffraction of x-rays key to studying crystals
        \end{itemize}
    \item crystals classified by certain physical properties
    \item a perfect crystal is assumed to be a regular array of repeating points
        \begin{itemize}
            \item we can construct a set of theoretical points in 3D (defined by vectors), called a lattice
        \end{itemize}
    \item lattice described by unit vectors \(\vec{a}_1,\vec{a}_2,\vec{a}_3\), called the lattice constants
    \item lattice given physical reality by placing atoms at lattice points
        \begin{itemize}
            \item these atoms are called a basis - there can be more than one atom in a basis, e.g. NaCl
        \end{itemize}
    \item use the relationship, where: \(\vec{r},\vec{r}'\) are points on the lattice; \(a_i\) are unit vectors; and \(n_i\) are scalar multiples -
        \begin{align*}
            \vec{r}' = \vec{r} + n_1\vec{a}_1 + n_2\vec{a}_2 + n_3\vec{a}_3
        \end{align*}
    \item it is called a primitive lattice if this equation cannot be reduced
    \item think of lattices as 3D constructs for filling space
    \item crystals have a high degree of symmetry
\end{itemize}

\subsection{Symmetry Operators}

\begin{enumerate}
    \item translation
    \item rotation
    \item reflection
    \item inversion
    \item combinations of above
\end{enumerate}

\begin{itemize}
    \item a lattice should remain invariant under specific symmetry operations
    \item point operators in 2D lead to 2D lattices, of which you can get different types -
        \begin{enumerate}
            \item square
            \item hexagonal
            \item rectangular
            \item centred rectangular
            \item oblique parallelogram
        \end{enumerate}
        \begin{itemize}
            \item cannot get a five-fold symmetry shape
        \end{itemize}
\end{itemize}

\subsection{Three-Dimensional Lattices}

\begin{itemize}
    \item there are seven basic crystal systems
        \begin{enumerate}
            \item triclinic
            \item monoclinic
            \item orthorhombic
            \item tetragonal
            \item rhombohedral (trigonal)
            \item hexagonal
            \item cubic
        \end{enumerate}
    \item use parameters to define these -
        \begin{itemize}
            \item p - primitive
            \item i - interstitial
            \item f - face-centred
            \item c - base-centred
        \end{itemize}
    \item by varying the parameters for each basic type (see table in lecture summary), get 14 Bravais lattices in 3D
        \begin{itemize}
            \item these are the basic building blocks of all crystals
        \end{itemize}
\end{itemize}

\subsection{Miller Indices}

\begin{itemize}
    \item \textbf{key concept for categorising crystals}
    \item describe a particular crystallographic plane or orthogonal direction in crystal
    \item effectively describes crystals as families of parallel planes
    \item method for determining the index:
        \begin{enumerate}
            \item find the intercepts of plane on crystal axes - the three lattice constants
            \item take the reciprocal of these constants
            \item reduce to 3 integers with the same ratio
            \item this gives the index of the plane, using the notation \((hkl)\), or \((\nu_1 \nu_2 \nu_3)\) in Kittel
            \item if one of the indices is negative, put a bar above the magnitude
        \end{enumerate}
    \item separation between planes:
        \begin{align*}
            d = \frac{1}{\sqrt{\frac{h^2}{a_1^2} + \frac{k^2}{a_2^2} + \frac{l^2}{a_3^2}}}
        \end{align*}
    \item For a cubic, this reduces to
        \begin{align*}
            d_{hkl} = \frac{a}{\sqrt{N}}
        \end{align*}
\end{itemize}

\section{}
\subsection{X-Ray Diffraction}

\begin{itemize}
    \item crystal is defined by a set of parallel planes separated by distance \(d\)
    \item waves incident on crystals will be diffracted - developed by Bragg and lead to the law of x-ray diffraction
    \item for each wave will experience specular reflection - small reduction in intensity
    \item path length difference \(A \to B \to C = 2d\sin\theta\)
    \item if path length is equal to an integer multiple of the wavelength of wave, get constructive interference
    \item Bragg Law:
        \begin{align*}
            2d\sin\theta = n\lambda
        \end{align*}
    \item typically, \(\lambda \approx 0.15\,nm\) for x-rays
    \item it is observed that each plane of atoms reflects \(10^{-3} - 10^{-5}\) of the intensity
    \item Bragg law is a consequence of periodic structure of crystals
    \item Fourier analysis is used
\end{itemize}

\subsection{Electron Density}

\begin{itemize}
    \item crystal lattice is defined by translation vector,
        \begin{align*}
            \underline{T} = n_1\vec{a}_1 + n_2\vec{a}_2 + n_3\vec{a}_3
        \end{align*}
    \item crystal is invariant under \(\underline{T}\) translation
    \item many physical properties related to electron density, \(n(\vec{r})\)
    \item crystal symmetry \(\implies n(\vec{r}) = n(\vec{r} + \underline{T})\), local electron environment is also invariant under \(\underline{T}\)
    \item consider electron density in one dimension:
        \begin{align*}
            n(x) = n_0 + \sum_p \left[c_p\cos\left(\frac{2\pi xp}{a}\right) + s_p\sin\left(\frac{2\pi xp}{a}\right)\right]
        \end{align*}
    \item \(p \in \mathbb{N}\); \(a =\) the lattice constant; and \(x =\) distance
    \item crystal symmetry also \(\implies n(x) = n(x + a)\)
\end{itemize}

\subsection{Reciprocal Lattice Points}

\begin{itemize}
    \item arguments of \(\sin\) and \(\cos\) are called reciprocal lattice points
        \begin{itemize}
            \item there is a factor of \(\frac{2\pi}{a}\) - requires functions to have correct periodicity
            \item units - \(\cos,\sin\) are dimensionless; \(\frac{2\pi p}{a}\) is the basis of summation
            \item only certain values are allowed by the relationship above
                \begin{align*}
                  n(x) = \sum_p n_p e^{\frac{i2\pi px}{a}}
                \end{align*}
        \end{itemize}
    \item allowed points in \(\sin()\) and \(\cos()\) are equivalent to families of planes described by Miller indices \((hkl)\)
    \item in 3D:
        \begin{align*}
            n(\vec{r}) = \sum_G n_G e^{\vec{G}\cdot\vec{r}}, ~ G = \text{reciprocal lattice vectors}
        \end{align*}
    \item \(G\) is defined as the family of reciprocal lattice points in 3D - each describing a family of crystal planes
    \item \(b_1,b_2,b_3\) are the reciprocal lattice unit vectors, units of frequency
        \begin{align*}
            b_1 = 2\pi\frac{\vec{a}_2 \times \vec{a}_3}{\vec{a}_1\cdot(\vec{a}_2 \times \vec{a}_3)} ~;~ b_2 = 2\pi\frac{\vec{a}_3 \times \vec{a}_1}{\vec{a}_1\cdot(\vec{a}_2 \times \vec{a}_3)} ~;~ b_3 = 2\pi\frac{\vec{a}_1 \times \vec{a}_2}{\vec{a}_1\cdot(\vec{a}_2 \times \vec{a}_3)}
        \end{align*}
    \item \(a_i\) are unit vectors of crystal
    \item \(a_i \cdot b_j = 2\pi\delta_{ij}\) - delta function
        \begin{align*}
            \underline{G} &= \nu_1\vec{b}_1 + \nu_2\vec{b}_2 + \nu_3\vec{b}_3 \\
            n(\vec{r} + \underline{T}) &= \sum_G n_G e^{i\underline{G}\cdot\vec{r}}\underbrace{e^{i\underline{G}\cdot\underline{T}}}_{2\pi \times p}
        \end{align*}
\end{itemize}

\section{}
\subsection{X-Ray Diffraction}

\begin{itemize}
    \item incident electromagnetic wave on crystal - \(\exp[9(\vec{k}\cdot \vec{r})]\)
        \begin{itemize}
            \item \(\vec{k}\) is the wavevector of the x-ray
        \end{itemize}
    \item elastic process - conservation of energy
    \item assume interaction between electric field of wave and electrons in atom
    \item electron density - \(n(\vec{r})\)
    \item scattered wave is described by
        \begin{align*}
            F &= \int n(\vec{r})\exp\left[i((\vec{k}-\vec{k}')\cdot\vec{r})\right]\,dV \\
              &= \int n(\vec{r})\exp\left[i(\vec{\Delta k}\cdot \vec{r})\right]\,dV
        \end{align*}
    \item \(\vec{\Delta k} = \vec{k} - \vec{k}'\)
    \item elastic scattering - \(|\vec{k}| = |\vec{k}'|\)
    \item \(\vec{\Delta k}\) - scattering vector
        \begin{itemize}
            \item for Bragg condition - \(\vec{\Delta k} = \vec{G}\)
        \end{itemize}
        \begin{align*}
            F = \sum_G \int n_G \exp\left[i(\vec{G} - \vec{\Delta k})\cdot\vec{r}\right]
        \end{align*}
    \item alternative formulation of Bragg condition -
        \begin{align*}
            \vec{k} + \vec{G} &= \vec{k}' \\
            \implies (\vec{k} + \vec{G})^2 &= |\vec{k}|^2 \\
            2\vec{k}\cdot\vec{G} + |\vec{G}|^2 &= 0 \\
            \implies 2\vec{k}\cdot\vec{G} &= |\vec{G}|^2
        \end{align*}
    \item n.b. \(\vec{\Delta k}\) has equivalent positive and negative values
\end{itemize}

\subsection{Brillouin Zones}

\begin{itemize}
    \item analogy to unit cells in reciprocal space
    \item first Brillouin zone is Wigner-Seitz primitive cell in reciprocal lattice
        \begin{itemize}
            \item used to describe a wide range of physical properties
        \end{itemize}
    \item construct Brillouin zone:
        \begin{enumerate}
            \item select origin in reciprocal space
            \item draw reciprocal lattice vector to all nearest neighbours
            \item perpendicular bisectors enclose first Brillouin zone
        \end{enumerate}
\end{itemize}

\paragraph{Examples of Reciprocal Lattices}

\begin{itemize}
    \item simple cubic lattice:
        \begin{align*}
            \vec{a}_1 &= a\hat{x} & \vec{a}_2 &= a\hat{y}, & \vec{a}_3 &= a\hat{z} \\
            \vec{b}_{1} &= \frac{2\pi}{a} \hat{x} & \vec{b}_2 &= \frac{2\pi}{a}\hat{y} & \vec{b}_3 &= \hat{z}
        \end{align*}
    \item reciprocal lattice is a simple cube with lattice constant \(\frac{2\pi}{a}\)
    \item body centred cubic lattice:
        \begin{align*}
            \vec{a}_1 &= \frac{1}{2}a(-\hat{x} + \hat{y} + \hat{z}) & \vec{a}_2 &= \frac{1}{2}a(\hat{x}-\hat{y}+\hat{z}) & \vec{a}_3 &= \frac{1}{2}a(\hat{x}+\hat{y}-\hat{z}) \\
            \vec{b}_1 &= \frac{2\pi}{a}(\hat{y} + \hat{z}) & \vec{b}_2 &= \frac{2\pi}{a}(\hat{x}+\hat{z}) & \vec{b}_3 &= \frac{2\pi}{a}(\hat{x}+\hat{y})
        \end{align*}
    \item these are primitive lattice vectors of fcc (face centre cubic) lattice
    \item similarly, reciprocal of fcc lattice is bcc lattice
\end{itemize}

\subsection{Structure Factor}

\begin{itemize}
    \item structure factor describes intensity of Bragg peaks
        \begin{itemize}
            \item arises because Bragg law considers parallel planes but can also get interference within unit cell
        \end{itemize}
    \item integral over unit cell describes total scattered intensity
        \begin{align*}
            F_G &= N \int_{cell} n(\vec{r}) \exp\left[-i\vec{G}\cdot\vec{r}\right] = NS_G
        \end{align*}
    \item \(N\) is the total number of cells; \(S_G\) is the structure factor for a single cell
    \item define origin at \(\vec{r} = 0\)
    \item consider \(n(\vec{r})\) as sum over all unique atoms in unit cell
        \begin{align*}
            n(\vec{r}) &= \sum_{j=1}^S n_j(\vec{r}-\vec{r}_j) \\
            S_G &= \sum_j \exp\left[-i\vec{G}\cdot\vec{r}\right] \int n_j(\vec{\rho}) \exp\left[-i\vec{G}\cdot\vec{\rho}\right] \\
            &= \sum_j f_j \exp\left[-i\vec{G}\cdot\vec{r}_j\right] \\
            S_G(\nu_1\nu_2\nu_3) &= \sum_j f_j \exp\left[-i2\pi (\nu_1x_j + \nu_2y_j +  \nu_3z_j)\right]
        \end{align*}
    \item \(\vec{\rho} = \vec{r} - \vec{r_j}, \vec{r}_j\) is position of unique atom in unit cell; \(f_j\) is atom form factor scattering of one atom
    \item \(\nu_1\nu_2\nu_3 = hkl\) describing Bragg peak; \((x_j,y_j,z_j)\) is position coordinates within unit cell
    \item for bcc lattice - have two unique atoms at coordinates \((000),(\frac{1}{2},\frac{1}{2},\frac{1}{2})\)
    \item evaluate \(S_G \implies S=0\) when \(\nu_1 + \nu_2+\nu_3 =\) odd integer; \(S=2f\) when \(\nu_1+\nu_2+\nu_3 =\) even integer
    \item fcc lattice - 4 atoms \((0,0,0),(0,\frac{1}{2},\frac{1}{2}),(\frac{1}{2},0,\frac{1}{2}),(\frac{1}{2},\frac{1}{2},0)\)
        \begin{itemize}
            \item \(S = 0\) when integers mixed
            \item \(S=4f\) when integers all odd or all even
        \end{itemize}
\end{itemize}

\section{}
\subsection{Crystal Bonding}

\begin{itemize}
    \item bonding is a stable equilibrium between attractive and repulsive force
    \item repulsive arises from electrons being fermions
        \begin{itemize}
            \item no two fermions occupy the same quantum state
            \item as electrons from adjacent atoms overlap, increases energy to satisfy Pauli exclusion principle
        \end{itemize}
    \item different types of bonds have different attractive forces:
\end{itemize}

\begin{enumerate}
    \item Van Der Waals bonding exists in almost all solid system - very weak force, usually only observed at low temperatures in noble gases
        \begin{itemize}
            \item attraction is between electric dipoles
                \begin{enumerate}
                    \item permanent
                    \item permanent-induced
                    \item two induced dipoles
                \end{enumerate}
            \item spherically symmetrical atom - when brought closer to another atom, electron distribution adjusts because of Coulomb potential
            \item can consider movement of charge as electric dipole:
                \begin{itemize}
                    \item amount of charge, \(q\), moving distance, \(L, \to\) dipole moment \(= p = qL\)
                    \item electric dipole consists of charge \(+q\) and \(-q\) separated by \(L\)
                    \item at arbitrary point, electric potential
                        \begin{align*}
                            V = \frac{Q}{4\pi \epsilon_0}\left(\frac{1}{r_b} - \frac{1}{r_a}\right)
                        \end{align*}
                    \item it can be shown that
                        \begin{align*}
                            V(r) = \frac{\vec{p}\cdot\hat{r_1}}{4\pi \epsilon_0 r^2}
                        \end{align*}
                    \item where \(\vec{p}\) is the electric dipole vector, and \(\hat{r}_1\) is the unit vector along \(\vec{r}\)
                    \item show potential energy and force are
                        \begin{align*}
                            U(r) &= \frac{A}{r^6} \\
                            F(r) &= -\frac{dU}{dr} = \frac{A}{r^7}
                        \end{align*}
                \end{itemize}
            \item modelling Pauli repulsion - very complex
                \begin{itemize}
                    \item approximate using empirical function
                    \item experimental data on solid gases shows that the function is of the form \(\frac{B}{r^{12}}\) fits data
                        \begin{align*}
                            U(r) = 4\epsilon\left[-\left(\frac{\sigma}{r}\right)^6 + \left(\frac{\sigma}{r}\right)^{12}\right]
                        \end{align*}
                    \item this is the Lennard Jones \(6-12\) potential - models interatomic potential in Van Der Waals solids
                        \begin{itemize}
                            \item \(4\epsilon\sigma^6 \equiv A\)
                            \item \(4\epsilon\sigma^{12} \equiv B\)
                        \end{itemize}
                \end{itemize}
        \end{itemize}
    \item other examples of bonding
        \begin{enumerate}
            \item ionic bonding - crystals made of positive and negative energy
                \begin{itemize}
                    \item many salts are under this (NaCl, LiF, MgCl)
                    \item overall energy of ionic crystal - ionisation energy, electron affinity
                    \item energy - electrostatic attraction
                        \begin{align*}
                            U(r) = -\frac{e^2}{4\pi\epsilon_0 r}
                        \end{align*}
                    \item this considers only nearest neighbours
                    \item in ionic crystals, energy must also consider other ions - not just nearest neighbours
                    \item interaction of all ions described by \(\underline{\text{modelling constant}}\)
                        \begin{itemize}
                            \item face centred cube crystal has modelling constant of \(1.7475\)
                        \end{itemize}
                \end{itemize}
          \item covalent crystals - sharing of electrons, generally occurs in systems of similar atoms (e.g. silicon semiconductor, diatomic gases)
              \begin{itemize}
                  \item covalent bonding can only be described using QM
                  \item two electrons, spin \(= \frac{1}{2}\)
                      \begin{itemize}
                          \item \(\uparrow \downarrow \to S = 0\) - spin antisymmetric
                          \item \(\uparrow \uparrow \to S = 1\) - spin symmetric
                          \item when spin is antisymmetric, position (wavefunction) is symmetric or vice versa
                          \item large electron density between atoms - forms a bond
                      \end{itemize}
              \end{itemize}
        \end{enumerate}
\end{enumerate}

\section{}

\begin{itemize}
    \item consider crystals as system of vibrating atoms
    \item family of excitations in solids, elastic waves - phonons
    \item range of phenomena suggests atoms vibrate
    \item describe crystal as series of parallel planes denoted by \(s,s\pm1,s\pm2,\cdots\)
    \item describe position of plane using coordinate \(u_s,u_{s\pm1},\cdots\),\(u_s\) is the displacement from the equilibrium
    \item longitudinal and transverse waves exist
    \item physics to describe motion?
        \begin{itemize}
            \item Hooke's law - elastic wave, restoring force, linear function of \(u_S\)
        \end{itemize}
    \item energy of oscillating system:
        \begin{align*}
            E = \frac{1}{2}kA^2
        \end{align*}
    \item spring constant - \(\omega = \sqrt{\frac{k}{m}}\)
    \item elastic energy is a quadratic function of displacement from mean position
    \item need to know the force exerted on individual planes
    \item \emph{assume only nearest neighbour interactions apply}
    \item forces acting on plane s:
        \begin{align*}
            F_s = c(u_{s+1}-u_s) + c(u_{s-1}-u_s)
        \end{align*}
    \item c is a force constant for the nearest neighbour
        \begin{align*}
            M\frac{d^2u_s}{dt^2} = c(u_{s+1}+u_{s-1}-2u_s)
        \end{align*}
    \item Assume SHM:
        \begin{align*}
            \frac{d^2u_s}{dt^2} = -\omega^2u_s
        \end{align*}
    \item Equation relates motion of planes:
        \begin{align*}
            -M\omega^2 u_s = c(u_{s+1}+u_{s-1}-2u_s)
        \end{align*}
    \item by substitution, the general form of equation is:
        \begin{align*}
            u_{s\pm1} = U\exp\left[i(s\pm1)ka\right] = U\exp\left(iska\right)\exp\left(\pm ika\right)
        \end{align*}
    \item \(U\) is the maximum amplitude, \(k\) is the wavevector of the elastic wave, and \(a\) is the spacing of adjacent planes
        \begin{align*}
            -\omega^2Mu\exp\left(iska\right) &= CU\left\{\exp\left[i(s+1)ka\right]+\exp\left[i(s-1)ka\right]-2\exp\left[iska\right]\right\} \\
            \omega^2M &= -C\left[\exp(ika) + \exp(-ika)-2\right] \\
            \omega(k)^2 &= \left(\frac{2c}{M}\right)(1 - \cos(ka)) \\
            \omega(k)^2 &= \frac{4c}{M}\sin^2\left(\frac{1}{2}ka\right) \\
            \implies \omega(k) &= \sqrt{\frac{4c}{M}}\left|\sin\left(\frac{1}{2}ka\right)\right|
        \end{align*}
    \item angular frequency depends on the wavevector - phenomena is known as dispersion
    \item waves of certain wavelength or wavevector travel at different velocities
\end{itemize}

\subsection{Group Velocity}

\begin{itemize}
    \item consider displacement of planes as a packet of elastic energy propagating through a crystal (phonons)
        \begin{align*}
            v_g = \frac{\partial \omega}{\partial k}
        \end{align*}
    \item velocity is related to the gradient of \(\omega(k)\) dispersion curve
        \begin{align*}
            v_g = \sqrt{\frac{Ca^2}{M}}\cos\left(\frac{1}{2}ka\right)
        \end{align*}
    \item low wavevector waves have higher velocity, waves at boundary of Brillouin zone have zero velocity
\end{itemize}

\subsection{Long Wavelength Limit}

\begin{itemize}
    \item applies to waves \(k \approx 0\), defined by \(ka \ll 1\)
    \item this corresponds to sound waves in crystal
    \item when \(ka \ll 1 \to \cos(ka) = 1 -\frac{1}{2}(ka)^2\)
    \item dispersion relation (long wavelength):
        \begin{align*}
            \omega^2 &= \left(\frac{c}{M}\right)k^2a^2 \\
            \omega &= \sqrt{\frac{c}{M}}ka
        \end{align*}
    \item \(\omega \propto k\) at long wavelengths
\end{itemize}

\section{}

\begin{itemize}
    \item consider two atom basis in phonon model - e.g. salts (NaCl), semiconductors (GaAs), etc
    \item use same equation of motion with \(M_1\) and \(M_2\) masses and \(U_{s,s\pm1\cdots},V_{s,s\pm1\cdots}\)
        \begin{align*}
            M_1\frac{d^2U_s}{dt^2} = c(V_s + V_{s-1} - 2U_s) ~;~ M_2\frac{d^2 V_s}{dt^2} = c(U_{s+1} + U_s - 2V_s)
        \end{align*}
    \item solutions is SHM - travelling wave, different amplitudes on adjacent planes \(u_s,v_s\)
    \item we define \(a\) as the distance between identical planes (\(M_1\) or \(M_2\))
        \begin{align*}
            U_s = U\exp(isKa)\exp(-i\omega t) ~;~ V_s = V\exp(isKa)\exp(-i\omega t)
        \end{align*}
    \item substitute travelling wave into equation of motion
        \begin{align*}
            -\omega^2 M_1U &= cv[1+\exp(-iKa)] - 2cu \\
            -\omega^2 M_2V &= cu[1+\exp(iKa)] - 2cv
        \end{align*}
    \item only solution obtained from determinant of matrix equation
        \begin{align*}
            \begin{vmatrix} 2c - M_1\omega^2 & -c[1+\exp(iKa)] \\ -c[1+\exp(iKa)] & 2c - M_2\omega^2 \end{vmatrix} &= 0 \\
            M_1M_2\omega^4 -2c(M_1+M_2)\omega^2 + 2c^2(1-\cos(Ka)) &= 0 \\
            \frac{c(M_1+M_2)}{M_!M_2} \pm \frac{C(M_1+M_2)}{M_1M_2}\sqrt{1 - 2\frac{M_1M_2(1-\cos(Ka)}{(M_1+M_2)^2}} &= \omega^2
        \end{align*}
    \item solution gives two branches in phonon dispersion relation
    \item consider two limits to illustrate general behaviour
        \begin{enumerate}
            \item when \(Ka \ll 1\) - long wavelength limit
            \item \(K = \pm\frac{\pi}{a}\) - boundary of first Brillouin zone
        \end{enumerate}
    \item for small \(Ka\) (long wavelength limit), \(\cos(Ka) \approx 1 - \frac{1}{2}K^2a^2\)
    \item two solutions of dispersion:
        \begin{align*}
            \omega^2 &\approx 2c\left(\frac{1}{M_1} + \frac{1}{M_2}\right) \\
            \omega^2 &\approx \frac{\frac{1}{2}c}{M_1M_2}K^2a^2
        \end{align*}
        \begin{enumerate}
            \item \(\omega\) is independent of \(K\) in the optical branch
                \begin{itemize}
                    \item two atoms out of phase
                \end{itemize}
            \item \(\omega \propto K\) in the acoustic branch
                \begin{itemize}
                    \item two atoms move in phase
                \end{itemize}
        \end{enumerate}
\end{itemize}

\subsection{Thermal Properties of Crystals}

\begin{itemize}
    \item phonon heat capacity:
        \begin{align*}
            C_V = \left(\frac{dU}{dT}\right)_V
        \end{align*}
    \item \(C_V\) used because no work done to change volume
    \item \(U\) is the total internal energy of the vibrating lattice
        \begin{align*}
            U_{tot} = \sum_k \sum_p U_{kp} = \sum_k\sum_p \langle n_{kp} \rangle \hbar \omega_{kp}
        \end{align*}
    \item where \(k\) is the wavevector, and \(p\) is the polarisation, and \(\langle n_{kp} \rangle\)
    \item \(\langle n_{kp} \rangle\) described by Planck distribution function:
        \begin{align*}
            n = \frac{1}{\exp\left(\frac{\hbar\omega}{k_BT}\right) - 1}
        \end{align*}
    \item number of vibrational nodes is called the density of states - number of vibrations per unit energy
        \begin{align*}
            D(\omega) = \frac{dN}{d\omega} = \left(\frac{vK^2}{2\pi^2}\right)\left(\frac{dK}{d\omega}\right)
        \end{align*}
    \item number of phonon nodes in a given frequency or energy range
\end{itemize}

\subsection{Debye Model}

\begin{itemize}
    \item assumption is that velocity of sound is constant
    \item Debye model dispersion relation: \(\omega = vK\)
    \item density of states goes to
        \begin{align*}
            D(\omega) = \frac{V\omega^2}{2\pi^2v^3} ~;~ D(\omega) \propto \omega^2
        \end{align*}
    \item maximum frequency range is Debye frequency:
        \begin{align*}
            \omega_D^3 = 6\pi^2v^3 \frac{N}{V}
        \end{align*}
    \item corresponds to Debye wavevector:
        \begin{align*}
            K_D = \frac{\omega_D}{v} = \left(6\pi^2\frac{N}{V}\right)^{1/3}
        \end{align*}
\end{itemize}

\subsection{Einstein Model}

\begin{itemize}
    \item assumes all phonons have the same frequency or energy
        \begin{align*}
            U = N\langle n\rangle\hbar\omega = \frac{N\hbar\omega}{\exp\left(\frac{\hbar\omega}{kT}\right) - 1}
        \end{align*}
    \item \(N\) is the total number of oscillators
\end{itemize}

\section{}

\subsection{Electrical Properties of Crystals from Classical Physics}

\begin{itemize}
    \item assumptions:
        \begin{enumerate}
            \item outer valence electrons are detached - free to move through the crystal
            \item electric field due to other electrons and nucleus cancel out
        \end{enumerate}
    \item Drude model - applied kinetic theory of gases to electrons
\end{itemize}

\begin{enumerate}
    \item specific heat capacity of electrons
        \begin{itemize}
            \item Mean kinetic energy \(E = \frac{3}{2}k_BT\)
            \item specific heat capacity per electron: \(C_V = \frac{dE}{dt} = \frac{3}{2}k_B\)
        \end{itemize}
    \item electrical conductivity
        \begin{itemize}
            \item begin with Ohm's Law, \(V = IR\)
            \item rewrite in dimensionless form, \(E = \rho J\)
            \item \(J = \sigma E\)
            \item Drude model assumes electrons collide with something
            \item describe using a mean time between collision events \(\tau\)
            \item equations of motion \(\underline{v} = \underline{v}_0 - \frac{|e|t\underline{E}}{m_e}\)
            \item electron velocity \(v_0\) is random - no overall contribution
                \begin{itemize}
                    \item considers only drift velocity in response to \(\underline{E}\)
                    \item electron drift velocity is average of \(-\frac{|e|t\underline{E}}{m_e}\)
                 \end{itemize}
                \begin{align*}
                    \bar{\underline{v}} &= -\frac{|e|\bar{t}\underline{E}}{m_e},~ \bar{t} = \tau \\
                    \underline{J} &= -n|e|\underline{v} = \frac{n|e|^2\tau}{m_e}\underline{E} \\
                    \implies \sigma &= \frac{n|e|^2\tau}{m_e}
                \end{align*}
            \item this is the Drude electrical conductivity formula
        \end{itemize}
    \item thermal conductivity of electrons
        \begin{itemize}
            \item temp gradient \(\frac{dT}{dz}\)
            \item assume electron is in thermal equilibrium at point of collision
            \item consider thermal energy carried by the electron
            \item thermal average is \(v_z^2 = \frac{k_BT}{m_e}\)
        \end{itemize}
        \begin{align*}
            Q &= -nv_z c_V v_z\tau \frac{dT}{dx} \\
              &= -\kappa \frac{dT}{dz} \\
            \kappa &= \frac{3}{2}n\frac{k_B^2T}{m_e}\tau
       \end{align*}
\end{enumerate}

\begin{itemize}
    \item comparison with ratio of thermal to electrical conductivity
\end{itemize}
\begin{align*}
    \frac{\kappa}{\sigma} = \frac{3}{2}\left(\frac{k}{e}\right)^2T
\end{align*}

\section{}
\subsection{Free Electron Model}

Assumptions:
\begin{enumerate}
    \item outer valence electrons detach - free to move around crystal
    \item effects of ions and electrons cancel - electrons move in region of no potential
\end{enumerate}
\begin{itemize}
    \item free electron model treats metal as empty box (zero potential) of dimensions (\(L_x,L_y,L_z\))
        \begin{itemize}
            \item inside the box, zero potential
            \item outside the box, infinite potential
        \end{itemize}
\end{itemize}

\paragraph{Periodic Boundary Conditions}

\begin{itemize}
    \item boundary used in this model - consequence of periodicity of crystals
        \begin{align*}
            \psi(\vec{r}) = \psi(\vec{r}+\vec{L}), \vec{L} = (L_x,L_y,L_z)
        \end{align*}
    \item wavefunction is assumed to be periodic with dimensions of sample space, \(\vec{L}\)
    \item this removes any limitation on the value of \(\vec{r}\)
        \begin{align*}
            \psi(x,y,z) = \psi(x+L_x,y,z) + \psi(x,y+L_y,z) + \psi(x,y,z+L_z)
        \end{align*}
\end{itemize}

\paragraph{Free Electron Wavefunction}

\begin{itemize}
    \item potential inside box is zero, so time-independent Schrodinger is
        \begin{align*}
            -\frac{\hbar^2}{2m_e}\nabla^2\psi(\vec{r}) &+ V(\vec{r})\psi(\vec{r}) = E\psi(\vec{r})\\
            \psi(x,y,z) &= A\exp\left[i(k_xx + k_yy + k_zz)\right], k_i = \frac{2\pi(l/m)}{L_i}\\
            E &= \frac{\hbar^2}{2m_e}\left(k_x^2 + k_y^2 + k_z^2\right)
        \end{align*}
    \item electron energy eigenstates are stationary (independent of time)
    \item amplitude \(A\) is constant, uncertainty in position coordinate, all energy states overlap
\end{itemize}

\paragraph{k-space}

\begin{itemize}
    \item reciprocal space
    \item can describe electrons using k-coordinate
    \item each electron has coordinate (\(k_x,k_y,k_z\)) in k-space, separated by \(\frac{2\pi}{l}\) in each dimension
    \item allowed points form mesh in k-space - each within a volume \(\left(\frac{2\pi}{L}\right)^3\)
        \begin{itemize}
            \item ``exclusion zone''
            \item no other allowed k-states within the volume
        \end{itemize}
    \item each k-state has 2 electron spin degeneracy - Pauli exclusion principle
    \item allowing for spin, we have
        \begin{align*}
            2\div\left(\frac{2\pi}{L}\right)^3 = \frac{L^3}{4\pi^3}
        \end{align*}
\end{itemize}

\paragraph{Fermi Energy and Surface}

\begin{itemize}
    \item maximum energy of system
    \item define Fermi energy as highest occupied energy level when system is in ground state (0 Kelvin)
        \begin{align*}
            E = \frac{\hbar^2k^2}{2m_e}
        \end{align*}
    \item surface of constant energy is constant, \(k^2\)
    \item Fermi surface is a sphere of radius \(k_F\)
    \item \(\frac{L^3}{4\pi^3}\) electron states per unit volume, so volume of sphere is \(V = \frac{4}{3}\pi k_F^3\)
    \item total number of electrons:
        \begin{align*}
            N &= \left(\frac{4}{3}\pi k_F\right)^3\left(\frac{L^3}{4\pi^3}\right) \\
            k_F &= \left(\frac{3N\pi^2}{L^3}\right)^{1/3} = (3\pi^2n)^{1/3}, n = \text{ electron density}\\
            E_F &= \frac{\hbar^2}{2m_e}k_F^2 = \frac{\hbar^2}{2m_e}(3\pi^2n)^{2/3}\\
        \end{align*}
\end{itemize}

\paragraph{Density of States}

\begin{itemize}
    \item number of electron energy states per unit energy range
    \item consider volume of k-space between \(k\) and \(k+\delta k\):
        \begin{itemize}
            \item volume is surface area \(\times \delta k = 4\pi k^2\delta^2\)
        \end{itemize}
    \item number of states between \(k\) and \(k+\delta k \to\)
        \begin{align*}
            n(k)\delta k - \frac{L^3}{4\pi^3}4\pi k^2 \delta k
        \end{align*}
    \item express energy:
        \begin{align*}
            n(E)\delta E = \frac{L^3}{\pi^2}k^2\delta k \implies n(E) = \sqrt{2}\frac{L^3}{\pi^2}\frac{n_e^{3/2}}{k^3}\sqrt{E}
        \end{align*}
\end{itemize}

\section{}
\subsection{Fermi-Dirac Distribution}

\begin{itemize}
    \item Fermi energy is energy of highest occupied state at 0 Kelvin (overall ground state)
    \item Fermi function describes occupation of energy levels
    \item at 0 Kelvin, states above \(E_F\) are empty \(f=0\), states below \(E_F\) are occupied \(f=1\)
    \item define an occupation number:
        \begin{align*}
            f(E) = \begin{cases} 1 & 0 < E \leq E_F \\ 0 & E > E_F \end{cases}
        \end{align*}
    \item can be considered a continuous distribution function
        \begin{align*}
            N = \int_0^{E_F} n(E)dE = \int_0^infty f(E)n(E)dE
        \end{align*}
    \item consider how function varies with temperature - energy range covering transition from \(f=1\) to \(f=0\) is broadened out at finite temperatures
    \item this is described by the Fermi-Dirac distribution function - derived by considering 3 constraints:
        \begin{enumerate}
            \item Conservation of Energy
            \item Conservation of Particle Number
            \item Subject to Pauli Exclusion principle
        \end{enumerate}
        \begin{align*}
            f(E) = \frac{1}{1 + \exp\left[\frac{(E-E_F)}{k_BT}\right]}
        \end{align*}
    \item this is a normalised statistical distribution function
    \item describes the probability of energy state \(E\) being occupied by an electron
\end{itemize}

\paragraph{Behaviour of Fermi-Dirac Function}

\begin{itemize}
    \item at low temperatures, \(k_BT \ll E_F\)
\end{itemize}
\begin{enumerate}
    \item when \(E < E_F \to \frac{E - E_F}{k_BT} \to\) large and negative
    \item when \(E > E_F \to \frac{E - E_F}{K_BT} \to\) large and positive, \(f(E) \approx 0\)
    \item when \(E \approx E_F\), transition from \(f(E) = 1 \to f(E) = 0\) occurs over narrow energy range around \(E_F\) - width is about \(k_BT\) on each side of \(E_F\)
\end{enumerate}
\begin{itemize}
    \item in systems with low densities of electrons, \(f(E) \ll 1\)
        \begin{itemize}
            \item approximation, when \(E_F \ll k_BT\):
                \begin{align*}
                    f(E) \approx \exp\left[-\left(\frac{E - E_F}{k_BT}\right)\right] \approx \exp\left(-\frac{E}{k_BT}\right)
                \end{align*}
            \item this behaves like the classical system, very low \(E_F\)
        \end{itemize}
\end{itemize}

\paragraph{Free Electron Heat Capacity}

\begin{itemize}
    \item can determine the electronic specific heat capacity using free electron model
    \item what happens when temperature is increased?
    \item only small proportions of electrons will increase their energy - those that are within \(k_BT\) of \(E_F\)
    \item we require am empty electron state for the excited electron to move to
    \item electrons within region \(k_BT\) of \(E_F\) will absorb thermal energy
    \item assume number of electrons with energy close to \(E_F\) is given by \(n(E_F)k_BT\)
    \item extra energy acquired by electron is \(k_BT\)
        \begin{align*}
            U(T) - U(0) = n(E_F)(k_BT)^2 - \text{ only for electrons}
        \end{align*}
    \item \(n(E_F)\) is the density of states at Fermi energy
        \begin{align*}
            C_V = \frac{dU}{dT} \approx 2n(E_F)k_B^2T
        \end{align*}
    \item note that this assumes \(n(E_F)\) is constant over energy range
        \begin{align*}
            n &= \frac{(E_F2m_e)^{3/2}}{3\pi^2\hbar^3} \\
            N &= n(E_F)E_FV \\
            n(E_F) = \frac{3}{2}\frac{N}{E_F} \\
            C_V \approx \frac{3}{2}k_BT\left(\frac{2k_BT}{E_F}\right)
        \end{align*}
    \item specific heat capacity is modified from classical value by bracketed factor
    \item electronic specific heat capacity is proportional to temperature
\end{itemize}

\section{}

\subsection{Magnetic Properties of Free Electrons}

\begin{itemize}
    \item Free electron model can predict magnetic properties
    \item how does metal respond when placed in magnetic field?
\end{itemize}

\paragraph{Magnetic Susceptibility of Metals}

\begin{itemize}
    \item metals develop an induced magnetic moment in magnetic fields
    \item interactions between B-field and electron spin
    \item it is known all materials show a weak paramagnetism which is independent of temperature
        \begin{itemize}
            \item this is parallel to applied field
        \end{itemize}
    \item use free electron model to demonstrate this observed effect
    \item electrons have a magnetic moment due to spin:
        \begin{align*}
            \mu_B = \frac{e\hbar}{2m_e} = 9.27\times10^{-24}\,J\,T^{-1}
        \end{align*}
    \item energy of electron will change in field by \(\pm \mu_B\) depending on spin
    \item assume equal numbers of \(\pm\) spin for electrons
        \begin{itemize}
            \item parallel
            \item anti-parallel
        \end{itemize}
    \item when B field applied:
        \begin{itemize}
            \item half of electrons increase energy by:
                \begin{align*}
                    + \frac{e\hbar}{2m_e}B \text{ - antiparallel}
                \end{align*}
            \item half of electrons reduce energy by:
                \begin{align*}
                    - \frac{e\hbar}{2m_e}B \text{ - parallel}
                \end{align*}
        \end{itemize}
    \item total energy of system can be reduced if some electrons reverse spins
    \item a proportion of electrons with antiparallel spins can reverse spins to reduce overall energy
    \item how many electrons reverse spin?
        \begin{itemize}
            \item need to have the same Fermi energy for spin up and spin down electrons
        \end{itemize}
    \item density of states function evaluated at \(E_F\) multiplied by change in energy gives number of electrons
    \item number of electrons within \(\mu_BB\) of the original Fermi energy: \(ne = \frac{1}{2}n(E_F)\mu_BB\)
    \item difference in population: \(n(E_F)\mu_BB\) - number of electrons with spin up increased by this amount
    \item net magnetic moment: \(n(E_F)\mu_B^2B\) - produces net magnetic moment per unit volume
        \begin{align*}
            M = \frac{\mu_B^2Bn(E_F)}{V}
        \end{align*}
    \item paramagnetic susceptibility, a measure of how easy it is to magnetise system:
        \begin{align*}
            \chi &= \frac{\partial M}{\partial H}, H = \frac{M}{\mu_0}
            \chi &= \mu_0\mu_B^2\frac{n(E_F)}{V}
        \end{align*}
    \item this is called the Pauli Paramagnetism - it is independent of temperature
    \item at finite temperatures, temperature dependence of Fermi distribution will lead to a small temp dependence
\end{itemize}

\subsection{Hall Effect}

\begin{itemize}
    \item observed in 1879
    \item consider current density, \(j\), flowing along bar in \(x\) direction:
    \item apply perpendicular magnetic field, \(B\)
    \item electrons experience Lorentz force - \(F = e(v\times B + E)\)
    \item electrons are pushed to one side of metal bar by this force
    \item electric field will compensate for motion due to Lorentz force - \(eE_y = -F \implies E_y = -v\times B\)
    \item current density, \(j = nev\)
        \begin{align*}
            E_y = -\frac{1}{ne}j\times B, \frac{1}{ne} = R_H
        \end{align*}
    \item \(R_H\) is the Hall coefficient
    \item the sign of the Hall coefficient shows the charge on the carriers
        \begin{itemize}
            \item some metals, however, have positive Hall coefficient
        \end{itemize}
\end{itemize}

\section{}
\subsection{Nearly Free Electron Model}

\begin{itemize}
    \item free electron model - ignored some interactions:
        \begin{enumerate}
            \item electron-atoms - free electron approximation
            \item electron-electron - independent electron approximation
        \end{enumerate}
    \item nearly free electron model includes electron-atom interactions
    \item failures of free electron model:
        \begin{itemize}
            \item temperature dependence of conductivity
            \item some metals have a positive Hall coefficient
        \end{itemize}
    \item interaction between electrons and crystal lattices?
        \begin{itemize}
            \item lattice \(\underline{R} = n_1\underline{a}_1 + n_2\underline{a}_2 + n_3\underline{a}_3\), where \(\underline{a}_1,\underline{a}_2,\underline{a}_3\) are lattice vectors
        \end{itemize}
\end{itemize}

\subsection{Bloch Theorem}

\begin{itemize}
    \item this is a consequence of periodic properties of crystals
        \begin{itemize}
            \item provides insights into behaviour of electrons in periodic potential Bloch states
            \item describes electrons moving in periodic potential
        \end{itemize}
    \item consider a 10 crystal (line of atoms). \(\psi(x)\) is solution satisfying time-independent Schrodinger equation
        \begin{itemize}
            \item Schrodinger equation has periodic potential \(V(x)\) representing atoms
            \item energy eigenvalues, \(\sigma\)
        \end{itemize}
    \item Schrodinger equation evaluated at \((x+R)\) must give same solution as as \((x)\)
        \begin{itemize}
            \item local electronic environment at \(x\) and \(x+R\) are identical
            \item have second solution \(\phi(x)\) which also satisfies the Schrodinger equation with energy \(E\)
            \item Assume \(\psi\) and \(\phi\) are unique solutions, can write \(\phi(x) = \psi(x+R)\)
            \item \(R\) is lattice vector \(=na\)
            \item \(\psi(x+R) = c(R)\psi(x)\) where \(c(R)\) is a constant equal to \(1\)
        \end{itemize}
    \item using series of lattice translations, \(c(R_1+R_2) = c(R_1)c(R_2)\)
        \begin{itemize}
            \item therefore \(c(nR) = [c(R)]^n,~ n \in \mathbb{N}\)
        \end{itemize}
    \item wavefunction must satisfy boundary conditions (periodic over M lattice translation where \(Ma = l\), the length of 10 crystals)
    \item Bloch theorem brings together two requirements to satisfy both periodic bonding conditions and lattice translation by \(n_1a_1\) where \(n_1\) is an integer
    \item from this we have \(\psi(x+Ma) = \psi(x) \implies [c(a)]^M = 1\)
        \begin{itemize}
            \item a functions that satisfies replacement for \(c(a) = \exp[ika]\), where \(ka = \frac{2\pi l}{M}\) (\(l \in \mathbb{Z}\))
        \end{itemize}
    \item for any lattice translation, \(\underline{R} = m\underline{a}\) (\(m \in \mathbb{Z},\underline{a}\) is lattice constant)
        \begin{itemize}
            \item \(c(R) = [c(a)]^M \equiv \exp[iMka] = \exp[ikR], R = \frac{2\pi l}{L}\) - \(l \in \mathbb{Z}\), \(L\) is the total dimension of sample
        \end{itemize}
    \item these statements set out Bloch's theorem - they explain the difference between free electron and nearly free electron models
    \item free electron model \(\psi(x) = C\exp[ikx]\) - plane wave with constant energy \(E\), \(k\) is the electron wavevector
    \item nearly free electron model \(\psi(x+R) = C\exp[ik(x+R)] = C\exp[ikR]\psi(x) = Cu_k(R)\psi(x),~ u_k(R)\) is the Bloch function
    \item the Bloch theorem tells us that nearly free electron wavefunctions (weak periodic potential) are composed of two parts:
        \begin{enumerate}
            \item plane wave free electron behaviour, \(C\exp[ikx]\)
            \item modulated in intensity by Bloch function, \(u_k(R)\), has periodicity of lattice
        \end{enumerate}
        \begin{itemize}
            \item fundamental nature of \(\psi\) is still free electron behaviour, but with a modification
        \end{itemize}
\end{itemize}

\paragraph{Consequences of Bloch Theorem}

\begin{itemize}
    \item adding multiples of \(\frac{2\pi}{a}\) to Bloch wavevectors does not alter solution \(\psi\)
        \begin{itemize}
            \item only \(k\) values in \(\frac{2\pi}{a}\) range are physically distinct
            \item all other values can be mapped into unique range
        \end{itemize}
    \item convention is to define this as \(-\frac{\pi}{a} \to \frac{\pi}{a}\)
        \begin{itemize}
            \item this corresponds to the first Brillouin zone
        \end{itemize}
\end{itemize}

\paragraph{Energy Band Diagrams}

\begin{itemize}
    \item shows electron behaviour in terms of energy and wavevector - free electron model \(E = \frac{\hbar^2k^2}{2m_e}\)
    \item branches of \(E(k)\) curve are moved into first Brillouin zone
\end{itemize}

\section{}

\subsection{Nearly Free Electron Energy Bands}

\begin{itemize}
    \item physical origin of energy gaps and energy bands:
        \begin{enumerate}
            \item Bragg reflection - electron waves can scatter from planes of atoms. Weak periodic potentiation Schrodinger equation. Gives corresponding values for energy gaps at \(k = \frac{n\pi}{a}\)
            \item interference at certain wavelengths, get interference between electron waves and atoms
        \end{enumerate}
\end{itemize}

\paragraph{Energy Bands}

\begin{itemize}
    \item describe relationship between energy and wavevector
    \item electrons of different `\(k\)' propagate at different velocities - dispersion
    \item travelling wave group velocity: $v_g = \frac{d\omega}{dk}$
    \item For electron: $v_g = \frac{1}{\hbar}\frac{dE(k)}{dk}$
    \item velocity of Block electrons (know \(E = \frac{\hbar^2k^2}{2m_e}\)). This gives:
        \begin{align*}
            v_g = \frac{1}{\hbar}\frac{d}{dk}\left(\frac{\hbar^2k^2}{2m_e}\right) = \frac{\hbar k}{m_e} = \frac{p}{m} = vk
        \end{align*}
    \item \emph{\(\hbar k\) is the crystal momentum}
    \item crystal momentum is the momentum an electron has as a result of interacting with periodic potential
    \item exhibits different physical parameters
\end{itemize}

\paragraph{Current Carried by Energy Bands}

\begin{itemize}
    \item we know that the current density is \(j = ne\langle v\rangle\), where \(n \equiv\) electron density, \(\langle v\rangle \equiv\) average velocity
        \begin{align*}
            \langle v\rangle = \frac{1}{\hbar}\int_{k=-\frac{\pi}{a}}^{k=\frac{\pi}{a}} \frac{dE}{dk}dk\,\frac{a}{2\pi}
        \end{align*}
    \item consider \(k = \frac{\pi}{a}\) nd \(k = -\frac{\pi}{a}\), these are physically equivalent states (from Block theorem)
        \begin{itemize}
            \item tells us that \(E\left(\frac{\pi}{a}\right) = 0 = E\left(-\frac{\pi}{a}\right)\)
        \end{itemize}
        \begin{align*}
            \langle v\rangle = \frac{a}{2\pi\hbar}\left[E\left(\frac{\pi}{a}\right) - E\left(-\frac{\pi}{a}\right)\right] = 0
        \end{align*}
    \item above implies average velocity of fulled energy band is zero, as is current density
    \item completely filled energy band carries no electrical current (insulators)
    \item current carried by partially filled bands (metals or semiconductors)
\end{itemize}

\subsection{Equation of Motion for Bloch Electrons}

\begin{itemize}
    \item consider force \(F\) applied to electron - \(Fv_g =\) rate of work being done
        \begin{align*}
            Fv_g &= \frac{dE}{dt} = \frac{dE}{dk}  \times \frac{dk}{dt} \\
            v_g &= \frac{1}{\hbar}\frac{dE}{dk} \implies F = \hbar\frac{dk}{dt}
        \end{align*}
    \item can predict how electron will respond in electric field
\end{itemize}

\subsection{Effective Mass}

\begin{itemize}
    \item electrons in Bloch states move as though the mass of electron is different form free electron masses
    \item consider \(E(k)\) relationship near band edge at \(k = k_0\) where \(\frac{dE}{dk} = 0\) (i.e. zero group velocity)
    \item general form given by \(E = E(k_0) + \frac{1}{2}A(k - k_0)^2\)
    \item group velocity: \(v_g(k) = \frac{A(k-k_0)}{\hbar}\) - compare with free electrons, where we have \(v_{free} = \frac{p}{m_e} = \frac{\hbar k}{m_e}\)
    \item electrons behave as though they have an effective mass of \(m_{eff} = \frac{\hbar^2}{A_{}}\)
    \item from Taylor's theorem, we can show that \(A = \frac{d^2E}{dk^2}\Big|_{k=k_0}\)
        \begin{align*}
            m_{eff} = \hbar^2\left[\frac{d^2E}{dk^2}\right]^{-1}_{k=k_0}
        \end{align*}
    \item second derivative is curvature - when \(E \propto k^2\) then \(m_{eff_{}}\) is constant
    \item some regions of \(m_{eff_{}}\) are positive, some regions are negative
    \item negative mass \(\implies\) electrons slow down in electric field, force is in opposite direction
\end{itemize}

\subsection{Electrons and Holes}

\begin{itemize}
    \item an energy band which is nearly filled has some vacant energy states near top of energy band
    \item consider vacant states of holes - charge of \(+e\), equivalent to negative effective mass
    \item hole wavevector \(k_h = -k_e\), energy \(E(k_h) = - E(k_e), v_h = v_e\)
\end{itemize}

\begin{enumerate}
    \item if energy band is full there is no current
    \item if energy band is partially full then electrons with $m_{eff}$ describe electrical response
    \item if energy band is almost completely filled, then holes with negative \(m_{eff_{}}\), positive charge
\end{enumerate}

\end{document}
