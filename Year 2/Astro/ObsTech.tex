\documentclass[a4paper,11pt,normalem]{article}
\usepackage{../../../LaTeX-Templates/Notes}

\titlecontents{section}
    [0pt]
    {}
    {Lecture \thecontentslabel\quad}
    {}
    {\dotfill\contentspage}
    \titleformat{\section}{\fontsize{12}{15}\normalfont}{\underline{\textbf{Lecture \thesection}}}{1em}{}

\rhead{}
\newcommand{\HRule}{\rule{\linewidth}{0.5mm}}

\begin{document}
{\centering
{\includegraphics[scale=0.5]{../../logo0.png}\hfill{\Large\bfseries Michaelmas 2017}}\\[1.5cm]
{\LARGE\bfseries Stars and Galaxies}\\[0.5cm]
\HRule \\[0.3cm]
{\huge\bfseries Observational Techniques}\\[0.1cm]
\HRule \\[1cm]}
\begin{center}
\begin{minipage}{0.4\textwidth}
    \begin{flushleft} \large
        \emph{Author:} \\ Matthew Rossetter
    \end{flushleft}
\end{minipage}~
\begin{minipage}{0.4\textwidth}
    \begin{flushright} \large
        \emph{Lecturer:} \\ Prof. Mark Swinbank
    \end{flushright}
\end{minipage}
\end{center}

\section{}
\begin{itemize}
    \item \emph{other stuff in notebook}
    \item Parts of atmosphere are opaque due to water vapour, \(O_3\), etc
    \item Correcting for atmospheric absorption:
\end{itemize}

\begin{align*}
    X &= 1 ~\text{airmass} \\
    X &= \sec(z) ~\text{airmasses} \\
    -\int_{I_C}^{I_O} \frac{dI}{I} &= \int_{0}^{X}k\,dX \\
    \ln{\frac{I_{obs}}{I_{corr}}} &= kX + c \\
    \frac{I_{obs}}{I_{corr}} &= e^{-kX} \\
    m_{obs} - m_{corr} &= -2.5\log{\frac{I_{obs}}{I_{corr}}} \\
    m_{obs} - m_{corr} &= -2.5\log{e^{-kX}} \\
                       &= 2.5kX\log{e} \\
    m_{corr} &= m_{obs} - A_\lambda(z = 0) \sec{z}
\end{align*}

\begin{itemize}
    \item Atmospheric refraction
        \begin{itemize}
            \item plane parallel atmosphere
            \item apply laws of refraction
            \item basic trig stuff
            \item always in small angle approx range
                \begin{equation*}
                    r = (n - 1)\tan{(z_0)}
                \end{equation*}
        \end{itemize}
    \item Refractive index also has wavelength dep
    \item atmos ref turns into an atmos dispersion
    \item disperses more for smaller wavelength
        \begin{itemize}
            \item 3 or 4 arcsecs
            \item a lot
        \end{itemize}
    \item Every object appears as a spectrum as colors separate
    \item atmos emission
        \begin{itemize}
            \item fluorescent emission
                \begin{itemize}
                    \item air glow
                \end{itemize}
            \item emits thermal radiation for TE
            \item Most emission is from OH molecules in upper atmos
                \begin{itemize}
                    \item vibrational and rotational movement
                \end{itemize}
        \end{itemize}
    \item want to try and stay away from regions with lots of this emission
    \item Other sources of emission:
        \begin{itemize}
            \item light pollution
                \begin{itemize}
                    \item from ground
                    \item from satellites and aircraft
                \end{itemize}
            \item zodiacal light
                \begin{itemize}
                    \item light scattered from interplanetary dust
                    \item in plane of the Solar System
                \end{itemize}
            \item scattered light
                \begin{itemize}
                    \item e.g. from the moon
                    \item telescope scheduling to dark, grat, and bright time
                \end{itemize}
        \end{itemize}
    \item more difficult observations at longer wavelengths
        \begin{itemize}
            \item more background issues
        \end{itemize}
    \item dust causes lots of interference
        \begin{itemize}
            \item at longer wavelengths, interaction between dust and photons is smaller
            \item interaction cross-section
        \end{itemize}
    \item easier to see through dust a lot easier and see other galaxies etc at longer wavelengths
    \item Atmospheric turbulence
        \begin{itemize}
            \item \emph{twinkle twinkle little star}
            \item Stars twinkle due to light getting bounced around in atmos
        \end{itemize}
    \item Angular resolution of telescope limited by Fraunhofer Diffraction
        \begin{itemize}
            \item \emph{see last year}
            \item Airy disk
            \item assume stars as point sources
            \item large telescope \(\implies\) small airy disk
            \item small telescope \(\implies\) large airy disk
            \item how close before two stars are seen as one?
        \end{itemize}
    \item Characterise resolution with Rayleigh criterion
        \begin{itemize}
            \item at some point the principle maximum of one star overlays with the principle minimum of the second
                \begin{itemize}
                    \item \emph{diffraction limit}
                \end{itemize}
            \item \(\theta_{dl} = 1.22\frac{\lambda}{D}\)
                \begin{itemize}
                    \item integrate round a cylinder using Bessel fns to get this
                    \item covered sort of later on in other module
                \end{itemize}
        \end{itemize}
    \item Atmos is constantly moving
        \begin{itemize}
            \item changing size, density, and temperature causes different path lengths ever dt for stars
            \item sum up over lots of dt for observing
                \begin{itemize}
                    \item causes blurring though
                \end{itemize}
            \item no longer airy disk, severely blurry
        \end{itemize}
    \item for atmos turbulence, the seeing is defined as minimum angle between two stars that can just be resolved
        \begin{itemize}
            \item typically in arcsec
            \item 50x worse than the diffraction limit
        \end{itemize}
    \item Detectors
        \begin{itemize}
            \item Charged Coupled Device
            \item little silicon micro-circuits
            \item little ray of capacitors
            \item discrete energy bands
                \begin{itemize}
                    \item conduction band and valence band
                    \item difference of \(\approx 1.1\)eV
                \end{itemize}
            \item upper cut-off wavelengths governed by band gap voltage difference
            \item lower wavelengths cut-off by absorption of photons into the silicon
            \item excellent Quantum efficiency
                \begin{itemize}
                    \item \(> 90\)\%
                \end{itemize}
            \item high dynamic range
            \item excellent linearity
            \item excellent stability
            \item still not enough pixels
        \end{itemize}
\end{itemize}

\section{}
\paragraph{Back to CCDs:}
\begin{itemize}
    \item Well Depth
        \begin{itemize}
            \item how many electrons can be stored in the upper state, usually 100s of thousands
        \end{itemize}
    \item use binary for how many levels for the signal
        \begin{itemize}
            \item i.e.~8 bit \(= 2^8 = 256\) levels
        \end{itemize}
    \item System Gain
        \begin{itemize}
            \item how many photo-electrons are required for digital output of 1
            \item small gain means reduced saturation signal
        \end{itemize}
\end{itemize}

\paragraph{Photometry}
\begin{itemize}
    \item Process of obtaining quantitive (numerical) values of the brightness of celestial objects
    \item CCD gives output prop to number of photons incident on each pixel
    \item Photometry takes raw data and corrects for noise from other sources
    \item Noise is just any interference for the image
    \item SNR (signal to noise ratio) defined as ratio of useful to non-useful data
    \item Poisson stats
        \begin{itemize}
            \item arrival of photons governed by this
            \item studied for how cameras observe sky stuff
            \item see stats last year
            \item Hughes and Hase and labs stuff
        \end{itemize}
        \begin{equation*}
            P(n,N) = \frac{\exp(-N)N^n}{n!}
        \end{equation*}
    \item High means approximates Gaussian stats
    \item mean is N
        \begin{itemize}
            \item also Variance
            \item std dev is \(\sqrt{N}\)
        \end{itemize}
    \item Telescope experiments can take eight hours or so
        \begin{itemize}
            \item so use Poisson errors for easy error in counts
        \end{itemize}
    \item Small error associated with read out
\end{itemize}

\paragraph{Basic Data Reduction to Correct for Background in
CCD}
\begin{itemize}
    \item Bias
        \begin{itemize}
            \item a zero second readout which results in a constant offset
            \item allows for understanding of the ``noise'' quantity
        \end{itemize}
    \item Dark
        \begin{itemize}
            \item CCD band stuff
            \item CCD will be in TE so will promote thermal photons
            \item thermal photons can hit detector and skew results
            \item this will increase in time
        \end{itemize}
    \item Flat Field
        \begin{itemize}
            \item variations in sensitivity
            \item varied energy ever so slightly across CCD
            \item quantum efficiency
            \item slight changes across the CCD in efficiency causes a non-uniform field across CCD
        \end{itemize}
    \item Also have sky background counts
        \begin{itemize}
            \item these are often the most significant contributor
        \end{itemize}
\end{itemize}
\begin{align*}
    \text{Final Frame} &= \frac{\text{Object Frame} - (\text{dark+bias})}{\text{Flat Field} - \text{(dark+bias)}} \\
                       &= \frac{\text{Object Frame} - (\text{dark+bias})}{\text{Flat Field} - \text{(dark+bias)}} - \frac{\text{Sky Frame} - (\text{dark+bias})}{\text{Flat Field} - (\text{dark+bias})} \\
                       &= \frac{\text{Object Frame} - \text{Sky Frame}}{\text{Flat Field} - \text{(dark+bias)}}
\end{align*}

\paragraph{Noise Sources}
\begin{itemize}
    \item Basic sources of noise are:
        \begin{enumerate}
            \item Readout noise, \(\sigma_{rd}\) electrons (Gaussian)
            \item Photon noise on the signal from the object (Poisson)
                \begin{itemize}
                    \item \(= \sqrt{f_{abj}t}\)
                \end{itemize}
            \item Photon noise on the signal from the sky background (Poisson)
                \begin{itemize}
                    \item \(= \sqrt{f_{bg}t}\)
                \end{itemize}
            \item Photon noise on the dark current (Poisson)
                \begin{itemize}
                    \item \(\sqrt{dt}\)
                \end{itemize}
        \end{enumerate}
    \item Uncorrelated noise sources can be added in quadrature
        \begin{itemize}
            \item \(\sigma_{\text{total}} = \sqrt{\sigma_{1}^{2} + \sigma_{2}^{2}}\)
        \end{itemize}
    \item Signal/Noise
        \begin{align*}
            SNR = \frac{S}{\sqrt{S + D + B + \sigma_{rd}^2}}
        \end{align*}
    \item S - signal
    \item B - background
    \item D - dark
    \item \(\sigma_{rd}\) - read error
    \item Prev equation assumes all the terms are in photo-electrons
    \item Will need to be accounted for if in ADU
    \item counts in number of photons
    \item gain can be set to more than 1
        \begin{itemize}
            \item confuses simple SNR eqn and changes what you plug in
        \end{itemize}
\end{itemize}

\paragraph{SNR Approximations}
\begin{itemize}
    \item Common approximations:
        \begin{enumerate}
            \item Photon noise limited on the object
                \begin{itemize}
                    \item signal dominates so can ignore other terms for SNR
                \end{itemize}
            \item Sky Limited
                \begin{itemize}
                    \item sky background dominates, only count background
                \end{itemize}
            \item Read Noise Limited
                \begin{itemize}
                    \item read background dominates, only count read term
                \end{itemize}
        \end{enumerate}
\end{itemize}

\section{}
\subsection{Spectroscopy}
\begin{itemize}
    \item Most useful tool in astro
    \item measurement of intensity of a light source
        \begin{itemize}
            \item function of wavelength
        \end{itemize}
    \item Different spectra:
        \begin{enumerate}
            \item light from source straight to detector
                \begin{itemize}
                    \item continuous spectrum
                \end{itemize}
            \item light from source travels through a cloud of gas straight to detector
                \begin{itemize}
                    \item continuous spectrum with dark lines
                \end{itemize}
            \item light from source travels into cloud and scatters through it to detector
                \begin{itemize}
                    \item bright line spectrum on black background
                \end{itemize}
        \end{enumerate}
    \item Types of spectrograph
        \begin{enumerate}
            \item Refraction (prisms)
            \item Diffraction gratings
            \item Interference (Fabry-Perot interferometer)
        \end{enumerate}
        \begin{itemize}
            \item focus on diffraction grating
        \end{itemize}
    \item Diffraction grating
        \begin{enumerate}
            \item Slit
                \begin{itemize}
                    \item need this to focus light from source of interest and block everything else
                \end{itemize}
            \item Collimating lens
                \begin{itemize}
                    \item make sure light lands parallel to diffraction grating
                \end{itemize}
            \item Diffraction grating
            \item Camera
        \end{enumerate}
    \item Condition for constructive interference:
        \begin{align*}
            n\lambda &= d\sin\theta \\
            \frac{d\theta}{d\lambda} &= \frac{n}{d\cos\theta}
        \end{align*}
    \item \(\frac{d\theta}{d\lambda}\) is known as angular dispersion (rad/nm)
        \begin{itemize}
            \item higher dispersions from higher spectral orders and smaller line spacings
            \item more convenient for Reciprocal Linear dispersion (\(\frac{d\lambda}{dx}\))
            \item measuring wavelength per unit x at detector (nm/mm)
            \item multiply \(\frac{d\theta}{d\lambda}\) by plate scale \(\frac{d\theta}{dx} = \frac{1}{f_{cam}}\)
        \end{itemize}
\end{itemize}
\begin{align*}
    \frac{d\lambda}{dx} = \frac{d\lambda}{d\theta}\frac{d\theta}{dx} = \frac{d}{f_{cam}n}\cos\theta
\end{align*}

\paragraph{Grating Equation}
\begin{itemize}
    \item For angles of incidence to grating
    \item For diffraction grating or reflection
\end{itemize}

\begin{align*}
    n\lambda &= d(\sin\alpha + \sin\beta) \\
    n\lambda\rho &= \sin\alpha + \sin\beta ~;~ \rho = \frac{1}{d}
\end{align*}

\paragraph{Resolving Power}
\begin{itemize}
    \item Recall angle for blurred star
        \begin{align*}
            \theta = 1.22\frac{\lambda}{D}
        \end{align*}
    \item Resolving power of a spectrograph is wavelength over band pass:
        \begin{itemize}
            \item \(\lambda\) is the wavelength
            \item \(\Delta\lambda\) is the minimum discernible difference in \(\lambda\)
        \end{itemize}
        \begin{align*}
            R &= \frac{\lambda}{\Delta\lambda} = nN \\
            R &= \frac{n\rho\lambda W}{\chi D_{T}}
        \end{align*}
    \item Where
        \begin{itemize}
            \item n is diffraction order\#
            \item N is number of lines
            \item \(\rho\) is the ruling density (lines/mm)
            \item \(\lambda\) is the wavelength
            \item W is the grating size
            \item \(\chi\) is the angular size of the image of a star on slit
            \item \(D_{T}\) is the telescope size
        \end{itemize}
    \item Don't want too narrow a slit
        \begin{itemize}
            \item optimise width of slit for photons from star
            \item spectral resolution gets blurred
        \end{itemize}
    \item Second equation above is for a practical spectrograph
        \begin{itemize}
            \item At most wavelengths, this value of R is much less than that given by \(nN\)
        \end{itemize}
\end{itemize}

\paragraph{CDs, DVDs, and Blu-Rays}

\begin{itemize}
    \item basically diffractions gratings
    \item DVDs store more info than CDs based on diffraction types
    \item Blu-Rays need UV light to make sense
\end{itemize}

\section{}
\subsection{Measuring Stars}

\begin{itemize}
    \item Black body radiation
        \begin{align*}
            E(\lambda, T) = \frac{2hc^2}{\lambda^5} \frac{1}{e^{\tfrac{hc}{\lambda kT}} - 1}
        \end{align*}
    \item Characteristic temperature is where \(\frac{dE}{d\lambda} = 0\), bump at top of curve
    \item Colours of stars depends on plot, nearest colour to peak is visible colour
        \begin{align*}
            L = 4\pi R^2 \sigma T^4
        \end{align*}
    \item Calc distance to star?
        \begin{itemize}
            \item use parallax
            \item define 1 parsec as distance corresponding to parallax of \(\theta = 1"\)
            \item 1 psc \(= 206265 AU\)
        \end{itemize}
\end{itemize}

\subsection{Interferometry}
\begin{itemize}
    \item Combines light from two telescopes
        \begin{itemize}
            \item makes it possible to measure stars
            \item interfere the light and measure phase difference
            \item diffraction limit: \(1.22\frac{\lambda}{D}\)
        \end{itemize}
    \item As star tracks across sky, path length changes
        \begin{itemize}
            \item phase will shift in and out of phase with movement
            \item more complicated for two light sources
            \item get a more complex fringe pattern
                \begin{itemize}
                    \item modulated by \(\frac{\lambda}{D}\) for each telescope
                \end{itemize}
        \end{itemize}
    \item Moving telescopes apart changes fringe pattern
        \begin{itemize}
            \item at some point apart, the fringe pattern will disappear and will resolve the star
            \item can then use maths to find \(\theta\) and find the radius using that and the distance away
            \item VLT uses more than two telescopes
        \end{itemize}
    \item Aperture synthesis
        \begin{itemize}
            \item a trick we need for observations
            \item path length will not change between two telescopes, if they come over parallel
            \item Will have a `y' pattern of telescope arrays so that path length will always be changing no matter what way it is passing over the sky
        \end{itemize}
\end{itemize}

\section{}
\begin{itemize}
    \item Zero-point mag gives one count
    \item \textbf{See example sheet from Lecture 6 for some good notes}
\end{itemize}

\subsection{Multi-Wavelength Techniques}
\begin{itemize}
    \item Missing a huge fraction of images outside visual
        \begin{itemize}
            \item how do we see the rest of it?
        \end{itemize}
    \item X-ray radiations
        \begin{itemize}
            \item electrons wizzing around
            \item Accelerated to high energies in plasma state
            \item effectively in about a million K
            \item protons will make electrons change path, and emit energy
            \item accretion disks generate some of this
        \end{itemize}
    \item Difficulties
        \begin{itemize}
            \item X-rays have too high energies
            \item mirrors absorb it and don't work
            \item very shallow angle mirrors focus instead
            \item Grazing incidence
        \end{itemize}
    \item UV radiation
        \begin{itemize}
            \item temperatures of around \(50 \, kK\)
            \item massive stars
            \item clumpy as all around clumps of new big stars forming in groups
        \end{itemize}
    \item Difficulties
        \begin{itemize}
            \item CCDs have lower QE for these lower energies
            \item hard to move energy level difference in CCDs to measure UV accurately
            \item swamped by other photons
            \item use a blocking filter to try and filter visual photons away and just get UV
        \end{itemize}
    \item Infra-red radiation
        \begin{itemize}
            \item begin to suffer from sky background here
            \item to do it accurately, you need to be in space
            \item see a `fuzz' tracing spiral arms on galaxies
                \begin{itemize}
                    \item hot dust in the interstellar medium being heated by stars
                    \item emission from cooler stars
                    \item globular clusters of old stars
                \end{itemize}
        \end{itemize}
    \item Sub-millimeter radiation
        \begin{itemize}
            \item looking at \(T = 3 \to 10 \, K\)
            \item challenging to detect such low energies
            \item very sensitive thermometers
            \item liquid helium at a few micro-Kelvin
            \item changes resistance and allows current to flow for a second
        \end{itemize}
    \item Why
        \begin{itemize}
            \item Pillars of Creation
            \item lots of dusty regions
                \begin{itemize}
                    \item actively forming stars in the dust clouds
                    \item carbonaceous material - graphite, diamonds etc
                    \item silicates
                    \item ices
                \end{itemize}
            \item optical photons increases dust temperature slightly, still around 10 K though
                \begin{itemize}
                    \item emits 100 micron wavelength photons to lose temperature
                \end{itemize}
            \item looking at Pillars in sub-millimeter shows clouds glowing now
            \item can observe nebulae very differently in sub-millimeter
        \end{itemize}
    \item Radio radiation
        \begin{itemize}
            \item 3 components
                \begin{itemize}
                    \item local thunderstorms
                    \item distant thunderstorms - radio waves bounce round atmosphere
                    \item constant hiss with period of 23 hours 56 minutes and 4.1 seconds
                    \item sidereal day
                \end{itemize}
            \item This hiss is the galactic emission
            \item surface of telescopes need to be `smooth'
            \item smoothness isn't as necessary for radios
                \begin{itemize}
                    \item easy to build big telescopes for radio without this concern
                \end{itemize}
            \item very difficult to get a high resolution radio telescope
        \end{itemize}
\end{itemize}

\section{}
\subsection{Radios Ctd}

\begin{itemize}
    \item Biggest telescope is FAST
        \begin{itemize}
            \item 500m diameter
        \end{itemize}
    \item Why observe in radio?
        \begin{itemize}
            \item 21cm
                \begin{itemize}
                    \item Neutral H emission
                \end{itemize}
            \item electron can have parallel or anti-parallel spin
                \begin{itemize}
                    \item two sub ground states
                \end{itemize}
            \item anti-parallel is lower energy than parallel so will eventually flip to this one
                \begin{itemize}
                    \item very small energy difference
                    \item hyper-fine energy splitting
                    \item this takes a few millions years though
                \end{itemize}
            \item lots of H in galaxies
                \begin{itemize}
                    \item probability adds up to observe this
                    \item pointing radio telescopes sees this
                \end{itemize}
        \end{itemize}
\end{itemize}

\subsection{Telescope Tech}

\begin{itemize}
    \item `Twinkling star'
        \begin{itemize}
            \item caused by atmosphere moving around and bumping image around
            \item break it up into sub-images
                \begin{itemize}
                    \item speckles
                \end{itemize}
            \item whole image will also move around
        \end{itemize}
    \item Fried parameter
        \begin{itemize}
            \item \(r_0 \approx 10\,cm\)
            \item size of turbulent cells
            \item coherence time
                \begin{itemize}
                    \item \(t_0 = \frac{r_0}{v}\)
                    \item v is wind speed
                    \item this means that a star will only be stable for about \(10\, ms\)
                \end{itemize}
        \end{itemize}
    \item Correcting this
        \begin{itemize}
            \item light comes in normally
            \item hits third mirror that can change angle with actuators
            \item then hits a beam splitter
                \begin{itemize}
                    \item 50\% to computer analyser
                    \item 50\% to somewhere else
                \end{itemize}
            \item computer constantly measures image and changes actuators to correct image for turbulence
                \begin{itemize}
                    \item uses fast Fourier transforms to get back to real image
                    \item happens every millisecond or so
                \end{itemize}
            \item this requires bright star though
            \item shine lasers up to \(15\,km\) into atmosphere to focus
                \begin{itemize}
                    \item this creates a fake star for corrections - `natural guide star'
                \end{itemize}
        \end{itemize}
\end{itemize}

\subsection{Exoplanets}

\begin{itemize}
    \item How do we observe planets against photon noise of stars?
        \begin{itemize}
            \item observe stellar spectrum and planet spectrum for comparison
            \item heavier molecules are more difficult to observe as they're lower down
                \begin{itemize}
                    \item refraction issues
                \end{itemize}
            \item detecting \(O_3\) would be a key trigger for life
                \begin{itemize}
                    \item not able to do it yet
                \end{itemize}
        \end{itemize}
\end{itemize}
\end{document}
