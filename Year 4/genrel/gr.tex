\documentclass[a4paper, 11pt, normalem]{report}

\usepackage{../../../LaTeX-Templates/Notes}
\usepackage{subfiles}
\usetikzlibrary{arrows.meta}
\usetikzlibrary{decorations.markings}
\usetikzlibrary{decorations.pathmorphing}
\tikzset{snake it/.style={decorate,decoration=snake}}

\newcommand\xo{\hat{X}_1}
\newcommand\xt{\hat{X}_2}
\newcommand\hsig{\hat{\sigma}}
\newcommand\hd{\hat{\unl{d}}}
\newcommand\hrho{\hat{\rho}}
\newcommand\vk{\unl{k}}

\title{General Relativity \vspace{-20pt}}
\author{Richard Bower}
\date{\vspace{-15pt}Epiphany Term 2020}
\rhead{\hyperlink{page.1}{Contents}}

\begin{document}

\maketitle
\tableofcontents
\chapter{}
Just intro stuff

\chapter{Introduction to Tensors}
\begin{itemize}
    \item Notation
    \item Coordinate transforms
    \item Contravariant tensors
    \item Covariant tensors
\end{itemize}

\section{Intro to Tensor Notation}
Consider the cartesian definition for $\vr$:
\begin{align}
    \vr = x\vec{i} + y\vec{j} + \vec{z}.
\end{align}
We have the basis vector $\{\vec{i},\vec{j},\vec{k}\}$ and coordinate values $\{x,y,z\}$.
We can write this in a different form as
\begin{align}
    \vr = x^1\ve_1 + x^2\ve_2 + x^3\ve_3.
\end{align}
Note: $x^2 \neq x*x$.
The 2 is an index, not a power. 
If we want to square something, we will write $(x^1)^2 = x^1x^1$.
We can rewrite the above again as
\begin{align}
    \vr = \sum_{i=1}^3 x^i\ve_i.
\end{align}
We can then simplify this further using the \textbf{Einstein summation convention}:
\begin{align}
    \vr = x^i\ve_i,
\end{align}
i.e. whenever there is a repeated index, we sum over them. 
Different letters will imply different things:
\begin{itemize}
    \item Roman letters $i,j,\dots$ - summing over 3D space
    \item Roman letters $a,b,c,\dots$ - summing over ND space
    \item Roman letters $A,B,\dots$ - summing over 2D space
    \item Greek letters $\alpha,\beta,\mu,\nu,\dots$ - summing over 4D space-time $\{x^0,x^1,x^2,x^3\}$, starting from 0 as time is different slightly, so $\{ct,x^i\}$
\end{itemize}

\section{Coordinate Transformation}
You may be used to 
\begin{align}
    x' = \gamma\left(x-\frac{vct}{c}\right),
\end{align}
where the extra $c$ factor to make time space-like. 
This notation can get confusing so instead we use:
\begin{align}
    x^{\bar{1}} = \gamma\left(x^1-\frac{v}{c}x^0\right),
\end{align}
where the 'bar' indicates new coordinate system. 

For a minute vector difference between points P and Q $d\vr$ in two coordinate systems, we can define $\ve_a$:
\begin{align}
    \vr(P) &= \ve_ax^a & \vr(P) &= \ve_{\bar{b}}x^{\bar{b}} \\
    d\vr &= dx^a\ve_a  \\
    \frac{\p\vr}{\p x^a} &= \ve_a & \frac{\p\vr}{\p x^{\bar{b}}} &= \ve_{\bar{b}}
\end{align}
But what is the relationship between these two coordinate systems?
Start with $x^{\bar{b}}=x^{\bar{b}}(x^a)$, and consider a general function 
\begin{align}
    f &= f(x^1,x^2,x^3) \\
    \Delta f &= \frac{\p f}{\p x^1}\Delta x' + \frac{\p f}{\p x^2}\Delta x^2 + \frac{\p f}{\p x^2}\Delta x^3 = \frac{\p f}{\p x^a}\Delta x^a
\end{align}
How do we get a small change in $x^{\bar{b}}$?
\begin{align}
    \Delta x^{\bar{b}} &= \frac{\p x^{\bar{b}}}{\p x^{a}}\Delta x^{a} \\
    dx^{\bar{b}} &= \frac{\p x^{\bar{b}}}{\p x^a}dx^a \\
    dx^{\bar{a}} &= \frac{\p x^{\bar{a}}}{\p x^b}dx^b
\end{align}
Notice how we can simply just switch round the indices - \textbf{these are all dummy variables and as long as the index notation is consistent, it is completely arbitrary which letter is used,} i.e. the letters themselves mean nothing.

\section{Tensors}
Any quantity which transforms as 
\begin{align}
    A^{\bar{b}} &= \frac{\p x^{\bar{b}}}{\p x^a}A^a
\end{align} 
is a Rank (1,0) or order 1 contravariant tensor.
What about $\ve_a$?
\begin{align}
    \vr &= x^a\ve_a = x^{\bar{b}}\ve_{\bar{b}} \\
    \ve_{\bar{b}} &= \frac{\p\vr}{\p x^{\bar{b}}} = \frac{\p\vr}{\p x^a} \frac{\p x^a}{\p x^{\bar{B}}} = \frac{\p x^a}{\p x^{\bar{b}}}\ve_a
\end{align}
So now we have reversed the position of the indices in Eq (2.15).

How do we define scalars? 
\begin{align}
    \del\phi &= \frac{\p\phi}{\p x^i}\ve_i \\
    \frac{\p\phi}{\p x^{\bar{j}}} &= \frac{\p x^i}{\p x^{\bar{j}}}\frac{\p\phi}{\p x^i}
\end{align}
In general, we have
\begin{align}
    A_{\bar{j}} &= \frac{\p x^i}{\p x^{\bar{j}}}A_i,
\end{align}
which we call a Rank (0,1) or order 1 covariant tensor.



\end{document}













