\documentclass[lasers.tex]{subfiles}

\begin{document}
\part{Quantum Information and Computing}
\chapter{}
\section{What is a quantum computer?}
It's like a classical computer, but we replace 'bits' (0s and 1s) with \emph{qubits.}
But what is a qubit?
A qubit is a 2-level quantum system, with the quantum levels referred to $|0\rangle$ and $|1\rangle$.
\emph{copy from notes.}

Examples of 2-level systems:
\begin{itemize}
    \item Spin-$\frac12$ particle: 2 states are spin 'up' and spin 'down'.
    \item Photon: 2 polarisations, e.g. vertical and horizontal or left-circular and right-circular.
    \item Atoms, ions, molecules with many energy levels and we can select 2 as our qubit states. 
    \item 'Artifical atoms' in solid state, e.g. quantum dots in semiconductors or LC resonator in a superconductor. 
\end{itemize}

List five physical implementations of qubits and their problems:
\begin{itemize}
    \item Two energy levels in an atom trapped by an optical tweezer - difficult to localise. 
    \item Two energy levels in an ion trapped using electrodes - it's only in one dimension (scaling problem). 
    \item Two energy levels of an impurity ion (spin) in a semi-conductor (e.g. phosphorous in Si) - interacts with surroundings, i.e. Silicon.
    \item Two energy levels of an LC circuit in a superconductor - very bulky and needs $10mK$ cryostat. 
    \item Two polarisation modes of a photon - photons don't interact. 
\end{itemize}

\chapter{}
\section{DiVincenzo Criteria}
The DiVincenzo criteria are often used to frame discussions about the advantages and disadvantages of different quantum computing platforms. 
The five criteria are:
\begin{enumerate}
    \item Initialisation (\textbf{state preparation}) - typically means the ability to prepare identical qubits (cooling) and address each qubit independently (localisation).
    \item A universal set of quantum \textbf{gates} - single- and two-qubit gates at minimum. 
    \item Measurement (\textbf{read out}) of $|0\rangle$ or $|1\rangle$.
    \item Low \textbf{decoherence} - qubits isolated from environment (external world).
    \item \textbf{Scalability} - the ability to scale up to say 100 or 1000 or more qubits.
\end{enumerate}

\section{Why Quantum Computing?}
\begin{enumerate}
    \item Moore's law - as transistor size is reduced, we approach atomic dimensions.  
    \item Energy efficiency - replace dissipative classical gates with reversible quantum gates.
    \item Quantum 'advantage' - quantum computers can store more information and compute (certain problems) much faster.
\end{enumerate}
A classical bit has states 0 and 1: N bits have $2N$ states; a qubit has states $0\rangle$ and $|0\rangle$: N qubits have $2^N$ states, i.e. exponential scaling. 
To see why, we need the \textbf{Qubit State Vector}.
\begin{align}
    |\psi\rangle = a|0\rangle + b|1\rangle,
\end{align}
where $a$ and $b$ are complex coefficients that may be time-dependent which obey the normalisation criterion. 

Now we want a 2 qubit state vector, where our two qubits are A and B, as a normalised product state:
\begin{align}
    |\Psi\rangle_{AB} &= (a|0\rangle_A + b|1\rangle_A)\otimes(c|0\rangle_B+d|1\rangle_B), \\
                      &= ac|00\rangle + ad|01\rangle + bc|10\rangle + bd|11\rangle.
\end{align}
In addition to product, we can have \textbf{entangled states} that are not factorisable into products.

What about 3 qubits, $A,B,C$?
\begin{align}
    |\Psi\rangle_{ABC} &= c_{000}|000\rangle + c_{001}|001\rangle + c_{010}|010\rangle + c_{011}|011\rangle + c_{100}|100\rangle + c_{101}|101\rangle + c_{110}|110\rangle + c_{111}|111\rangle,
\end{align}
which is $2^3=8$ states. 
For 4 qubits, we would have $2^4=16$ states; for $N$ qubits, $2^N$ states. 
For 40 qubits, $2^{40}\approx10^{12}$; for 100 qubits, $2^{100}\approx10^{30}$.

\chapter{Two-level quantum mechanics}
Can think of the state vector similar to spin:
\begin{align}
    |\psi\rangle &= a|0\rangle + b|1\rangle = a|\uparrow\rangle + b|\downarrow\rangle = \begin{pmatrix} a \\ b\end{pmatrix}.
\end{align}
How does the time-dependence appear?
$0\rangle$ and $|1\rangle$ are solutions of a Schrodinger equation,
\begin{align}
    i\hbar\dpt|\alpha\rangle &= H_0|\alpha\rangle,
\end{align}
with energies $E_0$ and $E_1$:
\begin{align}
    i\hbar\dpt|0\rangle &= E_0|0\rangle,\; i\hbar\dpt|1\rangle = E_1|1\rangle, \\
    |\psi(t=0)\rangle &= a(0)|0\rangle \implies a(t) = a(0)e^{-iE_0t/\hbar},\\
    |\psi\rangle &= ae^{-iE_0t/\hbar}|0\rangle + be^{-iE_1t/\hbar}|1\rangle, \\
                 &= e^{-E_0t/\hbar}\left(a|0\rangle + be^{-i\om_0 t}|1\rangle\right).
\end{align}
$\om_0$ is the angular resonant frequency of the qubit, $\om_0=(E_1-E_0)/\hbar$.
We define
\begin{align}
    G &\equiv e^{-iE_0t/\hbar} \text{ - global phase}, & R &\equiv e^{-i\om_0t} \text{ - relative phase}.
\end{align}
\begin{align}
    |a|^2 &+ |b|^2 = 1 \\
    |\psi\rangle &= a|0\rangle + be^{i\phi}|1\rangle,
\end{align}
So only have two free parameters in the ratio $\frac{a}{b}$ and the phase $\phi$.
\begin{align}
    |\psi\rangle &= \begin{pmatrix}\cos\frac{\theta}{2} \\ e^{i\phi}\sin\frac{\theta}{2}\end{pmatrix} = \begin{pmatrix} a \\ be^{i\phi}\end{pmatrix} 
\end{align}
$\tan\frac{\theta}{2} = \frac{b}{a}$ and $\theta$ and $\phi$ are angles (from $z$ down and $x$ round respectively as spherical coordinates. 
So we can talk about our state vector as a Bloch vector, with all possible states of the qubit are points on the Bloch sphere.

\section{Single-qubit gates}
All single qubits are rotations on the Bloch sphere. 
These rotations are described by unitary operator, e.g. $U$,
\begin{align}
    |\psi_f\rangle &= U|\psi_i\rangle,
\end{align}
where $U$ is a $2\times2$ matrix. 
We can write $U$ as a sum of Pauli spin matrices (and the identity matrix),
\begin{align}
    \hat{\sigma}_x &= \begin{pmatrix}0&1\\1&0\end{pmatrix}, & \hat{\sigma}_y &= \begin{pmatrix}0&-i\\i&0\end{pmatrix}, & \hat{\sigma}_z &= \begin{pmatrix}1&0\\0&-1\end{pmatrix}, & \hat{\sigma}_0 &= \begin{pmatrix}1&0\\0&1\end{pmatrix}.
\end{align}

\begin{example}[Density operator]
    The density operator is defined as:
    \begin{align}
        \hat{\rho} &= |\psi\rangle\langle\psi|.
    \end{align}
    We will write this in the form of the Pauli matrices and identity, then as a single matrix:
    \begin{align}
        \hat{\rho} &= \frac12\left(\hat{\sigma}_0 + u\hat{\sigma}_x + v\hat{\sigma}_y + w\hat{\sigma}_z\right) \\
                   &= \frac12\begin{pmatrix} 1+w & u-iv \\ u+iv & 1-w\end{pmatrix}.
    \end{align}
    $u,v,w$ are the expectation values of $\hat{\sigma}_x,\hat{\sigma}_y,\hat{\sigma}_z$ for the state $|\psi\rangle$.
\end{example}

\begin{example}[Short Question \#7]
    Let's take a particular state,
    \begin{align}
        |\psi\rangle &= \frac{1}{\sqrt{2}}\left(|0\rangle+e^{i\phi}|1\rangle\right).
    \end{align}
    What is the value of $\theta$?
    \begin{align}
        \cos\frac{\theta}{2} &= \sin\frac{\theta}{2} = \frac{1}{\sqrt{2}} \implies \frac{\theta}{2} = \frac{\pi}{4} \implies \theta = \frac{\pi}{2}.
    \end{align}
    Bloch vector is in equatorial plane. 
    Now calculate expectation values of $\hat{\sigma}_i$:
    \begin{align}
        \langle\psi|\hat{\sigma}_x|\psi\rangle &= \frac{1}{\sqrt{2}}\begin{pmatrix} 1 & e^{-i\phi}\end{pmatrix}\begin{pmatrix}0 & 1 \\ 1 & 0 \end{pmatrix}\frac{1}{\sqrt{2}}\begin{pmatrix}1 \\ e^{i\phi}\end{pmatrix} \\
                                               &= \frac12\left(e^{i\phi}+e^{-i\phi}\right) = \cos\phi.
    \end{align}
    $\phi=0\implies u=+1, \phi=\pi \implies u=-1$.
    \begin{align}
        \langle\psi|\hat{\sigma}_y|\psi\rangle &= \frac12\begin{pmatrix}1 & e^{-i\phi}\end{pmatrix}\begin{pmatrix}0 & -i \\ i & 0\end{pmatrix}\begin{pmatrix} 1 \\ e^{i\phi}\end{pmatrix} \\
                                               &= \frac{1}{2i}\left(e^{i\phi}+e^{-i\phi}\right) = \sin\phi. \\
        \langle\psi|\hat{\sigma}_z|\psi\rangle &= \frac12\begin{pmatrix} 1 & e^{-i\phi}\end{pmatrix}\begin{pmatrix}1 & 0 \\ 0 & -1\end{pmatrix}\begin{pmatrix}1 \\ e^{i\phi}\end{pmatrix} \\
                                               &= \frac12\left(1-1\right) = 0.
    \end{align}
    So the Bloch vector is in equatorial plain as thought. 
\end{example}













\end{document}



















