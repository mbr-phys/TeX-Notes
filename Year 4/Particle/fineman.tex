%NOTE: COMPILE IN TERMINAL USING LUALATEX
\documentclass[a4paper,11pt]{standalone}
%\documentclass[landscape,a4paper,11pt]{article}

\usepackage{graphics,graphicx,epsfig}
\usepackage{amsmath}
%\usepackage{cancel}
\usepackage{tikz-feynman}
%\usetikzlibrary{external}             %% Load the `external` library
%\tikzexternalize
%\immediate\write18{mkdir -p pgf-img}
%\tikzexternalize[                     %% Activate externalization
%  system call={                       %% Use lualatex in system call
%    lualatex \tikzexternalcheckshellescape -halt-on-error -interaction=batchmode -jobname="\image" "\texsource" || rm "\image.pdf"
%  },
%]

\begin{document}

%\feynmandiagram [large, horizontal=a to b] {
%    i1 [label=0:$\psi$] -- [fermion] a -- [fermion] i2 [label=0:$\bar{\psi}$],
%    a [label=90:$e$] -- [photon] b [label=90:$A_\mu$],
%    i1 -- [photon] i2,
%};
\feynmandiagram [small,baseline=(a.base),horizontal=z to y] {
    i1 -- [fermion] a [crossed dot] -- [anti fermion] i2,
    f1 -- [anti fermion] a -- [fermion] f2,
    i1 -- [opacity=0] z -- [opacity=0] i2,
    f1 -- [opacity=0] y -- [opacity=0] f2,
    z -- [opacity=0] a -- [opacity=0] y,
};
%        =
%        \feynmandiagram [horizontal=b to c,baseline=(c.base)] {
%            i3 -- [fermion] b [dot] -- [anti fermion] i4,
%            b -- [boson] c [dot],
%            f3 -- [anti fermion] c -- [fermion] f4,
%        };
%        +
%        \feynmandiagram [small,baseline=(a.base),horizontal=z to y] {
%            i1 -- [fermion] a [dot] -- [anti fermion] i2,
%            f1 -- [anti fermion] a -- [fermion] f2,
%            i1 -- [opacity=0] z -- [opacity=0] i2,
%            f1 -- [opacity=0] y -- [opacity=0] f2,
%            z -- [opacity=0] a -- [opacity=0] y,
%        };
%        +
%        \feynmandiagram [small,vertical=b to c,baseline=(c.base)] {
%            i3 -- [fermion] b [dot] -- [fermion] i4,
%            b -- [boson] c [dot],
%            f3 -- [fermion] c -- [fermion] f4,
%        }; 
%        +
%        \feynmandiagram [small,horizontal=b to c,baseline=(c.base)] {
%            i3 -- i5 [dot] -- b [dot] -- i4 [dot] -- i6,
%            b -- [boson] c [dot],
%            f3 -- c -- f4,
%            i5 -- [gluon] i4,
%            i3 -- [opacity=0] i6,
%        };
%        +
%        \feynmandiagram [small,horizontal=b to c,baseline=(f.base)] {
%            i1 -- a [dot] -- [boson] d [dot] -- i2,
%            i3 -- b [dot] -- [boson] c [dot] -- i4,
%            a -- [boson] b,
%            d -- [boson] c,
%            a -- [opacity=0] f -- [opacity=0] c,
%            b -- [opacity=0] f -- [opacity=0] d,
%        };
%        + \cdots

\end{document}
