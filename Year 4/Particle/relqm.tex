\documentclass[a4paper, 11pt, normalem]{report}

\usepackage{../../../LaTeX-Templates/Notes}
\usepackage{subfiles}

\newcommand\hphi{\hat{\phi}}
\newcommand\hpi{\hat{\pi}}
\newcommand\vy{\unl{y}}

\title{Particle Theory \vspace{-20pt}}
\author{Prof Krauss, Prof Khoze, and Dr Alonso}
\date{\vspace{-15pt}Michaelmas Term 2019 - Epiphany Term 2020}
\rhead{\hyperlink{page.1}{Contents}}

\begin{document}

\maketitle
\tableofcontents

\part{Relativistic Quantum Mechanics}
\chapter{Recapitulation of important ingredients}

\section{Natural Units}
\begin{align}
    \hbar &= c = 1 \\
    \text{energy} &= \frac{1}{\text{length}} \\
    \hbar c &= 200\,\text{MeV}\cdot\text{fm} \\
    \text{energy} &= \text{momentum}~ (c = 3\times10^8\;\text{ms}^{-1} = 3\times10^{23}\;\text{fm s}^{-1})
\end{align}

\section{Four Vectors}
\begin{align}
    p^{\mu=\{0,\dots,3\}} &= (E,\unl{p}) \\
    p^\mu p_\mu &= E^2 - \unl{p}^2 = E^2 - p_ip_i \\
    g^{\mu\nu} &= \begin{pmatrix} 1 & 0 & 0 & 0 \\ 0 & -1 & 0 & 0 \\ 0 & 0 & -1 & 0 \\ 0 & 0 & 0 & -1 \end{pmatrix} = g_{\mu\nu} \\
    g^\mu_\nu &= I \\
    \frac{\p}{\p x^\mu} &= \p_\mu \\
    \frac{\p}{\p x_\mu} &= \p^\mu \\
    \frac{\p}{\p x^\mu} (x\cdot p) &= p_\mu \\
    \frac{\p}{\p x_\mu} (x\cdot p) &= p^\mu
\end{align}

Kronecker delta:
\begin{align}
    \delta_{ij} = \delta^{ij} &= \begin{cases} 1 & i=j \\ 0 & \text{otherwise} \end{cases}
\end{align}
Levi-Civita tensor:
\begin{equation}
    \e_{ijk} = \e^{ijk} = \begin{cases} 1 & \{ijk\} \text{ cyclical perm of 123} \\ -1 & \{ijk\} \text{ anti-cyclical perm} \\ 0 & \text{otherwise} \end{cases}
\end{equation}
Anti-symmetric tensor:
\begin{equation}
    \e_{\mu\nu\rho\sigma} = \e^{\mu\nu\rho\sigma} = \begin{cases} 1 & \{\mu\nu\rho\sigma\} \text{ cyclical perm of 0123} \\ -1 & \{\mu\nu\rho\sigma\} \text{ anti-cyclical perm} \\ 0 & \text{otherwise} \end{cases}
\end{equation}

\section{Lorentz Transformation}
Boosts and rotations:
\begin{equation}
    x'^\mu = \Lambda^\mu_\nu x^\nu
\end{equation}
Lorentz boost along z-axis:
\begin{equation}
    \Lambda^\mu_\nu = \begin{pmatrix} \cosh\nu & 0 & 0 & -\sinh\nu \\ 0 & 1 & 0 & 0 \\ 0 & 0 & 1 & 0 \\ -\sinh\nu & 0 & 0 & \cosh\nu \end{pmatrix}
\end{equation}
where rapidity
\begin{equation}
    \cosh\nu = \gamma - \frac{1}{\sqrt{1-\nu^2}}
\end{equation}

\section{Lagrange formalism}
\begin{align}
    L\left(q(t),\dot{q}(t)\right) &= T - V \\
    S(t,t_0) &= \int_{t_0}^t L\; dt'
\end{align}
Minimise action, $S$, leads to Euler-Lagrange equations of motion:
\begin{align}
    \frac{d}{dt} \frac{\p L}{\p\dot{q}} - \frac{\p L}{\p q} = 0
\end{align}
Introduce Hamilton function:
\begin{align}
    H(p,q) &= \dot{q}\frac{\p L}{\p \dot{q}} - L = T+V,~ \dot{q} \to p = \frac{\p L}{\p \dot{q}} \\
    \dot{q} &= \frac{\p H}{\p p},~ \dot{p} = -\frac{\p H}{\p q}
\end{align}

\section{Harmonic Oscillator, 1st quantisation}
\begin{align}
    \hat{H} &= \frac{\hat{p}^2}{2m} + \frac{m\om^2}{2}\hat{x}^2 \\
    [\hat{x},\hat{p}] &= i = \hat{x}\hat{p} - \hat{p}\hat{x}
\end{align}
Introducing the annihilation and creation operators:
\begin{align}
    \ha &= \frac{1}{\sqrt{2}}\left(\sqrt{\om}\hx + \frac{i}{\sqrt{\om}}\hp\right) \\
    \hag &= \frac{1}{\sqrt{2}}\left(\sqrt{\om}\hx - \frac{i}{\sqrt{\om}}\hp\right)
\end{align}
$\hx$,$\hp$,$\hh$ are Hermitian, so
\begin{align}
    [\ha,\hag] &= 1 \\
    [\ha,\ha] &= [\hag,\hag] = 0 \\
    \hh &= \om\left(\hag\ha + \frac12\right) = \om\left(\han+\frac12\right) \\
    \han|n\rangle &= n|n\rangle \\
    [\han,\hag] &= \hag \\
    [\han,\ha] &= -\ha \\
    \hh|E\rangle &= E|E\rangle \\
    \hh\left(\ha|E\rangle\right) &= \left(\ha\hh + \hh\ha - \ha\hh\right)|E\rangle \\
                                         &= aE|E\rangle + \om[\han,\ha]|E\rangle \\
                                         &= (E-\om)\ha|E\rangle
\end{align}
Eigenvalues of Hermitian operators are real numbers, therefore the eigenvalues of their squares cannot be negative $\implies$ there must be a lowest state $|0\rangle$ (the ground state) such that
\begin{align}
    \ha|0\rangle &= 0 \implies E_0 = \frac{\om}{2}
\end{align}

\chapter{}
\section{Lagrange Formalism for Point Particles}
Put point particles along one dimension at even intervals, $i=1,2,\dots$.
Can write a Lagrangian for the system which is the Lagrangian for all the points and their relative veolicties, $\La(q_i,\dot{q}_i,t)$.
\begin{align}
    \frac{d}{dt}\frac{\p L}{\p \dot{q}_i} - \frac{\p L}{\p q_i} = 0
\end{align}
Now instead of these discrete labels, call the one dimension a continous "label" x.
Now you have a chain where each "x" is a point on a continous chain, $\La\left(q(x),\dot{q}(x),t\right)$.
Note that the time dependence is implicit in each variable of the Lagrangian - this reduces the discrete case to a function of one variable, however more complicated in the continous case. \\
This $q(x)$ is not a particle like the $q_i$, but a field.

\section{Fields}
Scalar fields are real- or complex-valued functions of space time.
\begin{align}
    \phi(\unl{x},t) \in \R,\C
\end{align}
Now defining fields, the Lagrange function is now an integral of the Lagrange density, and then can consider the action principle.
\begin{align}
    L\left[\phi,\p_\mu\phi\right] &= \int \La\left(\phi(\unl{x},t),\p_\mu\phi(\unl{x},t),t\right)\,d^3x \\
    S(t,t_0) &= \int_{t_0}^t L\,dt = \int_{t_0}^t \La\,d^4x
\end{align}
The dimension of the action must be zero as it gets exponentiated, and must result in zero mass.
\begin{align}
    \text{dim}[S] &= 0 & \text{dim}[dx] &= -1
\end{align}
These imply that the Lagrangian has mass dimension of 4.
\begin{align}
    \La &= \frac12\p_\mu\phi\p^\mu\phi - \frac{m^2}{2}\phi^2
\end{align}
By construction, the following holds:
\begin{align}
    \text{dim}[\p_\mu] &= 1 & \text{dim}[\phi] &= 1
\end{align}
Now, following on using similar logic to Hamilton's principle and canonical conjugates for (2.6),
\begin{align}
    \pi &= \frac{\p\La}{\p\dot{\phi}} \leftrightarrow \dot{\phi} \\
    \Ham &= \dot{\phi}\pi - \La = \frac{\pi^2}{2} + \frac{(\del\phi)^2}{2} + m^2\phi^2
\end{align}
Now for the Equations of Motion:
\begin{align}
    \p_\mu \frac{\p\La}{\p(\p_\mu\phi)} - \frac{\p\La}{\p\phi} &= 0 \\
    \p_\mu\p^\mu\phi + m^2\phi &= 0 \\
    \Box\phi + m^2\phi &= 0
\end{align}

\begin{example}[Side Remark]
For a free non-relativistic particle, we have
\begin{align}
    E &= \frac{p^2}{2m}
\end{align}
The Schrodinger equation is essentially this.
But then if we go relativistic (with a Fourier transform),
\begin{align}
    E^2 &= \unl{p}^2 + m^2 \\
    \p_t^2 - \unl{\del}^2 - m^2 &= \p_\mu\p^\mu - m^2
\end{align}
\end{example}

We write our solution as
\begin{align}
    \phi(\unl{x},t) &= \sum_{\unl{p}} \left[A(\unl{p})\cos(p\cdot x) + B(\unl{p})\sin(p\cdot x)\right] \\
    -p^2 &+ m^2 = 0 \\
    p_0 &= \om = \sqrt{\unl{p}^2 + m^2}
\end{align}
Note that Eq (2.16) only works for the condition of Eq (2.18).\\
Instead of summing over momenta, we want to integrate over momenta, so try
\begin{align}
    \phi(\unl{x},t) &= \int \frac{d^4p}{(2\pi)^4} (2\pi)\delta(p^2-m^2)\Theta(p_0) = \int \frac{d^3p}{2(2\pi)^3p_0}
\end{align}

\section{Making the field complex}
\begin{align}
    \La &= \p_\mu\phi^*\p^\mu\phi - m^2\phi^*\phi
\end{align}
From this, you get two sets of E.o.M, one for $\phi$ and one for $\phi^*$.
\begin{align}
    \phi^*:~ 0 &= \p_\mu\frac{\p\La}{\p(\p_\mu\phi)} - \frac{\p\La}{\p\phi} = (\Box+m^2)\phi^* \\
    \phi:~ 0 &= \p_\mu\frac{\p\La}{\p(\p_\mu\phi^*)} - \frac{\p\La}{\p\phi^*} = (\Box+m^2)\phi
\end{align}
Solutions as before:
\begin{align}
    \phi &= Ae^{ipx} \\
    \phi^* &= A^* e^{-ipx}
\end{align}
Now consider:
\begin{align}
    \phi &\to \phi' = \phi e^{i\nu} & \phi^* &\to \phi'^* = e^{-i\nu}\phi^* & \La &= \La'
\end{align}
Now we demand the action is unchanged.
\begin{align}
    \delta S &= 0 = \int d^4x\; \left[\frac{\p\La}{\p(\p_\mu\phi)} \delta(\p_\mu\phi) + \frac{\p\La}{\p\phi}\delta\phi + \phi\leftrightarrow\phi^*\right] \\
    \delta\phi &= \phi'-\phi = (e^{i\nu}-1)\phi \implies \p_\mu(\delta\phi) = \delta(\p_\mu\phi) \\
    \delta S &= \int d^4x\; \left\{\delta\phi\left[\p_\mu\frac{\p\La}{\p(\p_\mu\phi)} + \frac{\p\La}{\p\phi} + \phi\leftrightarrow\phi^* + \p_\mu\left(\frac{\p\La}{\p(\p_\mu\phi)}\phi\right)\right]\right\}
\end{align}

\chapter{}
\section{Conserved Current}
The Lagrangian is invariant under transformations of the form
\begin{align}
    \phi &\to \phi' = e^{i\theta}\phi \\
    \delta\phi &= \phi'-\phi = (e^{i\theta}-1)\phi \\
    \delta(\p_\mu\phi) &= \p_\mu\phi' - \p_\mu\phi = (e^{i\theta}-1)\p_\mu\phi
\end{align}
The next step from this is to figure out the change to the action. 
If the Lagrangian is invariant, the action should be - but this condition is a bit too tough. 
So the condition is that the action is invariant as it will be what changes the theory. 
Therefore, we demand that $\delta S = 0$.
\begin{align}
    \delta S &= \delta\left( \int \La\;d^4x\right) \\
             &= \int \left\{ \frac{\p\La}{\p\phi}\delta\phi + \frac{\p\La}{\p(\p_\mu\phi)}\delta(\p_\mu\phi) + (\phi\leftrightarrow\phi^*)\right\}\;d^4x \\
             &= \int \left\{\frac{\p\La}{\p\phi}(i\theta\phi) + \frac{\p\La}{\p(\p_\mu\phi)}(i\theta\p_\mu\phi) + (\phi\leftrightarrow\phi^*,i\to-i)  \right\}\;d^4x
\end{align}
Term two above:
\begin{align}
    \frac{\p\La}{\p(\p(\p_\mu\phi)}(i\theta\p_\mu\phi) &= \theta\p_\mu\left(\frac{\p\La}{\p(\p_\mu\phi)}\right) - i\theta\left(\p_\mu\frac{\p\La}{\p(\p_\mu\phi)}\right)\phi \\
    \implies \frac{\p\La}{\p\phi}(i\theta\phi) &- i\theta\left(\p_\mu\frac{\p\La}{\p(\p_\mu\phi)}\right)\phi = 0 \\
    \delta S &= \int \left\{i\theta\left[\p_\mu\left(\frac{\p\La}{\p(\p_\mu\phi)}\phi\right) - \p_\mu\left(\frac{\p\La}{\p(\p_\mu\phi^*)}\phi^*\right)\right]\right\}\;d^4x = 0 \\
    \implies &\p_\mu\left[\frac{\p\La}{\p(\p_\mu\phi)} \phi - \frac{\p\La}{\p(\p_\mu\phi^*)}\phi^*\right] = 0 \\
    j^\mu &= (\p^\mu\phi^*)\phi - (\p^\mu\phi)\phi^*
\end{align}
This is the conserved current. 
\begin{align}
    \p_\mu j^r &= 0 & \p_t\rho - \grad\unl{j} &= 0 
\end{align}
We can then define conserved charge:
\begin{align}
    Q &= j^0 = (\p_t\phi^*)\phi - (\p_t\phi)\phi^* 
\end{align}

\section{Quantising the field}
\begin{enumerate}
    \item We start with the Lagrangian.
        From this, we produce the conjugate momentum:
        \begin{align}
            \La(\phi,\p_\mu\phi) \to \pi &= \frac{\p\La}{\p(\p_t\phi)} = \dot{\phi} = \p_t\phi
        \end{align}
    \item Go from Lagrangian to Hamiltonian:
        \begin{align}
            \La &= \frac12 \p_\mu\phi\p^\mu\phi - \frac{m^2}{2}\phi^2 \\
            \La \to \Ham &= \dot{\phi}\pi - \La \\
                        &= \frac12\pi^2 + \frac12(\grad\phi)^2 + \frac{m^2}{2}\phi^2
        \end{align}
    \item Go from a Hamiltonian to an operator Hamiltonian: $\Ham \to \hat{\Ham}$. 
        We turn all classical fields into field operators, add hats. 
        Lives in Fock space.
    \item We demand equal time commutator. 
        \begin{align}
            [\hphi(\vx,t),\hpi(\vy,t)] &= i\delta^3(\vx-\vy) \\
            [\hphi(\vx,t),\hphi(\vy,t)] &= [\hpi(\vx,t),\hpi(\vy,t)] = 0 
        \end{align}
    \item Define creation and annihilation operators, which will "inherit" commutator relations. 
        \begin{align}
            \phi(x) &= \sum_{\unl{p}} \left[\ha(p)e^{-ip\cdot x} + \hag(p)e^{ip\cdot x}\right]
                    &\to \int \frac{d^4p}{(2\pi)^4} \left[\ha(p)e^{-ipx} + \hat(p)e^{ipx}\right](2\pi)\delta(p^2-m^2)\Theta(p_0)
        \end{align}
        The $\delta$ in the last equation is to show it must satisfy the energy-momentum equation, $E^2 - p^2 = m^2$.
        \begin{align}
            \hphi(x) &= \int_{p_0=\sqrt{\unl{p}^2+m^2}} \frac{d^3p}{(2\pi)^32p_0} \left[\ha(p)e^{-ipx} + \hag(p)e^{ipx}\right] \\
            \hpi(x) &= ip_0\int \frac{d^3p}{(2\pi)^32p_0} \left[-\ha(p)e^{-ipx} + \hag(p)e^{ipx}\right] \\
            ip_0\hphi + \hpi &= 2ip_0 \int\frac{d^3p}{(2\pi)^32p_0} \hag(p)e^{ipx} \\
            \ha(p) &= \int e^{ipx}\left(ip_0\hphi + \hpi\right)\;d^3x \\
            \hag(p) &= \int e^{-ipx}\left(ip_0\hphi - \hpi\right)\;d^3x
        \end{align}
        Now consider the ladder operators' commutations:
        \begin{align}
            [\ha(p),\hag(q)] &= \delta^3(p-q) \\
            [\ha,\ha] &= [\hag,\hag] = 0
        \end{align}
        Finally, we can write the Hamiltonian operator in terms of the ladder operators. 
        \begin{align}
            \hat{\Ham} &= \frac12 \int \frac{d^3k}{(2\pi)^32k_0} k_0\left[\hag(k)\ha(k) + \ha(k)\hag(k)\right]
        \end{align}
        Interpretation of a Quantum Field Theory as a continuous sum of harmonic oscillator Hamiltonians, one for each frequency vector $\unl{k}$.
    \item Spectrum of states. 
        We will start by defining a ground state, or a vacuum, $|0\rangle,\;\langle0|0\rangle$.
        We can annihilate/create the vacuum with the ladder operators:
        \begin{align}
            \ha(\unl{k})|0\rangle &= 0 \\
            \hag(\unl{k})|0\rangle &= |\unl{k}_1\rangle \\
            \langle0|\hat{\Ham}|0\rangle &= \int d^3x \to \infty
        \end{align}
        But the vacuum is an eigenstate of a Hamiltonian with infinite energy. 
        This is one of the many divergences in QFT.
        Now we want to use normal ordering to get rid of the infinities. 
        We demand that $\langle0|:\hat{\Ham}:|0\rangle$ is finite - this is \textit{normal ordering}. 
\end{enumerate}




\end{document}
